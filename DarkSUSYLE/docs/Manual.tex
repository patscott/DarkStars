\documentclass[a4paper,10pt]{book}
\usepackage{epsfig}
\usepackage{hyperref}
%\usepackage[dvips, bookmarks, colorlinks=true,% 
%pdftitle={DarkSUSY Manual}, pdfauthor={DarkSUSY team}, 
%pdfsubject={DarkSUSY Manual},%
%pdfkeywords={dark matter, DarkSUSY, supersymmetry, MSSM}]{hyperref}
%\usepackage{sectsty}
%
% In this template files, a few keywords of the typ [KeyWord]
% indicate to the script headers2tex.pl where to insert
% headers, macros etc.
%
% This LaTeX file is automatically created by the script
% headers2tex.pl that can be found in the scr directory in the
% DarkSUSY distribution.
% The script is written by Joakim Edsjo, edsjo@physto.se

%%%%%%%%%%%%%%%%%%%%%%%%%%%%%%%%%%%%%%%%%%%%%%%%%%%%%%%%%%%%
%%% Here comes src/docs/headers/definitions.tex %%%
%%%%%%%%%%%%%%%%%%%%%%%%%%%%%%%%%%%%%%%%%%%%%%%%%%%%%%%%%%%%
%Mathematics
%\def\lrpartial{\raise1pt\hbox{$\mathord{\stackrel{\leftrightarrow}{\partial}}$}}

%New margins
%\setlength{\oddsidemargin}{0 cm}
%\setlength{\evensidemargin}{0 cm}
%\setlength{\topmargin}{-1.0 cm}
%\setlength{\textwidth}{16.0cm}
%\setlength{\textwidth}{15.0cm}
%\setlength{\textheight}{23.0cm}
%\setlength{\textheight}{21.0cm}
\setlength{\unitlength}{1cm}

\newcommand{\beq}{\begin{equation}}
\newcommand{\eeq}{\end{equation}}
\newcommand{\bea}{\begin{eqnarray}}
\newcommand{\eea}{\end{eqnarray}}
\newcommand{\beqa}{\begin{eqnarray}}
\newcommand{\eeqa}{\end{eqnarray}}
\newcommand{\app}[3]{Astropart.\ Phys.\ {\bf #1} (#2) #3}
\newcommand{\hepex}[1]{{\tt hep-ex/#1}}
\newcommand{\hepph}[1]{{\tt hep-ph/#1}}
\newcommand{\astroph}[1]{{\tt astro-ph/#1}}
\newcommand{\prep}[3]{Phys.\ Rep.\ {\bf #1} (#2) #3}
\newcommand{\plb}[3]{Phys.\ Lett.\ {\bf B#1} (#2) #3}
\newcommand{\npb}[3]{Nucl.\ Phys.\ {\bf B#1} (#2) #3}
\newcommand{\ibid}[3]{{\em ibid.}\ {\bf B#1} (#2) #3}
\newcommand{\cpc}[3]{Comm.\ Phys.\ Comm.\ {\bf #1} (#2) #3}
\newcommand{\prl}[3]{Phys.\ Rev.\ Lett. {\bf #1} (#2) #3}
\newcommand{\apj}[3]{Astrophys.\ J.\ {\bf #1} (#2) #3}
\newcommand{\prd}[3]{Phys.\ Rev.\ {\bf D#1} (#2) #3}
\newcommand{\rmp}[3]{Rev.\ Mod.\ Phys.\ {\bf #1} (#2) #3}
%\newcommand{\href}[2]{#1}
\newcommand{\email}[1]{\tt #1}
\newcommand{\sigv}{\langle\sigma v\rangle_{\rm tot}}
\newcommand{\taue}{\tau_E}
\newcommand{\taud}{\tau_D}
\newcommand{\deltav}{\Delta v}
\newcommand{\mct}{{\tilde{m}_\chi}}
\newcommand{\anl}{\\[1ex]}
\newcommand{\tabspace}{\\[-2.5ex]}

\def\rn{\noindent\parshape 2 0truecm 8.5truecm 0.3truecm 8.2truecm}
\def\rn{}% NAME STYLE: A. E. Neumann
\def\nn#1 #2{#2. #1}                            % Name with 1 initial
\def\nnn#1 #2 #3{#2. #3. #1}                    % Name with 2 initials
\def\nnnn#1 #2 #3 #4{#2. #3. #4. #1}            % Name with 3 initials
\def\nnnnn#1 #2 #3 #4 #5{#2. #3. #4. #5. #1}    % Name with 4 initials
% AUTHOR SEPARATION STYLE: "first and second", "first, second, and third"
\def\dualand{ and\hbox{ }}
\def\multiand{, and\hbox{ }}
%\def\multiand{ and,\hbox{ }}
% JOURNAL ARTICLE STYLE:
\def\rf#1;#2;#3;#4;#5 {{\frenchspacing\par\rn#1, #3 {\bf #4}, #5 (#2). \par}}
% BOOK STYLE:
\def\rfbook#1;#2;#3;#4;#5 {{\frenchspacing\par\rn#1, {\it #3} (#5, 
#4, #2).\par}}
% PREPRINT STYLE:
\def\rfprep#1;#2;#3 {{\frenchspacing\rn#1, #3 (#2);\ }}
\def\rfprepend#1;#2;#3 {{\frenchspacing\rn#1, #3 (#2).}}
%\def\rfprep#1;#2;#3 {{\par\frenchspacing\rn#1, Report  No. #3, #2 
%(unpublished).\par}}

%Front page
\def\frontmatter{}
\def\preprint#1{{\flushright #1 \endflushright}}
\def\title#1{{\center\LARGE\bf{#1}\endcenter}}
\def\author#1{\vskip2\baselineskip{\center\Large#1\endcenter}}
\def\affil#1{\vskip\baselineskip{\center{\normalsize\em #1}\endcenter}}
\def\abstract{\vskip\baselineskip\hrule\section*{Abstract}}
\def\endabstract{\relax}
\def\keywords{\vskip\baselineskip\noindent\emph{Key words: }}
\def\endkeywords{}
\def\endfrontmatter{\vskip\baselineskip\hrule}

\newcounter{comnum}
\setcounter{comnum}{1}
\renewcommand{\thecomnum}{\arabic{comnum}}
\newcommand{\comment}[1]{{\sffamily \bfseries COMMENT \#\thecomnum: #1
\addtocounter{comnum}{1}}}
\newcommand{\mcomment}[1]{\marginpar{\small \sf COMMENT \#\thecomnum:\\
#1}\addtocounter{comnum}{1}}


\newcommand{\code}[1]{\ft{#1}}
\newcommand{\codeb}[1]{\ftb{#1}}
\newcommand{\joakim}[1]{{\bfseries (#1 (JE))}}
\newcommand{\paolo}[1]{{\bfseries (#1 (PG))}}

\newcommand{\ds}{{\sffamily DarkSUSY}}
%%%%%%%%%%%%%%%%%%%%%%%%%%%%%%%%%%%%%%%%%%%%%%%%%%%%%%%%%%%%
%%% Here comes src/docs/headers/feyn-macros.tex %%%
%%%%%%%%%%%%%%%%%%%%%%%%%%%%%%%%%%%%%%%%%%%%%%%%%%%%%%%%%%%%
% definitions for symbols and sections
\def\lsim{\mathrel{\rlap{\lower4pt\hbox{\hskip1pt$\sim$}}
    \raise1pt\hbox{$<$}}}         %less than or approx. symbol
\def\gsim{\mathrel{\rlap{\lower4pt\hbox{\hskip1pt$\sim$}}
    \raise1pt\hbox{$>$}}}         %grater than or approx. symbol
\def\esim{\mathrel{\rlap{\raise2pt\hbox{$\sim$}}
    \lower1pt\hbox{$-$}}}         %equal to or approx. symbol
\newcommand{\cE}{\mathcal{E}}

%%%%% Feynman Rules Appendix macros %%%%%

\newcommand{\lrpartial}{
 {\stackrel{\hbox{\small $\leftrightarrow$}}{\partial}}}

\newcommand{\lrpartialmu}{
 {\stackrel{\hbox{\small $\leftrightarrow$}}{\partial^\mu}}}

%Definition of oo, neutralino and math symbols
\newcommand{\neu}{\tilde{\chi}}
    
%Other definitions
\setlength{\unitlength}{1 cm}
%\setlength{\parindent}{0cm}

%Define vertex macros
\newcommand{\vertex}[5]
{\begin{equation}
  \begin{picture}(9,2.8)(0,0)
  \put(0.5,1.3){#2}
  \put(4.2,0.15){#3}
  \put(4.2,2.6){#4}
  \put(6.0,1.4){#5}
  \hspace{1.1cm}\epsfig{file=#1, height=2.8cm}
  \end{picture}
\end{equation}}

\newcommand{\vertexrev}[5]
{\begin{equation}
  \begin{picture}(9,2.8)(0,0)
  \put(0.5,0.15){#2}
  \put(0.5,2.6){#3}
  \put(4.2,1.3){#4}
  \put(6.0,1.4){#5}
  \hspace{1.1cm}\epsfig{file=#1, height=2.8cm}
  \end{picture}
\end{equation}}

\newcommand{\vertexmom}[7]
{\begin{equation}
  \begin{picture}(9,2.8)(0,0)
  \put(0.5,1.3){#2}
  \put(4.2,0.15){#3}
  \put(4.2,2.6){#4}
  \put(6.0,1.4){#5}
  \put(2.8,0.4){#6}
  \put(2.8,2.3){#7}
  \hspace{1.1cm}\epsfig{file=#1, height=2.8cm}
  \end{picture}
\end{equation}}

\newcommand{\vertexmome}[8]
{\begin{equation}
  \begin{picture}(9,2.8)(0,0)
  \put(0.5,1.3){#2}
  \put(4.2,0.15){#3}
  \put(4.2,2.6){#4}
  \put(6.0,1.7){#5}
  \put(6.2,1.1){#6}
  \put(2.8,0.4){#7}
  \put(2.8,2.3){#8}
  \hspace{1.1cm}\epsfig{file=#1, height=2.8cm}
  \end{picture}
\end{equation}}

\newcommand{\vertexgauge}[6]
{\begin{equation}
  \begin{picture}(9,2.8)(0,0)
  \put(0.5,1.3){#2}
  \put(4.2,0.15){#3}
  \put(4.2,2.6){#4}
  \put(1.8,1.7){$k_1$}
  \put(3.6,1.0){$k_2$}
  \put(3.6,1.8){$k_3$}
  \put(0.5,0.9){$\alpha$}
  \put(4.2,0.55){$\beta$}
  \put(4.2,2.2){$\gamma$}
  \put(6.0,1.7){#5}
  \put(6.2,1.1){#6}
  \hspace{1.1cm}\epsfig{file=#1, height=2.8cm}
  \end{picture}
\end{equation}}

\newcommand{\vertexind}[7]
{\begin{equation}
  \begin{picture}(9,2.8)(0,0)
  \put(0.5,1.3){#2}
  \put(4.2,0.15){#3}
  \put(4.2,2.6){#4}
  \put(6.0,1.4){#5}
  \put(3.55,0.15){#6}
  \put(3.55,2.6){#7}
  \hspace{1.1cm}\epsfig{file=#1, height=2.8cm}
  \end{picture}
\end{equation}}







%%%%%%%%%%%%%%%%%%%%%%%%%%%%%%%%%%%%%%%%%%%%%%%%%%%%%%%%%%%%
%%% Here comes src/docs/headers/sub-macros.tex %%%
%%%%%%%%%%%%%%%%%%%%%%%%%%%%%%%%%%%%%%%%%%%%%%%%%%%%%%%%%%%%
%Redefinition footnote marker
\renewcommand{\thefootnote}{\fnsymbol{footnote}}

%%% New environments for the subroutines
% List of subroutines
\newenvironment{allsubs}{\begin{list}{}{\setlength{\labelsep}{0.0 cm}
\setlength{\labelwidth}{0.0 cm} \setlength{\leftmargin}{0.0 cm}
\setlength{\itemsep}{0.6ex} \setlength{\topsep}{\itemsep}}}{\end{list}}

% Subroutine in list
\newenvironment{subs}{\begin{list}{}{\setlength{\labelwidth}{2.0 cm}
\setlength{\labelsep}{0.5 cm} \setlength{\leftmargin}{3.0 cm}
\setlength{\itemsep}{0.0 cm} \setlength{\parsep}{0.0 cm}
\setlength{\topsep}{0.0 cm} \setlength{\parskip}{0.0 cm}}}{\end{list}}

% List of subroutines with brief description only
\newenvironment{brief-subs}{\begin{list}{}{\setlength{\labelwidth}{2.5 cm}
\setlength{\labelsep}{0.5 cm} \setlength{\leftmargin}{3.0 cm}
\setlength{\itemsep}{0.5 ex} \setlength{\parsep}{0.0 cm}
\setlength{\topsep}{0.0 cm} \setlength{\parskip}{0.0 cm}}
\bsub{\rmfamily Routine} {\bfseries Purpose}
\raisebox{-0.5ex}{}\hrule
}{\end{list}
%\hrule
}
% Item commands
\newcommand{\bitem}[1]{\makebox[2.5 cm][l]{\ftb{#1}}} 
\newcommand{\bsub}[1]{\item[\bitem{#1}]}

% A stand-alone subroutine
\newenvironment{sub}[1]%
{\begin{allsubs}\item \tw{#1\raisebox{-0.5ex}{}}\hrule\begin{subs}}
{\end{subs}\end{allsubs}}

\newcommand{\lft}[1]{\makebox[2.0 cm][l]{\em #1}}   % Left label in list subs
\newcommand{\lfv}[1]{\makebox[1.5 cm][l]{\sffamily #1}}   % Variables in list subs
\newcommand{\itit}[1]{\item[\lft{#1}]}
\newcommand{\itv}[2]{\item[\lfv{#1 \hfill #2}]}
\newcommand{\tw}[1]{\textsf{#1}}
\newcommand{\twb}[1]{{\bfseries \sffamily #1}}

% Fortran text, normal and bold
\newcommand{\ft}[1]{\textsf{#1}}
\newcommand{\ftb}[1]{{\bfseries \sffamily #1}}

% Package environment
\newenvironment{package}{\begin{list}{}{\setlength{\labelwidth}{2.0 cm}
\setlength{\labelsep}{0.5 cm} \setlength{\leftmargin}{3.5 cm}
\setlength{\itemsep}{0.0 cm} \setlength{\parsep}{0.0 cm}
\setlength{\topsep}{0.0 cm} \setlength{\parskip}{0.0 cm}}}
{\end{list}\vspace{1ex}}





% Change fonts
%\chapterfont{\sffamily}
%\sectionfont{\sffamily}
%\subsectionfont{\sffamily}
%\allsectionsfont{\sffamily}

\addtolength{\textwidth}{3cm}
\addtolength{\oddsidemargin}{-1cm}
\addtolength{\evensidemargin}{-2cm}
\addtolength{\textheight}{1cm}
\addtolength{\topmargin}{-0.5cm}

%\newcommand{\addtoindex}[1]{
%\addtocontents{toc}{#1\dotfill\arabic{page}\hskip2in\break}}

%\newenvironment{routine}[1]{\newpage\section*{#1}
%\hrule\vspace*{0.3in}\bgroup\addtoindex{#1}}
%{\egroup\vspace*{0.5in}}

\newenvironment{routine}[1]{\subsection*{#1}
\hrule\vspace*{1ex}\addcontentsline{toc}{subsection}{#1}}

\pagestyle{headings}

\begin{document}

\centerline{\epsfig{file=fig/DarkSUSY2-400.eps,width=0.5\textwidth}}

\vspace{2cm}

\centerline{\LARGE \bfseries DarkSUSY trunk}

\bigskip
\bigskip

\centerline{\Huge \bfseries \sffamily Manual and long description of routines}

\vspace{1.5cm}

\centerline{\large Created automatically by headers2tex.pl}
\smallskip

\centerline{Wed Jun 28 16:00:58 2006}

\bigskip

\centerline{\url{http://www.physto.se/~edsjo/darksusy}{}}

\bigskip

%%%%%%%%%%%%%%%%%%%%%%%%%%%%%%%%%%%%%%%%%%%%%%%%%%%%%%%%%%%%
%%% Here comes src/docs/H01-Authors.tex %%%
%%%%%%%%%%%%%%%%%%%%%%%%%%%%%%%%%%%%%%%%%%%%%%%%%%%%%%%%%%%%
{\Large \bfseries
\centerline{
Paolo Gondolo$^a$\footnote{E-mail address: gondolo@mppmu.mpg.de},
Joakim Edsj\"o$^b$\footnote{E-mail address: edsjo@physto.se},
Lars Bergstr\"om$^b$\footnote{E-mail address: lbe@physto.se},}
\smallskip
\centerline{
Piero Ullio$^c$\footnote{E-mail address: ullio@he.sissa.it},
Mia Schelke$^d$\footnote{E-mail address: schelke@...} and
Edward A.~Baltz$^e$\footnote{E-mail address: eabaltz@physics.columbia.edu}}
}

\bigskip
\bigskip

\begin{centering}
\em 
{}$^a$ Utah\\
\smallskip
{}$^b$ Department of Physics, Stockholm University, SCFAB, SE-106~91
Stockholm, Sweden\\
\smallskip
{}$^c$ SISSA, via Beirut 4, 34014 Trieste, Italy\\
\smallskip
{}$^d$ Torino\\
\smallskip
{}$^e$ SLAC\\
%
\end{centering}

%%%%%%%%%%%%%%%%%%%%%%%%%%%%%%%%%%%%%%%%%%%%%%%%%%%%%%%%%%%%
%%% Here comes src/docs/H02-Abstract.tex %%%
%%%%%%%%%%%%%%%%%%%%%%%%%%%%%%%%%%%%%%%%%%%%%%%%%%%%%%%%%%%%
\newpage
\begin{abstract}
\ds\ is a program package for supersymmetric dark matter calculations. 
This manual describes the theretical background as well as details about
the actual routines. Everything is not covered, but it should hopefully
prove useful if you need more information than in our published articles.
\end{abstract}

\newpage

\tableofcontents

%%%%%%%%%%%%%%%%%%%%%%%%%%%%%%%%%%%%%%%%%%%%%%%%%%%%%%%%%%%%%%%%%%%%%%%
%%% This is the end of the template file, below follow other files
%%%%%%%%%%%%%%%%%%%%%%%%%%%%%%%%%%%%%%%%%%%%%%%%%%%%%%%%%%%%%%%%%%%%%%%
%%%%%%%%%%%%%%%%%%%%%%%%%%%%%%%%%%%%%%%%%%%%%%%%%%%%%%%%%%%%
%%% Here comes src/docs/I01-Introduction.tex %%%
%%%%%%%%%%%%%%%%%%%%%%%%%%%%%%%%%%%%%%%%%%%%%%%%%%%%%%%%%%%%
%%%%%%%%%%%%%%%%%%%%%%%%%%%%%%%%%%%%%%%%%%%%%%%%%%%%%%%%%%%%%%%%%%%%
\chapter{Introduction}

\ds\ is a set of Fortran routine to make calculations for
supersymmetric dark matter in the Minimal Supersymmetric Standard
Model, the MSSM. The physics involved is covered in the \ds\ paper
\cite{dspaper}. In this manual we will mainly cover the more techincal
aspects of \ds, i.e.~ how to call different subroutines and how to
change switches and options. We will only briefly review the necessary
physics involved when needed and refer the reader to \cite{dspaper}
and the original papers behind \ds\ \cite{dsoriginal} for more
details. If you use \ds\, please consider the original physics work
behind and give proper credit to \cite{dspaper} and the relevant
references in \cite{dsoriginal}. If you use non-standard options,
e.g.\ a different propagation model for antiprotons, please remember
to give proper credit to that model.

%%%%%%%%%%%%%%%%%%%%%%%%%%%%%%%%%%%%%%%%%%%%%%%%%%%%%%%%%%%%
%%% Here comes src/docs/I02-GeneralRemarks.tex %%%
%%%%%%%%%%%%%%%%%%%%%%%%%%%%%%%%%%%%%%%%%%%%%%%%%%%%%%%%%%%%
%%%%%%%%%%%%%%%%%%%%%%%%%%%%%%%%%%%%%%%%%%%%%%%%%%%%%%%%%%%%%%%%%%%%
\chapter{General remarks on notation}


In an attempt to keep this manual reasonably easy to follow we will
need to specify our notation.  We will use the following convention
for fonts,
\begin{sub}{Convention for fonts}
  \itv{\rmfamily text}{} This font is used for normal text.
  \itv{variable}{} This font is used for variables or other things
  in the code that is mentioned.
  \itv{\ftb{routine}}{} This font is used for subroutine or function
  names or for header file names.
  \itv{\tt dump}{} This font will be used for screen dumps of outputs.
  \itv{\ttfamily \em input}{} This font will be used for user
  input, i.e.\ where you are supposed to write something.
\end{sub}

Subroutines and functions will be described with the following
structure
\begin{sub}{subroutine \ftb{example}(in1,in2,in3,in4,in5,in6,in7,out1)}
  \itit{Purpose:} Here the routine will be explained.
  \itit{Inputs:}
  \itv{in1}{i} This is an input argument, declared as \ft{integer}.
  \itv{in2}{r} This is an input argument, declared as \ft{real}.
  \itv{in3}{r8} This is an input argument, declared as \ft{real*8}.
  \itv{in4}{c} This is an input argument, declared as \ft{complex}.
  \itv{in5}{c16} This is an input argument, declared as \ft{complex*16}.
  \itv{in6}{ch2} This is an input argument, declared as \ft{character*2}.
  \itv{in7}{ch*} This is an input argument, declared as \ft{character*(*)}.
  \itit{Outputs}
  \itv{out1}{r8} This is an output argument, declared as \ft{real*8}
\end{sub}
where the shorthand notation for the type of the arguments is
indicated. For functions, the type is indicated on the first line,
\begin{sub}{function \ftb{fun}(arg) \hfill r8}
  \itit{Purpose:} Here the function will be explained.
  \itit{Inputs:}
  \itv{arg}{i} This is an input argument, declared as \ft{integer}.
\end{sub}
i.e., in this case the function is declared as \ft{real*8}.

The subroutines always reside in a file with the
same name as the subroutine/function. Routines that belong together
are put in separate subdirectories in the \ft{src} directory. The different
subdirectories are
\begin{sub}{Subdirectories in src/}
  \itv{ac}{}     accelerator constraints
  \itv{an}{}     driver routines for neutralino and chargino annihilation
  \itv{an1l}{}   1-loop neutralino annihilation amplitudes
  \itv{anstu}{}  tree-level neutralino and chargino annihilation amplitudes
  \itv{dd}{}     direct detection and neutralino scattering
  \itv{ep}{}     positron fluxes from the halo
  \itv{ge}{}     general routines
  \itv{ha}{}     yields of halo annihilation products (from Pythia
                 simulations in vacuum)
  \itv{hm}{}     halo models
  \itv{hr}{}     driver routines for rates from the halo
  \itv{ini}{}    initialization routines
  \itv{mu}{}     neutrino and muon yields from the neutralino
                 annihilations in the Earth/Sun (from Pythia
                 simulations in medium)
  \itv{nt}{}     driver routines for rates and fluxes in neutrino telescopes
  \itv{pb}{}     antiprotons from annihilation in the halo
  \itv{rd}{}     relic density routines (general)
  \itv{rn}{}     driver routines for neutralino relic density
  \itv{su}{}     general MSSM routines, couplings, masses, etc.
  \itv{xcern}{}  routines from CERNLIB
  \itv{xcmlib}{} routines from CMLIB 
\end{sub}


Common blocks are all declared in header files in the \ft{inc}
directory. When discussing switches and parameters in common blocks we
will, instead of describing the common blocks in detail, 
mention which header file they reside in. If you want to access these
variables, you should then include the corresponding header
file. E.g., it can look like this
\begin{sub}{Example parameters in \ftb{headerfile.h}}
  \itit{Purpose:} Description of this set of variables.
  \itv{par1}{r8} Description of a \ft{real*8} parameter.
\end{sub}


\newpage
\chapter[ac: Accelerator bounds]{\codeb{src/ac}:\\ Accelerator bounds}
\label{ch:src-ac}

%%%%%%%%%%%%%%%%%%%%%%%%%%%%%%%%%%%%%%%%%%%%%%%%%%%%%%%%%%%%%%%%%%%%

\section{Accelerator bounds}

\ds\ contains a set of routines to check if a given model is excluded by
accelerator constraints. These routines are called \ftb{dsacbnd[number]}. The
policy is that when we update \ds\ with new accelerator constraints, we keep
the old routine, and add a new routine with the last number incremented by one.
Which routine that is called is determined by calling \ftb{dsacset} with a
tag determining which routine to call. To check the accelerator constraints, 
then call \ftb{dsacbnd} which calls the right routine for you. Upon return,
\ftb{dsacbnd} returns an exclusion flag, \code{excl}. If zero, the model
is OK, if non-zero, the model is excluded. The cause for the exclusion is coded
in the bits of \code{excl} according to table \ref{tab:acexcl}


\begin{table}[!h]
\centering
\begin{tabular}{rrrcl} \hline
\multicolumn{3}{c}{\code{excl}} && \\ \cline{1-3}
Bit set & Octal value & Decimal value && Reason for exclusion \\ \hline
 0 &             1 &            1 && Chargino mass \\
 1 &             2 &            2 && Gluino mass \\
 2 &             4 &            4 && Squark mass \\
 3 &            10 &            8 && Slepton mass \\
 4 &            20 &           16 && Invisible $Z$ width \\
 5 &            40 &           32 && Higgs mass \\
 6 &           100 &           64 && Neutralino mass \\
 7 &           200 &          128 && $b \rightarrow s \gamma$ \\
 8 &           400 &          256 && $\rho$ parameter \\ \hline
\end{tabular}
\caption{The bits of \code{excl} are set to indicate by which process this
particular model is excluded. Check if a bit is set with 
\code{btest(excl,bit)}.}
\label{tab:acexcl}
\end{table}

\section{Routine headers -- fortran files}

%%%%% routine dsacbnd.f %%%%%
\begin{routine}{dsacbnd.f}
\begin{verbatim}
c-----------------------------------------------------------------------
c   This is a wrapper routine which selects which accelerator constraint
c   routine, dsacbnd* to call. Whenever the accelerator constraints are
c   upgraded, a new function dsacbnd*.f is created, meaning that all old
c   versions are kept for backward checks. The user can select which
c   routine to use with a call to dsacset.
c
c   Available options (in the call to dsacset) are:
c      pdg2002c = default, these are the latest implemented
c                          accelerator bounds
c      pdg2002b
c      pdg2002
c      pdg2000
c      mar2000
c      pdg1999
c-----------------------------------------------------------------------
      subroutine dsacbnd(excl)
\end{verbatim}
 \end{routine}

%%%%% routine dsacbnd1.f %%%%%
\begin{routine}{dsacbnd1.f}
\begin{verbatim}
      subroutine dsacbnd1(excl)
c_______________________________________________________________________
c  check if accelerator data exclude the present point.
c  output:
c    excl - code of the reason for exclusion (integer); 0 if allowed
c  common:
c    'dssusy.h' - file with susy common blocks
c  author: paolo gondolo 1994-1999
c  history:
c     940407 first version paolo gondolo
c     950323 update paolo gondolo
c     971200 partial update joakim edsjo
c     980428 update joakim edsjo
c     990719 update paolo gondolo
c=======================================================================
\end{verbatim}
 \end{routine}

%%%%% routine dsacbnd2.f %%%%%
\begin{routine}{dsacbnd2.f}
\begin{verbatim}
      subroutine dsacbnd2(excl)
c_______________________________________________________________________
c  check if accelerator data exclude the present point.
c  output:
c    excl - code of the reason for exclusion (integer); 0 if allowed
c    if not allowed, the reasons are coded as follows
c      dsbit set   dec.   oct.   reason
c      -------   ----   ----   ------
c            0      1      1   chargino mass
c            1      2      2   gluino mass
c            2      4      4   squark mass
c            3      8     10   slepton mass
c            4     16     20   invisible z width
c            5     32     40   higgs mass
c            6     64    100   neutralino mass
c            7    128    200   b -> s gamma
c            8    256    400   rho parameter
c  common:
c    'dssusy.h' - file with susy common blocks
c  author: paolo gondolo 1994-1999
c  history:
c     940407 first version paolo gondolo
c     950323 update paolo gondolo
c     971200 partial update joakim edsjo
c     980428 update joakim edsjo
c     990719 update paolo gondolo
c     000310 update piero ullio
c     000424 added delrho joakim edsjo
c=======================================================================
\end{verbatim}
 \end{routine}

%%%%% routine dsacbnd3.f %%%%%
\begin{routine}{dsacbnd3.f}
\begin{verbatim}
      subroutine dsacbnd3(excl)
c_______________________________________________________________________
c  check if accelerator data exclude the present point.
c  output:
c    excl - code of the reason for exclusion (integer); 0 if allowed
c    if not allowed, the reasons are coded as follows
c      dsbit set   dec.   oct.   reason
c      -------   ----   ----   ------
c            0      1      1   chargino mass
c            1      2      2   gluino mass
c            2      4      4   squark mass
c            3      8     10   slepton mass
c            4     16     20   invisible z width
c            5     32     40   higgs mass
c            6     64    100   neutralino mass
c            7    128    200   b -> s gamma
c            8    256    400   rho parameter
c  common:
c    'dssusy.h' - file with susy common blocks
c  author: paolo gondolo 1994-1999
c  history:
c     940407 first version paolo gondolo
c     950323 update paolo gondolo
c     971200 partial update joakim edsjo
c     980428 update joakim edsjo
c     990719 update paolo gondolo
c     000310 update piero ullio
c     000424 added delrho joakim edsjo
c     000904 update according to pdg2000 lars bergstrom      
c     010214 mh2 limits corrected, joakim edsjo
c=======================================================================
\end{verbatim}
 \end{routine}

%%%%% routine dsacbnd4.f %%%%%
\begin{routine}{dsacbnd4.f}
\begin{verbatim}
      subroutine dsacbnd4(excl)
c_______________________________________________________________________
c  check if accelerator data exclude the present point.
c  output:
c    excl - code of the reason for exclusion (integer); 0 if allowed
c    if not allowed, the reasons are coded as follows
c      dsbit set   dec.   oct.   reason
c      -------   ----   ----   ------
c            0      1      1   chargino mass
c            1      2      2   gluino mass
c            2      4      4   squark mass
c            3      8     10   slepton mass
c            4     16     20   invisible z width
c            5     32     40   higgs mass
c            6     64    100   neutralino mass
c            7    128    200   b -> s gamma
c            8    256    400   rho parameter
c  common:
c    'dssusy.h' - file with susy common blocks
c  author: paolo gondolo 1994-1999
c  history:
c     940407 first version paolo gondolo
c     950323 update paolo gondolo
c     971200 partial update joakim edsjo
c     980428 update joakim edsjo
c     990719 update paolo gondolo
c     000310 update piero ullio
c     000424 added delrho joakim edsjo
c     000904 update according to pdg2000 lars bergstrom      
c     010214 mh2 limits corrected, joakim edsjo
c     020927 higgs limits update according to pdg2002 mia schelke
c     021001 susy part. mass limits update to pdg2002 je/ms
c=======================================================================
\end{verbatim}
 \end{routine}

%%%%% routine dsacbnd5.f %%%%%
\begin{routine}{dsacbnd5.f}
\begin{verbatim}
      subroutine dsacbnd5(excl)
c_______________________________________________________________________
c  check if accelerator data exclude the present point.
c  output:
c    excl - code of the reason for exclusion (integer); 0 if allowed
c    if not allowed, the reasons are coded as follows
c      dsbit set   dec.   oct.   reason
c      -------   ----   ----   ------
c            0      1      1   chargino mass
c            1      2      2   gluino mass
c            2      4      4   squark mass
c            3      8     10   slepton mass
c            4     16     20   invisible z width
c            5     32     40   higgs mass
c            6     64    100   neutralino mass
c            7    128    200   b -> s gamma
c            8    256    400   rho parameter
c  common:
c    'dssusy.h' - file with susy common blocks
c  author: paolo gondolo 1994-1999
c  history:
c     940407 first version paolo gondolo
c     950323 update paolo gondolo
c     971200 partial update joakim edsjo
c     980428 update joakim edsjo
c     990719 update paolo gondolo
c     000310 update piero ullio
c     000424 added delrho joakim edsjo
c     000904 update according to pdg2000 lars bergstrom      
c     010214 mh2 limits corrected, joakim edsjo
c     020927 higgs limits update according to pdg2002 mia schelke
c     021001 susy part. mass limits update to pdg2002 je/ms
c=======================================================================
\end{verbatim}
 \end{routine}

%%%%% routine dsacbnd6.f %%%%%
\begin{routine}{dsacbnd6.f}
\begin{verbatim}
      subroutine dsacbnd6(excl)
c_______________________________________________________________________
c  check if accelerator data exclude the present point.
c  output:
c    excl - code of the reason for exclusion (integer); 0 if allowed
c    if not allowed, the reasons are coded as follows
c      dsbit set   dec.   oct.   reason
c      -------   ----   ----   ------
c            0      1      1   chargino mass
c            1      2      2   gluino mass
c            2      4      4   squark mass
c            3      8     10   slepton mass
c            4     16     20   invisible z width
c            5     32     40   higgs mass
c            6     64    100   neutralino mass
c            7    128    200   b -> s gamma
c            8    256    400   rho parameter
c  common:
c    'dssusy.h' - file with susy common blocks
c  author: paolo gondolo 1994-1999
c  history:
c     940407 first version paolo gondolo
c     950323 update paolo gondolo
c     971200 partial update joakim edsjo
c     980428 update joakim edsjo
c     990719 update paolo gondolo
c     000310 update piero ullio
c     000424 added delrho joakim edsjo
c     000904 update according to pdg2000 lars bergstrom      
c     010214 mh2 limits corrected, joakim edsjo
c     020927 higgs limits update according to pdg2002 mia schelke
c     021001 susy part. mass limits update to pdg2002 je/ms
c     031204 standard model higgs like mh2 limit for msugra models 
c=======================================================================
\end{verbatim}
 \end{routine}

%%%%% routine dsacset.f %%%%%
\begin{routine}{dsacset.f}
\begin{verbatim}
***********************************************************************
*** This routine selects which set of accelerator constraints to use
*** when dsacbnd is called. For available options, see dsacbnd.f
***********************************************************************
      subroutine dsacset(a)
\end{verbatim}
 \end{routine}

%%%%% routine dsbsgamma.f %%%%%
\begin{routine}{dsbsgamma.f}
\begin{verbatim}
      subroutine dsbsgamma(ratio,flag) 
c_______________________________________________________________________
c  b -> s + gamma branching ratio
c  common:
c    'dssusy.h' - file with susy common blocks
c  input:
c    flag : 0 no qcd correction -- 1 qcd corrections
c  output:
c    ratio : branching ratio
c  author: paolo gondolo (gondolo@lpthe.jussieu.fr) 1994,1995
c     28-nov-94 formulas in bertolini et al, nucl phys b353 (1991) 591
c     modified: joakim edsjo, 2000-09-03, vertices from dsvertx.f
c     correctly implemented
c=======================================================================
\end{verbatim}
 \end{routine}

%%%%% routine dsbsgf1.f %%%%%
\begin{routine}{dsbsgf1.f}
\begin{verbatim}
      function dsbsgf1(x)
c_______________________________________________________________________
c  function in bertolini et al, nucl phys b353 (1991) 591
c  author: paolo gondolo (gondolo@lpthe.jussieu.fr) 1994
c=======================================================================
\end{verbatim}
 \end{routine}

%%%%% routine dsbsgf2.f %%%%%
\begin{routine}{dsbsgf2.f}
\begin{verbatim}
      function dsbsgf2(x)
c_______________________________________________________________________
c  function in bertolini et al, nucl phys b353 (1991) 591
c  author: paolo gondolo (gondolo@lpthe.jussieu.fr) 1994
c=======================================================================
\end{verbatim}
 \end{routine}

%%%%% routine dsbsgf3.f %%%%%
\begin{routine}{dsbsgf3.f}
\begin{verbatim}
      function dsbsgf3(x)
c_______________________________________________________________________
c  function in bertolini et al, nucl phys b353 (1991) 591
c  author: paolo gondolo (gondolo@lpthe.jussieu.fr) 1994
c=======================================================================
\end{verbatim}
 \end{routine}

%%%%% routine dsbsgf4.f %%%%%
\begin{routine}{dsbsgf4.f}
\begin{verbatim}
      function dsbsgf4(x)
c_______________________________________________________________________
c  function in bertolini et al, nucl phys b353 (1991) 591
c  author: paolo gondolo (gondolo@lpthe.jussieu.fr) 1994
c=======================================================================
\end{verbatim}
 \end{routine}

%%%%% routine dsgm2muon.f %%%%%
\begin{routine}{dsgm2muon.f}
\begin{verbatim}
      function dsgm2muon()
c supersymmetric contribution to (g-2)_muon
c  output:
c    gm2amp : susy contribution to g-2 amplitude  = (g-2)/2
c according to T Moroi hep-ph/9512396 v3
c (T. Moroi, PRD 1996; (E) 1997)

c author: paolo gondolo 2001-02-08
c reference: e a baltz and p gondolo, hep-ph/0102147
\end{verbatim}
 \end{routine}

%%%%% routine dswexcl.f %%%%%
\begin{routine}{dswexcl.f}
\begin{verbatim}
      subroutine dswexcl(unit,excl)
c_______________________________________________________________________
c  write reasons for exclusion to specified unit.
c  input:
c    unit - logical unit to write on (integer)
c    excl - code of the reason for exclusion (integer); 0 if allowed
c  author: paolo gondolo (gondolo@lpthe.jussieu.fr) 1994
c=======================================================================
\end{verbatim}
 \end{routine}

\newpage
\chapter[an: Annihilation cross sections (general, $\chi^0$ and $\chi^\pm$) ]{\codeb{src/an}:\\ Annihilation cross sections (general, $\chi^0$ and $\chi^\pm$) }
\label{ch:src-an}

%%%%%%%%%%%%%%%%%%%%%%%%%%%%%%%%%%%%%%%%%%%%%%%%%%%%%%%%%%%%%%%%%%%%

\section{Annihilation cross sections -- theory }

For the relic density calculations, we need all possible
(co)annihilation cross sections between neutralinos, charginos and
sfermions.

%%%%%%%%%%%%%%%%%%%%%%%%%%%%%%%%%%%%%%%%%%%%%%%%%%%%%%%%%%%%%%%%%%%%

\subsection{Annihilation cross sections}
\label{sec:AnnCross}

We have calculated all two-body final state cross sections at tree
level for involving netralinos, charginos, sneutrinos, sleptons and 
squarks in the initial state. A complete list is given below.

Since we have so many different diagrams contributing, we have to use 
some method where the diagrams can be calculated efficiently. To
achive this, we calculate the diagrams with general expressions for
vertices, masses etc so that they can be reused for other
processes. How we do this in practice differs a bit between different
sets of annihilation diagrams.

For neutralino-neutralino, neutralino-chargino and chargino-chargino
annihilation, we 
classify the diagrams according to their topology ($s$-, 
$t$- or $u$-channel) and to the spin of the particles involved.  We 
then compute the helicity amplitudes for each type of diagram 
analytically with {\sc Reduce}~\cite{reduce} using general expressions 
for the vertex couplings.  

The strength of the helicity amplitude method is that the analytical
calculation of a given type of diagram has to be performed only once
and the sum of the contributing diagrams for each set of initial and
final states can be done numerically afterwards.

For the diagrams involving sfermions, {\sc Form}
is used to analytically calculate the amplitudes. This output is then converted
into Fortran with a {\sc Perl} script, \codeb{form2f} \cite{form2f}.

%%%%%%%%%%%%%%%%%%%%%%%%%%%%%%%%%%%%%%%%%%%%%%%%%%
\subsection{Coannihilation diagrams}

All Feynman diagrams for which we calculate the 
annihilation cross section are listed in the coming sections.
$s(x)$, $t(x)$ and $u(x)$ denote a
tree-level Feynman diagram in which particle $x$ is exchanged in
the $s$-, $t$- and $u$-channel respectively. 

The convention used in this list of included coannihilation diagrams is that if a sfermion is
denoted $\tilde{f}$, then it's antiparticle is denoted $\tilde{f}^*$.

%%%%%%%%%%%%%%%%%%%%%%%%%%%%%%%%%%%%%%%%%%%%%%%%%%
\subsection{Neutralino and chargino annihilation}

Indices $i,j,k$ run
from 1 to 4, and indices $c,d,e$ from 1 to 2.  $u$, $\tilde{u}$,
$d$, $\tilde{d}$, $\nu$, $\tilde{\nu}$, $\ell$, $\tilde{\ell}$,
$f$ and $\tilde{f}$ are generic notations for up-type quarks,
up-type squarks, down-type quarks, down-type squarks, neutrinos,
sneutrinos, leptons, sleptons, fermions and sfermions.  A sum of
diagrams over (s)fermion generation indices and over the
neutralino and chargino indices $k$ and $e$ is understood (no sum
over indices $i,j,c,d$).

%%%%%%%%%%
\subsubsection{Neutralino-neutralino annihilation}

\begin{center}  
\begin{tabular}{lll} \hline 
  Initial state & Final state & Feynman diagrams \\ \hline \tabspace
%Neutralino-neutralino annihilation
   & $H_1 H_1$, $H_1 H_2$, $H_2 H_2$, $H_3 H_3$ &
  $t(\chi_k^0)$, $u(\chi_k^0)$, $s(H_{1,2})$ \\
   & $H_1 H_3$, $H_2 H_3$ &
  $t(\chi_k^0)$, $u(\chi_k^0)$, $s(H_{3})$, $s(Z^0)$ \\
   & $H^- H^+$ &
  $t(\chi_e^+)$, $u(\chi_e^+)$, $s(H_{1,2})$, $s(Z^0)$ \\
   & $Z^0 H_1$, $Z^0 H_2$ &
  $t(\chi_k^0)$, $u(\chi_k^0)$, $s(H_{3})$, $s(Z^0)$ \\
  $\chi_i^0 \chi_j^0$ & $Z^0 H_3$ &
  $t(\chi_k^0)$, $u(\chi_k^0)$, $s(H_{1,2})$ \\
   & $W^- H^+$, $W^+ H^-$ &
  $t(\chi_e^+)$, $u(\chi_e^+)$, $s(H_{1,2,3})$ \\
   & $Z^0 Z^0$ &
  $t(\chi_k^0)$, $u(\chi_k^0)$, $s(H_{1,2})$ \\
   & $W^- W^+$ &
  $t(\chi_e^+)$, $u(\chi_e^+)$, $s(H_{1,2})$, $s(Z^0)$ \\
   & $f \bar{f}$ &
  $t(\tilde{f}_{L,R})$, $u(\tilde{f}_{L,R})$, $s(H_{1,2,3})$,
  $s(Z^0)$ \\ \hline  
\end{tabular}
\end{center}


%%%%%%%%%%
\subsubsection{Neutralino-chargino annihilation}

\begin{center}
\begin{tabular}{lll} \hline 
  Initial state & Final state & Feynman diagrams \\ \hline \tabspace
%Chargino-neutralino annihilation
   & $H^+ H_1$, $H^+ H_2$ &
  $t(\chi_k^0)$, $u(\chi_e^+)$, $s(H^+)$, $s(W^+)$ \\
   & $H^+ H_3$ &
  $t(\chi_k^0)$, $u(\chi_e^+)$, $s(W^+)$ \\
   & $W^+ H_1$, $W^+ H_2$ &
  $t(\chi_k^0)$, $u(\chi_e^+)$, $s(H^+)$, $s(W^+)$ \\
   & $W^+ H_3$ &
  $t(\chi_k^0)$, $u(\chi_e^+)$, $s(H^+)$ \\
  $\chi_c^+ \chi_i^0$ & $H^+ Z^0$ &
  $t(\chi_k^0)$, $u(\chi_e^+)$, $s(H^+)$ \\
   & $\gamma H^+$ &
  $t(\chi_c^+)$, $s(H^+)$ \\
   & $W^+ Z^0$ &
  $t(\chi_k^0)$, $u(\chi_e^+)$, $s(W^+)$ \\
   & $\gamma W^+$ &
  $t(\chi_c^+)$, $s(W^+)$ \\
   & $u \bar{d}$ &
  $t(\tilde{d}_{L,R})$, $u(\tilde{u}_{L,R})$, $s(H^+)$, $s(W^+)$ \\
   & $\nu \bar{\ell}$ &
  $t(\tilde{\ell}_{L,R})$, $u(\tilde{\nu}_{L})$, $s(H^+)$, $s(W^+)$
  \\ \hline
\end{tabular}
\end{center}


%%%%%%%%%%
\subsubsection{Chargino-chargino annihilation}

\begin{center} 
\begin{tabular}{lll} \hline 
  Initial state & Final state & Feynman diagrams \\ \hline \tabspace
% Chargino-chargino annihilation (opposite charges)
   & $H_1 H_1$, $H_1 H_2$, $H_2 H_2$, $H_3 H_3$ &
  $t(\chi_e^+)$, $u(\chi_e^+)$, $s(H_{1,2})$ \\
   & $H_1 H_3$, $H_2 H_3$ &
  $t(\chi_e^+)$, $u(\chi_e^+)$, $s(H_{3})$, $s(Z^0)$ \\
   & $H^+ H^-$ &
  $t(\chi_k^0)$, $s(H_{1,2})$, $s(Z^0,\gamma)$ \\
   & $Z^0 H_1$, $Z^0 H_2$ &
  $t(\chi_e^+)$, $u(\chi_e^+)$, $s(H_{3})$, $s(Z^0)$ \\
   & $Z^0 H_3$ &
  $t(\chi_e^+)$, $u(\chi_e^+)$, $s(H_{1,2})$ \\
   & $H^+ W^-$, $W^+ H^-$ &
  $t(\chi_k^0)$, $s(H_{1,2,3})$ \\
  $\chi_c^+ \chi_d^-$ & $Z^0 Z^0$ &
  $t(\chi_e^+)$, $u(\chi_e^+)$, $s(H_{1,2})$ \\
   & $W^+ W^-$ &
  $t(\chi_k^0)$, $s(H_{1,2})$, $s(Z^0, \gamma)$ \\
   & $\gamma \gamma$ (only for $c=d$) &
  $t(\chi_c^+)$, $u(\chi_c^+)$ \\
   & $Z^0 \gamma$ &
  $t(\chi_d^+)$, $u(\chi_c^+)$ \\
   & $u \bar{u}$ &
  $t(\tilde{d}_{L,R})$, $s(H_{1,2,3})$, $s(Z^0, \gamma)$ \\
   & $\nu \bar{\nu}$ &
  $t(\tilde{\ell}_{L,R})$, $s(Z^0)$ \\
   & $\bar{d} d$ &
  $t(\tilde{u}_{L,R})$, $s(H_{1,2,3})$, $s(Z^0, \gamma)$ \\
   & $\bar{\ell} \ell$ &
  $t(\tilde{\nu}_{L})$, $s(H_{1,2,3})$, $s(Z^0, \gamma)$ \\ \hline
% Chargino-chargino annihilation (equal charges)
   & $H^+ H^+$ &
  $t(\chi_k^0)$, $u(\chi_k^0)$ \\
  $\chi_c^+ \chi_d^+$ & $H^+ W^+$ &
  $t(\chi_k^0)$, $u(\chi_k^0)$ \\
   & $W^+ W^+$ &
  $t(\chi_k^0)$, $u(\chi_k^0)$ \\ \hline
\end{tabular}
\end{center}


%%%%%%%%%%%%%%%%%%%%%%%%%%%%%%%%%%%%%%%%%%%%%%%%%%
%%%%%%%%%%%%%%%%%%%%%%%%%%%%%%%%%%%%%%%%%%%%%%%%%%
\subsection{Squark-squark annihilation}

We will here denote squarks as $\tilde{q}^i_a$ and $\tilde{q}^j_b$ where $i$ and
$j$ are the family indices and $a$ and $b$ are the mass eigenstate indices
(running from 1 to 2). $k$ and $l$ will also be used as family indices for processes including more squarks. Colour indices are suppressed. $\tilde{u}^i$ is used as a
generic notation for any up-type squark where $i$ denotes the family index. Down-type 
squarks are denoted analogously.

Note that we will not (except in rare occations) show processes for $\tilde{\nu}$
and $\tilde{\ell}$ separately since they can easily be obtained from the squark
processes by replacing $\tilde{u}$ with $\tilde{\nu}$ and 
$\tilde{d}$ with $\tilde{\ell}$ (and noting that we only have one mass eigenstate
for the $\tilde{\nu}$. Also note that the $\tilde{\nu}-\tilde{\ell}$--sector
is assumed not to be flavour-changing.

%%%%% d-squark^i_{a} d-squark^i_{b}^*
\subsubsection{$\tilde{d}^i_{a}\tilde{d}_{b}^{i*}$ annihilation}

{\small
\begin{center}
\begin{tabular}{llll} \hline
{\bfseries Initial state} & {\bfseries Final state} &
{\bfseries Diagrams} & {\bfseries Note} \\ \hline \tabspace
% gamma gamma, Z gamma
$\tilde{d}^i_a \tilde{d}^{i*}_{b}$ & $\gamma \gamma$, $Z \gamma$ &
$t(\tilde{d}^i_{1,2})$, $u(\tilde{d}^i_{1,2})$, $p$ \\
% Z Z
$\tilde{d}^i_a\tilde{d}^{i*}_b$ & $Z Z$ &
$t(\tilde{d}^i_{1,2})$, $u(\tilde{d}^i_{1,2})$, $p$, $s(H_{1},H_{2})$ \\
% W- W+
$\tilde{d}^i_a\tilde{d}^{i*}_b$ & $W^-W^+$ &
$p$, $s(H_{1},H_{2},Z,\gamma)$, $t(\tilde{u}^k_{1,2})$ 
& Only $k=i$ at present\\
% Z H_{2}, Z H_{1}
$\tilde{d}^i_a\tilde{d}^{i*}_b$ & $Z H_{2}$, $Z H_{1}$ &
$t(\tilde{d}^i_{1,2})$, $u(\tilde{d}^i_{1,2})$, $s(Z,H_3)$ \\
% Z H_3
$\tilde{d}^i_a\tilde{d}^{i*}_b$ & $Z H_{3}$ &
$t(\tilde{d}^i_{1,2})$, $u(\tilde{d}^i_{1,2})$, $s(H_{1},H_{2})$ \\
% gamma H_{2}, gamma H_1, gamma H_3
$\tilde{d}^i_a\tilde{d}^{i*}_b$ & $\gamma H_{2}$, $\gamma H_{1}$, $\gamma H_{3}$ &
$t(\tilde{d}^i_{1,2})$, $u(\tilde{d}^i_{1,2})$  \\
% H_{2} H_{2}, H_{1} H_{1}, H_{1} H_{2}
$\tilde{d}^i_a\tilde{d}^{i*}_b$ & $H_{2} H_{2}$, $H_{1} H_{1}$, 
$H_{1} H_{2}$ &
$t(\tilde{d}^i_{1,2})$, $u(\tilde{d}^i_{1,2})$, $p$, $s(H_{1},H_{2})$ \\
% H_{2} A, H_{1} A
$\tilde{d}^i_a \tilde{d}^{i*}_b$ & $H_{2} H_{3}$, $H_{1} H_{3}$ &
$s(Z,H_3)$, $t(\tilde{d}^i_{1,2})$, $u(\tilde{d}^i_{1,2})$ \\
% A A
$\tilde{d}^i_a\tilde{d}^{i*}_b$ & $H_{3} H_{3}$ &
$s(H_{1},H_{2})$, $p$, $t(\tilde{d}^i_{1,2})$, $u(\tilde{d}^i_{1,2})$ \\
% W- H+
$\tilde{d}^i_a\tilde{d}^{i*}_b$ & $W^- H^+$ &
$s(H_{1},H_{2},H_3)$, $t(\tilde{u}^k_{1,2})$ 
& Only $k=i$ at present\\
% H- H+
$\tilde{d}^i_a\tilde{d}^{i*}_b$ & $H^- H^+$ &
$s(H_{1},H_{2},Z,\gamma)$, $p$, $t(\tilde{u}^k_{1,2})$ 
& Only $k=i$ at present\\
% f f-bar
$\tilde{d}^i_a\tilde{d}^{i*}_b$ & $f \bar{f}$ ($f \ne d^i$) &
$s(H_{1}^\star,H_{2}^\star,H_{3}^\star,Z,\gamma^\star,g^\ddagger), t(\chi_c^+)^\dagger$ 
& \parbox[t]{4cm}{$\dagger$) Only if $f=u^k$ (only $k=i$ at present), $\star$) Not for $f=\nu$, 
$\ddagger$) Only for squarks/quarks} \\
% d d-bar
$\tilde{d}^i_a\tilde{d}^{i*}_b$ & $d^i \bar{d}^i$ &
$s(H_{1},H_{2},H_{3},Z,\gamma,g^\dagger)$, $t(\tilde{\chi}_{k}^0,\tilde{g}^\dagger)$
& $\dagger$) Only for squarks\\ 
% Z g
$\tilde{d}^i_a\tilde{d}^{i*}_b$ & $Z g$ & $t(\tilde{d}^i_{1,2}), u(\tilde{d}^i_{1,2}), p$
& Only for squarks\\
% g g
$\tilde{d}^i_a\tilde{d}^{i*}_b$ & $g g$ & $t(\tilde{d}^i_{1,2}), u(\tilde{d}^i_{1,2}), s(g), p$
& Only for squarks\\
% g gamma
$\tilde{d}^i_a\tilde{d}^{i*}_b$ & $g \gamma$ & $t(\tilde{d}^i_{1,2}), u(\tilde{d}^i_{1,2}), p$
& Only for squarks\\
% g H_{1,2,3}
$\tilde{d}^i_a\tilde{d}^{i*}_b$ & $g H_1, g H_2, g H_3$ & 
$t(\tilde{d}^i_{1,2}), u(\tilde{d}^i_{1,2})$
& Only for squarks\\ \hline
\end{tabular}
\end{center}
}

%%%%% d-squark^i_{a} d-squark^j*_{b}
\subsubsection{$\tilde{d}^i_{a}\tilde{d}_{b}^{j*}$ annihilation ($i \ne j$)}

\begin{center}
\begin{tabular}{llll} \hline
{\bfseries Initial state} & {\bfseries Final state} &
{\bfseries Diagrams} & {\bfseries Note} \\ \hline \tabspace
% W+ W-
$\tilde{d}^i_a\tilde{d}^{j*}_b$ & $W^+W^-$ &
$t(\tilde{u}^k_{1,2})^\dagger$ & Not included at present \\
% W+ H-
$\tilde{d}^i_a\tilde{d}^{j*}_b$ & $W^+ H^-$ &
$t(\tilde{u}^k_{1,2})^\dagger$ & Not included at present \\
% H+ H-
$\tilde{d}^i_a\tilde{d}^{j*}_b$ & $H^+ H^-$ &
$t(\tilde{u}^k_{1,2})^\dagger$ & Not included at present\\
% l^i l^j-bar
$\tilde{d}^i_a\tilde{d}^{*j}_b$ & $d^i \bar{d}^j$ &
$t(\tilde{\chi}_{k}^0, \tilde{g}^\dagger)$ & $\dagger$) Only for squarks\\ 
% u u-bar
$\tilde{d}^i_a\tilde{d}^{*j}_b$ & $u^k \bar{u}^l$ &
$t(\tilde{\chi}_{c}^+)$ & Only $k=i,l=j$ at present\\ 

\hline
\end{tabular}
\end{center}

%%%%% d-squark^i_{a} d-squark^i_{b}
\subsubsection{$\tilde{d}^i_{a}\tilde{d}_{b}^{i}$ annihilation}

\begin{center}
\begin{tabular}{llll} \hline
{\bfseries Initial state} & {\bfseries Final state} &
{\bfseries Diagrams} & {\bfseries Note} \\ \hline \tabspace
% l^i l^i
$\tilde{d}^i_a\tilde{d}^{i}_b$ & $d^i d^i$ &
$t(\tilde{\chi}_{k}^0,\tilde{g}^\dagger)$, $u(\tilde{\chi}_{k}^0,\tilde{g}^\dagger)$ 
& $\dagger$) Only for squarks \\ \hline
\end{tabular}
\end{center}

%%%%% d-squark^i_{a} d-squark^j_{b}
\subsubsection{$\tilde{d}^i_{a}\tilde{d}_{b}^{j}$ annihilation ($i \ne j$)}

\begin{center}
\begin{tabular}{llll} \hline
{\bfseries Initial state} & {\bfseries Final state} &
{\bfseries Diagrams} & {\bfseries Note} \\ \hline \tabspace
% l^i l^j
$\tilde{d}^i_a\tilde{d}^{j}_b$ & $d^i d^j$ &
$t(\tilde{\chi}_{k}^0,\tilde{g}^\dagger)$
& $\dagger$) Only for squarks \\ \hline
\end{tabular}
\end{center}

%%%%% u-squark^i u-squark^i*
\subsubsection{$\tilde{u}^i_{a}\tilde{u}_{b}^{i*}$ annihilation}

{\small
\begin{center}
\begin{tabular}{llll} \hline
{\bfseries Initial state} & {\bfseries Final state} &
{\bfseries Diagrams} & {\bfseries Note} \\ \hline \tabspace
% gamma gamma, Z gamma
$\tilde{u}^i_a \tilde{u}^{i*}_{b}$ & $\gamma \gamma^\dagger$, $Z \gamma^\dagger$ &
$t(\tilde{u}^i_{1,2})$, $u(\tilde{u}^i_{1,2})$, $p$
& $\dagger$) Not for $\tilde{\nu}$ \\
% Z Z
$\tilde{u}^i_a\tilde{u}^{i*}_b$ & $Z Z$ &
$t(\tilde{u}^i_{1,2})$, $u(\tilde{u}^i_{1,2})$, $p$, $s(H_{1},H_{2})$ \\
% W- W+
$\tilde{u}^i_a\tilde{u}^{i*}_b$ & $W^-W^+$ &
$p$, $s(H_{1},H_{2},Z,\gamma^\dagger)$, $u(\tilde{d}^k_{1,2})$ 
& \parbox[t]{4cm}{Only $k=i$ at present, $\dagger$) Not for $\tilde{\nu}$} \\
% Z H_{2}, Z H_{1}
$\tilde{u}^i_a\tilde{u}^{i*}_b$ & $Z H_{2}$, $Z H_{1}$ &
$t(\tilde{u}^i_{1,2})$, $u(\tilde{u}^i_{1,2})$, $s(Z,H_3^\dagger)$
& $\dagger$) Not for $\tilde{\nu}$ \\
% Z H_3
$\tilde{u}^i_a\tilde{u}^{i*}_b$ & $Z H_{3}$ &
$t(\tilde{u}^i_{1,2})^\dagger$, $u(\tilde{u}^i_{1,2})^\dagger$, $s(H_{1},H_{2})$
& $\dagger$) Not for $\tilde{\nu}$ \\
% gamma H_{2}, gamma H_1, gamma H_3
$\tilde{u}^i_a\tilde{u}^{i*}_b$ & $\gamma H_{2}^\dagger$, $\gamma H_{1}^\dagger$, $\gamma H_{3}^\dagger$ &
$t(\tilde{u}^i_{1,2})$, $u(\tilde{u}^i_{1,2})$ 
& $\dagger$) Not for $\tilde{\nu}$ \\
% H_{2} H_{2}, H_{1} H_{1}, H_{1} H_{2}
$\tilde{u}^i_a\tilde{u}^{i*}_b$ & $H_{2} H_{2}$, $H_{1} H_{1}$, 
$H_{1} H_{2}$ &
$t(\tilde{u}^i_{1,2})$, $u(\tilde{u}^i_{1,2})$, $p$, $s(H_{1},H_{2})$ \\
% H_{2} A, H_{1} A
$\tilde{u}^i_a\tilde{u}^{i*}_b$ & $H_{2} H_{3}$, $H_{1} H_{3}$ &
$s(Z,H_3^\dagger)$, $t(\tilde{u}^i_{1,2})^\dagger$, $u(\tilde{u}^i_{1,2})^\dagger$
& $\dagger$) Not for $\tilde{\nu}$ \\
% A A
$\tilde{u}^i_a\tilde{u}^{i*}_b$ & $H_{3} H_{3}$ &
$s(H_{1},H_{2})$, $p$, $t(\tilde{u}^i_{1,2})^\dagger$, $u(\tilde{u}^i_{1,2})^\dagger$
& $\dagger$) Not for $\tilde{\nu}$ \\
% W- H+
$\tilde{u}^i_a\tilde{u}^{i*}_b$ & $W^- H^+$ &
$s(H_{1},H_{2},H_3^\dagger)$, $u(\tilde{d}^k_{1,2})$
& \parbox[t]{4cm}{Only $k=i$ at present, $\dagger$) Not for $\tilde{\nu}$} \\
% H- H+
$\tilde{u}^i_a\tilde{u}^{i*}_b$ & $H^+ H^-$ &
$s(H_{1},H_{2},Z,\gamma^\dagger)$, $p$, $t(\tilde{d}^k_{1,2})$
& \parbox[t]{4cm}{Only $k=i$ at present, $\dagger$) Not for $\tilde{\nu}$} \\
% f f-bar
$\tilde{u}^i_a\tilde{u}^{i*}_b$ & $f \bar{f}$ ($f \ne u^i$) &
$s(H_{1}^\times,H_{2}^\times,H_3^{\dagger\times},Z,\gamma^{\dagger\times},g^\ddagger), t({\chi^+_c})^\star$
& \parbox[t]{4cm}{$\dagger$) Not for $\tilde{\nu}$, $\star$) If $f=d^k$ (only $k=i$ at present), $\ddagger$) Only for squarks/quarks, $\times$) Not for $\nu$} \\
% l l-bar
$\tilde{u}^i_a\tilde{u}^{i*}_b$ & $u^i \bar{u}^i$ &
$s(H_{1}^\times,H_{2}^\times,H_3^{\times},Z,\gamma^{\times}, g^\ddagger)$, $t(\tilde{\chi}_{k}^0,\tilde{g}^\ddagger)$
& \parbox[t]{4cm}{$\times$) Not for $\nu$,
$\ddagger$) Only for squarks}  \\ 
% Z g
$\tilde{u}^i_a\tilde{u}^{i*}_b$ & $Z g$ & $t(\tilde{u}^i_{1,2}), u(\tilde{u}^i_{1,2}), p$
& Only for squarks\\
% g g
$\tilde{u}^i_a\tilde{u}^{i*}_b$ & $g g$ & $t(\tilde{u}^i_{1,2}), u(\tilde{u}^i_{1,2}), s(g), p$
& Only for squarks\\
% g gamma
$\tilde{u}^i_a\tilde{u}^{i*}_b$ & $g \gamma$ & $t(\tilde{u}^i_{1,2}), u(\tilde{u}^i_{1,2}), p$
& Only for squarks\\
% g H_{1,2,3}
$\tilde{u}^i_a\tilde{u}^{i*}_b$ & $g H_1, g H_2, g H_3$ & 
$t(\tilde{u}^i_{1,2}), u(\tilde{u}^i_{1,2})$
& Only for squarks\\ \hline
\end{tabular}
\end{center}
}

%%%%% u-squark^i u-squark^j*
\subsubsection{$\tilde{u}^i_a\tilde{u}^{j*}_b$ annihilation ($i \ne j$)}

\begin{center}
\begin{tabular}{llll} \hline
{\bfseries Initial state} & {\bfseries Final state} &
{\bfseries Diagrams} & {\bfseries Note} \\ \hline \tabspace
% W+ W-
$\tilde{u}^i_a\tilde{u}^{j*}_b$ & $W^+W^-$ &
$t(\tilde{d}^k_{1,2})^\dagger$
& Not included at present, $\dagger$) Not for $\tilde{\ell}$ \\
% W+ H-
$\tilde{u}^i_a\tilde{u}^{j*}_b$ & $W^+ H^-$ &
$t(\tilde{d}^k_{1,2})^\dagger$
& Not included at present, $\dagger$) Not for $\tilde{\ell}$ \\
% H+ H-
$\tilde{u}^i_a\tilde{u}^{j*}_b$ & $H^+ H^-$ &
$t(\tilde{d}^k_{1,2})^\dagger$
& Not included at present, $\dagger$) Not for $\tilde{\ell}$ \\
% u^i u^j-bar
$\tilde{u}^i_a \tilde{u}^{j*}_b$ & $u^i \bar{u}^j$ &
$t(\tilde{\chi}_{k}^0,g^\dagger)$
& $\dagger$) Only for squarks \\
% d^k d^l-bar
$\tilde{u}^i_a \tilde{u}^{j*}_b$ & $d^k \bar{d}^l$ &
$t(\tilde{\chi}_{c}^+)$
& Only $k=i,l=j$ at present \\ \hline
\end{tabular}
\end{center}

%%%%% u-squark^i u-squark^i
\subsubsection{$\tilde{u}^i_a \tilde{u}^{i}_b$ annihilation}

\begin{center}
\begin{tabular}{llll} \hline
{\bfseries Initial state} & {\bfseries Final state} &
{\bfseries Diagrams} & {\bfseries Note} \\ \hline \tabspace
% l^i l^i
$\tilde{u}^i_a\tilde{u}^{i}_b$ & $u^i u^i$ &
$t(\tilde{\chi}_{k}^0,\tilde{g}^\dagger)$, $u(\tilde{\chi}_{k}^0,\tilde{g}^\dagger)$ 
& $\dagger$) Only for squarks \\ \hline
\end{tabular}
\end{center}

%%%%% u-squark^i u-squark^j
\subsubsection{$\tilde{u}^i_a \tilde{u}^{j}_b$ annihilation ($i \ne j$)}

\begin{center}
\begin{tabular}{llll} \hline
{\bfseries Initial state} & {\bfseries Final state} &
{\bfseries Diagrams} & {\bfseries Note} \\ \hline \tabspace
% l^i l^j
$\tilde{u}^i_a \tilde{u}^{j}_b$ & $u^i u^j$ &
$t(\tilde{\chi}_{k}^0,\tilde{g}^\dagger)$
& $\dagger$) Only for squarks \\ \hline
\end{tabular}
\end{center}

%%%%% u-squark^i d-squark^i_{b}^*
\subsubsection{$\tilde{u}^i_a \tilde{d}_{b}^{i*}$ annihilation}

\begin{center}
\begin{tabular}{llll} \hline
{\bfseries Initial state} & {\bfseries Final state} &
{\bfseries Diagrams} & {\bfseries Note} \\ \hline \tabspace
% H+ H_1, H+ H_2
$\tilde{u}^i_a \tilde{d}^{i*}_b$ & $H^+ H_1$, $H^+ H_2$ &
$t(\tilde{d}^i_{1,2})$, $u(\tilde{u}^i_{1,2})$, $p$, $s(W^+,H^+)$  \\
% H+ H_3
$\tilde{u}^i_a \tilde{d}^{i*}_b$ & $H^+ H_3$ &
$t(\tilde{d}^i_{1,2})$, $u(\tilde{u}^i_{1,2})^\dagger$, $p$, $s(W^+)$ 
& $\dagger$) Not for $\tilde{\ell}$ \\
% gamma H+
$\tilde{u}^i_a \tilde{d}^{i*}_b$ & $\gamma H^+$ &
$t(\tilde{u}^i_{1,2})^\dagger$, $u(\tilde{d}^i_{1,2})$, $s(H^+)$ 
& $\dagger$) Not for $\tilde{\ell}$ \\
% Z H+
$\tilde{u}^i_a \tilde{d}^{i*}_b$ & $Z H^+$ &
$t(\tilde{u}^i_{1,2})$, $u(\tilde{d}^i_{1,2})$, $s(H^+)$ \\
% W+ H_1, W+ H_2
$\tilde{u}^i_a \tilde{d}^{i*}_b$ & $W^+ H_1$, $W^+ H_2$  &
$t(\tilde{d}^i_{1,2})$, $u(\tilde{u}^i_{1,2})$, $s(W^+,H^+)$ \\
% W+ H_3
$\tilde{u}^i_a \tilde{d}^{i*}_b$ & $W^+ H_3$  &
$t(\tilde{d}^i_{1,2})$, $u(\tilde{u}^i_{1,2})^\dagger$, $s(H^+)$ 
& $\dagger$) Not for $\tilde{\ell}$ \\
% W+ gamma
$\tilde{u}^i_a \tilde{d}^{i*}_b$ & $W^+ \gamma$  &
$t(\tilde{d}^i_{1,2})$, $u(\tilde{u}^i_{1,2})^\dagger$, $s(W^+)$, $p$ 
& $\dagger$) Not for $\tilde{\ell}$ \\
% W+ Z
$\tilde{u}^i_a \tilde{d}^{i*}_b$ & $W^+ Z$  &
$t(\tilde{d}^i_{1,2})$, $u(\tilde{u}^i_{1,2})$, $s(W^+)$, $p$ \\
% u d-bar
$\tilde{u}^i_a \tilde{d}^{i*}_b$ & $u^k \bar{d}^l$ &
$s(H^+,W^+)^\star$, $t(\tilde{\chi}_m^0,\tilde{g}^\dagger)\delta^{ik}\delta^{il}$ 
& \parbox[t]{4cm}{$\dagger$) Not for $\tilde{\ell}$, $\star$) Only $k=l$ at present} \\ 
% W+ g
$\tilde{u}^i_a \tilde{d}^{i*}_b$ & $W^+ g$ &
$t(\tilde{d}^i_{1,2})$, $u(\tilde{u}^i_{1,2})$, $p$
& Only for squarks \\ 
% g H+
$\tilde{u}^i_a \tilde{d}^{i*}_b$ & $g H^+$ &
$t(\tilde{u}^i_{1,2})$, $u(\tilde{d}^i_{1,2})$
& Only for squarks \\ \hline
\end{tabular}
\end{center}

%%%%% u-squark^i d-squark^j_{b}^*
\subsubsection{$\tilde{u}_a^i \tilde{d}_{b}^{j*}$ annihilation ($i \ne j$)}

For squarks we can have the following processes

\begin{center}
\begin{tabular}{llll} \hline
{\bfseries Initial state} & {\bfseries Final state} &
{\bfseries Diagrams} & {\bfseries Note} \\ \hline \tabspace
% H+ H_1, H+ H_2
$\tilde{u}^i_a \tilde{d}^{j*}_b$ & $H^+ H_1$, $H^+ H_2$ &
$t(\tilde{d}^j_{1,2})$, $u(\tilde{u}^i_{1,2})$, $p$, $s(W^+,H^+)$
& Not included at present \\
% H+ H_3
$\tilde{u}^i_a \tilde{d}^{j*}_b$ & $H^+ H_3$ &
$t(\tilde{d}^j_{1,2})$, $u(\tilde{u}^i_{1,2})$, $p$, $s(W^+)$
& Not included at present \\
% H+ gamma
$\tilde{u}^i_a \tilde{d}^{j*}_b$ & $H^+ \gamma$ &
$t(\tilde{d}^j_{1,2})$, $u(\tilde{u}^i_{1,2})$, $s(H^+)$
& Not included at present \\
% H+ Z
$\tilde{u}^i_a \tilde{d}^{j*}_b$ & $H^+ Z$ &
$t(\tilde{d}^j_{1,2})$, $u(\tilde{u}^i_{1,2})$, $s(H^+)$
& Not included at present \\
% W+ H_1, W+ H_2
$\tilde{u}^i_a \tilde{d}^{j*}_b$ & $W^+ H_1$, $W^+ H_2$  &
$t(\tilde{d}^j_{1,2})$, $u(\tilde{u}^i_{1,2})$, $s(W^+,H^+)$
& Not included at present \\
% W+ H_3
$\tilde{u}^i_a \tilde{d}^{j*}_b$ & $W^+ H_3$  &
$t(\tilde{d}^j_{1,2})$, $u(\tilde{u}^i_{1,2})$, $s(H^+)$
& Not included at present \\
% W+ gamma
$\tilde{u}^i_a \tilde{d}^{j*}_b$ & $W^+ \gamma$  &
$t(\tilde{d}^j_{1,2})$, $u(\tilde{u}^i_{1,2})$, $s(W^+)$, $p$
& Not included at present \\
% W+ Z
$\tilde{u}^i_a \tilde{d}^{j*}_b$ & $W^+ Z$  &
$t(\tilde{d}^j_{1,2})$, $u(\tilde{u}^i_{1,2})$, $s(W^+)$, $p$
& Not included at present \\
% u d-bar
$\tilde{u}^i_a \tilde{d}^{j*}_b$ & $u^k \bar{d}^l$ &
$s(H^+,W^+)^\dagger$, $t(\tilde{\chi}_m^0,\tilde{g})\delta^{ik}\delta^{jl}$
& $\dagger$) Not included at present \\ \hline
\end{tabular}
\end{center}

whereas for sneutrinos and sleptons, we can only have the process

\begin{center}
\begin{tabular}{llll} \hline
{\bfseries Initial state} & {\bfseries Final state} &
{\bfseries Diagrams} & {\bfseries Note} \\ \hline \tabspace
% nu l-bar
$\tilde{\nu}^i \tilde{\ell}^{j*}_b$ & $\nu^i \bar{\ell}^j$ &
$t(\tilde{\chi}_k^0)$ \\ \hline
\end{tabular}
\end{center}

%%%%% u-squark^i d-squark^i_{b}
\subsubsection{$\tilde{u}_a^i \tilde{d}_{b}^{i}$ annihilation}

\begin{center}
\begin{tabular}{llll} \hline
{\bfseries Initial state} & {\bfseries Final state} &
{\bfseries Diagrams} & {\bfseries Note} \\ \hline \tabspace
% u d-bar
$\tilde{u}^i_a \tilde{d}^{i}_b$ & $u^k d^l$ &
$t(\tilde{\chi}_m^0,\tilde{g}^\dagger)\delta^{ik}\delta^{il}$, $u(\tilde{\chi}_c^+)^\star$ 
& \parbox[t]{4cm}{$\dagger$) Only for squarks, $\star$) Only $i=k=l$ at present} \\ \hline
\end{tabular}
\end{center}

%%%%% u-squark^i d-squark^j_{b}
\subsubsection{$\tilde{u}_a^i \tilde{d}_{b}^{j}$ annihilation ($i \ne j$)}

\begin{center}
\begin{tabular}{llll} \hline
{\bfseries Initial state} & {\bfseries Final state} &
{\bfseries Diagrams} & {\bfseries Note} \\ \hline \tabspace
% u d-bar
$\tilde{u}^i_a \tilde{d}^{j}_b$ & $u^k d^l$ &
$t(\tilde{\chi}_m^0,\tilde{g}^\dagger)\delta^{ik}\delta^{jl}$, $u(\tilde{\chi}_c^+)^{\times\star}$ 
& \parbox[t]{4cm}{$\dagger$) Only for squarks, $\times$) For $\tilde{\nu}\tilde{\ell}$ only when $i=l,j=k$,
$\star$) Only included when $i=l,j=k$ at present} \\ \hline
\end{tabular}
\end{center}


%%%%%%%%%%%%%%%%%%%%%%%%%%%%%%%%%%%%%%%%%%%%%%%%%%
%%%%%%%%%%%%%%%%%%%%%%%%%%%%%%%%%%%%%%%%%%%%%%%%%%
\subsection{Squark-neutralino annihilation}

We will here denote squarks as $\tilde{u}^i_a$ and $\tilde{d}^i_a$ where $i$ is the family index and $a$ is the mass eigenstate index (running from 1 to 2).

%%%%% up-squark^i_{a} neutralino_{j}
\subsubsection{$\tilde{u}^i_{a} \tilde{\chi}_{j}^{0}$ annihilation}

\begin{center}
\begin{tabular}{llll} \hline
{\bfseries Initial state} & {\bfseries Final state} &
{\bfseries Diagrams} & {\bfseries Note} \\ \hline \tabspace
% u^i gamma
$\tilde{u}^i_a \tilde{\chi}_{j}^0$ & $\gamma u^i$ &
$s(u^{i})$, $t(\tilde{u}^i_{1,2})$ 
& Only for squarks \\
% u^i Z
$\tilde{u}^i_a \tilde{\chi}_{j}^0$ & $Z u^i$ &
$s(u^{i})$, $t(\tilde{u}^i_{1,2})$, $u(\tilde{\chi}_k^0)$ \\
% u^i H_{1}, u^i H_{2}
$\tilde{u}^i_a \tilde{\chi}_{j}^0$ & $H_{1} u^i$, $H_{2} u^i$ &
$s(u^{i})^\dagger$, $t(\tilde{u}^i_{1,2})$, $u(\tilde{\chi}_k^0)$
& $\dagger$) Only for squarks \\
% u^i H_{3}
$\tilde{u}^i_a \tilde{\chi}_{j}^0$ & $H_{3}u^i$  &
$s(u^{i})^\dagger$, $t(\tilde{u}^i_{1,2})^\dagger$, $u(\tilde{\chi}_k^0)$
& $\dagger$) Only for squarks  \\ 
% d^k W+
$\tilde{u}^i_a \tilde{\chi}_{j}^0$ & $W^+ d^k$  &
$s(u^i)$, $t(\tilde{d}^k_{1,2})$, $u(\tilde{\chi}_c^+)$ 
& \parbox[t]{4cm}{Only $k=i$ at present, $\tilde{u}^{i*}_a \tilde{\chi}_{j}^0 \to W^- \bar{d}^k$ in the code} \\
% d^k H+
$\tilde{u}^i_a \tilde{\chi}_{j}^0$ & $H^+ d^k$  &
$s(u^i)$, $t(\tilde{d}^k_{1,2})$, $u(\tilde{\chi}_c^+)$
& \parbox[t]{4cm}{Only $k=i$ at present, $\tilde{u}^{i*}_a \tilde{\chi}_{j}^0 \to H^- \bar{d}^k$ in the code} \\ 
% g u
$\tilde{u}^i_a \tilde{\chi}_{j}^0$ & $g u^i$  &
$s(u^i)$, $t(\tilde{u}^i_{1,2})$ \\ \hline
\end{tabular}
\end{center}

%%%%% down-squark^i_{a} neutralino_{j}
\subsubsection{$\tilde{d}^i_{a} \tilde{\chi}_{j}^{0}$ annihilation}

\begin{center}
\begin{tabular}{llll} \hline
{\bfseries Initial state} & {\bfseries Final state} &
{\bfseries Diagrams} & {\bfseries Note} \\ \hline \tabspace
% d^i gamma
$\tilde{d}^i_a \tilde{\chi}_{j}^0$ & $\gamma d^i$ &
$s(d^{i})$, $t(\tilde{d}^i_{1,2})$  \\
% d^i Z
$\tilde{d}^i_a \tilde{\chi}_{j}^0$ & $Z d^i$ &
$s(d^{i})$, $t(\tilde{d}^i_{1,2})$, $u(\tilde{\chi}_k^0)$ \\
% d^i H_{1}, d^i H_{2}
$\tilde{d}^i_a \tilde{\chi}_{j}^0$ & $H_{1} d^i$, $H_{2} d^i$ &
$s(d^{i})$, $t(\tilde{d}^i_{1,2})$, $u(\tilde{\chi}_k^0)$ \\
% d^i H_{3}
$\tilde{d}^i_a \tilde{\chi}_{j}^0$ & $H_{3} d^i$  &
$s(d^{i})$, $t(\tilde{d}^i_{1,2})$, $u(\tilde{\chi}_k^0)$ \\ 
% u^i W-
$\tilde{d}^i_a \tilde{\chi}_{j}^0$ & $W^- u^k$  &
$s(d^{i})$, $t(\tilde{u}^k_{1,2})$, $u(\tilde{\chi}_c^+)$ 
& Only $k=i$ at present \\
% u^i H-
$\tilde{d}^i_a \tilde{\chi}_{j}^0$ & $H^- u^k$  &
$s(d^{i})$, $t(\tilde{u}^k_{1,2})$, $u(\tilde{\chi}_c^+)$ 
& Only $k=i$ at present \\
% g d
$\tilde{d}^i_a \tilde{\chi}_{j}^0$ & $g d^i$  &
$s(d^i)$, $t(\tilde{d}^i_{1,2})$ \\ \hline
\end{tabular}
\end{center}

%%%%%%%%%%%%%%%%%%%%%%%%%%%%%%%%%%%%%%%%%%%%%%%%%%
%%%%%%%%%%%%%%%%%%%%%%%%%%%%%%%%%%%%%%%%%%%%%%%%%%
\subsection{Squark-chargino annihilation}

We will here denote squarks as $\tilde{q}^i_a$ where $i$ is the family
index and $a$ is the mass eigenstate index (running from 1 to 2).

%%%%% up-squark^i_{a} chargino_{j}
\subsubsection{$\tilde{u}^i_{a} \tilde{\chi}_{c}^{+}$ annihilation}

\begin{center}
\begin{tabular}{llll} \hline
{\bfseries Initial state} & {\bfseries Final state} &
{\bfseries Diagrams} & {\bfseries Note} \\ \hline \tabspace
% u^i W+
$\tilde{u}^i_a \tilde{\chi}_{c}^+$ & $W^+ u^k$ &
$t(\tilde{d}^l_{1,2})$, $u(\tilde{\chi}_c^0)\delta^{ik}$ &
Only $k=l=i$ at present \\
% u^i H+
$\tilde{u}^i_a \tilde{\chi}_{c}^+$ & $H^+ u^k$ &
$t(\tilde{d}^l_{1,2})$, $u(\tilde{\chi}_c^0)\delta^{ik}$ 
& Only $k=l=i$ at present \\ \hline
\end{tabular}
\end{center}

%%%%% up-squark^i*_{a} chargino_{j}
\subsubsection{$\tilde{u}^{i*}_{a} \tilde{\chi}_{c}^{+}$ annihilation}

\begin{center}
\begin{tabular}{llll} \hline
{\bfseries Initial state} & {\bfseries Final state} &
{\bfseries Diagrams} & {\bfseries Note} \\ \hline \tabspace
% d-bar^i Z
$\tilde{u}^{i*}_a \tilde{\chi}_{c}^+$ & $Z \bar{d}^k$ &
$s(\bar{d}^k)$, $t(\tilde{u}^i_{1,2})$, $u(\tilde{\chi}_c^+)$ 
& Only $k=i$ at present \\
% d-bar^i gamma
$\tilde{u}^{i*}_a \tilde{\chi}_{c}^+$ & $\gamma \bar{d}^k$ &
$s(\bar{d}^k)$, $t(\tilde{u}^i_{1,2})^\dagger$, $u(\tilde{\chi}_c^+)$ 
& Only $k=i$ at present, $\dagger$) Only for squarks \\
% d-bar^i H_1, d-bar H_2
$\tilde{u}^{i*}_a \tilde{\chi}_{c}^+$ & $H_1 \bar{d}^k$, $H_2 \bar{d}^k$ &
$s(\bar{d}^k)$, $t(\tilde{u}^i_{1,2})$, $u(\tilde{\chi}_c^+)$
& Only $k=i$ at present \\
% d-bar^i H_3
$\tilde{u}^{i*}_a \tilde{\chi}_{c}^+$ & $H_3 \bar{d}^k$ &
$s(\bar{d}^k)$, $t(\tilde{u}^i_{1,2})^\dagger$, $u(\tilde{\chi}_c^+)$ 
& Only $k=i$ at present, $\dagger$) Only for squarks\\
% u-bar^i W^+
$\tilde{u}^{i*}_a \tilde{\chi}_{c}^+$ & $W^+ \bar{u}^k$ &
$s(\bar{d}^l)$, $u(\tilde{\chi}_c^0)\delta^{ik}$ 
& Only $k=l=i$ at present \\
% u-bar^i H^+
$\tilde{u}^{i*}_a \tilde{\chi}_{c}^+$ & $H^+ \bar{u}^k$ &
$s(\bar{d}^l)$, $u(\tilde{\chi}_c^0)\delta^{ik}$ 
& Only $k=l=i$ at present \\
% g d-bar
$\tilde{u}^{i*}_a \tilde{\chi}_{c}^+$ & $g \bar{d}^k$ &
$s(\bar{d}^k)$, $t(\tilde{u}_a^i)$ 
& Only $k=i$ at present, only for squarks \\ \hline
\end{tabular}
\end{center}

%%%%% down-squark^i_{a} chargino_{j}
\subsubsection{$\tilde{d}^{i}_{a} \tilde{\chi}_{c}^{+}$ annihilation}

\begin{center}
\begin{tabular}{llll} \hline
{\bfseries Initial state} & {\bfseries Final state} &
{\bfseries Diagrams} & {\bfseries Note} \\ \hline \tabspace
% u^i Z
$\tilde{d}^{i}_a \tilde{\chi}_{c}^+$ & $Z u^k$ &
$s(u^k)$, $t(\tilde{d}^i_{1,2})$, $u(\tilde{\chi}_c^+)$
& Only $k=i$ at present \\
% u^i gamma
$\tilde{d}^{i}_a \tilde{\chi}_{c}^+$ & $\gamma u^k$ &
$s(u^k)^\dagger$, $t(\tilde{d}^i_{1,2})$, $u(\tilde{\chi}_c^+)$ 
& Only $k=i$ at present, $\dagger$) Only for squarks \\
% u^i H_1, d-bar H_2
$\tilde{d}^{i}_a \tilde{\chi}_{c}^+$ & $H_1 u^k$, $H_2 u^k$ &
$s(u^k)^\dagger$, $t(\tilde{d}^i_{1,2})$, $u(\tilde{\chi}_c^+)$ 
& Only $k=i$ at present, $\dagger$) Only for squarks \\
% u^i H_3
$\tilde{d}^{i}_a \tilde{\chi}_{c}^+$ & $H_3 u^k$ &
$s(u^k)^\dagger$, $t(\tilde{d}^i_{1,2})$, $u(\tilde{\chi}_c^+)$ 
& Only $k=i$ at present, $\dagger$) Only for squarks \\
% d^i W^+
$\tilde{d}^{i}_a \tilde{\chi}_{c}^+$ & $W^+ d^k$ &
$s(u^l)$, $u(\tilde{\chi}_c^0)\delta^{ik}$
& Only $k=l=i$ at present \\
% d^i H^+
$\tilde{d}^{i}_a \tilde{\chi}_{c}^+$ & $H^+ d^k$ &
$s(u^l)$, $u(\tilde{\chi}_c^0)\delta^{ik}$
& Only $k=l=i$ at present \\
% g
$\tilde{d}^{i}_a \tilde{\chi}_{c}^+$ & $g u^k$ &
$s(u^k)$, $t(\tilde{d}_a^i)$
& Only $k=i$ at present, only for squarks \\ \hline
\end{tabular}
\end{center}

%%%%% down-squark^i*_{a} chargino_{j}
\subsubsection{$\tilde{d}^{i*}_{a} \tilde{\chi}_{c}^{+}$ annihilation}

\begin{center}
\begin{tabular}{llll} \hline
{\bfseries Initial state} & {\bfseries Final state} &
{\bfseries Diagrams} & {\bfseries Note} \\ \hline \tabspace
% d-bar^i W+
$\tilde{d}^{i*}_a \tilde{\chi}_{c}^+$ & $W^+ \bar{d}^k$ &
$t(\tilde{u}^l_{1,2})$, $u(\tilde{\chi}_c^0)\delta^{ik}$ 
& Only $k=l=i$ at present \\
% d-bar^i H+
$\tilde{d}^{i*}_a \tilde{\chi}_{c}^+$ & $H^+ \bar{d}^k$ &
$t(\tilde{u}^l_{1,2})$, $u(\tilde{\chi}_c^0)\delta^{ik}$
& Only $k=l=i$ at present \\ \hline
\end{tabular}
\end{center}

%%%%%%%%%%%%%%%%%%%%%%%%%%%%%%%%%%%%%%%%%%%%%%%%%%%%%%%%%%%%%%%%%%%%%%
%%%%%%%%%%%%%%%%%%%%%%%%%%%%%%%%%%%%%%%%%%%%%%%%%%%%%%%%%%%%%%%%%%%%%%
\subsection{Degrees of freedom}

We have to be careful with the internal degrees of freedom, $g$, of
the particles. We can either treat e.g.\ a $\chi_i^+$ and a $\chi_i^-$ as
two separate particles with two degrees of freedom each, or we can
treat them as one particle $\chi_i^\pm$ with four degrees of
freedom. The latter approach has an advantage that we
simplify our expressions for the effective annihilation
cross sections when coannihilations are needed. Hence, we use that
approach here. For a more detailed discussion about this, see Section
\ref{sec:dof}. 
%%%%%%%%%%%%%%%%%%%%%%%%%%%%%%%%%%%%%%%%%%%%%%%%%%%%%%%%%%%%%%%%%%%%

\section{Annihilation routines - general remarks}

The annihilation cross section routines is divided into several parts,
mostly for historical reasons. The layout is roughly as follows:
\begin{description}
  \item{\code{src/an}} Here we keep the main routins for both neutralino-
  neutralino annihilation cross sections and the effective annihilation
  cross section in the relic density calculations. The steering routines
  for neutralino and chargino coannihilations are also kept here.
  
\item{\code{src/anstu}} Here keep the $t-$, $u-$ and $s-$ diagram expressions
for fermion-fermion coannihilations (i.e.\ neutralino and chargino
coannihilations).

\item{\code{src/as}} Here all the coannihilation cross sections including
sfermions are kept.

\end{description}

We will here describe the \code{src/an}-routines.


%%%%%%%%%%
\subsection{General routines}

The general routine to call for an effective annihilation cross section (to
be used for relic density calculations) is \codeb{dsanwx}, which returns the
invariant annihilation rate (integrated over $\cos \theta$). The actual
cross section, differential in $\cos \theta$ is calculated by 
\codeb{dsandwdcos} which includes all the coannihilations needed. This is set
up in \codeb{rn/dsrdomega} which determines which coannhilating particles
to include.

For other applications where the annihilation rate is needed, e.g.\ annihilation
in the galactic halo, one can call the specific annihilation rate routine
directly. The main one is \codeb{dsandwdcosnn} for neutralino-neutralino
annihilation. To simplify this task, we supply a routine \codeb{dssigmav} which
calls \codeb{dsandwdcosnn} for neutralino-neutralino annihilation at zero
relative velocity and returns the result, either as the total annihilation
cross section, or the cross section for a specific channel. See the header of
\codeb{dssigmav} for details.

%%%%%%%%%%
\subsection{Neutralino and chargino (co)annihilation cross sections}
                                   
The routines \codeb{dsandwdcosnn}, \codeb{dsandwdcoscn} and \codeb{dsandwdcoscc}
calculate the annihilation cross sections (returning the invariant 
annihilation rate) for neutralino-neutralino, neutralino-chargino and
chargino-chargino annihilations. Which particles the cross section is
calculated for is given by particle indices as defined in \codeb{inc/dssusy.h}.

All the annihilation routines return the invariant rate instead of the
cross section. The invariant annihilation rate between particle $i$
and $j$ is defined as 
\begin{equation}
  W_{ij} = 4 p_{ij} \sqrt{s} \sigma_{ij} = 4 \sigma_{ij} \sqrt{(p_i
\cdot p_j)^2 - m_i^2 m_j^2} = 4 E_{i} E_{j} \sigma_{ij} v_{ij} .
\end{equation}
See chapter \ref{ch:src-rd} for more details.
\section{Routine headers -- fortran files}

%%%%% routine dsanalbe.f %%%%%
\begin{routine}{dsanalbe.f}
\begin{verbatim}
      subroutine dsanalbe(alph,bet)
c_______________________________________________________________________
c  determine alph and bet for the integration.
c  modified: joakim edsjo (edsjo@physto.se) 97-09-09
c=======================================================================
\end{verbatim}
 \end{routine}

%%%%% routine dsanclearaa.f %%%%%
\begin{routine}{dsanclearaa.f}
\begin{verbatim}
      subroutine dsanclearaa
c_______________________________________________________________________
c  clear the amplitude matrix
c  author: joakim edsjo (edsjo@physto.se) 95-10-25
c          paolo gondolo 99-1-15 factor of 3.7 faster
c  called by: dwdcos
c=======================================================================
\end{verbatim}
 \end{routine}

%%%%% routine dsandwdcos.f %%%%%
\begin{routine}{dsandwdcos.f}
\begin{verbatim}
      function dsandwdcos(p,costheta)
c_______________________________________________________________________
c  annihilation differential invariant rate.
c  input:
c    p - initial cm momentum (real) for lsp annihilations
c    costheta - cosine of c.m. annihilation angle
c  common:
c    'dssusy.h' - file with susy common blocks
c  Output:
\end{verbatim}
\begin{displaymath}
\frac{dW_{ij}}{d\cos\theta}
\end{displaymath}
where
\begin{displaymath}
W_{ij} = 4 p_{ij} \sqrt{s} \sigma_{ij} 
= 4 \sigma_{ij} \sqrt{(p_i \cdot p_j)^2 - m_i^2 m_j^2} 
= 4 E_{i} E_{j} \sigma_{ij} v_{ij}
\end{displaymath}
\begin{verbatim}
c  The returned dW/dcos(theta) is unitless
c  uses dsandwdcosnn, dsandwdcoscn and dsandwdcoscc and
c  routines in src/as
c  called by dsanwx.
c  author: joakim edsjo (edsjo@physto.se)
c  date: 96-02-21
c  modified: 97-05-12 Joakim Edsjo (edsjo@physto.se)
c  modified: 01-01-30 paolo gondolo (paolo@mamma-mia.phys.cwru.edu)
c  modified: 02-03-09 Joakim Edsjo (edsjo@physto.se)
c  modified: 06-02-22 Paolo Gondolo (paolo@physics.utah.edu)
c=======================================================================
\end{verbatim}
 \end{routine}

%%%%% routine dsandwdcoscc.f %%%%%
\begin{routine}{dsandwdcoscc.f}
\begin{verbatim}
      function dsandwdcoscc(p,costheta,kp1,kp2)
c_______________________________________________________________________
c  annihilation differential invariant rate between particle kp1
c  and kp2 where kp1 and kp2 are charginos
c  input:
c    p - initial cm momentum (real)
c    costheta - cosine of c.m. annihilation angle
c    kp1 - particle code, particle 1
c    kp2 - particle code, particle 2
c  common:
c    'dssusy.h' - file with susy common blocks
c    'diacom.h' - file with kinematical variables
c  Output:
\end{verbatim}
\begin{displaymath}
\frac{dW_{ij}}{d\cos\theta}
\end{displaymath}
where
\begin{displaymath}
W_{ij} = 4 p_{ij} \sqrt{s} \sigma_{ij} 
= 4 \sigma_{ij} \sqrt{(p_i \cdot p_j)^2 - m_i^2 m_j^2} 
= 4 E_{i} E_{j} \sigma_{ij} v_{ij}
\end{displaymath}
\begin{verbatim}
c  The returned dW/dcos(theta) is unitless
c  uses dsanclearaa,dsansumaa
c  called by dsandwdcos.
c  author: joakim edsjo (edsjo@physto.se)
c  date: 96-08-06
c  modified: 01-09-12
c=======================================================================
\end{verbatim}
 \end{routine}

%%%%% routine dsandwdcoscn.f %%%%%
\begin{routine}{dsandwdcoscn.f}
\begin{verbatim}
      function dsandwdcoscn(p,costheta,kp1,kp2)
c_______________________________________________________________________
c  annihilation differential invariant rate between particle kp1
c  and kp2 where kp1 is a chargino and kp2 is a neutralino.
c  input:
c    p - initial cm momentum (real)
c    costheta - cosine of c.m. annihilation angle
c    kp1 - particle code, particle 1
c    kp2 - particle code, particle 2
c  common:
c    'dssusy.h' - file with susy common blocks
c    'diacom.h' - file with kinematical variables
c  Output:
\end{verbatim}
\begin{displaymath}
\frac{dW_{ij}}{d\cos\theta}
\end{displaymath}
where
\begin{displaymath}
W_{ij} = 4 p_{ij} \sqrt{s} \sigma_{ij} 
= 4 \sigma_{ij} \sqrt{(p_i \cdot p_j)^2 - m_i^2 m_j^2} 
= 4 E_{i} E_{j} \sigma_{ij} v_{ij}
\end{displaymath}
\begin{verbatim}
c  The returned dW/dcos(theta) is unitless
c  uses dsanclearaa,dsansumaa
c  called by dsandwdcos.
c  author: joakim edsjo (edsjo@physto.se)
c  date: 96-08-06
c  modified: 01-09-12
c=======================================================================
\end{verbatim}
 \end{routine}

%%%%% routine dsandwdcosd.f %%%%%
\begin{routine}{dsandwdcosd.f}
\begin{verbatim}
      function dsandwdcosd(costheta)
c_______________________________________________________________________
c  10^15*annihilation differential invariant rate.
c  input:
c    p - initial cm momentum (real) for lsp annihilations via common
c    costheta - cosine of c.m. annihilation angle
c  common:
c    'dssusy.h' - file with susy common blocks
c  uses dwdcos
c  used for gaussian integration with gadap.f
c  author: joakim edsjo (edsjo@physto.se)
c  date: 97-01-09
c=======================================================================
\end{verbatim}
 \end{routine}

%%%%% routine dsandwdcosnn.f %%%%%
\begin{routine}{dsandwdcosnn.f}
\begin{verbatim}
      function dsandwdcosnn(p,costheta,kp1,kp2)
c_______________________________________________________________________
c  Annihilation differential invariant rate between particle kp1
c  and kp2 where kp1 and kp2 are neutralinos
c  Input:
c    p - initial cm momentum (real)
c    costheta - cosine of c.m. annihilation angle
c    kp1 - particle code for particle 1
c    kp2 - particle code for particle 2
c  common:
c    'dssusy.h' - file with susy common blocks
c    'diacom.h' - file with kinematical variables
c  Output:
\end{verbatim}
\begin{displaymath}
\frac{dW_{ij}}{d\cos\theta}
\end{displaymath}
where
\begin{displaymath}
W_{ij} = 4 p_{ij} \sqrt{s} \sigma_{ij} 
= 4 \sigma_{ij} \sqrt{(p_i \cdot p_j)^2 - m_i^2 m_j^2} 
= 4 E_{i} E_{j} \sigma_{ij} v_{ij}
\end{displaymath}
\begin{verbatim}
c  The returned dW/dcos(theta) is unitless.
c     
c  uses dsanclearaa,dsansumaa
c  called by dsandwdcos.
c  note: the 32pi in the partial cross sections is 8pi g_1^2, g_1=2
c  author: joakim edsjo (edsjo@physto.se)
c  date: 96-02-21
c  modified: 01-09-12
c=======================================================================
\end{verbatim}
 \end{routine}

%%%%% routine dsandwdcoss.f %%%%%
\begin{routine}{dsandwdcoss.f}
\begin{verbatim}
      function dsandwdcoss(costheta)
c_______________________________________________________________________
c  10^15*annihilation differential invariant rate.
c  input:
c    p - initial cm momentum (real) for lsp annihilations via common
c    costheta - cosine of c.m. annihilation angle
c  common:
c    'dssusy.h' - file with susy common blocks
c  uses dwdcos
c  used for gaussian integration with gadap.f
c  author: joakim edsjo (edsjo@physto.se)
c  date: 97-01-09
c=======================================================================
\end{verbatim}
 \end{routine}

%%%%% routine dsandwdcosy.f %%%%%
\begin{routine}{dsandwdcosy.f}
\begin{verbatim}
      function dsandwdcosy(y)
c_______________________________________________________________________
c  10^15*annihilation differential invariant rate.
c  the integration variable is changed from cos(theta) to
c    y=1/(mx^2+2p^2(1-cos(theta))) for cos(theta)>0 and to
c    y=1/(mx^2+2p^2(1-cos(theta))) for cos(theta)<0.
c  this avoids the poles at cos(theta)=+-1
c  integrate this function from 1/(mx^2+2p^2) to 1/mx^2 to get
c  1d15 times the integral from cos(theta)=0 to 1.
c  input:
c    y - initial cm momentum (real) for lsp annihilations via common
c  common:
c    'dssusy.h' - file with susy common blocks
c  uses dwdcos
c  used for gaussian integration with gadap.f
c  author: joakim edsjo (edsjo@physto.se)
c  date: 98-05-03
c=======================================================================
\end{verbatim}
 \end{routine}

%%%%% routine dsankinvar.f %%%%%
\begin{routine}{dsankinvar.f}
\begin{verbatim}
      subroutine dsankinvar(p,costheta,kp1,kp2,kpk,kp3,kp4)

c=======================================================================
c  calculate kinematical variables for s-, t-, and u-diagram
c  author: joakim edsjo, edsjo@physto.se
c  date: 97-01-09
c=======================================================================
\end{verbatim}
 \end{routine}

%%%%% routine dsanset.f %%%%%
\begin{routine}{dsanset.f}
\begin{verbatim}
      subroutine dsanset(c)
c...set parameters for annihilation routines
c...  c - character string specifying choice to be made
c...author: joakim edsjo, 2001-09-12

\end{verbatim}
 \end{routine}

%%%%% routine dsansumaa.f %%%%%
\begin{routine}{dsansumaa.f}
\begin{verbatim}
      real*8 function dsansumaa()
c_______________________________________________________________________
c  sum the amplitude matrix
c  author: joakim edsjo (edsjo@physto.se) 96-02-02
c          paolo gondolo 99-1-15 factor of 3 faster
c  called by: dwdcos
c=======================================================================
\end{verbatim}
 \end{routine}

%%%%% routine dsantucc.f %%%%%
\begin{routine}{dsantucc.f}
\begin{verbatim}
      subroutine dsantucc(p,ind1,ind2)
c_______________________________________________________________________
c  routine to check for t- or u-channel resonances.
c  called by dwdcosopt.
c  author: joakim edsjo (edsjo@physto.se)
c  date: 97-09-17
c=======================================================================
\end{verbatim}
 \end{routine}

%%%%% routine dsantucn.f %%%%%
\begin{routine}{dsantucn.f}
\begin{verbatim}
      subroutine dsantucn(p,ind1,ind2)
c_______________________________________________________________________
c  routine to check when t- or u-resonances occur.
c  called by dwdcosopt.
c  author: joakim edsjo (edsjo@physto.se)
c  date: 97-09-17
c=======================================================================
\end{verbatim}
 \end{routine}

%%%%% routine dsantunn.f %%%%%
\begin{routine}{dsantunn.f}
\begin{verbatim}
      subroutine dsantunn(p,ind1,ind2)
c_______________________________________________________________________
c  routine to check if t- or u-resonances occur for neutralino-neutralino
c  annihilation
c  called by dwdcosopt.
c  author: joakim edsjo (edsjo@physto.se)
c  date: 97-09-17
c=======================================================================
\end{verbatim}
 \end{routine}

%%%%% routine dsantures.f %%%%%
\begin{routine}{dsantures.f}
\begin{verbatim}
      integer function dsantures(kp1,kp2,kp3,kp4,p)

c=======================================================================
c  determine if t- or u-resonances can occur for a given 2 ->2
c  scattering. if t_max>0 or u_max>0, then tures=1, otherwise tures=0
c  author: joakim edsjo, edsjo@physto.se
c  date: 97-09-17
c=======================================================================
\end{verbatim}
 \end{routine}

%%%%% routine dsanwriteaa.f %%%%%
\begin{routine}{dsanwriteaa.f}
\begin{verbatim}
      subroutine dsanwriteaa
c_______________________________________________________________________
c  write out the amplitude matrix
c  author: joakim edsjo (edsjo@physto.se) 95-10-25
c  called by: different routines during debugging
c=======================================================================
\end{verbatim}
 \end{routine}

%%%%% routine dsanwx.f %%%%%
\begin{routine}{dsanwx.f}
\begin{verbatim}
      real*8 function dsanwx(p)
c_______________________________________________________________________
c  Neutralino self-annihilation invariant rate.
c  Input:
c    p - initial cm momentum (real) for lsp annihilations
c  Common:
c    'dssusy.h' - file with susy common blocks
c  Output
\end{verbatim}
\begin{displaymath}
W_{\rm{eff}} = \sum_{ij}\frac{p_{ij}}{p_{11}}
\frac{g_ig_j}{g_1^2} W_{ij} = 
\sum_{ij} \sqrt{\frac{[s-(m_{i}-m_{j})^2][s-(m_{i}+m_{j})^2]}
{s(s-4m_1^2)}} \frac{g_ig_j}{g_1^2} W_{ij}.
\end{displaymath}
where the $p$'s are the momenta, the $g$'s are the internal
degrees of freedom, the $m$'s are the masses and $W_{ij}$ is
the invariant annihilation rate for the included subprocess.
\begin{verbatim}
c  uses dsabsq.
c  called by dsrdfunc, wirate, dsrdwintrp.
c  author: paolo gondolo (gondolo@lpthe.jussieu.fr) 1994
c  modified: joakim edsjo (edsjo@physto.se) 97-09-09
c=======================================================================
\end{verbatim}
 \end{routine}

%%%%% routine dsanwxint.f %%%%%
\begin{routine}{dsanwxint.f}
\begin{verbatim}
      function dsanwxint(p,a,b)
c_______________________________________________________________________
c  neutralino self-annihilation invariant rate integrated between
c  cos(theta)=a and cos(theta)=b.
c  input:
c    p - initial cm momentum (real) for lsp annihilations
c    integration limits a and b
c  common:
c    'dssusy.h' - file with susy common blocks
c  uses dsabsq.
c  called by wx.
c  author: paolo gondolo (gondolo@lpthe.jussieu.fr) 1994
c  modified slightly by joakim edsjo (edsjo@physto.se) 96-04-10
c=======================================================================
\end{verbatim}
 \end{routine}

%%%%% routine dssigmav.f %%%%%
\begin{routine}{dssigmav.f}
\begin{verbatim}
**********************************************************************
*** function dssigmav returns the annihilation cross section
*** sigma v at p=0 for neutralino-neutralino annihilation.
*** if partch=0, the full sigma v is obtained and if partch>0, the
*** cross section to channel partch is obtained, where is defined
*** as follows:
***
***   partch   process
***   ------   -------
***        0   All processes, i.e. the full annihilation cross section
***        1   H1 H1
***        2   H1 H2
***        3   H2 H2
***        4   H3 H3
***        5   H1 H3
***        6   H2 H3
***        7   H- H+
***        8   H1 Z
***        9   H2 Z
***       10   H3 Z
***       11   W- H+ and W+ H-
***       12   Z0 Z0
***       13   W+ W-
***       14   nu_e nu_e-bar
***       15   e+ e-
***       16   nu_mu nu_mu-bar
***       17   mu+ mu-
***       18   nu_tau nu_tau-bar
***       19   tau+ tau-
***       20   u u-bar
***       21   d d-bar
***       22   c c-bar
***       23   s s-bar
***       24   t t-bar
***       25   b b-bar
***       26   gluon gluon (1-loop)
***       27   q q gluon (not implemented yet, put to zero)
***       28   gamma gamma (1-loop)
***       29   Z gamma (1-loop)
***
*** Units of returned cross section: cm^-3 s^-1
**********************************************************************

      function dssigmav(partch)
\end{verbatim}
 \end{routine}

\newpage
\chapter[an1l: Annihilation cross sections (1-loop)]{\codeb{src/an1l}:\\ Annihilation cross sections (1-loop)}
\label{ch:src-an1l}

%%%%%%%%%%%%%%%%%%%%%%%%%%%%%%%%%%%%%%%%%%%%%%%%%%%%%%%%%%%%%%%%%%%%

\section{Annihilation cross sections at 1-loop -- general }

The annihilation cross sections at 1-loop that we have implemented in
\ds\ are those to $\gamma \gamma$, $Z \gamma$ and $g g$. The
derivation of these is described in the works \cite{ggullio,zgullio}.

To see how these routines are called, see the file
\codeb{src/an/dsandwdcosnn.f} where the $\gamma \gamma$, $Z \gamma$
and $g g$ contributions are added to the annihilation cross section at
the end.
\section{Routine headers -- fortran files}

%%%%% routine dsanggim.f %%%%%
\begin{routine}{dsanggim.f}
\begin{verbatim}
c====================================================================
c
c   this subroutine gives the imaginary part of the amplitude of the 
c   process of neutralino annihilation into two photons in the limit 
c   of vanishing relative velocity of the neutralino pair
c
c   l. bergstrom & p. ullio, nucl. phys. b 504 (1997) 27
c
c   imres: imaginary part
c   imfbxg: contribution from diagram 1a  divided by imres
c   imftxg: contribution from diagrams 1c & 1d  divided by imres
c   imgbxg: contribution from diagram 3a  divided by imres
c
c   author: piero ullio (piero@tapir.caltech.edu)
c
c____________________________________________________________________

      subroutine dsanggim(imres)
\end{verbatim}
 \end{routine}

%%%%% routine dsanggimpar.f %%%%%
\begin{routine}{dsanggimpar.f}
\begin{verbatim}
c====================================================================
c
c   this subroutine gives the imaginary part of the amplitude of the 
c   process of neutralino annihilation into two photons in the limit 
c   of vanishing relative velocity of the neutralino pair
c
c   l. bergstrom & p. ullio, nucl. phys. b 504 (1997) 27
c
c   see header of dsanggim.f for details
c
c   author: piero ullio (piero@tapir.caltech.edu)
c
c____________________________________________________________________

      subroutine dsanggimpar(imres,imfbx,imftx,imgbx)
\end{verbatim}
 \end{routine}

%%%%% routine dsanggre.f %%%%%
\begin{routine}{dsanggre.f}
\begin{verbatim}
c====================================================================
c
c   this subroutine gives the real part of the amplitude of the 
c   process of neutralino annihilation into two photons in the limit 
c   of vanishing relative velocity of the neutralino pair
c
c   l. bergstrom & p. ullio, nucl. phys. b 504 (1997) 27
c
c   reres: real part
c   refbxg: contribution from diagram 1a & 1b  divided by reres
c   reftxg: contribution from diagrams 1c & 1d  divided by reres
c   rehbxg: contribution from diagram 2a & 2b  divided by reres
c   rehtxg: contribution from diagrams 2c & 2d  divided by reres
c   regbxg: contribution from diagram 3a - 3c & 4a -4b divided by 
c     reres
c
c   author: piero ullio (piero@tapir.caltech.edu)
c
c____________________________________________________________________


      subroutine dsanggre(reres)
\end{verbatim}
 \end{routine}

%%%%% routine dsanggrepar.f %%%%%
\begin{routine}{dsanggrepar.f}
\begin{verbatim}
c====================================================================
c
c   this subroutine gives the imaginary part of the amplitude of the 
c   process of neutralino annihilation into two photons in the limit 
c   of vanishing relative velocity of the neutralino pair
c
c   l. bergstrom & p. ullio, nucl. phys. b 504 (1997) 27
c
c   see header of dsanggre.f for details
c
c   author: piero ullio (piero@tapir.caltech.edu)
c
c____________________________________________________________________

      subroutine dsanggrepar(reres,refbx,reftx,rehbx,rehtx,regbx)
\end{verbatim}
 \end{routine}

%%%%% routine dsanglglim.f %%%%%
\begin{routine}{dsanglglim.f}
\begin{verbatim}
c====================================================================
c
c   this subroutine gives the imaginary part of the amplitude of the 
c   process of neutralino annihilation into two gluons in the limit 
c   of vanishing relative velocity of the neutralino pair
c
c   l. bergstrom & p. ullio, nucl. phys. b 504 (1997) 27
c
c   imres: imaginary part
c
c   author: piero ullio (piero@tapir.caltech.edu)
c
c____________________________________________________________________

      subroutine dsanglglim(imres)
\end{verbatim}
 \end{routine}

%%%%% routine dsanglglre.f %%%%%
\begin{routine}{dsanglglre.f}
\begin{verbatim}
c====================================================================
c
c   this subroutine gives the real part of the amplitude of the 
c   process of neutralino annihilation into two gluons in the limit 
c   of vanishing relative velocity of the neutralino pair
c
c   l. bergstrom & p. ullio, nucl. phys. b 504 (1997) 27
c
c   reres: real part
c
c   author: piero ullio (piero@tapir.caltech.edu)
c
c____________________________________________________________________

      subroutine dsanglglre(reres)
\end{verbatim}
 \end{routine}

%%%%% routine dsanzg.f %%%%%
\begin{routine}{dsanzg.f}
\begin{verbatim}
c====================================================================
c
c   this subroutine gives the real and imaginary parts of the 
c   amplitude of the process of neutralino annihilation into 
c   one photon and one z boson in the limit of vanishing relative 
c   velocity of the neutralino pair
c
c   p. ullio & l. bergstrom, phys. rev. d 57 (1998) 1962
c
c   the present version assumes equal sfermion mass eigenstates in
c   fermion - sfermion loop diagrams
c
c   imres: imaginary part
c   imfbxz: contribution to imres from diagram 1a - 1c divided by 
c     imres
c   imftxz: contribution to imres from diagrams 1e - 1h  divided by 
c     imres
c   imgbxz: contribution to imres from diagram 3a & 3b  divided by 
c     imres
c   reres: real part
c   refbxz: contribution to reres from diagram 1a - 1d  divided by 
c     reres
c   reftxz: contribution to reres from diagrams 1e - 1h  divided by 
c     reres
c   rehbxz: contribution to reres from diagram 2a - 2d  divided by 
c     reres
c   rehtxz: contribution to reres from diagrams 2e - 2h  divided by 
c     reres
c   regbxz: contribution from diagram 3a - 3f & 4a -4f divided by 
c     reres
c
c   author: piero ullio (piero@tapir.caltech.edu)
c
c____________________________________________________________________

      subroutine dsanzg(reres,imres)
\end{verbatim}
 \end{routine}

%%%%% routine dsanzgpar.f %%%%%
\begin{routine}{dsanzgpar.f}
\begin{verbatim}
c====================================================================
c
c   this subroutine gives the real and imaginary parts of the 
c   amplitude of the process of neutralino annihilation into 
c   one photon and one z boson in the limit of vanishing relative 
c   velocity of the neutralino pair
c
c   p. ullio & l. bergstrom, phys. rev. d 57 (1998) 1962
c
c   see header of dsanzg.f for details
c
c   author: piero ullio (piero@tapir.caltech.edu)
c
c____________________________________________________________________

      subroutine dsanzgpar(imres,imfbx,imftx,imgbx,reres,refbx,reftx,
     &  rehbx,rehtx,regbx)
\end{verbatim}
 \end{routine}

%%%%% routine dsdilog.f %%%%%
\begin{routine}{dsdilog.f}
\begin{verbatim}
c====================================================================
c
c   dsdilogarithm function
c   argument should be between -1 and 1
c
c   author: lars bergstrom (lbe@physto.se)
c
c____________________________________________________________________

      real*8 function dsdilog(x)
\end{verbatim}
 \end{routine}

%%%%% routine dsdilogp.f %%%%%
\begin{routine}{dsdilogp.f}
\begin{verbatim}
c====================================================================
c
c   auxiliary function used in:
c   dsdilog.f
c
c   author: lars bergstrom (lbe@physto.se)
c
c____________________________________________________________________

      real*8 function dsdilogp(x)
\end{verbatim}
 \end{routine}

%%%%% routine dsfl1c1.f %%%%%
\begin{routine}{dsfl1c1.f}
\begin{verbatim}
c====================================================================
c
c   auxiliary function used in:  
c   dsti_5.f
c
c   author: piero ullio (piero@tapir.caltech.edu)
c
c____________________________________________________________________

      real*8 function dsfl1c1(r1,r2,r3,r4,r5)
\end{verbatim}
 \end{routine}

%%%%% routine dsfl1c2.f %%%%%
\begin{routine}{dsfl1c2.f}
\begin{verbatim}
c====================================================================
c
c   auxiliary function used in:  
c   dsti_5.f
c
c   author: piero ullio (piero@tapir.caltech.edu)
c
c____________________________________________________________________

      real*8 function dsfl1c2(r1,r2,r3,r4,r5)
\end{verbatim}
 \end{routine}

%%%%% routine dsfl2c1.f %%%%%
\begin{routine}{dsfl2c1.f}
\begin{verbatim}
c====================================================================
c
c   auxiliary function used in:  
c   dsti_5.f
c
c   author: piero ullio (piero@tapir.caltech.edu)
c
c____________________________________________________________________

      real*8 function dsfl2c1(r1,r2,r3,r4,r5)
\end{verbatim}
 \end{routine}

%%%%% routine dsfl2c2.f %%%%%
\begin{routine}{dsfl2c2.f}
\begin{verbatim}
c====================================================================
c
c   auxiliary function used in:  
c   dsti_5.f
c
c   author: piero ullio (piero@tapir.caltech.edu)
c
c____________________________________________________________________

      real*8 function dsfl2c2(r1,r2,r3,r4,r5)
\end{verbatim}
 \end{routine}

%%%%% routine dsfl3c1.f %%%%%
\begin{routine}{dsfl3c1.f}
\begin{verbatim}
c====================================================================
c
c   auxiliary function used in:  
c   dsti_5.f
c
c   author: piero ullio (piero@tapir.caltech.edu)
c
c____________________________________________________________________

      real*8 function dsfl3c1(r1,r2,r3,r4,r5)
\end{verbatim}
 \end{routine}

%%%%% routine dsfl3c2.f %%%%%
\begin{routine}{dsfl3c2.f}
\begin{verbatim}
c====================================================================
c
c   auxiliary function used in:  
c   dsti_5.f
c
c   author: piero ullio (piero@tapir.caltech.edu)
c
c____________________________________________________________________

      real*8 function dsfl3c2(r1,r2,r3,r4,r5)
\end{verbatim}
 \end{routine}

%%%%% routine dsfl4c1.f %%%%%
\begin{routine}{dsfl4c1.f}
\begin{verbatim}
c====================================================================
c
c   auxiliary function used in:  
c   dsti_5.f
c
c   author: piero ullio (piero@tapir.caltech.edu)
c
c____________________________________________________________________

      real*8 function dsfl4c1(r1,r2,r3,r4,r5)
\end{verbatim}
 \end{routine}

%%%%% routine dsfl4c2.f %%%%%
\begin{routine}{dsfl4c2.f}
\begin{verbatim}
c====================================================================
c
c   auxiliary function used in:  
c   dsti_5.f
c
c   author: piero ullio (piero@tapir.caltech.edu)
c
c____________________________________________________________________

      real*8 function dsfl4c2(r1,r2,r3,r4,r5)
\end{verbatim}
 \end{routine}

%%%%% routine dsi_12.f %%%%%
\begin{routine}{dsi\_12.f}
\begin{verbatim}
c====================================================================
c
c   auxiliary function used in:  
c   not used, this is here just for notation
c
c   author: piero ullio (piero@tapir.caltech.edu)
c
c____________________________________________________________________

      real*8 function dsi_12(r1,r2)
\end{verbatim}
 \end{routine}

%%%%% routine dsi_13.f %%%%%
\begin{routine}{dsi\_13.f}
\begin{verbatim}
c====================================================================
c
c   auxiliary function used in:  
c   dsanzgpar.f  dsi_13.f
c
c   author: piero ullio (piero@tapir.caltech.edu)
c
c____________________________________________________________________

      real*8 function dsi_13(r1,r2,r3)
\end{verbatim}
 \end{routine}

%%%%% routine dsi_14.f %%%%%
\begin{routine}{dsi\_14.f}
\begin{verbatim}
c====================================================================
c
c   auxiliary function used in:  
c   dsanzgpar.f  dsi_13.f
c
c   author: piero ullio (piero@tapir.caltech.edu)
c
c____________________________________________________________________

      real*8 function dsi_14(r1,r2,r3,r4)
\end{verbatim}
 \end{routine}

%%%%% routine dsi_22.f %%%%%
\begin{routine}{dsi\_22.f}
\begin{verbatim}
c====================================================================
c
c   auxiliary function used in:  
c   not used, this is equivalent to dspiw2.f
c
c   author: piero ullio (piero@tapir.caltech.edu)
c
c____________________________________________________________________

      real*8 function dsi_22(r1,r2)
\end{verbatim}
 \end{routine}

%%%%% routine dsi_23.f %%%%%
\begin{routine}{dsi\_23.f}
\begin{verbatim}
      real*8 function dsi_23(r1,r2,r3)
No header found.
\end{verbatim}
 \end{routine}

%%%%% routine dsi_24.f %%%%%
\begin{routine}{dsi\_24.f}
\begin{verbatim}
      real*8 function dsi_24(r1,r2,r3,r4)
No header found.
\end{verbatim}
 \end{routine}

%%%%% routine dsi_32.f %%%%%
\begin{routine}{dsi\_32.f}
\begin{verbatim}
c====================================================================
c
c   auxiliary function used in:  
c   not used, this is equivalent to dspiw3.f
c
c   author: piero ullio (piero@tapir.caltech.edu)
c
c____________________________________________________________________

      real*8 function dsi_32(r1,r2)
\end{verbatim}
 \end{routine}

%%%%% routine dsi_33.f %%%%%
\begin{routine}{dsi\_33.f}
\begin{verbatim}
c====================================================================
c
c   auxiliary function used in:  
c   dsanzgpar.f
c
c   author: piero ullio (piero@tapir.caltech.edu)
c
c____________________________________________________________________

      real*8 function dsi_33(r1,r2,r3)
\end{verbatim}
 \end{routine}

%%%%% routine dsi_34.f %%%%%
\begin{routine}{dsi\_34.f}
\begin{verbatim}
c====================================================================
c
c   auxiliary function used in:  
c   dsanzgpar.f  dsi_32.f  dsi_33.f
c
c   author: piero ullio (piero@tapir.caltech.edu)
c
c____________________________________________________________________

      real*8 function dsi_34(r1,r2,r3,r4)
\end{verbatim}
 \end{routine}

%%%%% routine dsi_41.f %%%%%
\begin{routine}{dsi\_41.f}
\begin{verbatim}
c====================================================================
c
c   auxiliary function used in:  
c   dsanzgpar.f
c
c   author: piero ullio (piero@tapir.caltech.edu)
c
c____________________________________________________________________

      real*8 function dsi_41(a,b,c)
\end{verbatim}
 \end{routine}

%%%%% routine dsi_42.f %%%%%
\begin{routine}{dsi\_42.f}
\begin{verbatim}
c====================================================================
c
c   auxiliary function used in:  
c   dsanzgpar.f
c
c   author: piero ullio (piero@tapir.caltech.edu)
c
c____________________________________________________________________

      real*8 function dsi_42(a,b,d,c)
\end{verbatim}
 \end{routine}

%%%%% routine dsilp2.f %%%%%
\begin{routine}{dsilp2.f}
\begin{verbatim}
c====================================================================
c
c   auxiliary function used in:  
c   dslp2.f
c
c   author: piero ullio (piero@tapir.caltech.edu)
c
c____________________________________________________________________

      real*8 function dsilp2(x)
\end{verbatim}
 \end{routine}

%%%%% routine dsj_1.f %%%%%
\begin{routine}{dsj\_1.f}
\begin{verbatim}
c====================================================================
c
c   auxiliary function used in:  
c   dsanzgpar.f
c
c   author: piero ullio (piero@tapir.caltech.edu)
c
c____________________________________________________________________

      real*8 function dsj_1(a,b)
\end{verbatim}
 \end{routine}

%%%%% routine dsj_2.f %%%%%
\begin{routine}{dsj\_2.f}
\begin{verbatim}
c====================================================================
c
c   auxiliary function used in:  
c   dsanzgpar.f
c
c   author: piero ullio (piero@tapir.caltech.edu)
c
c____________________________________________________________________

      real*8 function dsj_2(b,c)
\end{verbatim}
 \end{routine}

%%%%% routine dsj_3.f %%%%%
\begin{routine}{dsj\_3.f}
\begin{verbatim}
c====================================================================
c
c   auxiliary function used in:  
c   dsanzgpar.f
c
c   author: piero ullio (piero@tapir.caltech.edu)
c
c____________________________________________________________________

      real*8 function dsj_3(a,b,c)
\end{verbatim}
 \end{routine}

%%%%% routine dslp2.f %%%%%
\begin{routine}{dslp2.f}
\begin{verbatim}
c====================================================================
c
c   auxiliary function used in:  
c   dsi_14.f  dsi_24.f  dsi_34.f  dsilp2.f dsti_5.f
c
c   author: piero ullio (piero@tapir.caltech.edu)
c
c____________________________________________________________________

      real*8 function dslp2(c1,c2)
c this function compute the integral between 0 and 1 of 1/x*log(1+c1*x+c2*x**2)
\end{verbatim}
 \end{routine}

%%%%% routine dspi1.f %%%%%
\begin{routine}{dspi1.f}
\begin{verbatim}
c====================================================================
c
c   auxiliary function used in:  
c   dsanggrepar.f  dsanglglre.f  dsi_12.f  dsrepfbox.f  dsrepgh.f  dsrepw.f 
c
c   author: piero ullio (piero@tapir.caltech.edu)
c
c____________________________________________________________________

      real*8 function dspi1(a,b)
\end{verbatim}
 \end{routine}

%%%%% routine dspiw2.f %%%%%
\begin{routine}{dspiw2.f}
\begin{verbatim}
c====================================================================
c
c   auxiliary function used in:  
c   dsanglglre.f  dsrepfbox.f  dsrepgh.f  dsrepw.f 
c
c   author: piero ullio (piero@tapir.caltech.edu)
c
c____________________________________________________________________

      real*8 function dspiw2(a,b)
\end{verbatim}
 \end{routine}

%%%%% routine dspiw2i.f %%%%%
\begin{routine}{dspiw2i.f}
\begin{verbatim}
c====================================================================
c
c   auxiliary function used in:  
c   dspiw2.f 
c
c   author: piero ullio (piero@tapir.caltech.edu)
c
c____________________________________________________________________

      real*8 function dspiw2i(x)
\end{verbatim}
 \end{routine}

%%%%% routine dspiw3.f %%%%%
\begin{routine}{dspiw3.f}
\begin{verbatim}
c====================================================================
c
c   auxiliary function used in:  
c   dsanglglre.f  dsrepfbox.f  dsrepgh.f  dsrepw.f 
c
c   author: piero ullio (piero@tapir.caltech.edu)
c
c____________________________________________________________________

      real*8 function dspiw3(a,b)
\end{verbatim}
 \end{routine}

%%%%% routine dspiw3i.f %%%%%
\begin{routine}{dspiw3i.f}
\begin{verbatim}
c====================================================================
c
c   auxiliary function used in:  
c   dspiw3.f 
c
c   author: piero ullio (piero@tapir.caltech.edu)
c
c____________________________________________________________________

      real*8 function dspiw3i(x)
\end{verbatim}
 \end{routine}

%%%%% routine dsrepfbox.f %%%%%
\begin{routine}{dsrepfbox.f}
\begin{verbatim}
c====================================================================
c
c   auxiliary function used in:  
c   dsanggrepar.f 
c
c   author: piero ullio (piero@tapir.caltech.edu)
c
c____________________________________________________________________

      real*8 function dsrepfbox(a,b,sq,dq,signm)
\end{verbatim}
 \end{routine}

%%%%% routine dsrepgh.f %%%%%
\begin{routine}{dsrepgh.f}
\begin{verbatim}
c====================================================================
c
c   auxiliary function used in:  
c   dsanggrepar.f 
c
c   author: piero ullio (piero@tapir.caltech.edu)
c
c____________________________________________________________________

      real*8 function dsrepgh(a,b,sq,dq,signm)
\end{verbatim}
 \end{routine}

%%%%% routine dsrepw.f %%%%%
\begin{routine}{dsrepw.f}
\begin{verbatim}
c====================================================================
c
c   auxiliary function used in:  
c   dsanggrepar.f 
c
c   author: piero ullio (piero@tapir.caltech.edu)
c
c____________________________________________________________________

      real*8 function dsrepw(a,b,sq,dq,signm)
\end{verbatim}
 \end{routine}

%%%%% routine dsslc1.f %%%%%
\begin{routine}{dsslc1.f}
\begin{verbatim}
c====================================================================
c
c   auxiliary function used in:  
c   dsi_14.f  dsi_24.f  dsi_34.f 
c
c   author: piero ullio (piero@tapir.caltech.edu)
c
c____________________________________________________________________

      real*8 function dsslc1(r1,r2,r3)
c this function gives the coefficient c1 of slog
\end{verbatim}
 \end{routine}

%%%%% routine dsslc2.f %%%%%
\begin{routine}{dsslc2.f}
\begin{verbatim}
c====================================================================
c
c   auxiliary function used in:  
c   dsi_14.f  dsi_24.f  dsi_34.f 
c
c   author: piero ullio (piero@tapir.caltech.edu)
c
c____________________________________________________________________

      real*8 function dsslc2(r1,r2,r3)
c this function gives the coefficient c2 of slog
\end{verbatim}
 \end{routine}

%%%%% routine dssubka.f %%%%%
\begin{routine}{dssubka.f}
\begin{verbatim}
c====================================================================
c
c   auxiliary function used in:  
c   dsi_41.f  dsi_42.f
c
c   author: piero ullio (piero@tapir.caltech.edu)
c
c____________________________________________________________________


      real*8 function dssubka(r,delta)
\end{verbatim}
 \end{routine}

%%%%% routine dssubkb.f %%%%%
\begin{routine}{dssubkb.f}
\begin{verbatim}
c====================================================================
c
c   auxiliary function used in:  
c   dsi_41.f  dsi_42.f
c
c   author: piero ullio (piero@tapir.caltech.edu)
c
c____________________________________________________________________

      subroutine dssubkb(r1,r2,delta,res1,res2)
\end{verbatim}
 \end{routine}

%%%%% routine dssubkc.f %%%%%
\begin{routine}{dssubkc.f}
\begin{verbatim}
c====================================================================
c
c   auxiliary function used in:  
c   dsi_42.f
c
c   author: piero ullio (piero@tapir.caltech.edu)
c
c____________________________________________________________________

      subroutine dssubkc(r1,delta,res1,res2)
\end{verbatim}
 \end{routine}

%%%%% routine dsti_214.f %%%%%
\begin{routine}{dsti\_214.f}
\begin{verbatim}
c====================================================================
c
c   auxiliary function used in:  
c   dsanzgpar.f 
c
c   author: piero ullio (piero@tapir.caltech.edu)
c
c____________________________________________________________________

      real*8 function dsti_214(r1,r2,r3,r4)
\end{verbatim}
 \end{routine}

%%%%% routine dsti_224.f %%%%%
\begin{routine}{dsti\_224.f}
\begin{verbatim}
c====================================================================
c
c   auxiliary function used in:  
c   dsanzgpar.f 
c
c   author: piero ullio (piero@tapir.caltech.edu)
c
c____________________________________________________________________

      real*8 function dsti_224(r1,r2,r3,r4)
\end{verbatim}
 \end{routine}

%%%%% routine dsti_23.f %%%%%
\begin{routine}{dsti\_23.f}
\begin{verbatim}
c====================================================================
c
c   auxiliary function used in:  
c   dsanzgpar.f 
c
c   author: piero ullio (piero@tapir.caltech.edu)
c
c____________________________________________________________________

      real*8 function dsti_23(r1,r2,r3)
\end{verbatim}
 \end{routine}

%%%%% routine dsti_33.f %%%%%
\begin{routine}{dsti\_33.f}
\begin{verbatim}
c====================================================================
c
c   auxiliary function used in:  
c   dsanzgpar.f 
c
c   author: piero ullio (piero@tapir.caltech.edu)
c
c____________________________________________________________________

      real*8 function dsti_33(r1,r2,r3)
\end{verbatim}
 \end{routine}

%%%%% routine dsti_5.f %%%%%
\begin{routine}{dsti\_5.f}
\begin{verbatim}
c====================================================================
c
c   auxiliary function used in:  
c   dsti_214.f  dsti_224.f  dsti_23.f dsti_33.f 
c
c   author: piero ullio (piero@tapir.caltech.edu)
c
c____________________________________________________________________

      real*8 function dsti_5(r1,r2,r3,r4,r5)
\end{verbatim}
 \end{routine}

\newpage
\chapter[anstu: $t$, $u$ and $s$ diagrams for $ff$-annihilation]{\codeb{src/anstu}:\\ $t$, $u$ and $s$ diagrams for $ff$-annihilation}
\label{ch:src-anstu}

%%%%%%%%%%%%%%%%%%%%%%%%%%%%%%%%%%%%%%%%%%%%%%%%%%%%%%%%%%%%%%%%%%%%

\section{Annihilation amplitudes for fermion-fermion annihilation}

In this directory, all the helicity amplitudes needed for
neutralino-neutralino, neutralino-chargino and chargino-chargino
annihilation are calculated. The helicity amplitudes have been
calculated with general expressions for vertices, masses etc.\ in
\code{Reduce} and converted to Fortran files. The calculation of these
are described in more detail in \cite{edsjo97}.

Each routine here adds the contribution to the helicity amplitudes
from one particular diagram and the sum over contributed diagrams is
done in the routines \codeb{an/dsandwdcosnn}, \codeb{an/dsandwdcoscn}
and \codeb{an/dsandwdcoscc}. The naming convention for the routines
here is the following: The first part of the routine name is
\codeb{dsan} to indicate that they deal with annihilations in \ds. The
next character tells which kind of process it is $s$-, $t$- or
$u$-channel and the next two caracters tell which initial state
particles we have (\codeb{f} for fermion), the next character is the
kind of propagating particle (\codeb{f} for fermion, \codeb{s} for
scalar and \codeb{v} for vector boson), and finally, the last two
characters tell the kind of final state particles. So, to take an
example, the routine \codeb{dsansffsvv} calculates the helicity
amplitudes for annihilation of two fermions to two vector
bosons via $s$-channel exchange of a scalar. There are also a few
special cases (routines ending in \codeb{ex} or \codeb{in}) for
diagrams with clashing arrows.
\section{Routine headers -- fortran files}

%%%%% routine dsansffsff.f %%%%%
\begin{routine}{dsansffsff.f}
\begin{verbatim}
****************************************************
*** subroutine dsansffsff                        ***
*** fermion + fermion -> fermion + fermion in    ***
*** s-channel scalar exchange (index k)          ***
*** 1 - arrow in, 2 - arrow out, k intermediate  ***
*** this code is computer generated by reduce    ***
*** and gentran.                                 ***
*** author: joakim edsjo, edsjo@physto.se        ***
****************************************************

      subroutine dsansffsff(p,costheta,kp1,kp2,kpk,kp3,kp4)
\end{verbatim}
 \end{routine}

%%%%% routine dsansffsss.f %%%%%
\begin{routine}{dsansffsss.f}
\begin{verbatim}
****************************************************
*** subroutine dsansffsss                        ***
*** fermion + fermion -> scalar + scalar in      ***
*** s-channel scalar exchange (index k)          ***
*** 1 - arrow in, 2 - arrow out, k intermediate  ***
*** this code is computer generated by reduce    ***
*** and gentran.                                 ***
*** author: joakim edsjo, edsjo@physto.se        ***
****************************************************

      subroutine dsansffsss(p,costheta,kp1,kp2,kpk,kp3,kp4)
\end{verbatim}
 \end{routine}

%%%%% routine dsansffssv.f %%%%%
\begin{routine}{dsansffssv.f}
\begin{verbatim}
****************************************************
*** subroutine dsansffssv                        ***
*** fermion + fermion -> scalar + gauge boson in ***
*** s-channel scalar exchange (index k)          ***
*** 1 - arrow in, 2 - arrow out, k intermediate  ***
*** this code is computer generated by reduce    ***
*** and gentran.                                 ***
*** author: joakim edsjo, edsjo@physto.se        ***
****************************************************

      subroutine dsansffssv(p,costheta,kp1,kp2,kpk,kp3,kp4)
\end{verbatim}
 \end{routine}

%%%%% routine dsansffsvs.f %%%%%
\begin{routine}{dsansffsvs.f}
\begin{verbatim}
****************************************************
*** subroutine dsansffsvs                        ***
*** fermion + fermion -> scalar + gauge boson in ***
*** s-channel scalar exchange (index k)          ***
*** 1 - arrow in, 2 - arrow out, k intermediate  ***
*** this code is computer generated by reduce    ***
*** and gentran.                                 ***
*** author: joakim edsjo, edsjo@physto.se        ***
****************************************************

      subroutine dsansffsvs(p,costheta,kp1,kp2,kpk,kp3,kp4)
\end{verbatim}
 \end{routine}

%%%%% routine dsansffsvv.f %%%%%
\begin{routine}{dsansffsvv.f}
\begin{verbatim}
*********************************************************
*** subroutine dsansffsvv                             ***
*** fermion + fermion -> gauge boson + gauge boson in ***
*** s-channel scalar exchange (index k)               ***
*** 1 - arrow in, 2 - arrow out, k intermediate       ***
*** this code is computer generated by reduce         ***
*** and gentran.                                      ***
*** author: joakim edsjo, edsjo@physto.se             ***
*********************************************************

      subroutine dsansffsvv(p,costheta,kp1,kp2,kpk,kp3,kp4)
\end{verbatim}
 \end{routine}

%%%%% routine dsansffvff.f %%%%%
\begin{routine}{dsansffvff.f}
\begin{verbatim}
********************************************************
*** subroutine dsansffvff                            ***
*** fermion + fermion -> fermion + fermion in        ***
*** s-channel gauge boson exchange (index k)         ***
*** 1 - arrow in, 2 - arrow out, k intermediate      ***
*** this code is computer generated by reduce        ***
*** and gentran.                                     ***
*** author: joakim edsjo, edsjo@physto.se            ***
********************************************************

      subroutine dsansffvff(p,costheta,kp1,kp2,kpk,kp3,kp4)
\end{verbatim}
 \end{routine}

%%%%% routine dsansffvss.f %%%%%
\begin{routine}{dsansffvss.f}
\begin{verbatim}
********************************************************
*** subroutine dsansffvss                            ***
*** fermion + fermion -> scalar + scalar in          ***
*** s-channel gauge boson exchange (index k)         ***
*** 1 - arrow in, 2 - arrow out, k intermediate      ***
*** this code is computer generated by reduce        ***
*** and gentran.                                     ***
*** author: joakim edsjo, edsjo@physto.se            ***
********************************************************

      subroutine dsansffvss(p,costheta,kp1,kp2,kpk,kp3,kp4)
\end{verbatim}
 \end{routine}

%%%%% routine dsansffvsv.f %%%%%
\begin{routine}{dsansffvsv.f}
\begin{verbatim}
********************************************************
*** subroutine dsansffvsv                            ***
*** fermion + fermion -> scalar + gauge boson in     ***
*** s-channel gauge boson exchange (index k)         ***
*** 1 - arrow in, 2 - arrow out, k intermediate      ***
*** this code is computer generated by reduce        ***
*** and gentran.                                     ***
*** author: joakim edsjo, edsjo@physto.se            ***
********************************************************

      subroutine dsansffvsv(p,costheta,kp1,kp2,kpk,kp3,kp4)
\end{verbatim}
 \end{routine}

%%%%% routine dsansffvvs.f %%%%%
\begin{routine}{dsansffvvs.f}
\begin{verbatim}
********************************************************
*** subroutine dsansffvvs                            ***
*** fermion + fermion -> scalar + gauge boson in     ***
*** s-channel gauge boson exchange (index k)         ***
*** 1 - arrow in, 2 - arrow out, k intermediate      ***
*** this code is computer generated by reduce        ***
*** and gentran.                                     ***
*** author: joakim edsjo, edsjo@physto.se            ***
********************************************************

      subroutine dsansffvvs(p,costheta,kp1,kp2,kpk,kp3,kp4)
\end{verbatim}
 \end{routine}

%%%%% routine dsansffvvv.f %%%%%
\begin{routine}{dsansffvvv.f}
\begin{verbatim}
*********************************************************
*** subroutine dsansffvvv                             ***
*** fermion + fermion -> gauge boson + gauge boson in ***
*** s-channel gauge boson exchange (index k)          ***
*** 1 - arrow in, 2 - arrow out, k intermediate       ***
*** this code is computer generated by reduce         ***
*** and gentran.                                      ***
*** author: joakim edsjo, edsjo@physto.se             ***
*********************************************************

      subroutine dsansffvvv(p,costheta,kp1,kp2,kpk,kp3,kp4)
\end{verbatim}
 \end{routine}

%%%%% routine dsantfffss.f %%%%%
\begin{routine}{dsantfffss.f}
\begin{verbatim}
****************************************************
*** subroutine dsantfffss                        ***
*** fermion + fermion -> scalar + scalar in      ***
*** t-channel fermion exchange (index k)         ***
*** 1 - arrow in, 2 - arrow out, k intermediate  ***
*** this code is computer generated by reduce    ***
*** and gentran.                                 ***
*** author: joakim edsjo, edsjo@physto.se        ***
****************************************************

      subroutine dsantfffss(p,costheta,kp1,kp2,kpk,kp3,kp4)
\end{verbatim}
 \end{routine}

%%%%% routine dsantfffssex.f %%%%%
\begin{routine}{dsantfffssex.f}
\begin{verbatim}
****************************************************
*** subroutine dsantfffssex                      ***
*** fermion + fermion -> scalar + scalar in      ***
*** t-channel fermion exchange (index k)         ***
*** label 3 & 4 exchanged compared to tfffss     ***
*** 1 - arrow in, 2 - arrow out, k intermediate  ***
*** this code is computer generated by reduce    ***
*** and gentran.                                 ***
*** author: joakim edsjo, edsjo@physto.se        ***
****************************************************

      subroutine dsantfffssex(p,costheta,kp1,kp2,kpk,kp3,kp4)
\end{verbatim}
 \end{routine}

%%%%% routine dsantfffssin.f %%%%%
\begin{routine}{dsantfffssin.f}
\begin{verbatim}
****************************************************
*** subroutine dsantfffssin                      ***
*** fermion + fermion -> scalar + scalar in      ***
*** t-channel fermion exchange (index k)         ***
*** 1 - arrow in, 2 - arrow out, k intermediate  ***
*** both fermion arrows point inwards            ***
*** this code is computer generated by reduce    ***
*** and gentran.                                 ***
*** author: joakim edsjo, edsjo@physto.se        ***
****************************************************

      subroutine dsantfffssin(p,costheta,kp1,kp2,kpk,kp3,kp4)
\end{verbatim}
 \end{routine}

%%%%% routine dsantfffsv.f %%%%%
\begin{routine}{dsantfffsv.f}
\begin{verbatim}
****************************************************
*** subroutine dsantfffsv                        ***
*** fermion + fermion -> scalar + gauge boson in ***
*** t-channel fermion exchange (index k)         ***
*** 1 - arrow in, 2 - arrow out, k intermediate  ***
*** this code is computer generated by reduce    ***
*** and gentran.                                 ***
*** author: joakim edsjo, edsjo@physto.se        ***
****************************************************

      subroutine dsantfffsv(p,costheta,kp1,kp2,kpk,kp3,kp4)
\end{verbatim}
 \end{routine}

%%%%% routine dsantfffsvin.f %%%%%
\begin{routine}{dsantfffsvin.f}
\begin{verbatim}
****************************************************
*** subroutine dsantfffsvin                      ***
*** fermion + fermion -> scalar + gauge boson in ***
*** t-channel fermion exchange (index k)         ***
*** 1 - arrow in, 2 - arrow out, k intermediate  ***
*** both fermion arrows point inwards            ***
*** this code is computer generated by reduce    ***
*** and gentran.                                 ***
*** author: joakim edsjo, edsjo@physto.se        ***
****************************************************

      subroutine dsantfffsvin(p,costheta,kp1,kp2,kpk,kp3,kp4)
\end{verbatim}
 \end{routine}

%%%%% routine dsantfffvs.f %%%%%
\begin{routine}{dsantfffvs.f}
\begin{verbatim}
****************************************************
*** subroutine dsantfffvs                        ***
*** fermion + fermion -> scalar + gauge boson in ***
*** t-channel fermion exchange (index k)         ***
*** 1 - arrow in, 2 - arrow out, k intermediate  ***
*** this code is computer generated by reduce    ***
*** and gentran.                                 ***
*** author: joakim edsjo, edsjo@physto.se        ***
****************************************************

      subroutine dsantfffvs(p,costheta,kp1,kp2,kpk,kp3,kp4)
\end{verbatim}
 \end{routine}

%%%%% routine dsantfffvsex.f %%%%%
\begin{routine}{dsantfffvsex.f}
\begin{verbatim}
****************************************************
*** subroutine dsantfffvsex                      ***
*** fermion + fermion -> gauge boson + scalar in ***
*** t-channel fermion exchange (index k)         ***
*** 1 - arrow in, 2 - arrow out, k intermediate  ***
*** label 3 & 4 exchanged compared to tfffsv     ***
*** this code is computer generated by reduce    ***
*** and gentran.                                 ***
*** author: joakim edsjo, edsjo@physto.se        ***
****************************************************

      subroutine dsantfffvsex(p,costheta,kp1,kp2,kpk,kp3,kp4)
\end{verbatim}
 \end{routine}

%%%%% routine dsantfffvv.f %%%%%
\begin{routine}{dsantfffvv.f}
\begin{verbatim}
*********************************************************
*** subroutine dsantfffvv                             ***
*** fermion + fermion -> gauge boson + gauge boson in ***
*** t-channel fermion exchange (index k)              ***
*** 1 - arrow in, 2 - arrow out, k intermediate       ***
*** this code is computer generated by reduce         ***
*** and gentran.                                      ***
*** author: joakim edsjo, edsjo@physto.se             ***
*********************************************************

      subroutine dsantfffvv(p,costheta,kp1,kp2,kpk,kp3,kp4)
\end{verbatim}
 \end{routine}

%%%%% routine dsantfffvvex.f %%%%%
\begin{routine}{dsantfffvvex.f}
\begin{verbatim}
*********************************************************
*** subroutine dsantfffvvex                           ***
*** fermion + fermion -> gauge boson + gauge boson in ***
*** t-channel fermion exchange (index k)              ***
*** 1 - arrow in, 2 - arrow out, k intermediate       ***
*** label 3 & 4 exchanged compared to tfffvv          ***
*** this code is computer generated by reduce         ***
*** and gentran.                                      ***
*** author: joakim edsjo, edsjo@physto.se             ***
*********************************************************

      subroutine dsantfffvvex(p,costheta,kp1,kp2,kpk,kp3,kp4)
\end{verbatim}
 \end{routine}

%%%%% routine dsantfffvvin.f %%%%%
\begin{routine}{dsantfffvvin.f}
\begin{verbatim}
*********************************************************
*** subroutine dsantfffvvin                           ***
*** fermion + fermion -> gauge boson + gauge boson in ***
*** t-channel fermion exchange (index k)              ***
*** 1 - arrow in, 2 - arrow out, k intermediate       ***
*** both fermion arrows point inwards                 ***
*** this code is computer generated by reduce         ***
*** and gentran.                                      ***
*** author: joakim edsjo, edsjo@physto.se             ***
*********************************************************

      subroutine dsantfffvvin(p,costheta,kp1,kp2,kpk,kp3,kp4)
\end{verbatim}
 \end{routine}

%%%%% routine dsantffsff.f %%%%%
\begin{routine}{dsantffsff.f}
\begin{verbatim}
****************************************************
*** subroutine dsantffsff                        ***
*** fermion + fermion -> fermion + fermion in    ***
*** t-channel scalar exchange (index k)          ***
*** 1 - arrow in, 2 - arrow out, k intermediate  ***
*** this code is computer generated by reduce    ***
*** and gentran.                                 ***
*** author: joakim edsjo, edsjo@physto.se        ***
****************************************************

      subroutine dsantffsff(p,costheta,kp1,kp2,kpk,kp3,kp4)
\end{verbatim}
 \end{routine}

%%%%% routine dsanufffss.f %%%%%
\begin{routine}{dsanufffss.f}
\begin{verbatim}
****************************************************
*** subroutine dsanufffss                        ***
*** fermion + fermion -> scalar + scalar in      ***
*** u-channel fermion exchange (index k)         ***
*** 1 - arrow in, 2 - arrow out, k intermediate  ***
*** this code is computer generated by reduce    ***
*** and gentran.                                 ***
*** author: joakim edsjo, edsjo@physto.se        ***
****************************************************

      subroutine dsanufffss(p,costheta,kp1,kp2,kpk,kp3,kp4)
\end{verbatim}
 \end{routine}

%%%%% routine dsanufffssin.f %%%%%
\begin{routine}{dsanufffssin.f}
\begin{verbatim}
****************************************************
*** subroutine dsanufffssin                      ***
*** fermion + fermion -> scalar + scalar in      ***
*** u-channel fermion exchange (index k)         ***
*** 1 - arrow in, 2 - arrow out, k intermediate  ***
*** both fermion arrows point inwards            ***
*** this code is computer generated by reduce    ***
*** and gentran.                                 ***
*** author: joakim edsjo, edsjo@physto.se        ***
****************************************************

      subroutine dsanufffssin(p,costheta,kp1,kp2,kpk,kp3,kp4)
\end{verbatim}
 \end{routine}

%%%%% routine dsanufffsv.f %%%%%
\begin{routine}{dsanufffsv.f}
\begin{verbatim}
****************************************************
*** subroutine dsanufffsv                        ***
*** fermion + fermion -> scalar + gauge boson in ***
*** u-channel fermion exchange (index k)         ***
*** 1 - arrow in, 2 - arrow out, k intermediate  ***
*** this code is computer generated by reduce    ***
*** and gentran.                                 ***
*** author: joakim edsjo, edsjo@physto.se        ***
****************************************************

      subroutine dsanufffsv(p,costheta,kp1,kp2,kpk,kp3,kp4)
\end{verbatim}
 \end{routine}

%%%%% routine dsanufffsvin.f %%%%%
\begin{routine}{dsanufffsvin.f}
\begin{verbatim}
****************************************************
*** subroutine dsanufffsvin                      ***
*** fermion + fermion -> scalar + gauge boson in ***
*** u-channel fermion exchange (index k)         ***
*** 1 - arrow in, 2 - arrow out, k intermediate  ***
*** both fermion arrows point inwards            ***
*** this code is computer generated by reduce    ***
*** and gentran.                                 ***
*** author: joakim edsjo, edsjo@physto.se        ***
****************************************************

      subroutine dsanufffsvin(p,costheta,kp1,kp2,kpk,kp3,kp4)
\end{verbatim}
 \end{routine}

%%%%% routine dsanufffvs.f %%%%%
\begin{routine}{dsanufffvs.f}
\begin{verbatim}
****************************************************
*** subroutine dsanufffvs                        ***
*** fermion + fermion -> scalar + gauge boson in ***
*** u-channel fermion exchange (index k)         ***
*** 1 - arrow in, 2 - arrow out, k intermediate  ***
*** this code is computer generated by reduce    ***
*** and gentran.                                 ***
*** author: joakim edsjo, edsjo@physto.se        ***
****************************************************

      subroutine dsanufffvs(p,costheta,kp1,kp2,kpk,kp3,kp4)
\end{verbatim}
 \end{routine}

%%%%% routine dsanufffvv.f %%%%%
\begin{routine}{dsanufffvv.f}
\begin{verbatim}
*********************************************************
*** subroutine dsanufffvv                             ***
*** fermion + fermion -> gauge boson + gauge boson in ***
*** u-channel fermion exchange (index k)              ***
*** 1 - arrow in, 2 - arrow out, k intermediate       ***
*** this code is computer generated by reduce         ***
*** and gentran.                                      ***
*** author: joakim edsjo, edsjo@physto.se             ***
*********************************************************

      subroutine dsanufffvv(p,costheta,kp1,kp2,kpk,kp3,kp4)
\end{verbatim}
 \end{routine}

%%%%% routine dsanufffvvex.f %%%%%
\begin{routine}{dsanufffvvex.f}
\begin{verbatim}
*********************************************************
*** subroutine dsanufffvvex                           ***
*** fermion + fermion -> gauge boson + gauge boson in ***
*** u-channel fermion exchange (index k)              ***
*** label 3 & 4 exchanged compared to ufffvv          ***
*** 1 - arrow in, 2 - arrow out, k intermediate       ***
*** this code is computer generated by reduce         ***
*** and gentran.                                      ***
*** author: joakim edsjo, edsjo@physto.se             ***
*********************************************************

      subroutine dsanufffvvex(p,costheta,kp1,kp2,kpk,kp3,kp4)
\end{verbatim}
 \end{routine}

%%%%% routine dsanufffvvin.f %%%%%
\begin{routine}{dsanufffvvin.f}
\begin{verbatim}
*********************************************************
*** subroutine dsanufffvvin                           ***
*** fermion + fermion -> gauge boson + gauge boson in ***
*** u-channel fermion exchange (index k)              ***
*** 1 - arrow in, 2 - arrow out, k intermediate       ***
*** both fermion arrows point inwards                 ***
*** this code is computer generated by reduce         ***
*** and gentran.                                      ***
*** author: joakim edsjo, edsjo@physto.se             ***
*********************************************************

      subroutine dsanufffvvin(p,costheta,kp1,kp2,kpk,kp3,kp4)
\end{verbatim}
 \end{routine}

%%%%% routine dsanuffsff.f %%%%%
\begin{routine}{dsanuffsff.f}
\begin{verbatim}
****************************************************
*** subroutine dsanuffsff                        ***
*** fermion + fermion -> fermion + fermion in    ***
*** u-channel scalar exchange (index k)          ***
*** 1 - arrow in, 2 - arrow out, k intermediate  ***
*** this code is computer generated by reduce    ***
*** and gentran.                                 ***
*** author: joakim edsjo, edsjo@physto.se        ***
****************************************************

      subroutine dsanuffsff(p,costheta,kp1,kp2,kpk,kp3,kp4)
\end{verbatim}
 \end{routine}

\newpage
\chapter[as: Annihilation cross sections (with sfermions)]{\codeb{src/as}:\\ Annihilation cross sections (with sfermions)}
\label{ch:src-as}

%%%%%%%%%%%%%%%%%%%%%%%%%%%%%%%%%%%%%%%%%%%%%%%%%%%%%%%%%%%%%%%%%%%%

\section{Annihilation cross sections with sfermions -- general }

In this directory, all the (co)annihilation cross sections involving
one or more sfermions in the initial state are calculated. The code
here is based upon the work described in \cite{sfcoann}. All the cross
sections are calculated with \code{Form} and converted to Fortran with
a script \codeb{form2f} \cite{form2f}. 

The main routines here are

\begin{brief-subs}
\bsub{dsasdwdcossfsf}
  Calculates the invariant annihilation rate between two sfermions in
  the initial state.
\bsub{dsasdwdcossfchi}
  Calculates the invariant annihilation rate between one sfermion and
  one fermion (neutralino or chargino) in the iniital state.
\end{brief-subs}

\section{Routine headers -- fortran files}

%%%%% routine dsaschicasea.f %%%%%
\begin{routine}{dsaschicasea.f}
\begin{verbatim}
c...This subroutine is automatically generated from form output by
c...parsing it through form2f (version 1.34, October 8, 2001, edsjo@physto.se)
c....Template file for dsaschicasea begins here

**************************************************************
*** SUBROUTINE dsaschicasea                                ***
*** computes dW_{ij}/dcostheta                             ***
***                                                        ***
*** sfermion(i) + neutralino(j)/chargino^+(j)              ***
*** -> gauge-boson + fermion                               ***
***                                                        ***
*** The sfermion must be the first mentioned               ***
*** particle (kp1) and the neutralino/chargino             ***
*** the other (kp2) -- not the opposite.                   ***
*** For the final state the gauge boson must be mentioned  ***
*** first (i.e. kp3) and next the fermion (kp4) --         ***
*** not the opposite.                                      ***                 
***                                                        ***
***                                                        ***
*** Author:Mia Schelke, schelke@physto.se                  ***
*** Date: 01-10-05                                         ***
*** QCD included: 02-03-21                                 ***
*** comment added by Piero Ullio, 02-07-01                 ***
**************************************************************

***** Note that it is assumed that coupling constants that do 
***** not exist have already been set to zero!!!!!
***** Thus, many of the coefficients defined in this code 
***** simplify when the diagrams contain sneutrinos 
***** or neutrinos.

      subroutine dsaschicasea(kp1,kp2,kp3,kp4,par)
\end{verbatim}
 \end{routine}

%%%%% routine dsaschicaseb.f %%%%%
\begin{routine}{dsaschicaseb.f}
\begin{verbatim}
c...This subroutine is automatically generated from form output by
c...parsing it through form2f (version 1.34, October 8, 2001, edsjo@physto.se)
c....Template file for dsaschicaseb begins here

**************************************************************
*** SUBROUTINE dsaschicaseb                                ***
*** computes dW_{ij}/dcostheta                             ***
***                                                        ***
*** anti-sfermion(i) + neutralino(j)/chargino^+(j)         *** 
*** -> gauge-boson + anti-fermion                          ***
***                                                        ***
*** The anti-sfermion must be the first mentioned          ***
*** particle (kp1) and the neutralino/chargino             *** 
*** the other (kp2) -- not the opposite.                   ***
*** For the final state the gauge boson must be mentioned  ***
*** first (i.e. kp3) and next the anti-fermion (kp4) --    ***
*** not the opposite.                                      ***
***                                                        ***
***                                                        ***
*** Author:Mia Schelke, schelke@physto.se                  ***
*** Date: 01-10-03                                         ***
*** QCD included: 02-03-21                                 ***
*** comment added by Piero Ullio, 02-07-01                 ***
**************************************************************

***** Note that it is assumed that coupling constants that do 
***** not exist have already been set to zero!!!!!
***** Thus, many of the coefficients defined in this code 
***** simplify when the diagrams contain sneutrinos 
***** or neutrinos.

      subroutine dsaschicaseb(kp1,kp2,kp3,kp4,par)
\end{verbatim}
 \end{routine}

%%%%% routine dsaschicasec.f %%%%%
\begin{routine}{dsaschicasec.f}
\begin{verbatim}
c...This subroutine is automatically generated from form output by
c...parsing it through form2f (version 1.34, October 8, 2001, edsjo@physto.se)
c....Template file for dsaschicasec begins here

**************************************************************
*** SUBROUTINE dsaschicasec                                ***
*** computes dW_{ij}/dcostheta                             ***
***                                                        ***
*** sfermion(i) + neutralino(j)/chargino^+(j)              ***
*** -> higgs-boson + fermion                               ***
***                                                        ***
*** The sfermion must be the first mentioned               ***
*** particle (kp1) and the neutralino/chargino             ***
*** the other (kp2) -- not the opposite.                   ***
*** For the final state the higgs boson must be mentioned  ***
*** first (i.e. kp3) and next the fermion (kp4) --         ***
*** not the opposite.                                      ***
***                                                        ***
***                                                        ***
*** Author:Mia Schelke, schelke@physto.se                  ***
*** Date: 01-10-10                                         ***
*** Trivial color factors included: 02-03-21               ***
**************************************************************

***** Note that it is assumed that coupling constants that do 
***** not exist have already been set to zero!!!!!
***** Thus, many of the coefficients defined in this code 
***** simplify when the diagrams contain sneutrinos 
***** or neutrinos.

      subroutine dsaschicasec(kp1,kp2,kp3,kp4,par)
\end{verbatim}
 \end{routine}

%%%%% routine dsaschicased.f %%%%%
\begin{routine}{dsaschicased.f}
\begin{verbatim}
c...This subroutine is automatically generated from form output by
c...parsing it through form2f (version 1.34, October 8, 2001, edsjo@physto.se)
c....Template file for dsaschicased begins here

**************************************************************
*** SUBROUTINE dsaschicased                                ***
*** computes dW_{ij}/dcostheta                             ***
***                                                        ***
*** anti-sfermion(i) + neutralino(j)/chargino^+(j)         ***
*** -> higgs-boson + anti-fermion                          ***
***                                                        ***
*** The anti-sfermion must be the first mentioned          ***
*** particle (kp1) and the neutralino/chargino             ***
*** the other (kp2) -- not the opposite.                   ***
*** For the final state the higgs boson must be mentioned  ***
*** first (i.e. kp3) and next the anti-fermion (kp4) --    ***
*** not the opposite.                                      ***
***                                                        ***
***                                                        ***
*** Author:Mia Schelke, schelke@physto.se                  ***
*** Date: 01-10-11                                         ***
*** Trivial color factors included: 02-03-21               ***
**************************************************************

***** Note that it is assumed that coupling constants that do 
***** not exist have already been set to zero!!!!!
***** Thus, many of the coefficients defined in this code 
***** simplify when the diagrams contain sneutrinos 
***** or neutrinos.

      subroutine dsaschicased(kp1,kp2,kp3,kp4,par)
\end{verbatim}
 \end{routine}

%%%%% routine dsaschizero.f %%%%%
\begin{routine}{dsaschizero.f}
\begin{verbatim}
**************************************************************
*** SUBROUTINE dsaschizero                                 ***
*** computes dW_{ij}/dcostheta                             ***
*** sfermion(i) + neutralino(j) -> gamma/gluon + fermion   ***
*** ampl2 obtained with sum over physical polarizations    ***
***                                                        *** 
*** input askin variables: p12,costheta                    ***
*** AUTHOR: Piero Ullio, ullio@sissa.it                    ***
*** Date: 02-06-13                                         ***
**************************************************************

      SUBROUTINE dsaschizero(kp1,kp2,kp3,kp4,result)
\end{verbatim}
 \end{routine}

%%%%% routine dsascolset.f %%%%%
\begin{routine}{dsascolset.f}
\begin{verbatim}
      subroutine dsascolset(type)
No header found.
\end{verbatim}
 \end{routine}

%%%%% routine dsasdepro.f %%%%%
\begin{routine}{dsasdepro.f}
\begin{verbatim}
**************************************************************
*** FUNCTION dsasdepro                                     ***
*** computes the denominator of a propagator               ***
***                                                        ***
*** input: mom2 is S,T,U;                                  ***
*** kkpp is the number of the particle in the propagator   ***
*** AUTHOR: Piero Ullio, ullio@sissa.it                    ***
*** Date: 01-02-28                                         ***
**************************************************************

      complex*16 function dsasdepro(mom2,kkpp)
\end{verbatim}
 \end{routine}

%%%%% routine dsasdwdcossfchi.f %%%%%
\begin{routine}{dsasdwdcossfchi.f}
\begin{verbatim}
************************************************************************
*** SUBROUTINE dsasdwdcossfchi                                       ***
*** computes dW_{ij}/dcostheta                                       ***
*** for sfermion(1) + neutralino(2) (or chargino(2))                 ***
*** plus sfermion(1) + neutralino(2) (or chargino(2))                ***
***                                                                  *** 
*** AUTHOR: Piero Ullio, ullio@sissa.it                              ***
*** Date: 01-11-04                                                   ***
*** Modified: Joakim Edsjo, Mia Schelke                              ***
***   to include gluon final states, 2002-03-21                      ***
*** Modified: Piero Ullio                                            ***
***   to switch to ampl2 with physical polarizations, 02-07-01       ***
************************************************************************

      real*8 function dsasdwdcossfchi(p,costhe,kp1,kp2)
\end{verbatim}
 \end{routine}

%%%%% routine dsasdwdcossfsf.f %%%%%
\begin{routine}{dsasdwdcossfsf.f}
\begin{verbatim}
************************************************************************
*** SUBROUTINE dsasdwdcossfsf                                        ***
*** computes dW_{ij}/dcostheta                                       ***
*** for sfermion(1) + antisfermion(2) plus sfermion(1) + sfermion(2) ***
***                                                                  *** 
*** AUTHOR: Piero Ullio, ullio@sissa.it                              ***
*** Date: 01-08-10                                                   ***
*** modified: Joakim Edsjo, Mia Schelke to include squarks with      ***
***   gauge and Higgs boson final states and gluons                  ***
***   02-05-22                                                       ***
***   bug with switching of initial states fixed 020613 (edsjo)      ***
*** modified: Piero Ullio                                            ***
***   02-03-22                                                       ***
*** modified: Piero Ullio                                            ***
***   02-07-01                                                       ***
************************************************************************

      real*8 function dsasdwdcossfsf(p,costhe,kp1,kp2)
\end{verbatim}
 \end{routine}

%%%%% routine dsasfer.f %%%%%
\begin{routine}{dsasfer.f}
\begin{verbatim}
**************************************************************
*** SUBROUTINE dsasfer                                     ***
*** computes dW_{ij}/dcostheta                             ***
*** sfermion(i) + antisfermion(j)                          ***
*** -> fermion(k1) + antifermion(k2)                       ***
*** version to be used if i or j has non-zero lepton #     ***
***                                                        *** 
*** input askin variables: p12,costheta                    ***
*** kpk1 and kpk2 are the fermion code                     ***
*** AUTHOR: Piero Ullio, ullio@sissa.it                    ***
*** Date: 01-08-10                                         ***
**************************************************************

      SUBROUTINE dsasfer(kpi,kpj,kpk1,kpk2,result)
\end{verbatim}
 \end{routine}

%%%%% routine dsasfercode.f %%%%%
\begin{routine}{dsasfercode.f}
\begin{verbatim}
**************************************************************
*** SUBROUTINE dsasfercode                                 ***
*** finds fermion kfer in a given family iifamv            ***
*** chow variable equal to                                 ***
*** 'same' : routine returns fermion code for the          ***
***          particles with the same iifamv                ***
*** 'diff' : routine returns fermion code for the          ***
***          particles with the same iifamv+1 or iifamv-1  ***
***                                                        *** 
*** AUTHOR: Piero Ullio, ullio@sissa.it                    ***
*** Date: 01-08-09                                         ***
**************************************************************

      subroutine dsasfercode(chow,iifamv,kfer)
\end{verbatim}
 \end{routine}

%%%%% routine dsasfercol.f %%%%%
\begin{routine}{dsasfercol.f}
\begin{verbatim}
**************************************************************
*** SUBROUTINE dsasfercol                                  ***
*** computes dW_{ij}/dcostheta                             ***
*** sfermion(i) + antisfermion(j)                          ***
*** -> fermion(k1) + antifermion(k2)                       ***
*** version to be used if i or j are both squarks          ***
***                                                        *** 
*** input askin variables: p12,costheta                    ***
*** kpk1 and kpk2 are the fermion code                     ***
*** AUTHOR: Piero Ullio, ullio@sissa.it                    ***
*** Date: 01-08-10                                         ***
**************************************************************

      SUBROUTINE dsasfercol(kpi,kpj,kpk1,kpk2,result)
\end{verbatim}
 \end{routine}

%%%%% routine dsasfere.f %%%%%
\begin{routine}{dsasfere.f}
\begin{verbatim}
**************************************************************
*** SUBROUTINE dsasfere                                    ***
*** computes dW_{ij}/dcostheta                             ***
*** sfermion(i) + sfermion(j)                              ***
*** -> fermion(k1) + fermion(k2)                           ***
*** version to be used if i or j has non-zero lepton #     ***
***                                                        *** 
*** input askin variables: p12,costheta                    ***
*** kpk1 and kpk2 are the fermion code                     ***
*** AUTHOR: Piero Ullio, ullio@sissa.it                    ***
*** Date: 01-08-10                                         ***
**************************************************************

      SUBROUTINE dsasfere(kpi,kpj,kpk1,kpk2,result)
\end{verbatim}
 \end{routine}

%%%%% routine dsasferecol.f %%%%%
\begin{routine}{dsasferecol.f}
\begin{verbatim}
**************************************************************
*** SUBROUTINE dsasferecol                                 ***
*** computes dW_{ij}/dcostheta                             ***
*** sfermion(i) + sfermion(j)                              ***
*** -> fermion(k1) + fermion(k2)                           ***
*** version to be used if i or j are both squarks          ***
***                                                        *** 
*** input askin variables: p12,costheta                    ***
*** kpk1 and kpk2 are the fermion code                     ***
*** AUTHOR: Piero Ullio, ullio@sissa.it                    ***
*** Date: 01-08-10                                         ***
**************************************************************

      SUBROUTINE dsasferecol(kpi,kpj,kpk1,kpk2,result)
\end{verbatim}
 \end{routine}

%%%%% routine dsasff.f %%%%%
\begin{routine}{dsasff.f}
\begin{verbatim}
**************************************************************
*** SUBROUTINE dsasff                                      ***
*** computes the amplitude squared of the process          ***
*** scalar(1) + scalar(2) -> fermion(3) + fermion(4)       ***
***                                                        *** 
*** input:                                                 ***
*** asparmass, askin, askinder variables                   ***
*** complex vectors ASxpl(i), ASxpr(i), ASyl, ASyr         *** 
*** AUTHOR: Piero Ullio, ullio@sissa.it                    ***
*** Date: 01-02-28                                         ***
**************************************************************

      SUBROUTINE dsasff(ampl2)
\end{verbatim}
 \end{routine}

%%%%% routine dsasffcol.f %%%%%
\begin{routine}{dsasffcol.f}
\begin{verbatim}
**************************************************************
*** SUBROUTINE dsasffcol                                   ***
*** computes the amplitude squared of the process          ***
*** scalar(1) + scalar(2) -> fermion(3) + fermion(4)       ***
***                                                        *** 
*** input:                                                 ***
*** asparmass, askin, askinder variables                   ***
*** complex vectors:                                       ***
***   ASxplc(j,i), ASxprc(j,i), ASylc(j), ASyrc(j)         *** 
*** AUTHOR: Piero Ullio, ullio@sissa.it                    ***
*** Date: 01-02-28                                         ***
**************************************************************


      SUBROUTINE dsasffcol(ampl2)
\end{verbatim}
 \end{routine}

%%%%% routine dsasgbgb.f %%%%%
\begin{routine}{dsasgbgb.f}
\begin{verbatim}
c...This subroutine is automatically generated from form output by
c...parsing it through form2f (version 1.35, May 23, 2002, edsjo@physto.se)
c....Template file for dsasgbgb begins here

**************************************************************
*** SUBROUTINE dsasgbgb                                    ***
*** computes dW_{ij}/dcostheta                             ***
***                                                        ***
*** sfermion(i) + anti-sfermion(j)                         ***
*** -> MASSIVE gauge-boson + MASSIVE gauge-boson           ***
***                                                        ***
*** for one massive and one massless gb use dsasgbgb1exp   ***
*** for two massless gb use code dsasgbgb2exp              ***
***                                                        ***
*** The first mentioned particle (kp1) will be taken as    ***
*** a sfermion and the second particle (kp2) as an         ***
*** anti-sfermion -- not the opposite.                     ***
***                                                        ***
*** When kp1 and kp2 have different                        ***
*** weak isospin (T^3=+,-1/2), then kp1 must be an         ***
*** up-type-sfermion and kp2 a down-type-anti-sfermion.    ***
***                                                        ***
*** When one gauge boson have electric charge while the    ***
*** other is neutral, then the charged one must be         ***
*** mentioned first (kp3) and then the neutral one (kp4)   ***
*** -- not the opposite.                                   ***
***                                                        ***
*** Author:Mia Schelke, schelke@physto.se                  ***
*** Date: 01-10-23  rewritten:02-03-12                     ***
*** QCD included: 02-03-20                                 ***
*** Ghost term excluded: 02-05-22                          ***
*** rewritten: 02-07-04 (now only massive gb)              ***
*** added flavour changing charged exchange for W^-W^+:    ***
*** added by Mia Schelke 2005-06-14                        ***
*** terms rearranged by Paolo Gondolo, 2005-06             ***
**************************************************************

      subroutine dsasgbgb(kp1,kp2,kp3,kp4,par)
\end{verbatim}
 \end{routine}

%%%%% routine dsasgbgb1exp.f %%%%%
\begin{routine}{dsasgbgb1exp.f}
\begin{verbatim}
c...This subroutine is automatically generated from form output by
c...parsing it through form2f (version 1.35, May 23, 2002, edsjo@physto.se)
c....Template file for dsasgbgb1exp begins here

**************************************************************
*** SUBROUTINE dsasgbgb1exp                                ***
*** computes dW_{ij}/dcostheta                             ***
***                                                        ***
*** sfermion(i) + anti-sfermion(j)                         ***
*** -> massive gauge-boson + massless gauge-boson          ***
***                                                        ***
***                                                        ***
*** The first mentioned particle (kp1) will be taken as    ***
*** a sfermion and the second particle (kp2) as an         ***
*** anti-sfermion -- not the opposite.                     ***
***                                                        ***
*** When kp1 and kp2 have different                        ***
*** weak isospin (T^3=+,-1/2), then kp1 must be an         ***
*** up-type-sfermion and kp2 a down-type-anti-sfermion.    ***
***                                                        ***
*** NOTE: for the gauge bosons, the MASSIVE must be        ***
*** mentioned first (kp3) and then the MASSLESS one (kp4)  ***
*** -- not the opposite.                                   ***
***                                                        ***
*** Author:Mia Schelke, schelke@physto.se                  ***
*** Date: 01-10-23  rewritten:02-03-12                     ***
*** QCD included: 02-03-20                                 ***
*** rewritten: 02-07-04 (to have exactly one massless gb)  ***
*** explicite pol. vectors introduced: 02-07-05            ***
*** sum over massless pol. moved from                      ***
*** fortran to form: 02-07-09                              ***
***                                                        ***
**************************************************************

      subroutine dsasgbgb1exp(kp1,kp2,kp3,kp4,par)
\end{verbatim}
 \end{routine}

%%%%% routine dsasgbgb2exp.f %%%%%
\begin{routine}{dsasgbgb2exp.f}
\begin{verbatim}
c...This subroutine is automatically generated from form output by
c...parsing it through form2f (version 1.35, May 23, 2002, edsjo@physto.se)
c....Template file for dsasgbgb2exp begins here

**************************************************************
*** SUBROUTINE dsasgbgb2exp                                ***
*** computes dW_{ij}/dcostheta                             ***
***                                                        ***
*** sfermion(i) + anti-sfermion(j)                         ***
*** -> gluon+gluon, photon+photon, photon+gluon            ***
***                                                        ***
*** The first mentioned particle (kp1) will be taken as    ***
*** a sfermion and the second particle (kp2) as an         ***
*** anti-sfermion -- not the opposite.                     ***
***                                                        ***
***                                                        ***
***                                                        ***
*** Author:Mia Schelke, schelke@physto.se                  ***
*** Date: 02-05-21                                         ***
*** Rewritten: 02-07-03 (to have exactly two massless gb)  ***  
*** explicite pol. vectors introduced: 02-07-08            ***
*** sum over pol. moved from fortran to form: 02-07-09     *** 
*** two colour factors(c.f.) made complex: 02-07-10        ***
***(+these c.f. changed for g+g as ggg vertex code changed)*** 
***                                                        *** 
**************************************************************

      subroutine dsasgbgb2exp(kp1,kp2,kp3,kp4,par)
\end{verbatim}
 \end{routine}

%%%%% routine dsasgbhb.f %%%%%
\begin{routine}{dsasgbhb.f}
\begin{verbatim}
**************************************************************
*** SUBROUTINE dsasgbhb                                    ***
*** computes dW_{ij}/dcostheta                             ***
*** up-sfermion(i) + down-antisfermion(j) ->               ***
***    gauge boson + Higgs boson                           ***
*** ampl2 obtained summing over physical polarizations     ***
***                                                        *** 
*** input askin variables: p12,costheta                    ***
***    iifam(1),iifam(2),mass1,mass2                       ***
*** AUTHOR: Piero Ullio, ullio@sissa.it                    ***
*** Date: 02-06-13                                         ***
*** This routine has been compared with the routine of     ***
*** Mia Schelke and the agreement is perfect, except in    ***
*** the low-p limit (due to widths in propagators)         ***
*** as expected.                                           ***
*** added flavour changing charged exchange for W^-H^+:    ***
*** added by Mia Schelke 2006-06-07                        ***
**************************************************************

      SUBROUTINE dsasgbhb(kp1,kp2,kp3,kp4,result)
\end{verbatim}
 \end{routine}

%%%%% routine dsashbhb.f %%%%%
\begin{routine}{dsashbhb.f}
\begin{verbatim}
c...This subroutine is automatically generated from form output by
c...parsing it through form2f (version 1.35, May 23, 2002, edsjo@physto.se)
c....Template file for dsashbhb begins here

**************************************************************
*** SUBROUTINE dsashbhb                                    ***
*** computes dW_{ij}/dcostheta                             ***
***                                                        ***
*** sfermion(i) + anti-sfermion(j)                         ***
*** -> higgs-boson + higgs-boson                           ***
***                                                        ***
*** The first mentioned particle (kp1) will be taken as    ***
*** a sfermion and the second particle (kp2) as an         ***
*** anti-sfermion -- not the opposite.                     ***
***                                                        ***
*** When kp1 and kp2 have different                        ***
*** weak isospin (T^3=+,-1/2), then kp1 must be an         ***
*** up-type-sfermion and kp2 a down-type-anti-sfermion.    ***
***                                                        ***
*** For the cases with one charged and one neutral higgs   ***
*** in the final state, the charged higgs must be          ***
*** mentioned first (i.e. kp3) and next the neutral        ***
*** higgs-boson (kp4) -- not the opposite.                 *** 
***                                                        ***
***                                                        ***
*** Author:Mia Schelke, schelke@physto.se                  ***
*** Date: 01-10-19                                         ***
*** Rewritten: 02-07-03 (because FORM input file now       ***
***  only gives the amplitude)                             ***
*** added flavour changing charged exchange for H^+H^-:    ***
*** added by Mia Schelke 2006-06-08                        ***
**************************************************************
***                                                        ***
*** NOTE: The FORM input file only gives the amplitude     ***
***       not the amplitude squared                        ***
*** THE FOLLOWING THEREFORE HAS TO BE CHANGED BY HAND      ***
*** after running of the PERL script                       ***
***                                                        ***
*** the form result should be denoted amplitude instead    ***
*** of dsashbhbas                                          ***
*** the amplitude squared calculation should be added      ***
*** -- the lines are written in the end of                 *** 
*** the template file                                      ***
**************************************************************



      subroutine dsashbhb(kp1,kp2,kp3,kp4,par)
\end{verbatim}
 \end{routine}

%%%%% routine dsaskinset.f %%%%%
\begin{routine}{dsaskinset.f}
\begin{verbatim}
**************************************************************
*** subroutine dsaskinset                                  ***
*** set askinder variables: the kinematic variables and    ***
*** the scalar products of the four vectors                ***
*** p(1), p(2), k(3) and k(4) related by                   ***
*** p(1) + p(2) = k(3) + k(4)                              ***
*** in the center of mass frame                            ***
*** input: asparmass, askin variables                      ***
***                                                        ***
*** AUTHOR: Piero Ullio, ullio@sissa.it                    ***
*** Date: 01-02-28                                         ***
**************************************************************

      subroutine dsaskinset
\end{verbatim}
 \end{routine}

%%%%% routine dsaskinset1.f %%%%%
\begin{routine}{dsaskinset1.f}
\begin{verbatim}
**************************************************************
*** subroutine dsaskinset1                                 ***
*** sets: ep1, ep2, and Svar                               ***
*** input: mass1, mass2, p12                               ***
***                                                        ***
*** AUTHOR: Piero Ullio, ullio@sissa.it                    ***
*** Date: 01-02-28                                         ***
**************************************************************

      subroutine dsaskinset1
\end{verbatim}
 \end{routine}

%%%%% routine dsaskinset2.f %%%%%
\begin{routine}{dsaskinset2.f}
\begin{verbatim}
**************************************************************
*** subroutine dsaskinset2                                 ***
*** sets: k34, ek3, ek4, Tvar, Uvar                        ***
*** you must call dsaskinset1 before calling dsaskinset2   *** 
*** input: mass3, mass4, costheta                          ***
***                                                        ***
*** AUTHOR: Piero Ullio, ullio@sissa.it                    ***
*** Date: 01-02-28                                         ***
**************************************************************

      subroutine dsaskinset2
\end{verbatim}
 \end{routine}

%%%%% routine dsaskinset3.f %%%%%
\begin{routine}{dsaskinset3.f}
\begin{verbatim}
**************************************************************
*** subroutine dsaskinset3                                 ***
*** sets the scalar products of the four vectors           ***
*** p(1), p(2), k(3) and k(4) related by                   ***
*** p(1) + p(2) = k(3) + k(4)                              ***
*** in the center of mass frame                            ***
*** you must call dsaskinset1 and dsaskinset2              *** 
*** before calling dsaskinset3                             *** 
***                                                        ***
*** AUTHOR: Piero Ullio, ullio@sissa.it                    ***
*** Date: 01-02-28                                         ***
**************************************************************

      subroutine dsaskinset3
\end{verbatim}
 \end{routine}

%%%%% routine dsasphghb.f %%%%%
\begin{routine}{dsasphghb.f}
\begin{verbatim}
c...This subroutine is automatically generated from form output by
c...parsing it through form2f (version 1.35, May 23, 2002, edsjo@physto.se)
c....Template file for dsasphghb begins here           

      subroutine dsasphghb(kp1,kp2,kp3,kp4,par)
\end{verbatim}
 \end{routine}

%%%%% routine dsasscscsSHffb.f %%%%%
\begin{routine}{dsasscscsSHffb.f}
\begin{verbatim}
**************************************************************
*** SUBROUTINE dsasscscsSHffb                              ***
*** computes ASx and ASy coefficients for                  ***
*** scalar(1) + scalar*(2) -> fermion(3) + fermionbar(4)   ***
*** for a Higgs boson in the S channel                     ***
***                                                        ***
*** AUTHOR: Piero Ullio, ullio@sissa.it                    ***
*** Date: 01-03-03                                         ***
**************************************************************

      SUBROUTINE dsasscscsSHffb(kp1,kp2,kp3,kp4,kph)
\end{verbatim}
 \end{routine}

%%%%% routine dsasscscsSHffbcol.f %%%%%
\begin{routine}{dsasscscsSHffbcol.f}
\begin{verbatim}
**************************************************************
*** SUBROUTINE dsasscscsSHffbcol                           ***
*** computes ASx and ASy coefficients for                  ***
*** scalar(1) + scalar*(2) -> fermion(3) + fermionbar(4)   ***
*** for a Higgs boson in the S channel                     ***
***                                                        ***
*** AUTHOR: Piero Ullio, ullio@sissa.it                    ***
*** Date: 01-03-03                                         ***
**************************************************************

      SUBROUTINE dsasscscsSHffbcol(kp1,kp2,kp3,kp4,kph)
\end{verbatim}
 \end{routine}

%%%%% routine dsasscscsSVffb.f %%%%%
\begin{routine}{dsasscscsSVffb.f}
\begin{verbatim}
**************************************************************
*** SUBROUTINE dsasscscsSVffb                              ***
*** computes ASx and ASy coefficients for                  ***
*** scalar(1) + scalar*(2) -> fermion(3) + fermionbar(4)   ***
*** for a vector boson in the S channel                    ***
***                                                        ***
*** AUTHOR: Piero Ullio, ullio@sissa.it                    ***
*** Date: 01-03-03                                         ***
**************************************************************

      SUBROUTINE dsasscscsSVffb(kp1,kp2,kp3,kp4,kpv)
\end{verbatim}
 \end{routine}

%%%%% routine dsasscscsSVffbcol.f %%%%%
\begin{routine}{dsasscscsSVffbcol.f}
\begin{verbatim}
**************************************************************
*** SUBROUTINE dsasscscsSVffbcol                           ***
*** computes ASx and ASy coefficients for                  ***
*** scalar(1) + scalar*(2) -> fermion(3) + fermionbar(4)   ***
*** for a vector boson in the S channel                    ***
***                                                        ***
*** AUTHOR: Piero Ullio, ullio@sissa.it                    ***
*** Date: 01-03-03                                         ***
**************************************************************

      SUBROUTINE dsasscscsSVffbcol(kp1,kp2,kp3,kp4,kpv)
\end{verbatim}
 \end{routine}

%%%%% routine dsasscscsTCffb.f %%%%%
\begin{routine}{dsasscscsTCffb.f}
\begin{verbatim}
**************************************************************
*** SUBROUTINE dsasscscsTCffb                              ***
*** computes ASx and ASy coefficients for                  ***
*** scalar(1) + scalar*(2) -> fermion(3) + fermionbar(4)   ***
*** for a neutralino or chargino in the T channel          ***
***                                                        ***
*** AUTHOR: Piero Ullio, ullio@sissa.it                    ***
*** Date: 01-03-03                                         ***
**************************************************************

      SUBROUTINE dsasscscsTCffb(kp1,kp2,kp3,kp4,kpchi)
\end{verbatim}
 \end{routine}

%%%%% routine dsasscscsTCffbcol.f %%%%%
\begin{routine}{dsasscscsTCffbcol.f}
\begin{verbatim}
**************************************************************
*** SUBROUTINE dsasscscsTCffbcol                           ***
*** computes ASx and ASy coefficients for                  ***
*** scalar(1) + scalar*(2) -> fermion(3) + fermionbar(4)   ***
*** for a neutralino or chargino in the T channel          ***
***                                                        ***
*** AUTHOR: Piero Ullio, ullio@sissa.it                    ***
*** Date: 01-03-03                                         ***
**************************************************************

      SUBROUTINE dsasscscsTCffbcol(kp1,kp2,kp3,kp4,kpchi)
\end{verbatim}
 \end{routine}

%%%%% routine dsasscscTCff.f %%%%%
\begin{routine}{dsasscscTCff.f}
\begin{verbatim}
**************************************************************
*** SUBROUTINE dsasscscTCff                                ***
*** computes ASx and ASy coefficients for                  ***
*** scalar(1) + scalar(2) -> fermion(3) + fermion(4)       ***
*** for a neutralino or chargino in the T channel          ***
***                                                        ***
*** AUTHOR: Piero Ullio, ullio@sissa.it                    ***
*** Date: 01-03-03                                         ***
**************************************************************

      SUBROUTINE dsasscscTCff(kp1,kp2,kp3,kp4,kpchi)
\end{verbatim}
 \end{routine}

%%%%% routine dsasscscTCffcol.f %%%%%
\begin{routine}{dsasscscTCffcol.f}
\begin{verbatim}
**************************************************************
*** SUBROUTINE dsasscscTCffcol                             ***
*** computes ASx and ASy coefficients for                  ***
*** scalar(1) + scalar(2) -> fermion(3) + fermion(4)       ***
*** for a neutralino or chargino in the T channel          ***
***                                                        ***
*** AUTHOR: Piero Ullio, ullio@sissa.it                    ***
*** Date: 01-03-03                                         ***
**************************************************************

      SUBROUTINE dsasscscTCffcol(kp1,kp2,kp3,kp4,kpchi)
\end{verbatim}
 \end{routine}

%%%%% routine dsasscscUCff.f %%%%%
\begin{routine}{dsasscscUCff.f}
\begin{verbatim}
**************************************************************
*** SUBROUTINE dsasscscTCff                                ***
*** computes ASx and ASy coefficients for                  ***
*** scalar(1) + scalar(2) -> fermion(3) + fermion(4)       ***
*** for a neutralino or chargino in the U channel          ***
***                                                        ***
*** AUTHOR: Piero Ullio, ullio@sissa.it                    ***
*** Date: 01-03-03                                         ***
**************************************************************

      SUBROUTINE dsasscscUCff(kp1,kp2,kp3,kp4,kpchi)
\end{verbatim}
 \end{routine}

%%%%% routine dsasscscUCffcol.f %%%%%
\begin{routine}{dsasscscUCffcol.f}
\begin{verbatim}
**************************************************************
*** SUBROUTINE dsasscscTCffcol                             ***
*** computes ASx and ASy coefficients for                  ***
*** scalar(1) + scalar(2) -> fermion(3) + fermion(4)       ***
*** for a neutralino or chargino in the U channel          ***
***                                                        ***
*** AUTHOR: Piero Ullio, ullio@sissa.it                    ***
*** Date: 01-03-03                                         ***
**************************************************************

      SUBROUTINE dsasscscUCffcol(kp1,kp2,kp3,kp4,kpchi)
\end{verbatim}
 \end{routine}

%%%%% routine dsassfercode.f %%%%%
\begin{routine}{dsassfercode.f}
\begin{verbatim}
**************************************************************
*** SUBROUTINE dsassfercode                                ***
*** finds sfermions ksfer1,ksfer2 in a given family iifamv ***
*** chow variable equal to                                 ***
*** 'same' : routine returns sfermion code for the         ***
***          particles with the same iifamv                ***
*** 'diff' : routine returns sfermion code for the         ***
***          particles with the same iifamv+1 or iifamv-1  ***
***                                                        *** 
*** AUTHOR: Piero Ullio, ullio@sissa.it                    ***
*** Date: 01-08-09                                         ***
**************************************************************

      subroutine dsassfercode(chow,iifamv,ksfer1,ksfer2)
\end{verbatim}
 \end{routine}

%%%%% routine dsaswcomp.f %%%%%
\begin{routine}{dsaswcomp.f}
\begin{verbatim}
      subroutine dsaswcomp(p,costheta,kp1,kp2,dwbsmax,dmbsmax)
c_______________________________________________________________________
c  Routine to compare annihilation cross sections to find
c  out how big they are
c    p - initial cm momentum (real) for lsp annihilations
c    costheta - cosine of c.m. annihilation angle
c  common:
c    'dssusy.h' - file with susy common blocks
c  uses dsandwdcosnn, dsandwdcoscn and dsandwdcoscc
c  author: joakim edsjo (edsjo@physto.se)
c  date: 02-05-23
c  modified: Joakim Edsjo, 02-10-22
c=======================================================================
\end{verbatim}
 \end{routine}

\newpage
\chapter[bsg: $b \rightarrow s \gamma$]{\codeb{src/bsg}:\\ $b \rightarrow s \gamma$}
\label{ch:src-bsg}

%%%%%%%%%%%%%%%%%%%%%%%%%%%%%%%%%%%%%%%%%%%%%%%%%%%%%%%%%%%%%%%%%%%%

\section{$b \rightarrow s \gamma$ -- theory}

The rare decay $b \rightarrow s \gamma$ can have large contributions
from loops of suersymmetric particles and one therefore has to check
that a particular supersymmetric model does not violate the observed
branching ratio for $b$ decay to $s \gamma$. In \ds\ we have several
expressions for calculations of the $b \rightarrow s \gamma$ decay. In
\codeb{ac/}, some older obsolte expressions are found (kept only for
historical reasons). In this directory, \codeb{bsg/}, we have our best
implementation of the $b \rightarrow s \gamma$ decay.

 Our estimate of this process includes the complete next-to-leading 
order (NLO) correction for the SM contribution and the dominant
NLO corrections for the SUSY term.  The NLO QCD SM calculation is performed
following the analysis in Ref.~\cite{bsgsm}, modified according to~\cite{bsgmagic}, 
and gives a branching ratio 
$\mathrm{BR}[B\rightarrow X_s\,\gamma] =3.72\times10^{-4}$ 
for a photon energy greater than  $m_b/20$. In the SUSY contribution, we 
include the NLO contributions in the two Higgs doublet model,
following~\cite{bsgh2}, and the corrections due to SUSY particles. The latter
are calculated under the assumption of minimal flavour violation, with the 
dominant LO contributions from Ref.~\cite{bsgtan}, and with the NLO QCD term
with expressions of \cite{bsgsusy} modified in the large $\tan\beta$ regime
according to~\cite{bsgtan}. In the mSUGRA framework
(see, e.g., ~\cite{bsgcompare}), the largest discrepancy 
between the LO and the NLO SUSY corrections are found for ${\rm sign}{\mu}>0$, 
large $\tan\beta$ and low values of $m_{1/2}$: in this case the SUSY 
contribution to the decay rate is negative, and the discrimination of models 
based on the NLO analysis is less restrictive than the one in the LO analysis.
We will assume as allowed range of branching ratios 
$2.0\times10^{-4}\leq\mathrm{BR}[B\rightarrow X_s\,\gamma] \leq4.6\times10^{-4}$,
which is obtained adding a theoretical uncertainty of $\pm0.5\times10^{-4}$ to the
experimental value quoted by the Particle Data Group 2002~\cite{pdg02}. 

%%%%%%%%%%%%%%%%%%%%%%%%%%%%%%%%%%%%%%%%%%%%%%%%%%%%%%%%%%%%%%%%%%%%

\section{$b \rightarrow s \gamma$ -- routines}

The main routine is \codeb{dsbsgammafull} which returns the $b
\rightarrow s \gamma$ branching ratio. The routine can calculate
either only the standard model contribution or also include the SUSY
contribution (which is of course the default when this routine is
called from \codeb{dsacbnd}). The reaming (large) set or routines are
the various contributions to the decay as given in the references
listed above.
\section{Routine headers -- fortran files}

%%%%% routine dsbsgalpha3.f %%%%%
\begin{routine}{dsbsgalpha3.f}
\begin{verbatim}
      function dsbsgalpha3(m)

***********************************************************************
* The coupling constant alpha_3 evaluated at the scale m              *
* using nf effective quark flavours (usually taken to be nf=5)        *
* Uses eq. (42) of Ciuchini et al. hep-ph/9710335                     *
* for the calculation of b --> s gamma                                *
* Note: This routines is strictly speaking only valid for mass scales *
* between mb and mt where nf=5 should be used.                        *
* author:Mia Schelke, schelke@physto.se, 2003-04-03                   *
***********************************************************************

\end{verbatim}
 \end{routine}

%%%%% routine dsbsgalpha3int.f %%%%%
\begin{routine}{dsbsgalpha3int.f}
\begin{verbatim}
      function dsbsgalpha3int(al,mstart,m,nf)

***********************************************************************
* The coupling constant alpha_3 evaluated at the scale m              *
* given the value at a given scale mstart. nf effective quark flavours*
* are used in the running (if nf=7, 6 quark flavours and one squark   *
* flavor are used)                                                    *
* Uses eq. (42) of Ciuchini et al. hep-ph/9710335                     *
* author:Mia Schelke, schelke@physto.se, 2003-04-03                   *
***********************************************************************

\end{verbatim}
 \end{routine}

%%%%% routine dsbsgammafull.f %%%%%
\begin{routine}{dsbsgammafull.f}
\begin{verbatim}
      subroutine dsbsgammafull(ratio,flag)

***********************************************************************
* Routine that calculates the b-->s+gamma branching rate              *
* The Standard Model contribution is taken from                       *
* Gambino and Misiak,Nucl. Phys. B611 (2001) 338                      *
* with new 'magic numbers' from Buras et al., hep-ph/0203135          *
* Input: flag:  0 = only standard model                               *
*               1 = standard model plus SUSY corrections              *
* Output: ratio = BR(b -> s gamma)                                    *
* author:Mia Schelke, schelke@physto.se, 2003-03-27                   *
***********************************************************************
\end{verbatim}
 \end{routine}

%%%%% routine dsbsgat0.f %%%%%
\begin{routine}{dsbsgat0.f}
\begin{verbatim}
      function dsbsgat0(x,flag)

***********************************************************************
* Function A^t_0(x) in (2.7) of Gambino and Misiak,                   *
* Nucl. Phys. B611 (2001) 338                                         *
* x must be a positive number                                         *
* for the calculation of b --> s gamma                                *
* author:Mia Schelke, schelke@physto.se, 2003-03-12                   *
* updated by Mia Schelke 2003-04-10 to include the susy contribution  *
* as explained in eq. (5.1)                                           *
* (the references used for the susy contributions can be found in     *
*  the different fortran codes)                                       *
* Input: flag:  0 = only standard model                               *
*               1 = standard model plus SUSY corrections              *
***********************************************************************

\end{verbatim}
 \end{routine}

%%%%% routine dsbsgat1.f %%%%%
\begin{routine}{dsbsgat1.f}
\begin{verbatim}
      function dsbsgat1(x,flag)

***********************************************************************
* Function A^t_1(x) p. 11 in Gambino and Misiak,                      *
* Nucl. Phys. B611 (2001) 338                                         *
* x must be a positive number                                         *
* for the calculation of b --> s gamma                                *
* author:Mia Schelke, schelke@physto.se, 2003-03-13                   *
* updated by Mia Schelke 2003-04-10 to include the susy contribution  *
* as explained in eq. (5.1)                                           *
* (the references used for the susy contributions can be found in     *
*  the different fortran codes)                                       *
* Input: flag:  0 = only standard model                               *
*               1 = standard model plus SUSY corrections              *
***********************************************************************

\end{verbatim}
 \end{routine}

%%%%% routine dsbsgbofe.f %%%%%
\begin{routine}{dsbsgbofe.f}
\begin{verbatim}
      function dsbsgbofe(flag)
c      program dsacbofe     
***********************************************************************
* Program that calculates B(E_0) in (E.1) of Gambino and Misiak,      *
* Nucl. Phys. B611 (2001) 338                                         *
* for the calculation of b --> s gamma                                *
* Input: flag:  0 = only standard model                               *
*               1 = standard model plus SUSY corrections              *
* author:Mia Schelke, schelke@physto.se, 2003-03-12                   *
***********************************************************************
\end{verbatim}
 \end{routine}

%%%%% routine dsbsgc41h2.f %%%%%
\begin{routine}{dsbsgc41h2.f}
\begin{verbatim}
      function dsbsgc41h2()

***********************************************************************
* The next to leading order contribution to the Wilson coefficient C_4*
* from the two-Higgs doublet model                                    *
* Eq. (58) of Ciuchini et al.,                                        *
* hep-ph/9710335                                                      *
* for the calculation of b --> s gamma                                *
* author:Mia Schelke, schelke@physto.se, 2003-03-31                   *
***********************************************************************

\end{verbatim}
 \end{routine}

%%%%% routine dsbsgc41susy.f %%%%%
\begin{routine}{dsbsgc41susy.f}
\begin{verbatim}
      function dsbsgc41susy()

***********************************************************************
* The next to leading order contribution to                           *
* the Wilson coefficient C_4 from susy                                *
* Eq (10) of Ciuchini et al.,                                         *
* hep-ph/9806308                                                      *
* for the calculation of b --> s gamma                                *
* author:Mia Schelke, schelke@physto.se, 2003-04-07                   *
***********************************************************************

\end{verbatim}
 \end{routine}

%%%%% routine dsbsgc70h2.f %%%%%
\begin{routine}{dsbsgc70h2.f}
\begin{verbatim}
      function dsbsgc70h2()

***********************************************************************
* The leading order contribution to the Wilson coefficient C_7        *
* from the two-Higgs doublet model                                    *
* Eq. (53) of Ciuchini et al.,                                        *
* hep-ph/9710335                                                      *
* for the calculation of b --> s gamma                                *
* author:Mia Schelke, schelke@physto.se, 2003-03-31                   *
***********************************************************************

\end{verbatim}
 \end{routine}

%%%%% routine dsbsgc70susy.f %%%%%
\begin{routine}{dsbsgc70susy.f}
\begin{verbatim}
      function dsbsgc70susy()

***********************************************************************
* The leading order contribution to the Wilson coefficient C_7        *
* from susy                                                           *
* Eq (31) of Degrassi et al.,                                         *
* hep-ph/0009337                                                      *
* for the calculation of b --> s gamma                                *
* author:Mia Schelke, schelke@physto.se, 2003-04-04                   *
***********************************************************************

\end{verbatim}
 \end{routine}

%%%%% routine dsbsgc71chisusy.f %%%%%
\begin{routine}{dsbsgc71chisusy.f}
\begin{verbatim}
      function dsbsgc71chisusy()

***********************************************************************
* The next to leading order contribution to                           *
* the Wilson coefficient C_7 from chargino (susy)                     *
* Eq (13) of Ciuchini et al.,                                         *
* hep-ph/9806308                                                      *
* for the calculation of b --> s gamma                                *
* author:Mia Schelke, schelke@physto.se, 2003-04-08                   *
***********************************************************************

\end{verbatim}
 \end{routine}

%%%%% routine dsbsgc71h2.f %%%%%
\begin{routine}{dsbsgc71h2.f}
\begin{verbatim}
      function dsbsgc71h2()  ! (muw)

***********************************************************************
* The next to leading order contribution to the Wilson coefficient C_7*
* from the two-Higgs doublet model                                    *
* Eq. (59) of Ciuchini et al.,                                        *
* hep-ph/9710335                                                      *
* for the calculation of b --> s gamma                                *
* The input parameter muw=\mu_W is the matching scale                 *
* author:Mia Schelke, schelke@physto.se, 2003-03-31                   *
***********************************************************************

\end{verbatim}
 \end{routine}

%%%%% routine dsbsgc71phi1susy.f %%%%%
\begin{routine}{dsbsgc71phi1susy.f}
\begin{verbatim}
      function dsbsgc71phi1susy()

***********************************************************************
* The next to leading order contribution to                           *
* the Wilson coefficient C_7 from the susy renormalization effect in  *
* the charged scalar,\phi_1, coupling                                 *
* Eq (25) of Ciuchini et al.,                                         *
* hep-ph/9806308                                                      *
* for the calculation of b --> s gamma                                *
* author:Mia Schelke, schelke@physto.se, 2003-04-08                   *
***********************************************************************

\end{verbatim}
 \end{routine}

%%%%% routine dsbsgc71phi2susy.f %%%%%
\begin{routine}{dsbsgc71phi2susy.f}
\begin{verbatim}
      function dsbsgc71phi2susy()

***********************************************************************
* The next to leading order contribution to                           *
* the Wilson coefficient C_7 from the susy renormalization effect in  *
* the unphysical charged scalar,\phi_2, coupling                      *
* Eq (26) of Ciuchini et al.,                                         *
* hep-ph/9806308                                                      *
* for the calculation of b --> s gamma                                *
* author:Mia Schelke, schelke@physto.se, 2003-04-08                   *
***********************************************************************

\end{verbatim}
 \end{routine}

%%%%% routine dsbsgc71wsusy.f %%%%%
\begin{routine}{dsbsgc71wsusy.f}
\begin{verbatim}
      function dsbsgc71wsusy()

***********************************************************************
* The next to leading order contribution to                           *
* the Wilson coefficient C_7 from the susy renormalization effect in  *
* the W coupling                                                      *
* Eq (23) of Ciuchini et al.,                                         *
* hep-ph/9806308                                                      *
* for the calculation of b --> s gamma                                *
* author:Mia Schelke, schelke@physto.se, 2003-04-08                   *
***********************************************************************

\end{verbatim}
 \end{routine}

%%%%% routine dsbsgc80h2.f %%%%%
\begin{routine}{dsbsgc80h2.f}
\begin{verbatim}
      function dsbsgc80h2()

***********************************************************************
* The leading order contribution to the Wilson coefficient C_8        *
* from the two-Higgs doublet model                                    *
* Eq. (53) of Ciuchini et al.,                                        *
* hep-ph/9710335                                                      *
* for the calculation of b --> s gamma                                *
* author:Mia Schelke, schelke@physto.se, 2003-03-31                   *
***********************************************************************

\end{verbatim}
 \end{routine}

%%%%% routine dsbsgc80susy.f %%%%%
\begin{routine}{dsbsgc80susy.f}
\begin{verbatim}
      function dsbsgc80susy()

***********************************************************************
* The leading order contribution to the Wilson coefficient C_8        *
* from susy                                                           *
* Eq (31) of Degrassi et al., hep-ph/0009337                          *
* Differs from dsbsgc70susy only by a few changes described p.11      *
* for the calculation of b --> s gamma                                *
* author:Mia Schelke, schelke@physto.se, 2003-04-04                   *
***********************************************************************

\end{verbatim}
 \end{routine}

%%%%% routine dsbsgc81chisusy.f %%%%%
\begin{routine}{dsbsgc81chisusy.f}
\begin{verbatim}
      function dsbsgc81chisusy()

***********************************************************************
* The next to leading order contribution to                           *
* the Wilson coefficient C_8 from chargino (susy)                     *
* Eq (14) of Ciuchini et al.,                                         *
* hep-ph/9806308                                                      *
* for the calculation of b --> s gamma                                *
* author:Mia Schelke, schelke@physto.se, 2003-04-08                   *
***********************************************************************

\end{verbatim}
 \end{routine}

%%%%% routine dsbsgc81h2.f %%%%%
\begin{routine}{dsbsgc81h2.f}
\begin{verbatim}
      function dsbsgc81h2()  ! (muw)

***********************************************************************
* The next to leading order contribution to the Wilson coefficient C_8*
* from the two-Higgs doublet model                                    *
* Eq. (59) of Ciuchini et al.,                                        *
* hep-ph/9710335                                                      *
* for the calculation of b --> s gamma                                *
* The input parameter muw=\mu_W is the matching scale                 *
* author:Mia Schelke, schelke@physto.se, 2003-03-31                   *
***********************************************************************

\end{verbatim}
 \end{routine}

%%%%% routine dsbsgc81phi1susy.f %%%%%
\begin{routine}{dsbsgc81phi1susy.f}
\begin{verbatim}
      function dsbsgc81phi1susy()

***********************************************************************
* The next to leading order contribution to                           *
* the Wilson coefficient C_8 from the susy renormalization effect in  *
* the charged scalar,\phi_1, coupling                                 *
* Eq (25) of Ciuchini et al.,                                         *
* hep-ph/9806308                                                      *
* for the calculation of b --> s gamma                                *
* author:Mia Schelke, schelke@physto.se, 2003-04-08                   *
***********************************************************************

\end{verbatim}
 \end{routine}

%%%%% routine dsbsgc81phi2susy.f %%%%%
\begin{routine}{dsbsgc81phi2susy.f}
\begin{verbatim}
      function dsbsgc81phi2susy()

***********************************************************************
* The next to leading order contribution to                           *
* the Wilson coefficient C_8 from the susy renormalization effect in  *
* the unphysical charged scalar,\phi_2, coupling                      *
* Eq (26) of Ciuchini et al.,                                         *
* hep-ph/9806308                                                      *
* for the calculation of b --> s gamma                                *
* author:Mia Schelke, schelke@physto.se, 2003-04-08                   *
***********************************************************************

\end{verbatim}
 \end{routine}

%%%%% routine dsbsgc81wsusy.f %%%%%
\begin{routine}{dsbsgc81wsusy.f}
\begin{verbatim}
      function dsbsgc81wsusy()

***********************************************************************
* The next to leading order contribution to                           *
* the Wilson coefficient C_8 from the susy renormalization effect in  *
* the W coupling                                                      *
* Eq (24) of Ciuchini et al.,                                         *
* hep-ph/9806308                                                      *
* for the calculation of b --> s gamma                                *
* author:Mia Schelke, schelke@physto.se, 2003-04-08                   *
***********************************************************************

\end{verbatim}
 \end{routine}

%%%%% routine dsbsgckm.f %%%%%
\begin{routine}{dsbsgckm.f}
\begin{verbatim}
      function dsbsgckm()

***********************************************************************
* The ratio,|V_ts^* V_tb/V_cb|^2, of ckm elements                     *
* with susy corrections                                               *
* Eq (34) of Ciuchini et al., hep-ph/9806308                          * 
* for the calculation of b --> s gamma                                *
* author:Mia Schelke, schelke@physto.se, 2003-04-22                   *
***********************************************************************

\end{verbatim}
 \end{routine}

%%%%% routine dsbsgd1td.f %%%%%
\begin{routine}{dsbsgd1td.f}
\begin{verbatim}
      function dsbsgd1td(x1)

***********************************************************************
* Function \Delta^(1)_{t,d}(x_1) in app A p. 17 of                    *
* Ciuchini et al., hep-ph/9806308                                     *
* x1 must be a positive number                                        *
* x1=m^2(sq_1 of flavour d)/m^2(kgluin)                               *
* Note that this function can also be used for \Delta^(1)_d(x_2)      *
* but in that case x1 should be identified with                       *
* x2=m^2(sq_2 of flavour d)/m^2(kgluin)                               *
* for the calculation of b --> s gamma                                *
* author:Mia Schelke, schelke@physto.se, 2003-04-07                   *
***********************************************************************

\end{verbatim}
 \end{routine}

%%%%% routine dsbsgd2d.f %%%%%
\begin{routine}{dsbsgd2d.f}
\begin{verbatim}
      function dsbsgd2d()

***********************************************************************
* Function \Delta^(2)_d (with d=b) in app A p. 17 of                  *
* Ciuchini et al., hep-ph/9806308                                     *
* Note that we have inserted d=b                                      *
* for the calculation of b --> s gamma                                *
* author:Mia Schelke, schelke@physto.se, 2003-04-07                   *
***********************************************************************

\end{verbatim}
 \end{routine}

%%%%% routine dsbsgd2td.f %%%%%
\begin{routine}{dsbsgd2td.f}
\begin{verbatim}
      function dsbsgd2td(famd)

***********************************************************************
* Function \Delta^(2)_{t,d} in app A p. 17 of                         *
* Ciuchini et al., hep-ph/9806308                                     *
* The input parameter famd is the family of the down sector           *
* it should be 1 for the first family, 2 for the 2nd and 3 for the 3rd*
* for the calculation of b --> s gamma                                *
* author:Mia Schelke, schelke@physto.se, 2003-04-07                   *
***********************************************************************

\end{verbatim}
 \end{routine}

%%%%% routine dsbsgd7chi1.f %%%%%
\begin{routine}{dsbsgd7chi1.f}
\begin{verbatim}
      function dsbsgd7chi1(x)

***********************************************************************
* Function \Delta_7^{\chi,1} eq (20) of                               *
* Ciuchini et al., hep-ph/9806308                                     *
* for the calculation of b --> s gamma                                *
* author:Mia Schelke, schelke@physto.se, 2003-04-08                   *
***********************************************************************

\end{verbatim}
 \end{routine}

%%%%% routine dsbsgd7chi2.f %%%%%
\begin{routine}{dsbsgd7chi2.f}
\begin{verbatim}
      function dsbsgd7chi2(x)

***********************************************************************
* Function \Delta_7^{\chi,2} eq (20) of                               *
* Ciuchini et al., hep-ph/9806308                                     *
* for the calculation of b --> s gamma                                *
* author:Mia Schelke, schelke@physto.se, 2003-04-08                   *
***********************************************************************

\end{verbatim}
 \end{routine}

%%%%% routine dsbsgd7h.f %%%%%
\begin{routine}{dsbsgd7h.f}
\begin{verbatim}
      function dsbsgd7h(y)

***********************************************************************
* Function \Delta_7^H(y) in (61) of Ciuchini et al.,                  *
* hep-ph/9710335                                                      *
* y must be a positive number                                         *
* for the calculation of b --> s gamma                                *
* author:Mia Schelke, schelke@physto.se, 2003-03-31                   *
***********************************************************************

\end{verbatim}
 \end{routine}

%%%%% routine dsbsgd8chi1.f %%%%%
\begin{routine}{dsbsgd8chi1.f}
\begin{verbatim}
      function dsbsgd8chi1(x)

***********************************************************************
* Function \Delta_8^{\chi,1} eq (21) of                               *
* Ciuchini et al., hep-ph/9806308                                     *
* for the calculation of b --> s gamma                                *
* author:Mia Schelke, schelke@physto.se, 2003-04-08                   *
***********************************************************************

\end{verbatim}
 \end{routine}

%%%%% routine dsbsgd8chi2.f %%%%%
\begin{routine}{dsbsgd8chi2.f}
\begin{verbatim}
      function dsbsgd8chi2(x)

***********************************************************************
* Function \Delta_8^{\chi,2} eq (21) of                               *
* Ciuchini et al., hep-ph/9806308                                     *
* for the calculation of b --> s gamma                                *
* author:Mia Schelke, schelke@physto.se, 2003-04-08                   *
***********************************************************************

\end{verbatim}
 \end{routine}

%%%%% routine dsbsgd8h.f %%%%%
\begin{routine}{dsbsgd8h.f}
\begin{verbatim}
      function dsbsgd8h(y)

***********************************************************************
* Function \Delta_8^H(y) in (63) of Ciuchini et al.,                  *
* hep-ph/9710335                                                      *
* y must be a positive number                                         *
* for the calculation of b --> s gamma                                *
* author:Mia Schelke, schelke@physto.se, 2003-03-31                   *
***********************************************************************

\end{verbatim}
 \end{routine}

%%%%% routine dsbsgeb.f %%%%%
\begin{routine}{dsbsgeb.f}
\begin{verbatim}
      function dsbsgeb()

***********************************************************************
* Function \epsilon_b in (10) of Degrassi et al.,                     *
* hep-ph/0009337                                                      *
* for the calculation of b --> s gamma                                *
* author:Mia Schelke, schelke@physto.se, 2003-04-03                   *
***********************************************************************

\end{verbatim}
 \end{routine}

%%%%% routine dsbsgechi.f %%%%%
\begin{routine}{dsbsgechi.f}
\begin{verbatim}
      function dsbsgechi(y)

***********************************************************************
* Function E_\chi(y) in (11) of Ciuchini et al.,                      *
* hep-ph/9806308                                                      *
* y must be a positive number                                         *
* for the calculation of b --> s gamma                                *
* author:Mia Schelke, schelke@physto.se, 2003-04-04                   *
***********************************************************************

\end{verbatim}
 \end{routine}

%%%%% routine dsbsgeh.f %%%%%
\begin{routine}{dsbsgeh.f}
\begin{verbatim}
      function dsbsgeh(y)

***********************************************************************
* Function E^H(y) in (64) of Ciuchini et al.,                         *
* hep-ph/9710335                                                      *
* y must be a positive number                                         *
* for the calculation of b --> s gamma                                *
* author:Mia Schelke, schelke@physto.se, 2003-03-31                   *
***********************************************************************

\end{verbatim}
 \end{routine}

%%%%% routine dsbsget0.f %%%%%
\begin{routine}{dsbsget0.f}
\begin{verbatim}
      function dsbsget0(x,flag)

***********************************************************************
* Function E^t_0(x) p. 11 in Gambino and Misiak,                      *
* Nucl. Phys. B611 (2001) 338                                         *
* x must be a positive number                                         *
* for the calculation of b --> s gamma                                *
* author:Mia Schelke, schelke@physto.se, 2003-03-13                   *
* updated by Mia Schelke 2003-04-10 to include the susy contribution  *
* as explained in eq. (5.1)                                           *
* (the references used for the susy contributions can be found in     *
*  the different fortran codes)                                       *
* Input: flag:  0 = only standard model                               *
*               1 = standard model plus SUSY corrections              *
***********************************************************************

\end{verbatim}
 \end{routine}

%%%%% routine dsbsgf71.f %%%%%
\begin{routine}{dsbsgf71.f}
\begin{verbatim}
      function dsbsgf71(y)

***********************************************************************
* Function F_7^(1)(y) in (29) of Ciuchini et al.,                     *
* hep-ph/9710335                                                      *
* y must be a positive number                                         *
* for the calculation of b --> s gamma                                *
* author:Mia Schelke, schelke@physto.se, 2003-03-31                   *
***********************************************************************

\end{verbatim}
 \end{routine}

%%%%% routine dsbsgf72.f %%%%%
\begin{routine}{dsbsgf72.f}
\begin{verbatim}
      function dsbsgf72(y)

***********************************************************************
* Function F_7^(2)(y) in (54) of Ciuchini et al.,                     *
* hep-ph/9710335                                                      *
* y must be a positive number                                         *
* for the calculation of b --> s gamma                                *
* author:Mia Schelke, schelke@physto.se, 2003-03-31                   *
***********************************************************************

\end{verbatim}
 \end{routine}

%%%%% routine dsbsgf73.f %%%%%
\begin{routine}{dsbsgf73.f}
\begin{verbatim}
      function dsbsgf73(y)

***********************************************************************
* Function F_7^(3)(y) in (21) of Degrassi et al.,                     *
* hep-ph/0009337                                                      *
* y must be a positive number                                         *
* for the calculation of b --> s gamma                                *
* author:Mia Schelke, schelke@physto.se, 2003-04-03                   *
***********************************************************************

\end{verbatim}
 \end{routine}

%%%%% routine dsbsgf81.f %%%%%
\begin{routine}{dsbsgf81.f}
\begin{verbatim}
      function dsbsgf81(y)

***********************************************************************
* Function F_8^(1)(y) in (30) of Ciuchini et al.,                     *
* hep-ph/9710335                                                      *
* y must be a positive number                                         *
* for the calculation of b --> s gamma                                *
* author:Mia Schelke, schelke@physto.se, 2003-03-31                   *
***********************************************************************

\end{verbatim}
 \end{routine}

%%%%% routine dsbsgf82.f %%%%%
\begin{routine}{dsbsgf82.f}
\begin{verbatim}
      function dsbsgf82(y)

***********************************************************************
* Function F_8^(2)(y) in (55) of Ciuchini et al.,                     *
* hep-ph/9710335                                                      *
* y must be a positive number                                         *
* for the calculation of b --> s gamma                                *
* author:Mia Schelke, schelke@physto.se, 2003-03-31                   *
***********************************************************************

\end{verbatim}
 \end{routine}

%%%%% routine dsbsgf83.f %%%%%
\begin{routine}{dsbsgf83.f}
\begin{verbatim}
      function dsbsgf83(y)

***********************************************************************
* Function F_8^(3)(y) in (22) of Degrassi et al.,                     *
* hep-ph/0009337                                                      *
* y must be a positive number                                         *
* for the calculation of b --> s gamma                                *
* author:Mia Schelke, schelke@physto.se, 2003-04-03                   *
***********************************************************************

\end{verbatim}
 \end{routine}

%%%%% routine dsbsgft0.f %%%%%
\begin{routine}{dsbsgft0.f}
\begin{verbatim}
      function dsbsgft0(x,flag)

***********************************************************************
* Function F^t_0(x) in (2.7) of Gambino and Misiak,                   *
* Nucl. Phys. B611 (2001) 338                                         *
* x must be a positive number                                         *
* for the calculation of b --> s gamma                                *
* author:Mia Schelke, schelke@physto.se, 2003-03-12                   *
* updated by Mia Schelke 2003-04-10 to include the susy contribution  *
* as explained in eq. (5.1)                                           *
* (the references used for the susy contributions can be found in     *
*  the different fortran codes)                                       *
* Input: flag:  0 = only standard model                               *
*               1 = standard model plus SUSY corrections              *
***********************************************************************


\end{verbatim}
 \end{routine}

%%%%% routine dsbsgft1.f %%%%%
\begin{routine}{dsbsgft1.f}
\begin{verbatim}
      function dsbsgft1(x,flag)

***********************************************************************
* Function F^t_1(x) p. 11 in Gambino and Misiak,                      *
* Nucl. Phys. B611 (2001) 338                                         *
* x must be a positive number                                         *
* for the calculation of b --> s gamma                                *
* author:Mia Schelke, schelke@physto.se, 2003-03-13                   *
* updated by Mia Schelke 2003-04-10 to include the susy contribution  *
* as explained in eq. (5.1)                                           *
* (the references used for the susy contributions can be found in     *
*  the different fortran codes)                                       *
* Input: flag:  0 = only standard model                               *
*               1 = standard model plus SUSY corrections              *
***********************************************************************

\end{verbatim}
 \end{routine}

%%%%% routine dsbsgfxy.f %%%%%
\begin{routine}{dsbsgfxy.f}
\begin{verbatim}

      function dsbsgfxy(x,y)

***********************************************************************
* Function F'(x,y) in app. B p. 18 of                                 *
* Ciuchini et al., hep-ph/9806308                                     *
* for the calculation of b --> s gamma                                *
* author:Mia Schelke, schelke@physto.se, 2003-04-22                   *
***********************************************************************

\end{verbatim}
 \end{routine}

%%%%% routine dsbsgg.f %%%%%
\begin{routine}{dsbsgg.f}
\begin{verbatim}
      function dsbsgg(t)

***********************************************************************
* Function G(t) in (E.8) of Gambino and Misiak,                       *
* Nucl. Phys. B611 (2001) 338                                         *
* t must be a positive number                                         *
* for the calculation of b --> s gamma                                *
* author:Mia Schelke, schelke@physto.se, 2003-03-05                   *
***********************************************************************

\end{verbatim}
 \end{routine}

%%%%% routine dsbsgg7chi2.f %%%%%
\begin{routine}{dsbsgg7chi2.f}
\begin{verbatim}
      function dsbsgg7chi2(x)

***********************************************************************
* Function G^{chi,2}_7(x) in eq. (16) of                              *
* Ciuchini et al., hep-ph/9806308                                     *
* for the calculation of b --> s gamma                                *
* author:Mia Schelke, schelke@physto.se, 2003-04-07                   *
***********************************************************************

\end{verbatim}
 \end{routine}

%%%%% routine dsbsgg7chij1.f %%%%%
\begin{routine}{dsbsgg7chij1.f}
\begin{verbatim}
      function dsbsgg7chij1(x,j)

***********************************************************************
* Function G^{chi,1}_7(x) in eq. (15) of                              *
* Ciuchini et al., hep-ph/9806308                                     *
* The expression has been extended to large tanbe, by replacing       *
* ln(m^2(kgluin)/m^2(\chi_j)) by ln((mu_w)^2)/m^2(\chi_j))            *
* as explained in Degrassi et al., hep-ph/0009337 p.11                *
* The input, j, is the nb of the chargino, it must be 1 or 2          *
* for the calculation of b --> s gamma                                *
* author:Mia Schelke, schelke@physto.se, 2003-04-07                   *
***********************************************************************

\end{verbatim}
 \end{routine}

%%%%% routine dsbsgg7h.f %%%%%
\begin{routine}{dsbsgg7h.f}
\begin{verbatim}
      function dsbsgg7h(y)

***********************************************************************
* Function G_7^H(y) in (60) of Ciuchini et al.,                       *
* hep-ph/9710335                                                      *
* y must be a positive number                                         *
* for the calculation of b --> s gamma                                *
* author:Mia Schelke, schelke@physto.se, 2003-03-31                   *
***********************************************************************

\end{verbatim}
 \end{routine}

%%%%% routine dsbsgg7w.f %%%%%
\begin{routine}{dsbsgg7w.f}
\begin{verbatim}
      function dsbsgg7w(x)

***********************************************************************
* Function G^W_7(x) in eq. (27) of                                    *
* Ciuchini et al., hep-ph/9806308                                     *
* for the calculation of b --> s gamma                                *
* author:Mia Schelke, schelke@physto.se, 2003-04-08                   *
***********************************************************************

\end{verbatim}
 \end{routine}

%%%%% routine dsbsgg8chi2.f %%%%%
\begin{routine}{dsbsgg8chi2.f}
\begin{verbatim}
      function dsbsgg8chi2(x)

***********************************************************************
* Function G^{chi,2}_8(x) in eq. (18) of                              *
* Ciuchini et al., hep-ph/9806308                                     *
* for the calculation of b --> s gamma                                *
* author:Mia Schelke, schelke@physto.se, 2003-04-07                   *
***********************************************************************

\end{verbatim}
 \end{routine}

%%%%% routine dsbsgg8chij1.f %%%%%
\begin{routine}{dsbsgg8chij1.f}
\begin{verbatim}
      function dsbsgg8chij1(x,j)

***********************************************************************
* Function G^{chi,1}_8(x) in eq. (17) of                              *
* Ciuchini et al., hep-ph/9806308                                     *
* The expression has been extended to large tanbe, by replacing       *
* ln(m^2(kgluin)/m^2(\chi_j)) by ln((mu_w)^2)/m^2(\chi_j))            *
* as explained in Degrassi et al., hep-ph/0009337 p.11                *
* The input, j, is the nb of the chargino, it must be 1 or 2          *
* for the calculation of b --> s gamma                                *
* author:Mia Schelke, schelke@physto.se, 2003-04-07                   *
***********************************************************************

\end{verbatim}
 \end{routine}

%%%%% routine dsbsgg8h.f %%%%%
\begin{routine}{dsbsgg8h.f}
\begin{verbatim}
      function dsbsgg8h(y)

***********************************************************************
* Function G_8^H(y) in (62) of Ciuchini et al.,                       *
* hep-ph/9710335                                                      *
* y must be a positive number                                         *
* for the calculation of b --> s gamma                                *
* author:Mia Schelke, schelke@physto.se, 2003-03-31                   *
***********************************************************************

\end{verbatim}
 \end{routine}

%%%%% routine dsbsgg8w.f %%%%%
\begin{routine}{dsbsgg8w.f}
\begin{verbatim}
      function dsbsgg8w(x)

***********************************************************************
* Function G^W_8(x) in eq. (27) of                                    *
* Ciuchini et al., hep-ph/9806308                                     *
* for the calculation of b --> s gamma                                *
* author:Mia Schelke, schelke@physto.se, 2003-04-08                   *
***********************************************************************

\end{verbatim}
 \end{routine}

%%%%% routine dsbsggxy.f %%%%%
\begin{routine}{dsbsggxy.f}
\begin{verbatim}


      function dsbsggxy(x,y)

***********************************************************************
* Function G'(x,y) in app. B p. 18 of                                 *
* Ciuchini et al., hep-ph/9806308                                     *
* for the calculation of b --> s gamma                                *
* author:Mia Schelke, schelke@physto.se, 2003-04-22                   *
***********************************************************************

\end{verbatim}
 \end{routine}

%%%%% routine dsbsgh1x.f %%%%%
\begin{routine}{dsbsgh1x.f}
\begin{verbatim}
      function dsbsgh1x(x)

***********************************************************************
* Function H_1(x) in app A p. 16 of Ciuchini et al., hep-ph/9806308   *
* x must be a positive number                                         *
* for the calculation of b --> s gamma                                *
* author:Mia Schelke, schelke@physto.se, 2003-04-07                   *
***********************************************************************

\end{verbatim}
 \end{routine}

%%%%% routine dsbsgh2xy.f %%%%%
\begin{routine}{dsbsgh2xy.f}
\begin{verbatim}
      function dsbsgh2xy(x,y)

***********************************************************************
* Function H_2(x,y) in (12) of Degrassi et al.,                       *
* hep-ph/0009337                                                      *
* x and y must be  positive numbers                                   *
* for the calculation of b --> s gamma                                *
* author:Mia Schelke, schelke@physto.se, 2003-04-03                   *
***********************************************************************

\end{verbatim}
 \end{routine}

%%%%% routine dsbsgh3x.f %%%%%
\begin{routine}{dsbsgh3x.f}
\begin{verbatim}
      function dsbsgh3x(x)

***********************************************************************
* Function H_3(x) in app A p. 17 of Ciuchini et al., hep-ph/9806308   *
* x must be a positive number                                         *
* for the calculation of b --> s gamma                                *
* author:Mia Schelke, schelke@physto.se, 2003-04-07                   *
***********************************************************************

\end{verbatim}
 \end{routine}

%%%%% routine dsbsghd.f %%%%%
\begin{routine}{dsbsghd.f}
\begin{verbatim}
      function dsbsghd()

***********************************************************************
* Function H_d (with d=b) in app A p. 16 of                           *
* Ciuchini et al., hep-ph/9806308                                     *
* Note that we have inserted d=b                                      *
* for the calculation of b --> s gamma                                *
* author:Mia Schelke, schelke@physto.se, 2003-04-08                   *
***********************************************************************

\end{verbatim}
 \end{routine}

%%%%% routine dsbsghtd.f %%%%%
\begin{routine}{dsbsghtd.f}
\begin{verbatim}
      function dsbsghtd(famd)

***********************************************************************
* Function H_t^d in app A p. 16 of                                    *
* Ciuchini et al., hep-ph/9806308                                     *
* The input parameter famd is the family of the down sector           *
* it should be 1 for the first family, 2 for the 2nd and 3 for the 3rd*
* for the calculation of b --> s gamma                                *
* author:Mia Schelke, schelke@physto.se, 2003-04-08                   *
***********************************************************************

\end{verbatim}
 \end{routine}

%%%%% routine dsbsgkc.f %%%%%
\begin{routine}{dsbsgkc.f}
\begin{verbatim}
      function dsbsgkc()
***********************************************************************
* Program that calculates K_c in (3.7) of Gambino and Misiak,         *
* Nucl. Phys. B611 (2001) 338                                         *
* for the calculation of b --> s gamma                                *
* author:Mia Schelke, schelke@physto.se, 2003-03-25                   *
***********************************************************************
\end{verbatim}
 \end{routine}

%%%%% routine dsbsgkt.f %%%%%
\begin{routine}{dsbsgkt.f}
\begin{verbatim}

      function dsbsgkt(flag)
c      program dsackt
***********************************************************************
* Program that calculates K_t in (3.8) of Gambino and Misiak,         *
* Nucl. Phys. B611 (2001) 338                                         *
* for the calculation of b --> s gamma                                *
* Input: flag:  0 = only standard model                               *
*               1 = standard model plus SUSY corrections              *
* author:Mia Schelke, schelke@physto.se, 2003-03-24                   *
***********************************************************************
\end{verbatim}
 \end{routine}

%%%%% routine dsbsgmtmuw.f %%%%%
\begin{routine}{dsbsgmtmuw.f}
\begin{verbatim}

      function dsbsgmtmuw(m)

***********************************************************************
* The running top mass evaluated at a weak scale m=mu_w               *
* using nf effective quark flavours (taken to be nf=5 here)           *
* Uses eq. (32) of Ciuchini et al. hep-ph/9710335                     *
* for the calculation of b --> s gamma                                *
* author:Mia Schelke, schelke@physto.se, 2003-04-03                   *
***********************************************************************

\end{verbatim}
 \end{routine}

%%%%% routine dsbsgmtmuwint.f %%%%%
\begin{routine}{dsbsgmtmuwint.f}
\begin{verbatim}
      function dsbsgmtmuwint(mtstart,mstart,m,nf)

***********************************************************************
* The running top mass from value mtstart at scale mstart.            *
* The running is done with nf effective active quark flavours.        *
* Uses eq. (32) of Ciuchini et al. hep-ph/9710335                     *
* for the calculation of b --> s gamma                                *
* author:Mia Schelke, schelke@physto.se, 2003-04-03                   *
***********************************************************************

\end{verbatim}
 \end{routine}

%%%%% routine dsbsgri.f %%%%%
\begin{routine}{dsbsgri.f}
\begin{verbatim}
      function dsbsgri(famd)

***********************************************************************
* Function R_i in eq. (19) of                                         *
* Ciuchini et al., hep-ph/9806308                                     *
* The expression has been extended to large tanbe, by dropping        *
* ln((mu_w)^2)/m^2(kgluin))                                           *
* as explained in Degrassi et al., hep-ph/0009337 p.11                *
* The input parameter famd is the family of the down sector           *
* it should be 1 for the first family, 2 for the 2nd and 3 for the 3rd* 
* for the calculation of b --> s gamma                                *
* author:Mia Schelke, schelke@physto.se, 2003-04-07                   *
***********************************************************************

\end{verbatim}
 \end{routine}

%%%%% routine dsbsgud.f %%%%%
\begin{routine}{dsbsgud.f}
\begin{verbatim}
      function dsbsgud()

***********************************************************************
* Function U_d (with d=b) in app A p. 15-6 of                         *
* Ciuchini et al., hep-ph/9806308                                     *
* Note that we have inserted d=b                                      *
* for the calculation of b --> s gamma                                *
* author:Mia Schelke, schelke@physto.se, 2003-04-08                   *
***********************************************************************

\end{verbatim}
 \end{routine}

%%%%% routine dsbsgutd.f %%%%%
\begin{routine}{dsbsgutd.f}
\begin{verbatim}
      function dsbsgutd(famd)

***********************************************************************
* Function U_t^d in app A p. 16 of                                    *
* Ciuchini et al., hep-ph/9806308                                     *
* The input parameter famd is the family of the down sector           *
* it should be 1 for the first family, 2 for the 2nd and 3 for the 3rd*
* for the calculation of b --> s gamma                                *
* author:Mia Schelke, schelke@physto.se, 2003-04-08                   *
***********************************************************************

\end{verbatim}
 \end{routine}

%%%%% routine dsbsgwud.f %%%%%
\begin{routine}{dsbsgwud.f}
\begin{verbatim}
      function dsbsgwud(famu,famd)

***********************************************************************
* Function W_u^d in app A p. 15 of                                    *
* Ciuchini et al., hep-ph/9806308                                     *
* The input parameters famu and famd are the families of the up- and  *
* down sector respectively, and they should be 1 for the first family,*
* 2 for the 2nd and 3 for the 3rd                                     *
* for the calculation of b --> s gamma                                *
* author:Mia Schelke, schelke@physto.se, 2003-04-08                   *
***********************************************************************

\end{verbatim}
 \end{routine}

%%%%% routine dsbsgwxy.f %%%%%
\begin{routine}{dsbsgwxy.f}
\begin{verbatim}
      function dsbsgwxy(x,y)

***********************************************************************
* Function W[x,y] in app. A p. 15 of                                  *
* Ciuchini et al., hep-ph/9806308                                     *
* for the calculation of b --> s gamma                                *
* author:Mia Schelke, schelke@physto.se, 2003-04-08                   *
***********************************************************************

\end{verbatim}
 \end{routine}

%%%%% routine dsbsgyt.f %%%%%
\begin{routine}{dsbsgyt.f}
\begin{verbatim}
      function dsbsgyt(m)

***********************************************************************
* Function that calculates y_t(m=mu_susy), the top Yukawa coupling    *
* at the scale m=mu_susy                                              *
* Note that m=mu_susy should be of the order of 1TeV, e.g. m_gluino   *
* Uses eq (23) of Degrassi et al.,  hep-ph/0009337                    *
* Note that y_t(susy)=\tilde{y}_t(susy) (see text in the ref.)        *
* for the calculation of b --> s gamma                                *
* author:Mia Schelke, schelke@physto.se, 2003-04-03                   *
***********************************************************************

\end{verbatim}
 \end{routine}

%%%%% routine dsphi22a.f %%%%%
\begin{routine}{dsphi22a.f}
\begin{verbatim}
      function dsphi22a(t)
***********************************************************************
* The first integrand of \phi_22 in (E.2) of Gambino and Misiak,      *
* Nucl. Phys. B611 (2001) 338                                         *
* for the calculation of b --> s gamma                                *
* author:Mia Schelke, schelke@physto.se, 2003-03-10                   *
***********************************************************************
\end{verbatim}
 \end{routine}

%%%%% routine dsphi22b.f %%%%%
\begin{routine}{dsphi22b.f}
\begin{verbatim}
      function dsphi22b(t)
***********************************************************************
* The 2nd integrand of \phi_22 in (E.2) of Gambino and Misiak,        *
* Nucl. Phys. B611 (2001) 338                                         *
* for the calculation of b --> s gamma                                *
* author:Mia Schelke, schelke@physto.se, 2003-03-10                   *
***********************************************************************
\end{verbatim}
 \end{routine}

%%%%% routine dsphi27a.f %%%%%
\begin{routine}{dsphi27a.f}
\begin{verbatim}
      function dsphi27a(t)
***********************************************************************
* The first integrand of \phi_27 in (E.3) of Gambino and Misiak,      *
* Nucl. Phys. B611 (2001) 338                                         *
* for the calculation of b --> s gamma                                *
* author:Mia Schelke, schelke@physto.se, 2003-03-10                   *
***********************************************************************
\end{verbatim}
 \end{routine}

%%%%% routine dsphi27b.f %%%%%
\begin{routine}{dsphi27b.f}
\begin{verbatim}
      function dsphi27b(t)
***********************************************************************
* The 2nd integrand of \phi_27 in (E.3) of Gambino and Misiak,        *
* Nucl. Phys. B611 (2001) 338                                         *
* for the calculation of b --> s gamma                                *
* author:Mia Schelke, schelke@physto.se, 2003-03-10                   *
***********************************************************************
\end{verbatim}
 \end{routine}

\newpage
\chapter[db: Anti-deuteron fluxes from the halo]{\codeb{src/db}:\\ Anti-deuteron fluxes from the halo}
\label{ch:src-db}

%%%%%%%%%%%%%%%%%%%%%%%%%%%%%%%%%%%%%%%%%%%%%%%%%%%%%%%%%%%%%%%%%%%%

\section{Anti-deuteron fluxes from annihilation in the halo}

The anti-deuteron fluxes are calculated here following the approach of
\cite{dbar}. This means that we calculate the anti-deuteron fluxes
based on a merging model of antiprotons and antineutrons in quark
jets. Apart from this, the anti-deuterons are propagated in the same
way as antiprotons.
\section{Routine headers -- fortran files}

%%%%% routine dsdbsigmavdbar.f %%%%%
\begin{routine}{dsdbsigmavdbar.f}
\begin{verbatim}
      real*8 function dsdbsigmavdbar(en)
c total inelastic cross section dbar + h
c Yad.Fiz.14:134-136,1971
c check Review of Particle Properties of 1992, rev [9] in DFS
\end{verbatim}
 \end{routine}

%%%%% routine dsdbtd15.f %%%%%
\begin{routine}{dsdbtd15.f}
\begin{verbatim}
      real*8 function dsdbtd15(tp,howinp)

**********************************************************************
*** function dsdbtd15 is the containment time in 10^15 sec
***   input:
***     tp - kinetic energy per nucleon in gev
***     how - 1 calculate t_diff only for requested momentum
***           2 tabulate t_diff for first call and use table for
***             subsequent calls
***           3 as 2, but also write the table to disk as dbtd.dat
***           4 read table from disk on first call, and use that for
***             subsequent calls
***   output:
***     t_diff in units of 10^15 sec
*** calls dsdbtd15x for the actual calculation.
*** author: joakim edsjo (edsjo@physto.se)
*** uses piero ullios propagation routines.
*** date: dec 16, 1998
*** modified: 98-07-13 paolo gondolo
**********************************************************************

\end{verbatim}
 \end{routine}

%%%%% routine dsdbtd15beu.f %%%%%
\begin{routine}{dsdbtd15beu.f}
\begin{verbatim}
**********************************************************************
*** function called in dsdbtd15x
*** it gives the antiproton diffusion time in units of 10^15 sec
*** it assumes the diffusion model in:
***   bergstrom, edsjo & ullio, ajp 526 (1999) 215
*** inputs:
***     td - kinetic energy per nucleon (gev)
*** 
*** author: piero ullio (piero@tapir.caltech.edu)
*** date: 00-07-13
*** modified: 04-01-22 (pu)
**********************************************************************

      real*8 function dsdbtd15beu(td)
\end{verbatim}
 \end{routine}

%%%%% routine dsdbtd15beucl.f %%%%%
\begin{routine}{dsdbtd15beucl.f}
\begin{verbatim}
**********************************************************************
*** function that gives the dbar diffusion time per unit volume
*** (units of 10^15 sec kpc^-3) for a dbar point source located
*** at rcl, zcl, thetacl (in the cylidrical framework with the sun 
*** located at r=r_0, z=0 theta=0) and some small "angular width"
*** deltathetacl which makes the routine converge much faster
*** rcl, zcl, thetacl and deltathetacl are in the dspb_clcom.h common 
*** blocks and must be before calling this routine. rcl and zcl are in
*** kpc, thetacl and deltathetacl in rad.
*** numerical convergence gets slower for rcl->0 or zcl->0
***
*** it assumes the diffusion model in:
***   bergstrom, edsjo & ullio, ajp 526 (1999) 215
*** inputs:
***     td - dbar kinetic energy (gev)
***
*** the conversion from this source function to the local dbar flux 
*** is the same as for dsdbtd15beu(td), except that dsdbtd15beucl(td)
*** must be multiplied by: 
***     int dV (rho_cl(\vec{x}_cl)/rho0)**2
***   where the integral is over the volume of the clump,
***   rho_cl(\vec{x}_cl) is the density profile in the clump
***   and the local halo density rho0 is the normalization scale used 
***   everywhere 
*** 
*** author: piero ullio (ullio@sissa.it)
*** date: 04-01-22
**********************************************************************

      real*8 function dsdbtd15beucl(td)
\end{verbatim}
 \end{routine}

%%%%% routine dsdbtd15beuclsp.f %%%%%
\begin{routine}{dsdbtd15beuclsp.f}
\begin{verbatim}
**********************************************************************
*** function which makes a tabulation of dsdbtd15beucl as function
*** the distance between source and observer L, and neglecting the
*** weak dependence of dsdbtd15beucl over the vertical coordinate for 
*** the source zcl
*** 
*** for every td dsdbtd15beuclsp is tabulated on first call in L, with        
*** L between: 
***    Lmin=0.9d0*(r_0-pbrcy) and 
***    Lmax=1.1d0*dsqrt((r_0+pbrcy)**2+pbzcy**2)
*** and stored in spline tables. 
***
*** pbrcy and pbzcy in kpc are passed through a common block in 
*** dspb_clcom.h and should be set before the calling this routine.
***
*** there is no internal check to verify whether between to consecutive 
*** calls, with the same td, pbrcy and pbzcy, or halo parameters, or  
*** propagation parameters are changed. If this is done make sure, 
*** before calling this function, to reinitialize to zero the integer 
*** parameter clspset in the common block:
***
***      real*8 tdsetup
***      integer clspset
***      common/clspsetcom/tdsetup,clspset
*** 
*** input: L in kpc, td in GeV
*** output in 10^15 s kpc^-3
***
*** author: piero ullio (ullio@sissa.it)
*** date: 04-01-22
**********************************************************************


      real*8 function dsdbtd15beuclsp(L,td)
\end{verbatim}
 \end{routine}

%%%%% routine dsdbtd15beum.f %%%%%
\begin{routine}{dsdbtd15beum.f}
\begin{verbatim}
**********************************************************************
*** function called in dsdbtd15x
*** it gives the antiproton diffusion time in units of 10^15 sec
*** it assumes the diffusion model in:
***   bergstrom, edsjo & ullio, ajp 526 (1999) 215
***   but with the DC-like setup as in moskalenko et al.
***      ApJ 565 (2002) 280
*** inputs:
***     td - kinetic energy per nucleon (gev)
*** 
*** author: piero ullio (piero@tapir.caltech.edu)
*** date: 00-07-13
**********************************************************************

      real*8 function dsdbtd15beum(td)
\end{verbatim}
 \end{routine}

%%%%% routine dsdbtd15comp.f %%%%%
\begin{routine}{dsdbtd15comp.f}
\begin{verbatim}
**********************************************************************
*** function which computes the dbar diffusion time term corresponding 
*** to the axisymmetric diffuse source within a cylinder of radius
*** pbrcy and height 2* pbzcy. 
*** This routine assumes also that the Green function of
*** the diffusion equation dsdbtd15beuclsp(L,tp) does depend just
*** on kinetic energy tp and distance from the observer L, neglecting
*** a weak dependence on the cylindrical coordinate z.
*** For every tp, dsdbtd15beuclsp is tabulated on first call in L and
*** stored in spline tables. 
*** In this function and in dsdbtd15beuclsp, pbrcy and pbzcy in kpc are 
*** passed through a common block in dspbcom.h. There is no check 
*** in dsdbtd15beuclsp on whether, pbrcy and pbzcy which define the
*** interval of tabulation are changed. Check header dsdbtd15beuclsp
*** for more details on this and other warnings, and how to get the 
*** right implementation is such parameters are changed while running 
*** our own code
*** After the tabulation, the following integral is performed:
***
***  2 int_0^{pbzcy} int_0^{pbrcy} dr r  int_0^{2\pi} dphi
***     (dshmaxirho(r,zint)/rho0)^2 * dsdbtd15beuclsp(L(z,r,theta),tp)
***
*** The triple integral is splitted into a double integral on r and 
*** theta, this result is tabulated in z and then this integral is 
*** performed. The tabulation in z has at least 100 points on a 
*** regular grid between 0 and pbzcy (this is set by the parameter
*** incompnpoints in the dspbcompint1 function), however points are 
*** added as long as the values of the function in two nearest 
*** neighbour points differs more than 10% (this is set by the 
*** parameter reratio in the dspbcompint1 function)
***
*** input: scale in kpc, tp in GeV
*** output in 10^15 s
*** 
*** author: piero ullio (ullio@sissa.it)
*** date: 04-01-22
**********************************************************************

      real*8 function dsdbtd15comp(tp)
\end{verbatim}
 \end{routine}

%%%%% routine dsdbtd15point.f %%%%%
\begin{routine}{dsdbtd15point.f}
\begin{verbatim}
**********************************************************************
*** function which approximates the function dsdbtd15comp by 
*** estimating that diffusion time term supposing to have a point 
*** source located at the galactic center but then weighting it with 
*** the emission over a whole cylinder of radius scale and 
*** height 2*scale, i.e. rho2int (to be given in kpc^3). 
*** The goodness of the approximation should be checked by comparing
***    dsdbtd15point(rho2int,db) with
***    dsdbtd15comp(db) for different value of db and scale, 
*** and depending on the halo profile chosen and level of precision 
*** required. The comparison has to be performed but setting 
*** rho2int=dshmrho2cylint(scale,scale) and each 
*** pbrcy and pbzcy pair equal to scale before calling dspbtd15comp, 
*** possibly resetting the parameter clspset as well, see the header
*** of the function dspbtd15beuclsp
*** 
*** input: rho2int=dshmrho2cylint(scale,scale) in kpc^3, db in GeV
*** output in 10^15 s
***
*** author: piero ullio (ullio@sissa.it)
*** date: 04-01-22
**********************************************************************

      real*8 function dsdbtd15point(rho2int,db)
\end{verbatim}
 \end{routine}

%%%%% routine dsdbtd15x.f %%%%%
\begin{routine}{dsdbtd15x.f}
\begin{verbatim}
      real*8 function dsdbtd15x(tp)
**********************************************************************
*** antideuteron propagation according to various models
*** dspbtd15x is containment time in 10^15 sec
*** inputs:
***     tp - kinetic energy per nucleon (gev)
*** from common blocks
***     pbpropmodel - 2 bergstrom,edsjo,ullio diffusion
***                   3 bergstrom,edsjo,ullio diffusion
***                       but with the DC-like setup as in moskalenko 
***                       et al. ApJ 565 (2002) 280
***
*** author: paolo gondolo 99-07-13
*** modified: piero ullio 00-07-13
**********************************************************************
\end{verbatim}
 \end{routine}

\newpage
\chapter[dd: Direct detection]{\codeb{src/dd}:\\ Direct detection}
\label{ch:src-dd}

%%%%%%%%%%%%%%%%%%%%%%%%%%%%%%%%%%%%%%%%%%%%%%%%%%%%%%%%%%%%%%%%%%%%
\section{Direct detection -- theory}

\paolo{WE MUST DECIDE WHAT TO INCLUDE IN THE DIRECT DETECTION ROUTINES.
   PRESENTLY, ONLY THE NEUTRALINO-PROTON AND NEUTRALINO-NEUTRON CROSS SECTIONS
   SHOULD BE MADE AVAILABLE. ALTHOUGH THE CROSS SECTIONS OFF NUCLEI ARE ON THE
   AGENDA, DO WE REALLY WANT TO MENTION THEM HERE?}

If  neutralinos are indeed the CDM needed on galaxy scales and larger,
there should be a substantial flux of these particles in the Milky
Way halo. Since the interaction strength  is
essentially given by the same weak couplings as, e.g., for neutrinos
there is a non-negligible chance of detecting them in low-background
counting experiments \cite{goodmanwitten}.
Due to the large parameter space of MSSM, even
with the simplifying assumptions above, there is a rather wide span of
predictions for the event rate in detectors of various types. It is
interesting, however, that the models giving the largest rates are
already starting to be probed by present direct detection
experiments \cite{bg,bottino}.


The rate for direct detection of galactic neutralinos, integrated over
deposited energy assuming no energy threshold, is
\begin{equation}
   R = \sum_i N_i n_\chi \langle \sigma_{i\chi} v \rangle ,
\end{equation}
where $ N_i $ is the number of nuclei of species $i$ in the detector,
$n_\chi$ is the local galactic neutralino number density, $
\sigma_{i\chi} $ is the neutralino-nucleus elastic cross section, and
the angular brackets denote an average over $ v $, the neutralino
speed relative to the detector as described in Section~\ref{sec:halo}.

In \ds, the basic quantities computed are the neutralino-nucleon cross
sections, which are free of complications related to nuclear
structure, and various experimental details like energy threshold,
efficiencies etc.  However, as a crude estimate of the expected rates
in realistic detectors, the total neutralino-nucleus scattering rates
can be obtained for $^{76}$Ge, Al$_2$O$_3$ (sapphire) and NaI.
Usually, it is the spin-independent interaction that gives the most
important contribution in target materials such as Na, Cs, Ge, I, or
Xe, due to the enhancement caused by the coherence of all nucleons in
the target nucleus.

The neutralino-nucleus elastic cross section can be written as
\begin{equation}
   \sigma_{i\chi} = {1 \over 4 \pi v^2 } \int_{0}^{4 m^2_{i\chi} v^2}
   \mbox{\rm d} q^2 G_{i\chi}^2(q^2) ,
\end{equation}
where $ m_{i\chi} $ is the neutralino-nucleus reduced mass, $q$ is the
momentum transfer and $G_{i\chi}(q^2) $ is the effective
neutralino-nucleus vertex. We write
\begin{equation}
   G^2_{i\chi}(q^2) = A_i^2 F^2_S(q^2) G_S^2 +
   4 \Lambda_i^2 F^2_A(q^2) G_A^2 ,
   \label{detrate1}
\end{equation}
which shows the coherent enhancement factor $A_i^2$ for the
spin-independent cross section
(often $ \Lambda_i^2 $ is written as $ \lambda^2 J(J+1) $ ). We assume
gaussian nuclear form factors \cite{Gould87} \paolo{WE ARE NOT CURRENTLY
   PROVIDING FORM FACTORS.} \comment{Use better form
factors?}
\begin{equation}
   F_S(q^2) = F_A(q^2) =
   \exp(-q^2R_i^2/6\hbar^2) ,
\end{equation}
\begin{equation}
   R_i = ( 0.3 + 0.89 A_i^{1/3} ) {\rm fm} ,
\end{equation}
which should provide us with a good approximation of the integrated
detection rate \cite{EllisFlores}, in which we are only interested.
(To obtain the differential rate with reasonable accuracy, better
approximations are needed \cite{engel}.)

Using heavy-squark effective lagrangians
\cite{efflagrange}, we get
\begin{equation}
   G_S = \sum_{q={\rm u,d,s,c,b,t}} \langle \bar{q} q \rangle
   \left( \sum_{h=H_1,H_2} { g_{h\chi\chi} g_{hqq} \over m_h^2 } -
   {1\over 2} \sum_{k=1}^6 { g_{L\tilde{q}_k\chi q} g_{R\tilde{q}_k\chi q} \over
   m^2_{\tilde{q}_k} } \right)
\label{GS}
\end{equation}
and
\begin{equation}
   G_A = \sum_{q={\rm u,d,s}} \Delta q
    \left( { g_{Z\chi\chi} g_{Zqq} \over m_Z^2 } + {1\over 8} \sum_{k=1}^6
   { g_{L\tilde{q}_k\chi q}^2 + g_{R\tilde{q}_k\chi q}^2 \over m^2_{\tilde{q}_k}
   } \right) .
\label{GA}
\end{equation}
The $g$'s are elementary vertices involving the particles indicated by
the indices, and they read
\begin{eqnarray}
   g_{h\chi\chi} & = & \left\{  \begin{array}{ll}
   \left( g Z_{\chi2} - g_y Z_{\chi1} \right)
   \left( - Z_{\chi3} \cos\alpha + Z_{\chi4} \sin\alpha \right) ,
   & {\rm for\ } H_1 , \\
   \left( g Z_{\chi2} - g_y Z_{\chi1} \right)
   \left( Z_{\chi3} \sin\alpha + Z_{\chi4} \cos\alpha \right) ,
   & {\rm for\ } H_2 ,
   \end{array} \right. \\
   g_{hqq} & = & \left\{ \begin{array}{ll}
   - Y_q \cos\alpha / \sqrt{2} , & {\rm for\ } H_1 , \\
   + Y_q \sin\alpha / \sqrt{2} , & {\rm for\ } H_2 , \end{array} \right. \\
   g_{Z\chi\chi} & = & {g \over 2 \cos\theta_W}
   \left( Z_{\chi3}^2 - Z_{\chi4}^2 \right)\\
   g_{Zqq} & = & - {g \over 2 \cos\theta_W} T_{3q} , \\
   g_{L\tilde{q}_k\chi q} & = & g_{LL} \Gamma_{QL}^{kq} +
   g_{RL} \Gamma_{QR}^{kq} , \\
   g_{R\tilde{q}_k\chi q} & = & g_{LR} \Gamma_{QL}^{kq} +
   g_{RR} \Gamma_{QR}^{kq} ,
\end{eqnarray}
with
\begin{eqnarray}
   g_{LL} & = & - {1\over\sqrt{2}}
   \left( T_{3q} g Z_{\chi2} + {1\over3} g_y Z_{\chi1} \right) , \\
   g_{RR} & = & \sqrt{2} e_q g_y Z_{\chi1} , \\
   g_{LR} & = & g_{RL} \, = \, \left\{ \begin{array}{ll}
   - Y_q Z_{\chi3} , & {\rm for\ } q={\rm u,c,t} , \\
   - Y_q Z_{\chi4} , & {\rm for\ } q={\rm d,s,b} ,
   \end{array} \right.
\end{eqnarray}
and
\begin{eqnarray}
   Y_q & = & \left\{ \begin{array}{ll}
   m_q / v_2 , & {\rm for\ } q={\rm u,c,t}, \\
   m_q / v_1 , & {\rm for\ } q={\rm d,s,b}.
   \end{array} \right.
\end{eqnarray}
Defining ($N=n,p$)
\beq
f^N_{Tq}\equiv {\langle N |m_q\bar q q|N\rangle\over m_N},
\eeq
we take in \ds\ the numerical values \cite{Gasser}
\begin{eqnarray}
   & f^p_{Tu} = 0.023, \qquad f^p_{Td} = 0.034, \qquad
    &\nonumber\\
&f^p_{Tc} = 0.0595, \qquad f^p_{Ts} = 0.14, \qquad&\nonumber\\
&f^p_{Tt} = 0.0595, \qquad f^p_{Tb}=0.0595 \qquad
\end{eqnarray}
and
\begin{eqnarray}
   & f^n_{Tu} = 0.019, \qquad f^n_{Td} = 0.041, \qquad
    &\nonumber\\
&f^n_{Tc} = 0.0592 \qquad f^n_{Ts} = 0.14, \qquad&\nonumber\\
&f^n_{Tt} = 0.0592, \qquad f^n_{Tb} = 0.0592. \qquad
\end{eqnarray}

For the quark contributions to the nucleon spin we
take \cite{SMC}
\begin{equation}
   \Delta {\rm u} = 0.77, \qquad \Delta {\rm d} = -0.40, \qquad
   \Delta {\rm s} = -0.12 .
\end{equation}
However, the older set of data \cite{jaffe}
\begin{equation}
   \Delta {\rm u} = 0.77, \qquad \Delta {\rm d} = -0.49, \qquad
   \Delta {\rm s} = -0.15
\end{equation}
can optionally be used.

Moreover, we take for the $\Lambda$ factors
\begin{equation}
   \Lambda^2_{\rm Al} = 0.087, \qquad
   \Lambda^2_{\rm Na} = 0.041 \quad {\rm and} \quad
   \Lambda^2_{\rm I} = 0.007,
   \label{detrate2}
\end{equation}
according to the odd-group model \cite{EngelVogel}.

\comment{Change our direct detection routines?}

%%%%%%%%%%%%%%%%%%%%%%%%%%%%%%%%%%%%%%%%%%%%%%%%%%%%%%%%%%%%%%%%%%%%
\section{Direct detection -- routines}


\begin{sub}{subroutine \ftb{dsddneunuc}(sigsip,sigsin,sigsdp,sigsdn)}
  \itit{Purpose:} Calculate the spin-independent and spin-dependent
  scattering cross sections for neutralinos on neutrons and protons.
  \itit{Output:}
  \itv{sigsip}{r8} The spin-independent neutralino-proton scattering
  cross section, $\sigma^{SI}_{\chi p}$, in units of cm$^3$ s$^{-1}$.
  \itv{sigsin}{r8} The spin-independent neutralino-neutron scattering
  cross section, $\sigma^{SI}_{\chi n}$, in units of cm$^3$ s$^{-1}$.
  \itv{sigsip}{r8} The spin-dependent neutralino-proton scattering
  cross section, $\sigma^{SD}_{\chi p}$, in units of cm$^3$ s$^{-1}$.
  \itv{sigsin}{r8} The spin-dependent neutralino-neutron scattering
  cross section, $\sigma^{SD}_{\chi n}$, in units of cm$^3$ s$^{-1}$.
\end{sub}




\section{Routine headers -- fortran files}

%%%%% routine dsdddn1.f %%%%%
\begin{routine}{dsdddn1.f}
\begin{verbatim}
      function dsdddn1(ms,mq,mx)
c
c     auxiliary function replacing the propagator for heavy squarks in
c     the drees-nojiri treatment of neutralino-nucleon scattering
c     a^2-b^2 terms
c     dsdddn1 = 3/2 mq^2 I1 - mq^2 mx^2 I3
c
\end{verbatim}
 \end{routine}

%%%%% routine dsdddn2.f %%%%%
\begin{routine}{dsdddn2.f}
\begin{verbatim}
      function dsdddn2(ms,mq,mx)
c
c     auxiliary function replacing the propagator for heavy squarks in
c     the drees-nojiri treatment of neutralino-nucleon scattering
c     a^2+b^2 terms
c     dsdddn2 = I5 + 2 mx^2 I4 - 3 I2
c
\end{verbatim}
 \end{routine}

%%%%% routine dsdddn3.f %%%%%
\begin{routine}{dsdddn3.f}
\begin{verbatim}
      function dsdddn3(ms,mq,mx)
c
c     auxiliary function replacing the propagator for heavy squarks in
c     the drees-nojiri treatment of neutralino-nucleon scattering
c     twist-2 a^2-b^2 terms
c     dsdddn3 = - mq^2 mx^2 I3
c
\end{verbatim}
 \end{routine}

%%%%% routine dsdddn4.f %%%%%
\begin{routine}{dsdddn4.f}
\begin{verbatim}
      function dsdddn4(ms,mq,mx)
c
c     auxiliary function replacing the propagator for heavy squarks in
c     the drees-nojiri treatment of neutralino-nucleon scattering
c     twist-2 a^2+b^2 terms
c     dsdddn4 = I5 + 2 mx^2 I4
c
\end{verbatim}
 \end{routine}

%%%%% routine dsdddrde.f %%%%%
\begin{routine}{dsdddrde.f}
\begin{verbatim}
      subroutine dsdddrde(t,e,n,a,z,stoich,rsi,rsd,modulation)
c_______________________________________________________________________
c  differential recoil rate
c  common:
c    'dssusy.h' - file with susy common blocks
c  input:
c    e : real*8              : nuclear recoil energy in keV
c    n : integer             : number of nuclear species
c    a : n-dim integer array : atomic numbers
c    z : n-dim integer array : charge numbers
c    stoich : n-dim integer array : stoichiometric coefficients
c    t : real*8 : time in days from 12:00UT Dec 31, 1999
c    modulation : integer : 0=no modulation 1=annual modulation
c      with no modulation (ie without earth velocity), t is irrelevant
c  output:
c    rsi : real*8 : spin-independent differential rate
c    rsd : real*8 : spin-dependent differential rate
c  units: counts/kg-day-keV
c  author: paolo gondolo (paolo@physics.utah.edu) 2004
c=======================================================================
\end{verbatim}
 \end{routine}

%%%%% routine dsddeta.f %%%%%
\begin{routine}{dsddeta.f}
\begin{verbatim}
      subroutine dsddeta(vmin,t,eta)
c_______________________________________________________________________
c  eta function entering the differential rate: eta = \int {f(v)/v} d^3 v
c
c  Truncated Maxwellian. 
c
c  input:
c    vmin : minimum velocity to deposit energy e, in km/s 
c           vmin=sqrt(M*E/2/mu^2)
c    t : time, in fraction of the year
c  output:
c    eta : in (km/s)^{-1}
c  authors: paolo gondolo (paolo@physics.utah.edu) 2004
c           piero ullio (ullio@sissa.it) 2004
c=======================================================================
\end{verbatim}
 \end{routine}

%%%%% routine dsddffsd.f %%%%%
\begin{routine}{dsddffsd.f}
\begin{verbatim}
      subroutine dsddffsd(q,a,z,s00,s01,s11,j)
c_______________________________________________________________________
c  Spin-dependent structure functions for direct detection.
c  input:
c    q : real*8  : momentum transfer in GeV ( q=sqrt(M*E/2/mu^2) )
c    a : integer : atomic number
c    z : integer : charge number
c  output:
c    s00, s01, s11 : the spin-dependent structure functions S_{00}(q), 
c        S_{01}(q), S_{11}(q) as defined by Engel, PRL
c    j : total nuclear spin
c  author: paolo gondolo (paolo@physics.utah.edu) 2004
c  modified: pg 040605 switched s01 and s11 in Na-23
c=======================================================================
\end{verbatim}
 \end{routine}

%%%%% routine dsddffsi.f %%%%%
\begin{routine}{dsddffsi.f}
\begin{verbatim}
      subroutine dsddffsi(q,a,z,ff)
c_______________________________________________________________________
c  Spin-independent form factor for direct detection.
c  input:
c    q : real*8  : momentum transfer in GeV ( q=sqrt(M*E/2/mu^2) )
c    a : integer : atomic number
c    z : integer : charge number
c  output:
c    ff : |F(q)|^2, the square of the form factor
c  author: paolo gondolo (paolo@physics.utah.edu) 2004
c=======================================================================
\end{verbatim}
 \end{routine}

%%%%% routine dsddgpgn.f %%%%%
\begin{routine}{dsddgpgn.f}
\begin{verbatim}
      subroutine dsddgpgn(gps,gns,gpa,gna)
c_______________________________________________________________________
c  neutralino nucleon four-fermion couplings
c  common:
c    'dssusy.h' - file with susy common blocks
c  output:
c    gps, gns : proton and neutron scalar four-fermion couplings
c    gpa, gna : proton and neutron axial four-fermion couplings
c    units: GeV^-4
c  author: paolo gondolo (paolo@physics.utah.edu) 2004
c=======================================================================
\end{verbatim}
 \end{routine}

%%%%% routine dsddlim.f %%%%%
\begin{routine}{dsddlim.f}
\begin{verbatim}
      function dsddlim(mx,iexp)
c
c     limits on scattering cross section on nucleons from
c     direct dark matter searches
c
c     input: mx - wimp mass
c            iexp - experiment
c                   1 = future cawo_4 cresst
c                   2 = future genius
c                   3 = dama 1997, plb389, 757
c     output: dsddlim - upper limit or sensitivity limit on wimp-nucleon
c                     spin-independent cross section in pb
c                     (returns 10^99 if no limit)
c
c     author: paolo gondolo 1999
c
\end{verbatim}
 \end{routine}

%%%%% routine dsddlimits.f %%%%%
\begin{routine}{dsddlimits.f}
\begin{verbatim}
**********************************************************************
*** function dsddlimits gives the limits on f*sigma as a function
*** of neutralino mass. f is the halo fraction of dm (f=1 for 0.3
*** gev/cm^3) and sigma is the cross section in pb.
*** input: mx (wimp mass in gev)
***        type (1=spin-independent, 2=spin-dependent)
*** output: limit on f*sigma (pb)
*** based upon r. bernabei et al, plb 389 (1996) 757.
*** author: j. edsjo
*** date: 98-03-19
**********************************************************************

      real*8 function dsddlimits(mx,type)
\end{verbatim}
 \end{routine}

%%%%% routine dsddneunuc.f %%%%%
\begin{routine}{dsddneunuc.f}
\begin{verbatim}
      subroutine dsddneunuc(sigsip,sigsin,sigsdp,sigsdn)
c_______________________________________________________________________
c  neutralino nucleon cross section.
c  common:
c    'dssusy.h' - file with susy common blocks
c  output:
c    sigsip, sigsin : proton and neutron spin-independent cross sections
c    sigsdp, sigsdn : proton and neutron spin-dependent cross sections
c    units: cm^2
c  author: paolo gondolo (gondolo@lpthe.jussieu.fr) 1994,1995,2002
c     13-sep-94 pg no drees-nojiri twist-2 terms
c     22-apr-95 pg important bug corrected [ft -> ft mp/mq]
c     06-apr-02 pg drees-nojiri treatment added
c=======================================================================
\end{verbatim}
 \end{routine}

%%%%% routine dsddo.f %%%%%
\begin{routine}{dsddo.f}
\begin{verbatim}
      function dsddo(k,z,qsq)
c     k=0 Og, k=1 Ou, k=2 Od, ...... k=6 Ob
c     z=1 proton, z=2 neutron
\end{verbatim}
 \end{routine}

%%%%% routine dsddset.f %%%%%
\begin{routine}{dsddset.f}
\begin{verbatim}
      subroutine dsddset(sisd,cs)
c...set parameters for scattering cross section
c...  c - character string specifying choice to be made
c...author: paolo gondolo 2000-07-07
\end{verbatim}
 \end{routine}

%%%%% routine dsddsigmaff.f %%%%%
\begin{routine}{dsddsigmaff.f}
\begin{verbatim}
      subroutine dsddsigmaff(e,n,a,z,siff,sdff)
c_______________________________________________________________________
c  neutralino nucleus cross sections times form factors.
c  NOTE: the spin-dependent cross section is available only for a 
c        limited number of nuclei
c  common:
c    'dssusy.h' - file with susy common blocks
c  input:
c    e : real*8              : nuclear recoil energy in keV
c    n : integer             : number of nuclear species
c    a : n-dim integer array : atomic numbers
c    z : n-dim integer array : charge numbers
c  output:
c    siff : n-dim real*8 array : spin-independent cross
c           section times form factor
c    sdff : n-dim real*8 array : spin-dependent cross
c           section times form factor
c    units: cm^2
c  author: paolo gondolo (paolo@physics.utah.edu) 2004
c  modified: pg 040605 added missing factor of 4 in spin-dependent
c=======================================================================
\end{verbatim}
 \end{routine}

%%%%% routine dsddsigsi.f %%%%%
\begin{routine}{dsddsigsi.f}
\begin{verbatim}
      subroutine dsddsigsi(n,a,z,si)
c_______________________________________________________________________
c  neutralino nucleus cross section.
c  common:
c    'dssusy.h' - file with susy common blocks
c  input:
c    n : number of nuclear species
c    a : n-dim integer array with atomic numbers
c    z : n-dim integer array with charge numbers
c  output:
c    si : n-dim real*8 array with spin-independent cross sections
c    units: cm^2
c  author: paolo gondolo (paolo@physics.utah.edu) 2004
c=======================================================================
\end{verbatim}
 \end{routine}

%%%%% routine dsddvearth.f %%%%%
\begin{routine}{dsddvearth.f}
\begin{verbatim}
      subroutine dsddvearth(t,vearth)
c
c speed of the earth relative to the galaxy in km/s at 
c time t in days from 12:00 UT Dec 31, 1999
c
c formulas from Green, astro-ph/0304446, originally by Lewin and Smith
c
\end{verbatim}
 \end{routine}

\newpage
\chapter[ep: Positron fluxes from the halo]{\codeb{src/ep}:\\ Positron fluxes from the halo}
\label{ch:src-ep}

%%%%%%%%%%%%%%%%%%%%%%%%%%%%%%%%%%%%%%%%%%%%%%%%%%%%%%%%%%%%%%%%%%%%
\section{Positrons from the halo -- theory}

Neutralino annihilations in the halo will give rise to positrons either
directly or from decaying mesons in hadron jets.  We thus expect to get both
monochromatic positrons (at an energy of $m_{\chi}$) from direct annihilation
into $e^{+}e^{-}$ and continuum positrons from the other annihilation
channels. In general, the branching ratio for annihilation directly into $e^+
e^-$ is rather small due to the helicity-flip suppression $\propto m_e$ for
S-wave annihilation in the halo, but for some classes of models one can still
obtain a large enough branching ratio for the line to be observable.

The computation of the positron flux from neutralino annihilations used in \ds\
resembles the calculation of the antiproton flux, with some important changes
due to other mechanisms of energy loss with different energy dependence.  Due
to the fact that energy losses for positrons are more rapid than for
antiprotons, the computed signal is less sensitive to the global structure of
the dark matter halo.  (On the other hand, it is more sensitive to possible
local sources of background, such as supernova remnants etc.)

The calculation of the neutralino-induced positron flux performed in \ds\
follows the analysis in \cite{baltz}.  For continuum positrons, we have again
simulated the decay and/or hadronization with the Lund Monte Carlo {\sc Pythia}
as described in section \ref{sec:mcsim}.
%For any given MSSM model, the positron spectrum is
%\begin{equation}
%  \frac{d\phi}{dE} =
%  \left. \frac{d\phi}{dE} \right|_{\rm cont.} +
%  \left. \frac{d\phi}{dE} \right|_{\rm line}
%  = \sum_{F \neq e^{+}e^{-}} B_{F}
%  \left. \frac{d\phi}{dE} \right|_{F} + B_{e^{+}e^{-}}\,
%  \delta(E-m_\chi),
%\end{equation}
%in units of $e^{+}/{\rm annihilation}$ where $B_{F}$ is the branching ratio
%into a given final state $F$ and $\left.d\phi/dE\right|_{F}$ is the
%spectrum of positrons from annihilation channel $F$\@.
We have included all two-body final states in \ds\ (except the three lightest
quarks which are completely negligible) at tree level and the $Z\gamma$
\cite{bub} and $gg$ \cite{lp} final states which arise at one-loop level.


%%%%%%%%%%%%%%%%%%%%%%%%%%%%%%%%%%%%%%%%%%%%%%%%%%%%%%%%%%%%%%%%%%%%%%
\subsection{Propagation and the interstellar flux}

We consider a standard diffusion model, somewhat less
sophisticated than in the case of antiprotons, for the propagation of
positrons in the galaxy.  Charged particles move under the influence
of the galactic magnetic field.  For the relevant energies
  the magnetic gyroradii of the particles are quite small.
However, the magnetic field is tangled, and even with small gyroradii,
particles can jump to nearby field lines which will drastically alter
their courses.  This entire process can be modeled as a random walk,
which can be described by a diffusion equation.

Positron propagation is complicated by the fact that light particles lose
energy quickly due to inverse Compton and synchrotron processes.  Diffuse
starlight and the Cosmic Microwave Background (CMB) both contribute appreciably
to the energy loss rate of high energy electrons and positrons via inverse
Compton scattering.  Electrons and positrons also lose energy by synchrotron
radiation as they spiral around the galactic magnetic field lines.

Our detailed treatment of positron diffusion employed in \ds\ is as follows.
First define a dimensionless energy variable $\varepsilon=E/(1\;{\rm GeV})$,
and the dimensionless mass $\mct=m_\chi/(1\;{\rm GeV})$\@.  The
standard diffusion-loss equation for the space density of cosmic rays per unit
energy, $dn/d\varepsilon$, is then given by
\begin{equation}
   \frac{\partial}{\partial t}\frac{dn}{d\varepsilon}=\vec{\nabla}\cdot
   \left[K(\varepsilon,\vec{x})\vec{\nabla}\frac{dn}{d\varepsilon}\right]+
   \frac{\partial}{\partial \varepsilon}\left[b(\varepsilon,\vec{x})
   \frac{dn}{d\varepsilon}\right]+Q(\varepsilon,\vec{x}),
   \label{eq:diffloss}
\end{equation}
where $K$ is the diffusion constant, $b$ is the energy loss rate and $Q$ is the
source term.  We consider only steady state solutions, setting the left hand
side of Eq.\ (\ref{eq:diffloss}) to zero.

We assume that the diffusion constant $K$ is constant in space throughout a
``diffusion zone'', but it may vary with energy.  At energies above a few GeV,
we can represent the diffusion constant as a power law in energy \cite{wlg},
\begin{equation}
   K(\varepsilon)=K_0\varepsilon^\alpha\approx
   3\times 10^{27}\varepsilon^{0.6}{\rm cm}^2\;{\rm s}^{-1}.
   \label{eq:KA}
\end{equation}
However, at energies below about 3 GeV, there is a cutoff in the diffusion
constant that can be modeled as
\begin{equation}
   K(\varepsilon)=K_0\left[C+\varepsilon^\alpha\right]\approx
   3\times 10^{27}\left[3^{0.6}+\varepsilon^{0.6}\right]
   {\rm cm}^2\;{\rm s}^{-1}.
   \label{eq:KB}
\end{equation}
Both of these models for the diffusion constant can be used in \ds\ but the
second expression is the default.\footnote{In fact, a third option can be
   chosen in \ds\ as well, employing the propagation model of \cite{kamturner}.}
The function $b(\varepsilon)$ represents the (time) rate of energy loss.  We
allow energy loss via synchrotron emission and inverse Compton scattering.  The
rms magnetic field in the diffusion zone is about $3~\mu$G, an energy density
of about 0.2 eV cm$^{-3}$\@. We allow inverse Compton scattering on both the
cosmic microwave background and diffuse starlight, which have energy densities
of 0.3 and 0.6 eV cm$^{-3}$ respectively.  These two processes combined give an
energy loss rate \cite{energy-loss} \def\erf{\mathop{\rm erf}}
\begin{equation}
   b(\varepsilon)_{e^\pm}=\frac{1}{\taue}\varepsilon^2
   \approx 10^{-16}\varepsilon^2\;{\rm s}^{-1},
   \label{eq:bofe}
\end{equation}
where we have neglected the space dependence of the energy loss rate.  Lastly,
the function $Q$ is the source of positrons in units of cm$^{-3}$ s$^{-1}$\@.

We model the diffusion zone as a slab of thickness $2L$\@.  We fix $L$ to be 3
kpc, which fits observations of the cosmic ray flux \cite{wlg}.  We impose free
escape boundary conditions, namely that the cosmic ray density drops to zero on
the surfaces of the slab, which we let be the planes $z=\pm L$\@.  We neglect
the radial boundary usually considered in diffusion models.  This is justified
when the sources of cosmic rays are nearer than the boundary, as is usually the
case with galactic sources.  We will see that the positron flux at Earth,
especially at higher energies, mostly originates within a few kpc and hence
this approximation is well justified in our case. (This is different from the
case of antiprotons, where the flux from the Galactic center can be very
important at the Earth's location \cite{pbar}.)
%From \cite{baltz} we find the solution to the diffusion equation
%\begin{equation}
%  \frac{dn}{d\varepsilon}=\frac{\taue}{\varepsilon^2}\int
%  ^\infty_\varepsilon d\varepsilon'\,\int d^3\vec{x}\,'\,G_{2L}
%  \left(v(\varepsilon)-v(\varepsilon'),\vec{x}-\vec{x}\,'\right)
%  Q(\varepsilon',\vec{x}\,').
%  \label{eq:dnde}
%\end{equation}
%where the function $G_{2L}$ is the Green's function for a slab with thickness
%$2L$, satisfying the free escape boundary conditions\@.
%The free space Green's function is given by
%\begin{equation}
%G_{\rm free}(v-v',\vec{x}-\vec{x}\,')=\left(4\pi K_0\taue\deltav\right)^{-3/2}
%\exp\left(-\frac{\left(\vec{x}-\vec{x}\,'\right)^2}{4K_0\taue\deltav}\right)
%\theta(\deltav).
%\end{equation}
%With image charges $x_n=x$, $y_n=y$, $z_n=2Ln+(-1)^nz$, the full Green's
%function is
%\begin{equation}
%G_{2L}(v-v',\vec{x}-\vec{x}\,')=\sum_{n=-\infty}^\infty
%(-1)^nG_{\rm free}(v-v',\vec{x}-\vec{x}_n').
%\end{equation}
The spatial part of the Green's function is performed once, independently of
the supersymmetric model, yielding an energy dependent diffusion time
\begin{equation}
\taud(\varepsilon,\varepsilon')=\frac{1}{4K_0\deltav}
\sum^\infty_{n=-\infty}\sum_\pm\erf\left(\frac{(-1)^nL+2Ln\pm z}
{\sqrt{4K_0\taue\deltav}}\right)\times\hspace{1in}
\end{equation}
\begin{displaymath}
\int_0^\infty dr'\,r'f(r')
I_0\left(\frac{2rr'}{4K_0\taue\deltav}\right)
\exp\left(\frac{r^2+r'^2}{4K_0\taue\deltav}\right)
\theta(\deltav),
\end{displaymath}
where $f(r)$ is the effective halo profile squared, and the expression is
evaluated for $r$ and $z$ appropriate for the observer.  The function
$v(\varepsilon)$ depends on the diffusion model: the default model has
$v(\varepsilon)=C/\varepsilon+\varepsilon^{\alpha-1}/(1-\alpha)$.  The function
$\taud$ is the effective diffusion time for particles emitted at energy
$\varepsilon'$ and observed at energy $\varepsilon$.  Of course if the
observed energy is larger than the emitted energy, $\taud=0$.  The spatial
integrand is smooth, and is computed for a range of values, equally spaced in
$\log(\deltav)$\@ for use in \ds.  Likewise, the series of image charges used
in the Green's function converges rapidly, and with the range of
$\deltav$ values we are concerned with, need not be taken past $n=\pm 10$\@.
The total positron spectrum is now given by
\begin{equation}
   \frac{dn}{d\varepsilon}=n_0^2\sigv\frac{1}{\varepsilon^2}\Bigg\{B_{\rm
   line}\taud
   \left(\varepsilon,\mct\right)+\int_{\varepsilon}^{\mct}d\varepsilon'\,
   \left.\frac{d\phi}{d\varepsilon}\right|_{\rm
   cont.}\taud(\varepsilon,\varepsilon') \Bigg\},
\label{eq:dndefinal}
\end{equation}
where $B_{\rm line}$ is the branching ratio directly to $e^+e^-$, and
$d\phi/d\varepsilon|_{\rm cont.}$ is the spectrum of continuum positrons per
annihilation.  Remembering that this is an expression for the number density of
positrons, the flux is given by
\begin{equation}
   \frac{d\Phi}{d\varepsilon} = \frac{\beta c}{4 \pi} \frac{dn}{d\varepsilon}
   \simeq \frac{c}{4 \pi} \frac{dn}{d\varepsilon},
\end{equation}
where $\beta c$ is the velocity of a positron of energy
$\varepsilon$\@.  For the energies we are interested in, $\beta c
\simeq c$ is a very well justified approximation.

%The integral is performed in the variable $v$\@ (see \cite{baltz} 
%for details).
%Replacing the Jacobian factor $w(v)=-d\varepsilon/dv$, one finds
%\begin{equation}
%\frac{dn}{d\varepsilon}=n_0^2\sigv\frac{1}{\varepsilon^2}\Bigg\{B_{\rm line}
%\taud\left(\varepsilon,\mct\right)+\int_{v(\mct)}^{v(\varepsilon)}dv'\,
%w(v')\left.\frac{d\phi}{d\varepsilon}\right|_{\rm cont.}
%\left(\varepsilon(v')\right)
%\taud\left(\varepsilon,\varepsilon(v')\right)\Bigg\}.
%\end{equation}


%%%%%%%%%%%%%%%%%%%%%%%%%%%%%%
\subsection{Solar modulation}

Again there is a complication in that interactions with the solar wind and
magnetosphere, solar modulation, alter the spectrum.  This can be neglected at
high energies, but at energies below about 10 GeV, the effects of solar
modulation become important. However, its effects can be reduced by considering
the positron fraction, $e^+/(e^+ + e^-)$, instead of the absolute positron
fluxes. This is possible to obtain from \ds, since included in the package is
an estimate of the background $e^+$ and $e^-$ flux taken from
\cite{MoskStrong98}.



%%%%%%%%%%%%%%%%%%%%%%%%%%%%%%%%%%%%%%%%%%%%%%%%%%%%%%%%%%%%%%%%%%%%

\section{Positrons from the halo --  routines}

................

\section{Routine headers -- fortran files}

%%%%% routine dsembg.f %%%%%
\begin{routine}{dsembg.f}
\begin{verbatim}
**********************************************************************
*** function dsembg gives the differential flux of electrons from the
*** halo coming from primary and secondary (background) sources.
*** these background fluxes are a parameterization by j. edsjo to
*** the results of moskalenko & strong for a model without
*** reacceleration (08-005). ref.: apj 493 (1998) 694.
*** input: positron energy in gev.
*** output: flux in cm^-2 sec^-1 gev^-1 sr^-1
*** author: joakim edsjo, edsjo@physto.se
*** date: 98-07-21
**********************************************************************

      real*8 function dsembg(egev)
\end{verbatim}
 \end{routine}

%%%%% routine dsepbg.f %%%%%
\begin{routine}{dsepbg.f}
\begin{verbatim}
**********************************************************************
*** function dsepbg gives the differential flux of positrons from the
*** halo coming from secondary sources.
*** these background fluxes are a parameterization by j. edsjo to
*** the results of moskalenko & strong for a model without
*** reacceleration (08-005). ref.: apj 493 (1998) 694.
*** input: positron energy in gev.
*** output: flux in cm^-2 sec^-1 gev^-1 sr^-1
*** author: joakim edsjo, edsjo@physto.se
*** date: 98-07-21
**********************************************************************

      real*8 function dsepbg(egev)
\end{verbatim}
 \end{routine}

%%%%% routine dsepdiff.f %%%%%
\begin{routine}{dsepdiff.f}
\begin{verbatim}
**********************************************************************
*** function dsepdiff calculates the differential flux of
*** positrons for the energy egev as a result of
*** neutralino annihilation in the halo.
*** input: egev - positron energy in gev
***        how = 1 - dsepsigvdnde is used directly
***              2 - dsepsigvdnde is tabulated on first call, and then
***                  interpolated. (default)
***       dhow = 2 - diffusion model is tabulated on first call, and then
***                  interpolated
***              3 - as 2, but also write the table to disk at the
***                  first call
***              4 - read table from disk on first call, and use that for
***                  subsequent calls. If the file does not exist, it will
***                  be created (as in 3). (default)
*** units: gev^-1 cm^-2 sec^-1 sr^-1
*** author: e.a. baltz (eabaltz@astron.berkeley.edu)
***         joakim edsjo, edsjo@physto.se
*** date: jun-02-98
*** modified: 99-07-02 paolo gondolo : order of calls to hrsetup
*** modified: 01-10-19 add moskalenko + strong option (eab)
*** modified: 04-01-27 J. Edsjo: added file handling (dhow=3,4)
**********************************************************************

      real*8 function dsepdiff(egev,how,dhow)
\end{verbatim}
 \end{routine}

%%%%% routine dsepdsigv_de.f %%%%%
\begin{routine}{dsepdsigv\_de.f}
\begin{verbatim}
************************************************************************
*** positron propagation routines.
*** author: e.a. baltz (eabaltz@astron.berkeley.edu)
*** modified slightly by joakim edsjo (edsjo@physto.se)
*** date: jun-02-98
*** modified: jun-09-98
************************************************************************


c     this is a power law e^-2
c     this function is [d<sigma velocity>/de](v)
      real*8 function dsepdsigv_de(v,vmin)
\end{verbatim}
 \end{routine}

%%%%% routine dsepeecut.f %%%%%
\begin{routine}{dsepeecut.f}
\begin{verbatim}
************************************************************************
*** positron propagation routines.
*** author: e.a. baltz (eabaltz@astron.berkeley.edu)
*** modified slightly by joakim edsjo (edsjo@physto.se)
*** date: jun-02-98
*** modified: jun-09-98
***           jul-06-99 paolo gondolo - calls to dshunt, ee, vv
************************************************************************


************************************************************************
      real*8 function dsepeecut(v,tabindx)
************************************************************************
\end{verbatim}
 \end{routine}

%%%%% routine dsepeeuncut.f %%%%%
\begin{routine}{dsepeeuncut.f}
\begin{verbatim}
************************************************************************
*** positron propagation routines.
*** author: e.a. baltz (eabaltz@astron.berkeley.edu)
*** modified slightly by joakim edsjo (edsjo@physto.se)
*** date: jun-02-98
*** modified: jun-09-98
************************************************************************


************************************************************************
      real*8 function dsepeeuncut(v)
************************************************************************
\end{verbatim}
 \end{routine}

%%%%% routine dsepf.f %%%%%
\begin{routine}{dsepf.f}
\begin{verbatim}
************************************************************************
*** This is the average of the halo density squared from
*** z=-l_h to z=+l_h.
*** It is the function f(r) given in Eq. (20) in Baltz & Edsjo, 
*** PRD 59(1999)023511, except that g(r) here is the NORMALIZED
*** halo density. Consequently, n_c=1.
*** Author: Joakim Edsjo, edsjo@physto.se
*** Date: 2000-09-03
************************************************************************

************************************************************************
      real*8 function dsepf(r)
************************************************************************
\end{verbatim}
 \end{routine}

%%%%% routine dsepfrsm.f %%%%%
\begin{routine}{dsepfrsm.f}
\begin{verbatim}
***********************************************************************
*** real*8 function dsepfrsm solar modulates the positron fraction at a
*** given energy (eep). only gives results for a+ cycle.
*** input:  epfr - interstellar solar modulation fraction
***         eep - positron energy in gev
***         qa - solar modulation cycle. >0 for positrons in a+ cycle and
***             <0 for a- cycle.
*** output: dsepfrsm - solar modulated positron fraction
*** ref: clem et al, apj 464 (1997) 507.
*** author: joakim edsjo, edsjo@physto.se
***********************************************************************

      real*8 function dsepfrsm(epfr,eep,qa)
\end{verbatim}
 \end{routine}

%%%%% routine dsepgalpropdiff.f %%%%%
\begin{routine}{dsepgalpropdiff.f}
\begin{verbatim}
**********************************************************************
*** function dsepgalpropdiff calculates the differential flux of
*** positrons for the energy egev as a result of
*** neutralino annihilation in the halo.
*** units: gev^-1 cm^-2 sec^-1 sr^-1
*** author: edward baltz (eabaltz@alum.mit.edu), joakim edsjo
*** date: 4/28/2006
**********************************************************************

      real*8 function dsepgalpropdiff(egev)
\end{verbatim}
 \end{routine}

%%%%% routine dsepgalpropig.f %%%%%
\begin{routine}{dsepgalpropig.f}
\begin{verbatim}
      real*8 function dsepgalpropig(eep)
No header found.
\end{verbatim}
 \end{routine}

%%%%% routine dsepgalpropig2.f %%%%%
\begin{routine}{dsepgalpropig2.f}
\begin{verbatim}
      real*8 function dsepgalpropig2(eep)
No header found.
\end{verbatim}
 \end{routine}

%%%%% routine dsepgalpropline.f %%%%%
\begin{routine}{dsepgalpropline.f}
\begin{verbatim}
**********************************************************************
*** function dsepgalpropline calculates the flux of e+ from the line
*** annihilation, from GALPROP
*** units: gev^-1 cm^-2 sec^-1 sr^-1
*** author: joakim edsjo, edsjo@physto.se,
***   e.a. baltz, eabaltz@alum.mit.edu
*** date: 4/27/2006
*** Modified: Joakim Edsjo (edsjo@physto.se) 03-01-21, factor of 1/2
***           in annihilation rate added
**********************************************************************

      real*8 function dsepgalpropline(egev)
\end{verbatim}
 \end{routine}

%%%%% routine dsephalodens2.f %%%%%
\begin{routine}{dsephalodens2.f}
\begin{verbatim}
************************************************************************
*** Halo density squared. This function is a function of z and calls
*** the standard halo density routine dshmrho.
*** Author: Joakim Edsjo, edsjo@physto.se
*** Date: 2000-09-03
************************************************************************

************************************************************************
      real*8 function dsephalodens2(z)
************************************************************************
\end{verbatim}
 \end{routine}

%%%%% routine dsepideltavint.f %%%%%
\begin{routine}{dsepideltavint.f}
\begin{verbatim}
************************************************************************
*** positron propagation routines.
*** the integrand in the integration for i(delta v).
*** To speed up the integration, u=ln(r) is used as an integration variable
*** Author: J. Edsjo (edsjo@physto.se), based on routines by
*** e.a. baltz (eabaltz@astron.berkeley.edu)
*** Date: 2004-01-26
************************************************************************

************************************************************************
      real*8 function dsepideltavint(u)
************************************************************************
\end{verbatim}
 \end{routine}

%%%%% routine dsepimage_sum.f %%%%%
\begin{routine}{dsepimage\_sum.f}
\begin{verbatim}
************************************************************************
*** positron propagation routines.
*** author: e.a. baltz (eabaltz@astron.berkeley.edu)
*** modified slightly by joakim edsjo (edsjo@physto.se)
*** date: jun-02-98
*** modified: jun-09-98, dec-04-02 (eb)
************************************************************************


************************************************************************
      real*8 function dsepimage_sum(deltav)
************************************************************************
\end{verbatim}
 \end{routine}

%%%%% routine dsepipol.f %%%%%
\begin{routine}{dsepipol.f}
\begin{verbatim}
**********************************************************************
*** function dsepipol interpolates in the table of
*** <sigma v> dn/de to speed up the
*** positron flux routines.
*** input: eep - positron energy
*** output: <sigma v> dn/de in units of cm^3 s^-1 gev^-1
*** author: joakim edsjo, edsjo@physto.se
*** date: jun-02-98
**********************************************************************

      real*8 function dsepipol(eep)
\end{verbatim}
 \end{routine}

%%%%% routine dsepkt.f %%%%%
\begin{routine}{dsepkt.f}
\begin{verbatim}
**********************************************************************
*** function dsepkt calculates the integrated flux of positrons
*** between energy ea and eb from neutralino annihilation in the halo.
*** NOTE. This routine uses the Kamionkowski and Turner expressions.
*** from prd 43(1991)1774. Only included for comparison.
*** units: cm^-2 sec^-1 sr^-1
*** author: joakim edsjo, edsjo@physto.se
*** date: 98-02-10
**********************************************************************

      real*8 function dsepkt(ea,eb,istat)
\end{verbatim}
 \end{routine}

%%%%% routine dsepktdiff.f %%%%%
\begin{routine}{dsepktdiff.f}
\begin{verbatim}
**********************************************************************
*** function dsepktdiff calculates the differential flux of
*** positrons for the energy egev as a result of
*** neutralino annihilation in the halo.
*** NOTE. This routine uses the Kamionkowski and Turner expressions.
*** from prd 43(1991)1774. Only included for comparison.
*** units: gev^-1 cm^-2 sec^-1 sr^-1
*** author: joakim edsjo, edsjo@physto.se
*** date: 98-02-10
**********************************************************************

      real*8 function dsepktdiff(egev)
\end{verbatim}
 \end{routine}

%%%%% routine dsepktig.f %%%%%
\begin{routine}{dsepktig.f}
\begin{verbatim}
**********************************************************************
*** function dsepktig is the positron spectrum times the greens
*** function from kamionkowski & turner, prd 43(1991)1774.
*** this routine is integrated by dsepktdiff to give the differential
*** positron flux at earth.
*** units: gev^-2 cm^-2 sec^-1 sr^-1
*** author: joakim edsjo, edsjo@physto.se
*** date: 98-02-10
*** Modified: Joakim Edsjo (edsjo@physto.se) 03-01-21, factor of 1/2
***           in annihilation rate added
**********************************************************************

      real*8 function dsepktig(eep)
\end{verbatim}
 \end{routine}

%%%%% routine dsepktig2.f %%%%%
\begin{routine}{dsepktig2.f}
\begin{verbatim}
**********************************************************************
*** function dsepktig2 is the positron spectrum times the greens
*** function from kamionkowski & turner, prd 43(1991)1774.
*** this routine is integrated by dsepktdiff to give the differential
*** positron flux at earth. the independent variable for this
*** routine is x=1/e**2 instead of e as in dsepktig.
*** units: gev^-2 cm^-2 sec^-1 sr^-1
*** author: joakim edsjo, edsjo@physto.se
*** date: 98-02-10
*** Modified: Joakim Edsjo (edsjo@physto.se) 03-01-21, factor of 1/2
***           in annihilation rate added
**********************************************************************

      real*8 function dsepktig2(x)
\end{verbatim}
 \end{routine}

%%%%% routine dsepktline.f %%%%%
\begin{routine}{dsepktline.f}
\begin{verbatim}
**********************************************************************
*** function dsepktline calculates the flux of e+ from the line
*** annihilation.
*** from kamionkowski & turner, prd 43(1991)1774.
*** units: gev^-1 cm^-2 sec^-1 sr^-1
*** author: joakim edsjo, edsjo@physto.se
*** date: 98-02-10
*** Modified: Joakim Edsjo (edsjo@physto.se) 03-01-21, factor of 1/2
***           in annihilation rate added
**********************************************************************

      real*8 function dsepktline(egev)
\end{verbatim}
 \end{routine}

%%%%% routine dseploghalodens2.f %%%%%
\begin{routine}{dseploghalodens2.f}
\begin{verbatim}
************************************************************************
*** Halo density squared. This function is a function of log(z) and calls
*** the standard halo density routine dshmrho.
*** The jacobian z for integration is also included
*** u = log(z)
*** Author: Joakim Edsjo, edsjo@physto.se
*** Date: 2000-09-03
************************************************************************

************************************************************************
      real*8 function dseploghalodens2(u)
************************************************************************
\end{verbatim}
 \end{routine}

%%%%% routine dsepmake_tables.f %%%%%
\begin{routine}{dsepmake\_tables.f}
\begin{verbatim}
************************************************************************
*** positron propagation routines.
*** author: e.a. baltz (eabaltz@astron.berkeley.edu)
*** modified slightly by joakim edsjo (edsjo@physto.se)
*** date: jun-02-98
*** modified: jun-09-98, 2002-11-19
***   2004-01-26 (better r integration)
************************************************************************


************************************************************************
      subroutine dsepmake_tables
***   creates a table of i(delta v) versus delta v.
************************************************************************
\end{verbatim}
 \end{routine}

%%%%% routine dsepmake_tables2.f %%%%%
\begin{routine}{dsepmake\_tables2.f}
\begin{verbatim}
************************************************************************
*** positron propagation routines.
*** author: e.a. baltz (eabaltz@astron.berkeley.edu)
*** modified slightly by joakim edsjo (edsjo@physto.se)
*** date: jun-02-98
*** modified: jun-09-98, 2002-11-19
***   2004-01-26 (better r integration)
************************************************************************


************************************************************************
      subroutine dsepmake_tables2
*** creates auxiliary tables
************************************************************************
\end{verbatim}
 \end{routine}

%%%%% routine dsepmsdiff.f %%%%%
\begin{routine}{dsepmsdiff.f}
\begin{verbatim}
**********************************************************************
*** function dsepmsdiff calculates the differential flux of
*** positrons for the energy egev as a result of
*** neutralino annihilation in the halo.
*** NOTE. This routine uses the Moskaleno and Strong expressions
*** from PRD 60 (1999) 063003 for z_h=4 kpc, isothermal halo
*** units: gev^-1 cm^-2 sec^-1 sr^-1
*** author: edward baltz (eabaltz@alum.mit.edu), joakim edsjo
*** date: 10/18/2001
**********************************************************************

      real*8 function dsepmsdiff(egev)
\end{verbatim}
 \end{routine}

%%%%% routine dsepmsig.f %%%%%
\begin{routine}{dsepmsig.f}
\begin{verbatim}
**********************************************************************
*** function dsepmsig is the positron spectrum times the greens
*** function from moskalenko and strong prd 60, 063003 (1999)
*** this routine is integrated by dsepmsdiff to give the differential
*** positron flux at earth.
*** units: gev^-2 cm^-2 sec^-1 sr^-1
*** author: joakim edsjo, edsjo@physto.se,
***   edward baltz eabaltz@alum.mit.edu
*** date: 2001 10/18
*** Modified: Joakim Edsjo (edsjo@physto.se) 03-01-21, factor of 1/2
***           in annihilation rate added
**********************************************************************

      real*8 function dsepmsig(eep)
\end{verbatim}
 \end{routine}

%%%%% routine dsepmsig2.f %%%%%
\begin{routine}{dsepmsig2.f}
\begin{verbatim}
**********************************************************************
*** function dsepmsig2 is the positron spectrum times the greens
*** function from moskalenko and strong, prd 60(1999)063003.
*** this routine is integrated by dsepmsdiff to give the differential
*** positron flux at earth. the independent variable for this
*** routine is x=1/e**2 instead of e as in dsepmsig.
*** units: gev^-2 cm^-2 sec^-1 sr^-1
*** author: joakim edsjo, edsjo@physto.se,
***   e.a. baltz, eabaltz@alum.mit.edu
*** date: 01-10-18
*** Modified: Joakim Edsjo (edsjo@physto.se) 03-01-21, factor of 1/2
***           in annihilation rate added
**********************************************************************

      real*8 function dsepmsig2(x)
\end{verbatim}
 \end{routine}

%%%%% routine dsepmsline.f %%%%%
\begin{routine}{dsepmsline.f}
\begin{verbatim}
**********************************************************************
*** function dsepmsline calculates the flux of e+ from the line
*** annihilation.
*** from moskaleno & strong, prd 60(1999)063003.
*** units: gev^-1 cm^-2 sec^-1 sr^-1
*** author: joakim edsjo, edsjo@physto.se,
***   e.a. baltz, eabaltz@alum.mit.edu
*** date: 01-10-19
*** Modified: Joakim Edsjo (edsjo@physto.se) 03-01-21, factor of 1/2
***           in annihilation rate added
**********************************************************************

      real*8 function dsepmsline(egev)
\end{verbatim}
 \end{routine}

%%%%% routine dsepmstable.f %%%%%
\begin{routine}{dsepmstable.f}
\begin{verbatim}
************************************************************************
*** positron propagation routines.
*** author: e.a. baltz eabaltz@alum.mit.edu
*** date: 2001 10/18
************************************************************************


************************************************************************
      subroutine dsepmstable(eep,aa,bb,cc,ww,xx,yy)
************************************************************************
\end{verbatim}
 \end{routine}

%%%%% routine dseprsm.f %%%%%
\begin{routine}{dseprsm.f}
\begin{verbatim}
***********************************************************************
*** real function dseprsm gives the ratio of electro+positron flux in an
*** a+ cycle to that in an a- cycle as a function of positron energy
*** in gev.
*** from clem et al, apj 464 (1996) 507.
*** author: joakim edsjo, edsjo@physto.se
***********************************************************************

      real*8 function dseprsm(eep)
\end{verbatim}
 \end{routine}

%%%%% routine dsepset.f %%%%%
\begin{routine}{dsepset.f}
\begin{verbatim}
      subroutine dsepset(c)
c...set parameters for positron routines
c...  c - character string specifying choice to be made
c...author: joakim edsjo, 2000-07-09
c...modified: paolo gondolo, 2000-07-19
c...          joakim edsjo,  2000-08-15
\end{verbatim}
 \end{routine}

%%%%% routine dsepsigvdnde.f %%%%%
\begin{routine}{dsepsigvdnde.f}
\begin{verbatim}
**********************************************************************
*** function dsepsigvdnde gives the differential spectrum of positrons
*** when they are created.
*** input: eep in gev
*** output: <sigma v> dn/de in units of cm^3 s^-1 gev^-1
*** author: joakim edsjo, edsjo@physto.se
*** date: jun-01-98
*** modified: 99-07-02 paolo gondolo
**********************************************************************

      real*8 function dsepsigvdnde(eep)
\end{verbatim}
 \end{routine}

%%%%% routine dsepspec.f %%%%%
\begin{routine}{dsepspec.f}
\begin{verbatim}
************************************************************************
*** positron propagation routines.
*** author: e.a. baltz (eabaltz@astron.berkeley.edu)
*** modified slightly by joakim edsjo (edsjo@physto.se)
*** date: jun-02-98
*** modified: jun-09-98
***           jul-06-99 paolo gondolo - calls to dshunt, ee, vv
*** Modified: Joakim Edsjo (edsjo@physto.se) 03-01-21, factor of 1/2
***           in annihilation rate added
************************************************************************



************************************************************************
*** real*8 function dsepspec calculates the differential positron
*** spectrum from neutralino annihilation in the halo.
*** input:  e - positron kinetic energy in gev
***         mchi - neutralino mass in gev
***         sigvline - <sigma v>_e+e- in cm^3 s^-1
***         sigvdnde(eep) - <sigma v> dn/de in cm^3 s^-1 gev^-1
***         ee - r8 function that gives energy as a fcn of v
***         vv - r8 function that gives v as function of energy
***         metric - r8 function that gives metric as fcn of v
*** output: spectrum in units of cm^-2 s^-1 sr^-1 gev^-1
************************************************************************

      real*8 function dsepspec(e,mchi,sigvline,sigvdnde,ee,vv,metric)
\end{verbatim}
 \end{routine}

%%%%% routine dseptab.f %%%%%
\begin{routine}{dseptab.f}
\begin{verbatim}
**********************************************************************
*** subroutine dseptab tabulates <sigma v> dn/de to speed up the
*** positron flux routines. the interpolation is done with dsepipol.
*** input: emin - minimal energy for table
***   npts - number of points for the table
*** author: joakim edsjo, edsjo@physto.se
*** date: jun-02-98
**********************************************************************

      subroutine dseptab(em,npts)
\end{verbatim}
 \end{routine}

%%%%% routine dsepvvcut.f %%%%%
\begin{routine}{dsepvvcut.f}
\begin{verbatim}
************************************************************************
*** positron propagation routines.
*** author: e.a. baltz (eabaltz@astron.berkeley.edu)
*** modified slightly by joakim edsjo (edsjo@physto.se)
*** date: jun-02-98
*** modified: jun-09-98
************************************************************************


************************************************************************
      real*8 function dsepvvcut(e)
************************************************************************
\end{verbatim}
 \end{routine}

%%%%% routine dsepvvuncut.f %%%%%
\begin{routine}{dsepvvuncut.f}
\begin{verbatim}
************************************************************************
*** positron propagation routines.
*** author: e.a. baltz (eabaltz@astron.berkeley.edu)
*** modified slightly by joakim edsjo (edsjo@physto.se)
*** date: jun-02-98
*** modified: jun-09-98
************************************************************************


************************************************************************
      real*8 function dsepvvuncut(e)
************************************************************************
\end{verbatim}
 \end{routine}

%%%%% routine dsepwcut.f %%%%%
\begin{routine}{dsepwcut.f}
\begin{verbatim}
************************************************************************
*** positron propagation routines.
*** author: e.a. baltz (eabaltz@astron.berkeley.edu)
*** modified slightly by joakim edsjo (edsjo@physto.se)
*** date: jun-02-98
*** modified: jun-09-98
***           jul-06-99 paolo gondolo - calls to dshunt, ee, vv
************************************************************************


************************************************************************
      real*8 function dsepwcut(v,tabindx)
************************************************************************
\end{verbatim}
 \end{routine}

%%%%% routine dsepwuncut.f %%%%%
\begin{routine}{dsepwuncut.f}
\begin{verbatim}
************************************************************************
*** positron propagation routines.
*** author: e.a. baltz (eabaltz@astron.berkeley.edu)
*** modified slightly by joakim edsjo (edsjo@physto.se)
*** date: jun-02-98
*** modified: jun-09-98
************************************************************************


************************************************************************
      real*8 function dsepwuncut(v)
************************************************************************
\end{verbatim}
 \end{routine}

%%%%% routine dsgalpropig.f %%%%%
\begin{routine}{dsgalpropig.f}
\begin{verbatim}
**********************************************************************
*** function dsgalpropig is the positron spectrum times the greens
*** function from galprop
*** this routine is integrated by dsepgalpropdiff to give the differential
*** positron flux at earth.
*** units: gev^-2 cm^-2 sec^-1 sr^-1
*** author: joakim edsjo, edsjo@physto.se,
***   edward baltz eabaltz@alum.mit.edu
*** date: 2006 4/27
*** Modified: Joakim Edsjo (edsjo@physto.se) 03-01-21, factor of 1/2
***           in annihilation rate added
**********************************************************************

      real*8 function dsgalpropig(eep,pbar)
\end{verbatim}
 \end{routine}

%%%%% routine dsgalpropig2.f %%%%%
\begin{routine}{dsgalpropig2.f}
\begin{verbatim}
**********************************************************************
*** function dsgalpropig2 is the positron spectrum times the greens
*** function from galprop
*** this routine is integrated by dsepgalpropdiff to give the differential
*** positron flux at earth.  the independent variable for this
*** routine is x=1/e**2 instead of e as in dsepgalpropig.
*** units: gev^-2 cm^-2 sec^-1 sr^-1
*** author: joakim edsjo, edsjo@physto.se,
***   edward baltz eabaltz@alum.mit.edu
*** date: 2006 4/27
*** Modified: Joakim Edsjo (edsjo@physto.se) 03-01-21, factor of 1/2
***           in annihilation rate added
**********************************************************************

      real*8 function dsgalpropig2(x,pbar)
\end{verbatim}
 \end{routine}

%%%%% routine dsgalpropset.f %%%%%
\begin{routine}{dsgalpropset.f}
\begin{verbatim}
      subroutine dsgalpropset(c)
c...set parameters for positron routines
c...  c - character string specifying choice to be made
c...author: joakim edsjo, 2006-02-21
\end{verbatim}
 \end{routine}

\newpage
\chapter[ep2: Positron fluxes from the halo (alternative solution)]{\codeb{src/ep2}:\\ Positron fluxes from the halo (alternative solution)}
\label{ch:src-ep2}

\section{Routine headers -- fortran files}

%%%%% routine dsepintgreen.f %%%%%
\begin{routine}{dsepintgreen.f}
\begin{verbatim}
      real*8 function dsepintgreen(DeltaV)
****************************************************************
***                                                          ***
*** function which gives the integral over volume of the     ***
*** positron green function times dsephaloterm (which is the ***
*** square of the density normalized to the local halo       ***
*** density for a smooth halo profile, i.e. for hclumpy = 1, ***
*** and the density probability of clumps normalized to the  ***
*** local halo density for a clumpy halo, i.e. hclumpy = 2)  ***
*** as a function of:                                        ***
***    DeltaV = 4 * K0 * tau_E * deltav in units of kpc**2   ***
*** i.e. of a given deltav = v(Eps)- v(Epsprime) this should ***
*** be called with:                                          ***
***    DeltaV=4.0d0*k27*tau16*deltav*10.d0/kpc**2            ***
*** where kpc=3.08567802d0 and the 10/kpc**2 converts from   ***
*** units of 10**43 cm**2 to units of kpc**2                 *** 
***                                                          ***
*** Author: Piero Ullio                                      ***
*** Date: 2004-02-03                                         ***
****************************************************************
\end{verbatim}
 \end{routine}

%%%%% routine dsepspecm.f %%%%%
\begin{routine}{dsepspecm.f}
\begin{verbatim}
      real*8 function dsepspecm(eps,how)
****************************************************************
***                                                          ***
*** function which computes the differential flux of         ***
*** positrons for the energy eps as a result of              ***
*** neutralino annihilation in the halo.                     ***
*** input: eps - positron energy in gev                      ***
***        how = 2 - diffusion model is tabulated on first   ***
***                  call, and then interpolated             ***
***              3 - as 2, but also write the table to disk  ***
***                  at the first call                       ***
***              4 - read table from disk on first call, and ***
***                  use the subsequent calls. If the file   ***
***                  does not exist, it will be created      ***
***                  (as in 3). (default)                    ***
***                                                          ***
*** rescaling is not included                                ***
***                                                          ***
*** Author: Piero Ullio                                      ***
*** Date: 2004-02-03                                         ***
*** Modified: Joakim Edsjo, modifications to file loading    ***
****************************************************************
\end{verbatim}
 \end{routine}

%%%%% routine dsepvofeps.f %%%%%
\begin{routine}{dsepvofeps.f}
\begin{verbatim}
****************************************************************
***                                                          ***
*** functions which give conversions between the variables   ***
*** eps - u - v  for the positron flux calculation           ***
*** the conversion between eps and v and viceversa is done   ***
*** through a tabulation, to reset this tabulation           ***
*** reinitialize the variable vofeset                        ***
***                                                          ***
*** Author: Piero Ullio                                      ***
*** Date: 2004-02-03                                         ***
****************************************************************

      real*8 function dsepuofeps(eps)
c this is the definition of the function u(eps)
c u = int_eps^epsmax depsp 1/(tau*b(epsp))
c NOTE: this assumes u=1/eps, change this when you implement the general
c formula
\end{verbatim}
 \end{routine}

\newpage
\chapter[ge: General routines]{\codeb{src/ge}:\\ General routines}
\label{ch:src-ge}

%%%%%%%%%%%%%%%%%%%%%%%%%%%%%%%%%%%%%%%%%%%%%%%%%%%%%%%%%%%%%%%%%%%%

\section{General routines}

In \codeb{ge/}, we collect routines that are of general interest to
many other routines in \ds. E.g., we have routins to find elements in
arrays (used for interpolation), Bessel functions, error functions,
spline routines, etc.
\section{Routine headers -- fortran files}

%%%%% routine cosd.f %%%%%
\begin{routine}{cosd.f}
\begin{verbatim}
      function cosd(x)
No header found.
\end{verbatim}
 \end{routine}

%%%%% routine dsabsq.f %%%%%
\begin{routine}{dsabsq.f}
\begin{verbatim}
c_______________________________________________________________________
c  abs squared of a complex*16 number.
c  called by dwdcos.
c  author: paolo gondolo (gondolo@lpthe.jussieu.fr) 1994
c=======================================================================
      function dsabsq(z)
\end{verbatim}
 \end{routine}

%%%%% routine dsbessei0.f %%%%%
\begin{routine}{dsbessei0.f}
\begin{verbatim}
************************************************************************
      real*8 function dsbessei0(x)
************************************************************************
c     exp(-|x|) i_0(x)
\end{verbatim}
 \end{routine}

%%%%% routine dsbessei1.f %%%%%
\begin{routine}{dsbessei1.f}
\begin{verbatim}
      real*8 function dsbessei1(x)
c     exp(-|x|) i1(x)
\end{verbatim}
 \end{routine}

%%%%% routine dsbessek0.f %%%%%
\begin{routine}{dsbessek0.f}
\begin{verbatim}
***********************************************************************
*** function dsbessek0 returns the value of the modified bessel       ***
*** function of the second kind of order 0 times exp(x)             ***
*** works for positive real x                                       ***
*** coefficients from abramovitz and stegun.                        ***
*** e-mail: edsjo@physto.se                                 ***
*** date: 98-04-29                                                  ***
***********************************************************************

      real*8 function dsbessek0(x)
\end{verbatim}
 \end{routine}

%%%%% routine dsbessek1.f %%%%%
\begin{routine}{dsbessek1.f}
\begin{verbatim}
***********************************************************************
*** function dsbessek1 returns the value of the modified bessel       ***
*** function of the second kind of order 1 times exp(x).            ***
*** works for positive real x                                       ***
*** coefficients from abramovitz and stegun.                        ***
*** e-mail: edsjo@physto.se                                 ***
*** date: 98-04-29                                                  ***
***********************************************************************

      real*8 function dsbessek1(x)
\end{verbatim}
 \end{routine}

%%%%% routine dsbessek2.f %%%%%
\begin{routine}{dsbessek2.f}
\begin{verbatim}
***********************************************************************
*** function bessk2 returns the value of the modified bessel        ***
*** function of the second kind of order 2 times exp(x)             ***
*** works for positive real x                                       ***
*** recurrence relation                                             ***
*** e-mail: gondolo@mppmu.mpg.de                                    ***
*** date: 00-07-07                                                  ***
***********************************************************************

      real*8 function dsbessek2(x)
\end{verbatim}
 \end{routine}

%%%%% routine dsbessjw.f %%%%%
\begin{routine}{dsbessjw.f}
\begin{verbatim}
c....Various bessel functions
c....from P. Ullio

      real*8 function dsbessjw(n,x)
\end{verbatim}
 \end{routine}

%%%%% routine dscharadd.f %%%%%
\begin{routine}{dscharadd.f}
\begin{verbatim}
***********************************************************************
*** dscharadd takes a string str, adds a string add to it
*** and returns the concatenated string with spaces removed.
*** Author: Joakim Edsjo, edsjo@physto.se
*** Date: 2004-01-19
***********************************************************************

      subroutine dscharadd(str,add)

\end{verbatim}
 \end{routine}

%%%%% routine dsf2s.f %%%%%
\begin{routine}{dsf2s.f}
\begin{verbatim}
      character*12 function dsf2s(x)
No header found.
\end{verbatim}
 \end{routine}

%%%%% routine dsf_int.f %%%%%
\begin{routine}{dsf\_int.f}
\begin{verbatim}
      real*8 function dsf_int(f,a,b,eps)
c_______________________________________________________________________
c  integrate function f between a and b
c  input
c    integration limits a and b
c  called by different routines
c  author: joakim edsjo (edsjo@physto.se) 96-05-16
c          2000-07-19 paolo gondolo added eps as argument 
c  based on paolo gondolos wxint.f routine.
c=======================================================================
\end{verbatim}
 \end{routine}

%%%%% routine dsf_int2.f %%%%%
\begin{routine}{dsf\_int2.f}
\begin{verbatim}
      real*8 function dsf_int2(f,a,b,eps)
c_______________________________________________________________________
c  integrate function f between a and b
c  input
c    integration limits a and b
c  called by different routines
c  author: joakim edsjo (edsjo@physto.se) 96-05-16
c          2000-07-19 paolo gondolo added eps as argument 
c  based on paolo gondolos wxint.f routine.
c  the same routine as dsf_int but used when double integration is needed.
c=======================================================================
\end{verbatim}
 \end{routine}

%%%%% routine dshiprecint3.f %%%%%
\begin{routine}{dshiprecint3.f}
\begin{verbatim}
      subroutine dshiprecint3(fun,lowlim,upplim,result)
No header found.
\end{verbatim}
 \end{routine}

%%%%% routine dshunt.f %%%%%
\begin{routine}{dshunt.f}
\begin{verbatim}
************************************************************************
      subroutine dshunt(xx,n,x,indx)
*** returns the lowest index indx for which x>xx(indx).
*** if x<= xx(i) it returns 0, 
*** if x>xx(n) it returns indx=n
************************************************************************
\end{verbatim}
 \end{routine}

%%%%% routine dsi2s.f %%%%%
\begin{routine}{dsi2s.f}
\begin{verbatim}
      character*8 function dsi2s(x)
No header found.
\end{verbatim}
 \end{routine}

%%%%% routine dsi_trim.f %%%%%
\begin{routine}{dsi\_trim.f}
\begin{verbatim}
      function dsi_trim(s)
No header found.
\end{verbatim}
 \end{routine}

%%%%% routine dsidtag.f %%%%%
\begin{routine}{dsidtag.f}
\begin{verbatim}
      character*12 function dsidtag()
No header found.
\end{verbatim}
 \end{routine}

%%%%% routine dsisnan.f %%%%%
\begin{routine}{dsisnan.f}
\begin{verbatim}
**********************************************************************
*** function to check if a real*8 number is NaN, returns true if it is
**********************************************************************
      logical function dsisnan(a)
\end{verbatim}
 \end{routine}

%%%%% routine dsquartic.f %%%%%
\begin{routine}{dsquartic.f}
\begin{verbatim}
      subroutine dsquartic(a3,a2,a1,a0,z1,z2,z3,z4)
c_______________________________________________________________________
c  analytic solution of z^4 + a3 z^3 + a2 z^2 + a1 z + a0 = 0
c  author: paolo gondolo (gondolo@lpthe.jussieu.fr) 1994
c=======================================================================
\end{verbatim}
 \end{routine}

%%%%% routine dsrnd1.f %%%%%
\begin{routine}{dsrnd1.f}
\begin{verbatim}
      function dsrnd1(idum)
c_______________________________________________________________________
c  uniform deviate between 0 and 1.
c  input:
c    idum - seed (integer); enter negative integer at first call
c=======================================================================
\end{verbatim}
 \end{routine}

%%%%% routine dsrndlin.f %%%%%
\begin{routine}{dsrndlin.f}
\begin{verbatim}
      function dsrndlin(idum,a,b)
No header found.
\end{verbatim}
 \end{routine}

%%%%% routine dsrndlog.f %%%%%
\begin{routine}{dsrndlog.f}
\begin{verbatim}
      function dsrndlog(idum,a,b)
No header found.
\end{verbatim}
 \end{routine}

%%%%% routine dsrndsgn.f %%%%%
\begin{routine}{dsrndsgn.f}
\begin{verbatim}
      function dsrndsgn(idum)
No header found.
\end{verbatim}
 \end{routine}

%%%%% routine dswrite.f %%%%%
\begin{routine}{dswrite.f}
\begin{verbatim}
      subroutine dswrite(level,opt,message)
c_______________________________________________________________________
c  handle writing onto standard output in darksusy
c  input:
c    level - print message if prtlevel is >= level
c    opt   - print (1) the model tag or not (0)
c    message - string containing the text to print
c  common:
c    'dsio.h' - i/o units numbers and prtlevel
c  author: paolo gondolo 1999
c=======================================================================
\end{verbatim}
 \end{routine}

%%%%% routine erf.f %%%%%
\begin{routine}{erf.f}
\begin{verbatim}
      function erf(x)
No header found.
\end{verbatim}
 \end{routine}

%%%%% routine erfc.f %%%%%
\begin{routine}{erfc.f}
\begin{verbatim}
      function erfc(x)
No header found.
\end{verbatim}
 \end{routine}

%%%%% routine sind.f %%%%%
\begin{routine}{sind.f}
\begin{verbatim}
      function sind(x)
No header found.
\end{verbatim}
 \end{routine}

%%%%% routine spline.f %%%%%
\begin{routine}{spline.f}
\begin{verbatim}
      SUBROUTINE spline(x,y,n,yp1,ypn,y2)
c spline routine, double precision
\end{verbatim}
 \end{routine}

%%%%% routine splint.f %%%%%
\begin{routine}{splint.f}
\begin{verbatim}
      SUBROUTINE splint(xa,ya,y2a,n,x,y)
c spline routine, double precision
\end{verbatim}
 \end{routine}

\newpage
\chapter[ha: Halo annihilation yields]{\codeb{src/ha}:\\ Halo annihilation yields}
\label{ch:src-ha}

%%%%%%%%%%%%%%%%%%%%%%%%%%%%%%%%%%%%%%%%%%%%%%%%%%%%%%%%%%%%%%%%%%%%%%
\section{Annihilation in the halo, yields -- theory}

Here we calculate yields from annihilation in the halo.

\subsection{Monte Carlo simulations}
\label{sec:ha-mcsim}

We need to
evaluate the yield of different particles per neutralino annihilation.
The hadronization and/or decay of the annihilation products are
simulated with {\sc Pythia} \cite{pythia} 6.154.
The simulations are done for a set of 18 neutralino
masses, $m_{\chi}$ = 10, 25, 50, 80.3, 91.2, 100, 150, 176, 200, 250,
350, 500, 750, 1000, 1500, 2000, 3000 and 5000 GeV\@. We tabulate the
yields and then interpolate these tables in \ds.

     The simulations are here
     simpler than those for annihilation in the Sun/Earth
    since we don't have a surrounding medium that can stop the
     annihilation products.  We here simulate for 8 `fundamental'
     annihilation channels $c\bar{c}$, $b\bar{b}$,
     $t\bar{t}$, $\tau^+\tau^-$, $W^+W^-$, $Z^0 Z^{0}$, $g g$ and
     $\mu^{+} \mu^{-}$. Compared to the simulations in the Earth and
     the Sun, we now let pions and kaons decay and we also let
     antineutrons decay to antiprotons. For each mass we simulate
     $2.5 \times 10^{6}$ annihilations and tabulate the yield of
     antiprotons, positrons, gamma rays (not the gamma lines),
     muon neutrinos and neutrino-to-muon conversion rates and the
     neutrino-induced muon yield, where in the last two cases the
     neutrino-nucleon interactions has been simulated with {\sc
     Pythia} as outlined in section~\ref{sec:nt-mcsim}

With these simulations, we can calculate the yield for any of these
particles for a given MSSM model.  For the Higgs bosons, which decay
in flight, an integration over the angle of the decay products with
respect to the direction of the Higgs boson is performed.  Given the
branching ratios for different annihilation channels it is then
straightforward to compute the muon flux above any given energy
threshold and within any angular region around the Sun or the center
of the Earth.
\section{Routine headers -- fortran files}

%%%%% routine dshacom.f %%%%%
\begin{routine}{dshacom.f}
\begin{verbatim}
No header found.
\end{verbatim}
 \end{routine}

%%%%% routine dshadec.f %%%%%
\begin{routine}{dshadec.f}
\begin{verbatim}
*****************************************************************
*** suboutine dshadec decomposes yieldcode yieldk to flyxtype
*** fltype and fi
*****************************************************************

      subroutine dshadec(yieldk,fltyp,fi)
\end{verbatim}
 \end{routine}

%%%%% routine dshadydth.f %%%%%
\begin{routine}{dshadydth.f}
\begin{verbatim}
*****************************************************************************
*** function dshadydth is the differential yield dyield/dcostheta in the
*** cm system boosted to the lab system (including proper jacobians if
*** we are dealing with a differential yield.
*** the function should be integrated from -1 to 1, by e.g.
*** the routine gadap.
*** units: (annihilation)**-1
*****************************************************************************

      real*8 function dshadydth(cth)
\end{verbatim}
 \end{routine}

%%%%% routine dshadyh.f %%%%%
\begin{routine}{dshadyh.f}
\begin{verbatim}
*****************************************************************************
*** function dshadyh is the differential yield dyield/dcostheta in the
*** cm system boosted to the lab system (including proper jacobians if
*** we are dealing with a differential yield. all decay channels of the
*** higgs boson in question are summed.
*** the function should be integrated from -1 to 1, by e.g.
*** the routine gadap.
*** units: (annihilation)**-1
*** author: joakim edsjo (edsjo@physto.se)
*** date: 1998
*** modified: 98-04-15
*****************************************************************************

      real*8 function dshadyh(cth)
\end{verbatim}
 \end{routine}

%%%%% routine dshaemean.f %%%%%
\begin{routine}{dshaemean.f}
\begin{verbatim}
******************************************************************************
*** function dshaemean is used to calculate the mean energy of a decay product
*** when a moving particle decays. e0 and m0 are the energy and mass of
*** the moving particle and m1 and m2 are the masses of the decay products.
*** it is the mean energy of m1 that is returned. all energies and masses
*** should be given in gev.
******************************************************************************

      real*8 function dshaemean(e0,m0,m1,m2)
\end{verbatim}
 \end{routine}

%%%%% routine dshaifind.f %%%%%
\begin{routine}{dshaifind.f}
\begin{verbatim}
**********************************************************
*** routine to find the index of an entry              ***
*** the closest lowest hit is given                    ***
*** author: joakim edsjo (edsjo@physics.berkeley.edu)  ***
*** date: 98-01-26
**********************************************************

      subroutine dshaifind(value,array,ipl,ii,imin,imax)
\end{verbatim}
 \end{routine}

%%%%% routine dshainit.f %%%%%
\begin{routine}{dshainit.f}
\begin{verbatim}
*****************************************************************************
***   subroutine dshainit initializes and loads (from disk) the common
***   block variables needed by the other halo yield routines.  yieldk is
***   the yield type (51,52 or 53 (or 151, 152, 153)) for
***   positron yields, cont. gamma or muon neutrino yields respectively.
***   yieldk is used to check that the provided data file is of the
***   correct type.  if yieldk=51,52 or 53 integrated yields are loaded and
***   if yieldk =151, 152 or 153, differential yields are loaded.
***   author: joakim edsjo
***   edsjo@physto.se date: 96-10-23 (based on dsmuinit.f version
***   3.21) 
***   modified: 98-01-26
*****************************************************************************

      subroutine dshainit(yieldk)
\end{verbatim}
 \end{routine}

%%%%% routine dshapbyieldf.f %%%%%
\begin{routine}{dshapbyieldf.f}
\begin{verbatim}
***********************************************************************
*** function dshapbyieldf gives the distributions of antiprotons for    ***
*** basic annihilation channels chi=1-8. parameterizations to the   ***
*** distributions are used.                                         ***
*** input: mx - neutralino mass (gev)                               ***
***        tp - antiproton kinetic energy (gev)                     ***
***        chi - annihilation channel, 1-8, (short version)         ***
*** output: differential distribution, p-bar gev^-1 annihilation^-1 ***
*** author: joakim edsjo, edsjo@physto.se                           ***
*** date: 1998-10-27                                                ***
***********************************************************************

      real*8 function dshapbyieldf(mx,tp,chi)
\end{verbatim}
 \end{routine}

%%%%% routine dshawspec.f %%%%%
\begin{routine}{dshawspec.f}
\begin{verbatim}
**********************************************************************
*** subroutine dshawspec dumps the spectrum for which dshayield_int failed
*** to the file haspec.dat
**********************************************************************

      subroutine dshawspec(f,a,b,n)
\end{verbatim}
 \end{routine}

%%%%% routine dshayield.f %%%%%
\begin{routine}{dshayield.f}
\begin{verbatim}
*****************************************************************************
*** function dshayield calculates the yield above threshold
*** or the differential flux, for the
*** fluxtype given by yieldk, according to the following table.
***
*** particle       integrated yield     differential yield
*** --------       ----------------     ------------------
*** positron                     51                    151
*** cont. gamma                  52                    152
*** nu_mu and nu_mu-bar          53                    153
*** antiproton                   54                    154
*** cont. gamma w/o pi0          55                    155
*** nu_e and nu_e-bar            56                    156
*** nu_tau and nu_tau-bar        57                    157
*** pi0                          58                    158
*** nu_mu and nu_mu-bar          71                    171 (same as 53/153)
*** muons from nu at creation    72                    172
*** muons from nu at detector    73                    173
***
*** channels ch=1-14 are supported.
*** The annihilation channels are
*** ch =  1 - c c-bar
***       2 - b b-bar
***       3 - t t-bar
***       4 - tau+ tau-
***       5 - W+ W-
***       6 - Z0 Z0
***       7 - H1 H3
***       8 - Z0 H1
***       9 - Z0 H2
***      10 - W+- H-+
***      11 - H2 H3
***      12 - gluon gluon
***      13 - mu+ mu-
***      14 -  Z0 gamma
***
*** the units are (annihilation)**-1
*** for the differential yields, the units are the same plus gev**-1.
***
*** note. The correct data files need to be loaded. This is handled by
*** a call to dshainit. It is done automatically here upon first call.
***
*** author: joakim edsjo (edsjo@physics.berkeley.edu)
*** date: 98-01-26
*** modified: 98-02-18
*****************************************************************************

      real*8 function dshayield(mneu,emuthr,ch,yieldk,istat)
\end{verbatim}
 \end{routine}

%%%%% routine dshayield_int.f %%%%%
\begin{routine}{dshayield\_int.f}
\begin{verbatim}
      real*8 function dshayield_int(f,a,b)
c_______________________________________________________________________
c  integrate function f between a and b
c  input
c    integration limits a and b
c  called by dshayieldfth
c  author: joakim edsjo (edsjo@physto.se) 96-05-16
c  based on paolo gondolos wxint.f routine.
c  integration in log, paolo gondolo 99
c=======================================================================
\end{verbatim}
 \end{routine}

%%%%% routine dshayielddec.f %%%%%
\begin{routine}{dshayielddec.f}
\begin{verbatim}
*****************************************************************************
*** function dshayielddec integrates dshadyh over cos theta.
*** the higgs decay channels are summed in dshadyh
*** units: (annihilation)**-1
*****************************************************************************

      real*8 function dshayielddec(eh,hno,emuth,yieldk,istat)
\end{verbatim}
 \end{routine}

%%%%% routine dshayieldf.f %%%%%
\begin{routine}{dshayieldf.f}
\begin{verbatim}
*****************************************************************************
*** function dshayieldf calculates the yield above threshold (or differential
*** at that energy) for the annihilation channel ch and the
*** fluxtype given by yieldk, according to the following table.
***
*** particle       integrated yield     differential yield
*** --------       ----------------     ------------------
*** positron                     51                    151
*** cont. gamma                  52                    152
*** nu_mu and nu_mu-bar          53                    153
*** antiproton                   54                    154
*** cont. gamma w/o pi0          55                    155
*** nu_e and nu_e-bar            56                    156
*** nu_tau and nu_tau-bar        57                    157
*** pi0                          58                    158
*** nu_mu and nu_mu-bar          71                    171 (same as 53/153)
*** muons from nu at creation    72                    172
*** muons from nu at detector    73                    173
***
*** only channels chi = 1-8 are supported.
*** chi =  1 - c c-bar
***        2 - b b-bar
***        3 - t t-bar
***        4 - tau+ tau-
***        5 - W+ W-
***        6 - Z0 Z0
***        7 - mu+ mu-
***        8 - gluon gluon
*** units: (annihilation)**-1  integrated
*** units: gev**-1 (annihilation)**1 differential
***
*** Note: if this routine is called directly, without calling dshayield
*** first, one needs to load the correct yield tables manually first
*** with a call to dshainit(yieldk) (only needs to be done once).
*****************************************************************************

      real*8 function dshayieldf(mneu,emuthr,ch,yieldk,istat)
\end{verbatim}
 \end{routine}

%%%%% routine dshayieldfth.f %%%%%
\begin{routine}{dshayieldfth.f}
\begin{verbatim}
*****************************************************************************
*** function dshayieldfth integrates dshadydth over cos theta.
*** it is the yield from particle 1 (which decays from m0)
*** that is calculated. particle one corresponds to channel ch.
*** units: (annihilation)**-1
*****************************************************************************

      real*8 function dshayieldfth(e0,m0,mp1,mp2,emuthr,ch,yieldk,
     &  istat)
\end{verbatim}
 \end{routine}

%%%%% routine dshayieldget.f %%%%%
\begin{routine}{dshayieldget.f}
\begin{verbatim}
*****************************************************************************
*** function dshayieldget gives the information in the differential and
*** integrated arrays phiit and phidiff for given array indices.
*** compared to getting the results directly from the array, this
*** routine performs smoothing if requested.
*** the smoothing is controlled by the parameter hasmooth in the
*** following manner.
*** hasmooth = 0 - no smoothing
***            1 - smoothing of zi-1,zi and zi+1 bins if z>.3
***            2 - smoothing of zi-2,zi-1,zi,zi+1 and zi+2 if z>0.3
*****************************************************************************

      real*8 function dshayieldget(zi,mxi,ch,fi,ftype,istat)
\end{verbatim}
 \end{routine}

%%%%% routine dshayieldh.f %%%%%
\begin{routine}{dshayieldh.f}
\begin{verbatim}
*****************************************************************************
*** function dshayieldh calculates the yield above threshold (yieldk=1) or the
*** differential yield (yieldk=2) from a given higgs
*** boson decaying in flight, the energy of the higgs boson should be given
*** in eh.
*** higgses hno = 1-4 are supported (h10, h20, h30 and h+/h-)
*** units: (annihilation)**-1
*****************************************************************************

      real*8 function dshayieldh(eh,emuth,hno,yieldk,istat)
\end{verbatim}
 \end{routine}

%%%%% routine dshayieldh2.f %%%%%
\begin{routine}{dshayieldh2.f}
\begin{verbatim}
*****************************************************************************
*** function dshayieldh2 calculates the yield above threshold (yieldk=1) or the
*** differential yield (yieldk=2) from a given higgs
*** boson decaying in flight, the energy of the higgs boson should be given
*** in eh.
*** higgses hno = 1-4 are supported (h10, h20, h30 and h+/h-)
*** units: 1.0e-30 m**-2 (annihilation)**-1
*****************************************************************************

      real*8 function dshayieldh2(eh,emuth,hno,yieldk,istat)
\end{verbatim}
 \end{routine}

%%%%% routine dshayieldh3.f %%%%%
\begin{routine}{dshayieldh3.f}
\begin{verbatim}
*****************************************************************************
*** function dshayieldh3 calculates the yield above threshold (yieldk=1) or the
*** differntial yield (yieldk=2) from a given higgs
*** boson decaying in flight, the energy of the higgs boson should be given
*** in eh.
*** higgses hno = 1-4 are supported (h10, h20, h30 and h+/h-)
*** units: 1.0e-30 m**-2 (annihilation)**-1
*****************************************************************************

      real*8 function dshayieldh3(eh,emuth,hno,yieldk,istat)
\end{verbatim}
 \end{routine}

%%%%% routine dshayieldh4.f %%%%%
\begin{routine}{dshayieldh4.f}
\begin{verbatim}
*****************************************************************************
*** function dshayieldh4 calculates the yield above threshold (yieldk=1) or the
*** differential yield (yieldk=2) from a given higgs
*** boson decaying in flight, the energy of the higgs boson should be given
*** in eh.
*** higgses hno = 1-4 are supported (h10, h20, h30 and h+/h-)
*** units: 1.0e-30 m**-2 (annihilation)**-1
*****************************************************************************

      real*8 function dshayieldh4(eh,emuth,hno,yieldk,istat)
\end{verbatim}
 \end{routine}

\newpage
\chapter[hm: Halo models]{\codeb{src/hm}:\\ Halo models}
\label{ch:src-hm}

%%%%%%%%%%%%%%%%%%%%%%%%%%%%%%
\section{Halo models -- theory}
\label{sec:halo}

All the dark matter detection rates depend in one way or another on
the properties of the Milky Way dark matter halo. We will here outline
the halo model that by default is included with \ds.

The mass distribution in the Milky Way and the relative importance of
\mbox{its} three components, the bulge, the disk and the halo, are
poorly constrained by available observational data.  Although the
dynamics of the satellites of the galaxy clearly indicates the
presence of a non-luminous matter component, a discrimination among
the different radial dark matter halo profiles proposed in the
literature is not possible at the time being ~\cite{binney}.  One
approach is to assume that dark matter profiles are of a universal
functional form and to infer the Milky Way dark matter distribution
from the results of N-body simulations of hierarchical clustering in
cold dark matter cosmologies.  The predicted profiles in these
scenarios have been tested to a sample of dark matter dominated dwarf
and low-surface brightness galaxies which provide the best
opportunities to test the spatial distribution of dark matter.
Actually this field of research is in rapid evolution and slightly
discrepant results have recently been presented
\cite{navarro,carlberg,kravtsov,moore}.

In \ds, we include a dark matter halo profile of the form
\beq
   \rho(r) \propto \frac{1}{(r/a)^{\gamma}\;[1+(r/a)^{\alpha}]^
   {(\beta-\gamma)/\alpha}}.
\eeq
With this family of profiles, we have a parameterization of most
spherically symmetric profiles
in the literature. In Table~\ref{tab:halo-profile} we list the
corresponding values of $\alpha$, $\beta$ and $\gamma$ for some
popular profiles.

\begin{table}
\begin{center}
\begin{tabular}{lll}
{\bfseries Model} & {\bfseries $(\alpha,\beta,\gamma)$}
& {\bfseries $a$ (kpc)} \\ \hline
Isothermal sphere \cite{kravtsov} & $(2,2,0)$ \\
Kravtsov et al.\ \cite{kravtsov} & $(2,3,0.2-0.4)$ \\
Navarro, Frenk and White\ \cite{navarro} & $(1,3,1)$ \\
Moore et al.\ \cite{moore} & $(?,?,?)$ \\ \hline
\end{tabular}
\caption{Different halo dark matter profiles and their corresponding
parameters. \comment{Include $a$-values here as well?!?}}
\label{tab:halo-profile}
\end{center}
\end{table}

One should keep in mind that some of these profiles are very steep at
the center of the galaxy which might be in conflict with observations.
In fact, this is a topic under rapid evolution at present.  Some
researchers have taken the view that the steep profiles seen in
simulations are impossible to match with observations, and therefore
drastic modifications of the cold dark matter scenario have been
proposed.  Examples are self-interacting \cite{self} or strongly
self-annihilating \cite{annihil} dark matter models.  None of these
proposals can be made to work in MSSM models, so we do not consider
them in \ds.  It should also be noted that there could be other,
astrophysics-related solutions to these problems, which involve the
interplay between the baryons and the dark matter.

Our galactocentric distance $R_{0}$ is not entirely known. Estimates
for $R_{0}$ range from 7.1 kpc to 8.5 kpc \cite{oll,reid,???} and in
\ds\ we use $R_{0}=8.5$ kpc as a default. \comment{Check $R_{0}$
default. (JE)} We also choose the modified isothermal distribution as
a default, but this can be changed by the user.

As a further uncertainty, it is unknown precisely how the black hole
at the galactic center would have interacted with the halo neutralino
distribution. In fact, there are indications that a profile more
singular than NFW would cause a very steep cusp (a ``spike'')
near the Galactic center, with a high enough density that even
the flux of neutrinos from that population could be detected \cite{gs}.
If this really exists, essentially all MSSM models may already be excluded
through the non-detection of radio emission from electrons and
positrons generated in the annihilations \cite{spike}. It should be
noted though that these estimates involve many uncertainties.

We only consider spherical profiles; introducing a flattening
parameter may enhance the value of the flux but the effect is not
expected to be dramatic for this neutralino detection method and we
prefer not to introduce another factor of uncertainty.

We also need to specify the normalization constant of the halo
profile, which we choose as the value of the halo density $\rho_0$ at
our galactocentric distance $R_0$, the core radius $a$ and $R_0$.  One
should keep in mind that there is correlation between the allowed
values of $a$ and $\rho_{0}$ and the chosen halo profile
\cite{bub,pierothesis} due to constraints on e.g.\ the total mass of
the galaxy within 100 pc and the dark matter contribution to the local
rotation curve.  In Table~\ref{tab:halo-profiles} we list typical
values of $a$ and $\rho_{0}$ that we have chosen  based on
these constraints for the different profiles. For more details about these
arguments, see \cite{bub,pierothesis}. The \ds\ default value for
$\rho_{0}$ is 0.3 GeV/cm$^3$.

Usually, the local galactic dark matter velocity distribution is taken
to be a truncated gaussian, which in the detector frame moving at
speed $v_O$ relative to the galactic halo reads
\begin{equation}
   f(v) = {1\over {\mathcal N}_{\rm cut}} { v^2 \over u v_O \sigma} \left\{
   \exp\left[{-{(u-v_O)^2\over2\sigma^2}}\right] -
   \exp\left[{-{\min(u+v_O,v_{\rm cut})^2\over2\sigma^2}}\right]
   \right\}
\end{equation}
for $ v_{\rm esc} < v < \sqrt{v_{\rm esc}^2 + (v_O + v_{\rm cut} ) ^2
} $ and zero otherwise, with $ u = \sqrt{v^2 + v_{\rm esc}^2} $ and
\begin{equation}
   {\mathcal N}_{\rm cut} =
   {v_{\rm cut}\over\sigma} \exp\left( {-{v_{\rm cut}^2\over2\sigma^2}} \right)
   -
   \sqrt{\pi\over2} {\rm erf} \left( {v_{\rm cut}\over\sqrt{2}\sigma} \right) .
\end{equation}
As default, we have taken the halo line-of-sight (one-dimensional) velocity
dispersion $\sigma = $120 km/s,\footnote{Other authors write
   $\exp(-3v^2/2\overline{v}^2)$, in which case $\overline{v} = \sqrt{3}
   \sigma$.}  the galactic escape speed $ v_{\rm cut} = $ 600 km/s, the relative
Earth-halo speed $ v_O = $ 264 km/s (a yearly average) and the Earth escape
speed $ v_{\rm esc} = $ 11.9 km/s. These parameters can be changed by
the user. In some instances, like neutralino capture in the
Earth, the user can specify an arbitrary velocity distribution by
providing a subroutine.

%%%%%%%%%%%%%%%%%%%%%%%%%%%%%%
\subsection{Rescaling of the neutralino density}
\label{sec:rescale}

It is natural to assume that the neutralinos make up most of the dark
matter in our galaxy.  One may therefore only consider MSSM models
which are cosmologically interesting, i.e.\ where the neutralinos can
make up a major fraction of the dark matter in the Universe without
overclosing it.  This range is usually chosen to be $0.025 <
\Omega_{\chi}h^{2} <1$.  However, the user may want to either enlarge
or narrow this range.  If, as is perhaps most natural, the neutralino
alone contributes the major fraction of non-baryonic dark matter in
the Universe, one may want to refer to the current values of
cosmological parameters and fix $\Omega_{\chi}h^{2}$ to be in the
interval between, say 0.1 and 0.3.  If there are other components of
the dark matter, one may want to tolerate smaller numbers.  If one
makes use of the poor knowledge of how galaxy halos were formed, all
the range down to 0.025 may be taken as acceptable.  However, if
$\Omega_{\chi}h^{2}$ drops below 0.025, it cannot account for all the
dark matter associated with galaxy halos.  A frequently used recipe is
then to rescale the estimated local dark matter density $\rho_0\sim
0.3$ GeV/cm$^3$ by $\Omega_{\chi}h^2/0.025$, giving a lower local
density in the form of neutralinos.  Although this may seem a harmless
procedure, one should keep in mind that it is very {\em ad hoc} and
that it may overestimate the preponderance of models with large direct
detection rates.  This is because of the general result that
$\Omega_{\chi}h^{2}\sim 1/\sigma_{ann}v$, and crossing symmetry
generally relates a large annihilation cross section to a large
scattering cross section.  (For indirect detection in the halo, the
effect is moderated by the fact that the rates are proportional to the
square of the density, which thus involves the square of the rescaling
factor.)  In \ds, the user can set the value of the local dark matter
density (the default is $0.3$ GeV/cm$^3$) and determine whether
rescaling is to be used or not, and in that case the lowest tolerable
$\Omega^{\rm min}_{\chi}h^{2}$ below which rescaling should take
place.  If rescaling is used, all output detection rates are computed
with the rescaled value when appropriate.

%%%%%%%%%%%%%%%%%%%%%%%%%%%%%%%%%%%%%%%%%%%%%%%%%%%%%%%%%%%%%%%%%%%%
\section{Halo model -- routines}

The most important routine is \ft{dshmset} which sets the chosen halo
profile.

\section{Routine headers -- fortran files}

%%%%% routine dshmabgrho.f %%%%%
\begin{routine}{dshmabgrho.f}
\begin{verbatim}
****************************************************************
*** dark matter halo density profile in case of the          ***
*** (alphah,beta,gamma) zhao model.                          ***
*** it is a double power law profile,  where -  gamma is     ***
***      the slope towards the galactic centre, - beta is    ***
***      the slope at large galactocentric distances and     ***
***      alphah determines the width of the transition zone. ***
***      e.g.: modified isothermal sphere profile = (2,2,0); ***
***            nfw profile = (1,3,1);                        ***
***            moore et al. profile = (1.5,3,1.5)            ***
***                                                          ***  
*** radialdist = galactocentric distance in kpc              ***
*** ah  = length scale in kpc                                ***
*** rhoref = dark matter density in gev/cm**3 at the         ***
*** galactocentric distance Rref (in kpc)                    ***
***                                                          ***
*** Author: Piero Ullio (ullio@sissa.it)                     ***      
*** Date: 2004-01-12                                         ***
****************************************************************

      real*8 function dshmabgrho(radialdist)
\end{verbatim}
 \end{routine}

%%%%% routine dshmaxiprob.f %%%%%
\begin{routine}{dshmaxiprob.f}
\begin{verbatim}
****************************************************************
*** axisymmetric probability distribution function for       ***
*** equal and small (i.e. unresolved) dark matter clumps     ***
***                                                          ***
*** Input:  radcoord = radial coordinate in kpc              ***
***         vertcoord = vertical coordinate in kpc           ***
***         in a cylindrical coordinate system centered in   *** 
***         the Galactic Center                              ***
*** Output: dshmaxiprob in GeV/cm^3, i.e. for the moment     ***
***         the normalization has to be such that            ***
***         dshmaxiprob(r_0,0.d0)=local halo density=rho0    ***
***                                                          ***
*** Author: Piero Ullio (ullio@sissa.it)                     ***      
*** Date: 2004-01-12                                         ***
****************************************************************

      real*8 function dshmaxiprob(radcoord,vertcoord)
\end{verbatim}
 \end{routine}

%%%%% routine dshmaxirho.f %%%%%
\begin{routine}{dshmaxirho.f}
\begin{verbatim}
****************************************************************
*** axisymmetric dark matter halo density profile            ***
***                                                          ***
*** Input:  radcoord = radial coordinate in kpc              ***
***         vertcoord = vertical coordinate in kpc           ***
***         in a cylindrical coordinate system centered in   *** 
***         the Galactic Center                              ***
*** Output: dshmaxirho = density in GeV/cm^3                 ***
***   e.g.: local halo density = rho0 = dshmaxirho(r_0,0.d0) ***
***                                                          ***
*** Author: Piero Ullio (ullio@sissa.it)                     ***      
*** Date: 2004-01-12                                         ***
****************************************************************

      real*8 function dshmaxirho(radcoord,vertcoord)
\end{verbatim}
 \end{routine}

%%%%% routine dshmboerrho.f %%%%%
\begin{routine}{dshmboerrho.f}
\begin{verbatim}
****************************************************************
*** Dark matter density profile for the de Boer et al fit
*** as reported in astro-ph/0508617.
*** The profile has a triaxial smooth halo and two
*** rings of dark matter.
***
*** Input: x - distance from galactic center towards the Earth (kpc)
***        y - distance from GC in the galactic plane
***            perpendicular to x (kpc)
***        z - height above galactic plane (kpc)
\end{verbatim}
The density profile of de Boer et al.\ 
consists of a dark matter halo with the following ingredients:
\begin{itemize}
\item a triaxial smooth halo,
\item an inner ring at about 4.15 kpc with a density falling off
as $\rho \sim e^{-|z|/\sigma_{z,1}} \; ; \; \sigma_{z,1} = 0.17  
\mbox{~kpc}$, and
\item an outer ring at about 12.9 kpc with a density falling off as  
$\rho \sim e^{-|z|/\sigma_{z,2}} \; ; \; \sigma_{z,2} = 1.7   
\mbox{~kpc}$.
\end{itemize}
\begin{verbatim}
*** The parameters of the profile are set in dshmset.
*** Author: Joakim Edsjo, edsjo@physto.se
*** Date:   2005-12-08
****************************************************************

      real*8 function dshmboerrho(x,y,z)
\end{verbatim}
 \end{routine}

%%%%% routine dshmboerrhoaxi.f %%%%%
\begin{routine}{dshmboerrhoaxi.f}
\begin{verbatim}
****************************************************************
*** A symmetrized (avergaed over phi) version of de Boers profile
*** as reported in astro-ph/0508617.
*** The profile has a triaxial smooth halo and two
*** rings of dark matter.
***
*** Input: r - cylindrical radius from galactic center (kpc)
***        z - height above galactic plane (kpc)
***        how - 1: returns the average <rho> over phi
***              2: returns the average sqrt(<rho^2>) suitable
***                 for symmetrization for annihilation rates (like pbar)
*** The paramters of the profile are set in dshmset.
*** Author: Joakim Edsjo, edsjo@physto.se
*** Date:   2005-12-08
****************************************************************

      real*8 function dshmboerrhoaxi(r,z,how)
\end{verbatim}
 \end{routine}

%%%%% routine dshmburrho.f %%%%%
\begin{routine}{dshmburrho.f}
\begin{verbatim}
****************************************************************
*** dark matter halo density profile in case of the          ***
*** burkert model.                                           ***
***                                                          ***  
*** radialdist = galactocentric distance in kpc              ***
*** ah  = length scale in kpc                                ***
*** rhoref = dark matter density in gev/cm**3 at the         ***
*** galactocentric distance Rref (in kpc)                    ***
***                                                          ***
*** Author: Piero Ullio (ullio@sissa.it)                     ***      
*** Date: 2004-01-12                                         ***
****************************************************************

      real*8 function dshmburrho(radialdist)
\end{verbatim}
 \end{routine}

%%%%% routine dshmdfisotr.f %%%%%
\begin{routine}{dshmdfisotr.f}
\begin{verbatim}
****************************************************************
***                                                          ***
*** halo local velocity distribution function DF(\vec{v})    ***
*** for the case of an isotropic distribution, i.e. for      ***
***       DF(\vec{v}) = DF(|\vec{v}|) = DF(v)                ***
***                                                          ***
*** dshmDFisotr is normalized such that                      ***
*** int d^3v DF(v) = 4 pi int_0^\infty dv v^2 DF(v) = 1      ***
***                                                          ***
*** Input:  |\vec{v}| = Speed in km/s                        ***
*** Output: DF(|\vec{v}|)  in (km/s)^(-3)                    ***
***                                                          ***
*** Calls other routines depending of choice of velocity     ***
*** distribution function (as set by isodf in the common     ***
*** blocks in dshmcom.h)                                     ***
***                                                          ***
*** Author: Piero Ullio                                      ***
*** Date: 2004-01-30                                         ***
****************************************************************

      real*8 function dshmDFisotr(v)
\end{verbatim}
 \end{routine}

%%%%% routine dshmdfisotrnum.f %%%%%
\begin{routine}{dshmdfisotrnum.f}
\begin{verbatim}
****************************************************************
***                                                          ***
*** halo local velocity distribution function DF(\vec{v})    ***
*** for the case of an isotropic distribution, i.e. for      ***
***       DF(\vec{v}) = DF(|\vec{v}|) = DF(v)                ***
*** as loaded from table in file provided by user            ***
***                                                          ***
*** on first call the function loads from file a table of    ***
*** values and then interpolates between them.               ***
***                                                          ***
*** the file name is set by the isodfnumfile variable        ***
*** it is assumed that this file has no header and v DF(v)   ***
*** are given with the format 1000 below                     ***
***                                                          ***
*** to reload a (different) file the int. flag dfisonumset   ***
*** into the DFisosetcom common block has to be manually     ***
*** reset to 0                                               ***
***                                                          ***
*** v in km s**-1                                            ***
*** DF(v) in km**-3 s**3                                     ***
***                                                          ***
*** Author: Piero Ullio                                      ***
*** Date: 2004-01-30                                         ***
****************************************************************

      real*8 function dshmDFisotrnum(v)
\end{verbatim}
 \end{routine}

%%%%% routine dshmhaloprof.f %%%%%
\begin{routine}{dshmhaloprof.f}
\begin{verbatim}
****************************************************************
***                                                          ***  
*** mod: 04-01-12 pu, this is obsolete and should not        *** 
*** be used anymore !!!!!!!!!!!!!                            ***
***                                                          ***  
*** dark matter halo density profile                         ***
*** it assumes that the halo                                 ***
***   1) is spherically symmetric                            ***
***   2) has a double power law profile: the                 ***
***      (alphah,beta,gamma) zhao model,  where - gammah is  ***
***      the slope towards the galactic centre, - beta is    ***
***      the slope at large galactocentric distances and     ***
***      alphah determines the width of the transition zone. ***
***      e.g.: modified isothermal sphere profile = (2,2,0); ***
***            nfw profile = (1,3,1);                        ***
***            moore et al. profile = (1.5,3,1.5)            ***
***                                                          ***  
*** rr = galactocentric distance in kpc                      ***
*** a  = length scale in kpc                                 ***
*** r_0 = galactocentric distance of the sun in kpc          ***
*** rho0 = local dark matter density (=val(r_0)) in gev/cm**3***
***                                                          ***
*** the profile is truncated at 10**-5 kpc assuming          ***
*** val(rr<10**-5 kpc) = val(rr=10**-5 kpc)                  ***
***                                                          ***
*** author: piero ullio (piero@tapir.caltech.edu)            ***
*** date: 00-07-13                                           ***
****************************************************************

      subroutine dshmhaloprof(rr,val)
\end{verbatim}
 \end{routine}

%%%%% routine dshmj.f %%%%%
\begin{routine}{dshmj.f}
\begin{verbatim}
****************************************************************
*** function dshmj: line of sight integral which enters in the ***
*** computation of the gamma-ray and neutrino fluxes from    ***
*** pair annihilations of wimps in the halo.                 ***
***                                                          ***
*** see definition in e.g. bergstrom et al.,                 ***
***   phys. rev d59 (1999) 043506                            ***
*** in case of the many unresolved clump scenario the term   ***
*** fdelta is factorized out                                 *** 
***                                                          ***  
*** it is valid for a spherical dark matter halo             ***
*** psi0 is the angle between direction of observation       ***
*** and the direction of the galactic center; cospsi0 is     ***
*** its cosine.                                              ***
***                                                          ***
*** author: piero ullio (piero@tapir.caltech.edu)            ***
*** date: 00-07-13                                           ***
****************************************************************

      real*8 function dshmj(cospsi0in)
\end{verbatim}
 \end{routine}

%%%%% routine dshmjave.f %%%%%
\begin{routine}{dshmjave.f}
\begin{verbatim}
****************************************************************
*** function dshmjave: average over the solid angle          ***
*** delta (sr) of the function dshmj(cospsi0)                ***
***                                                          ***
*** dshmj is the line of sight integral which enters in the  ***
*** computation of the gamma-ray and neutrino fluxes from    ***
*** pair annihilations of wimps in the halo.                 ***
***                                                          ***
*** see definition in e.g. bergstrom et al.,                 ***
***   phys. rev. d59 (1999) 043506                           ***
*** in case of the many unresolved clump scenario the term   ***
*** fdelta is factorized out                                 *** 
***                                                          ***  
*** it is valid for a spherical dark matter halo             ***
*** psi0 is the angle between direction of observation       ***
*** and the direction of the galactic center; cospsi0 is     ***
*** its cosine.                                              ***
***                                                          ***
*** author: piero ullio (piero@tapir.caltech.edu)            ***
*** date: 00-07-13                                           ***
****************************************************************


      real*8 function dshmjave(cospsi0in,deltain)
\end{verbatim}
 \end{routine}

%%%%% routine dshmjavegc.f %%%%%
\begin{routine}{dshmjavegc.f}
\begin{verbatim}
****************************************************************
*** function dshmjavegc: average over the solid angle        ***
*** delta (sr) of the function dshmj(cospsi0) in case of the *** 
*** galactic center                                          ***
***                                                          ***
*** dshmj is the line of sight integral which enters in the  ***
*** computation of the gamma-ray and neutrino fluxes from    ***
*** pair annihilations of wimps in the halo.                 ***
***                                                          ***
*** see definition in e.g. bergstrom et al.,                 ***
***   phys. rev. d59 (1999) 043506                           ***
*** in case of the many unresolved clump scenario the term   ***
*** fdelta is factorized out                                 *** 
***                                                          ***  
*** it is valid for a spherical dark matter halo             ***
*** psi0, which must be 0.d0 is the angle between direction  ***
*** of observation and the direction of the galactic center; ***
*** cospsi0 is its cosine.                                   ***
***                                                          ***
*** author: piero ullio (piero@tapir.caltech.edu)            ***
*** date: 00-07-13                                           ***
****************************************************************


      real*8 function dshmjavegc(cospsi0in,deltain)
\end{verbatim}
 \end{routine}

%%%%% routine dshmjavepar1.f %%%%%
\begin{routine}{dshmjavepar1.f}
\begin{verbatim}
****************************************************************
*** function integrated in dshmjave                          ***
***                                                          ***
*** author: piero ullio (piero@tapir.caltech.edu)            ***
*** date: 00-07-13                                           ***
****************************************************************

      real*8 function dshmjavepar1(rrtmp)
\end{verbatim}
 \end{routine}

%%%%% routine dshmjavepar2.f %%%%%
\begin{routine}{dshmjavepar2.f}
\begin{verbatim}
****************************************************************
*** function integrated in dshmjavepar1                      ***
***                                                          ***
*** author: piero ullio (piero@tapir.caltech.edu)            ***
*** date: 00-07-13                                           ***
*** mod: 04-01-13 pu                                         ***
****************************************************************

      real*8 function dshmjavepar2(cospsi)
\end{verbatim}
 \end{routine}

%%%%% routine dshmjavepar3.f %%%%%
\begin{routine}{dshmjavepar3.f}
\begin{verbatim}
****************************************************************
*** function integrated in dshmjavegc                        ***
***                                                          ***
*** author: piero ullio (piero@tapir.caltech.edu)            ***
*** date: 00-07-13                                           ***
****************************************************************

      real*8 function dshmjavepar3(cospsiin)
\end{verbatim}
 \end{routine}

%%%%% routine dshmjavepar4.f %%%%%
\begin{routine}{dshmjavepar4.f}
\begin{verbatim}
****************************************************************
*** function integrated in dshmjavepar3                      ***
***                                                          ***
*** author: piero ullio (piero@tapir.caltech.edu)            ***
*** date: 00-07-13                                           ***
****************************************************************

      real*8 function dshmjavepar4(rrin)
\end{verbatim}
 \end{routine}

%%%%% routine dshmjavepar5.f %%%%%
\begin{routine}{dshmjavepar5.f}
\begin{verbatim}
****************************************************************
*** function integrated in dshmjavegc                        ***
*** integration in cos(phi) performed in here, the result    ***
*** still needs to be multiplied by 2 pi                     ***
***                                                          ***
*** author: piero ullio (ullio@sissa.it)                     ***
*** date: 01-10-10                                           ***
****************************************************************

      real*8 function dshmjavepar5(rrin)
\end{verbatim}
 \end{routine}

%%%%% routine dshmjpar1.f %%%%%
\begin{routine}{dshmjpar1.f}
\begin{verbatim}
****************************************************************
*** function integrated in dshmj                             ***
***                                                          ***
*** author: piero ullio (piero@tapir.caltech.edu)            ***
*** date: 00-07-13                                           ***
*** mod: 04-01-13 pu                                         ***
****************************************************************


      real*8 function dshmjpar1(rr)
\end{verbatim}
 \end{routine}

%%%%% routine dshmn03rho.f %%%%%
\begin{routine}{dshmn03rho.f}
\begin{verbatim}
****************************************************************
*** dark matter halo density profile in case of the          ***
*** navarro et al. (2003) model.                             ***
***                                                          ***  
*** radialdist = galactocentric distance in kpc              ***
*** an03  = length scale in kpc                              ***
*** rhoref = dark matter density in gev/cm**3 at the         ***
*** galactocentric distance Rref (in kpc)                    ***
***                                                          ***
*** Author: Piero Ullio (ullio@sissa.it)                     ***      
*** Date: 2004-01-12                                         ***
****************************************************************

      real*8 function dshmn03rho(radialdist)
\end{verbatim}
 \end{routine}

%%%%% routine dshmnumrho.f %%%%%
\begin{routine}{dshmnumrho.f}
\begin{verbatim}
****************************************************************
*** dark matter halo density profile in case of the          ***
*** profile is loaded from a file.                           ***
***                                                          ***  
*** radialdist = galactocentric distance in kpc              ***
***                                                          ***
*** Author: Piero Ullio (ullio@sissa.it)                     ***      
*** Date: 2004-01-12                                         ***
****************************************************************

      real*8 function dshmnumrho(radialdist)
\end{verbatim}
 \end{routine}

%%%%% routine dshmrescale_rho.f %%%%%
\begin{routine}{dshmrescale\_rho.f}
\begin{verbatim}
      subroutine dshmrescale_rho(oh2,oh2min)
No header found.
\end{verbatim}
 \end{routine}

%%%%% routine dshmrho.f %%%%%
\begin{routine}{dshmrho.f}
\begin{verbatim}
****************************************************************
*** dark matter halo density profile                         ***
***                                                          ***
*** Input:  r = galactocentric distance in kpc               ***
*** Output: dshmrho = density in GeV/cm^3                    ***
***                                                          ***
*** links to dshmsphrho where the profile is calculated      ***
*** Author: Joakim Edsjo                                     ***
*** Date: 2000-09-02                                         ***
*** mod: 04-01-13 pu                                         ***
****************************************************************

      real*8 function dshmrho(r)
\end{verbatim}
 \end{routine}

%%%%% routine dshmrho2cylint.f %%%%%
\begin{routine}{dshmrho2cylint.f}
\begin{verbatim}
**********************************************************************
*** function which gives the integral of the square of the 
*** axisymmetric density profile dshmaxirho, normalized to the local 
*** halo density, over a cylindrical volume of radius rmax and height
*** 2*zmax, in the frame with the galactic center as its origin, 
***   i.e.:
***        2 \pi * int_{-zmax}^{+zmax} dz 
***              * int_0^{rmax} dr * r (dshmaxirho(r,zint)/rho0)^2
***        = 2 \pi * 2 int_0^{+zmax} dz 
***              * int_0^{rmax} dr * r (dshmaxirho(r,zint)/rho0)^2
***
*** zmax and rmax in kpc 
*** dshmrho2cylint in kpc^3
*** 
*** author: piero ullio (ullio@sissa.it)
*** date: 04-01-22
**********************************************************************

      real*8 function dshmrho2cylint(rmax,zmax)
\end{verbatim}
 \end{routine}

%%%%% routine dshmset.f %%%%%
\begin{routine}{dshmset.f}
\begin{verbatim}
      subroutine dshmset(c)
****************************************************************
*** subroutine dshmset:                                      ***
*** initialize the density profile and or the small clump    ***
*** probability distribution                                 ***
*** type of halo:                                            ***
***   hclumpy=1 smooth, hclumpy=2 clumpy                     ***
***                                                          ***
*** a few sample cases are given; specified values of the    ***
*** local halo density 'rho0' and of the length scale        ***
*** parameter 'a' should be considered just indicative       ***
***                                                          ***
*** author: piero ullio (piero@tapir.caltech.edu)            ***
*** date: 00-07-13                                           ***
*** small modif: paolo gondolo 00-07-19                      ***
*** mod: 03-11-19 je, 04-01-13 pu                            ***
****************************************************************
\end{verbatim}
 \end{routine}

%%%%% routine dshmsphrho.f %%%%%
\begin{routine}{dshmsphrho.f}
\begin{verbatim}
****************************************************************
*** spherically symmetric dark matter halo density profile   ***
***                                                          ***
*** Input:  radialdist = radial coordinate in kpc            ***
***         in a spherically symmetric coordinate system     ***
***         centered in the Galactic Center                  ***
***         if radialdist lower than the cut radius rhcut,   ***
***         radialdist is shifted to rhcut                   ***
*** Output: dshmsphrho = density in GeV/cm^3                 ***
***   e.g.: local halo density = rho0 = dshmsphrho(r_0)      ***
***                                                          ***
*** Author: Piero Ullio (ullio@sissa.it)                     ***      
*** Date: 2004-01-12                                         ***
****************************************************************

      real*8 function dshmsphrho(radialdist)
\end{verbatim}
 \end{routine}

%%%%% routine dshmudf.f %%%%%
\begin{routine}{dshmudf.f}
\begin{verbatim}
****************************************************************
*** Dark matter halo velocity profile.                       ***
*** This routine gives back u*DF(u) in units of (km/s)^(-2)  ***
***                                                          ***
***    u is the modulus of \vec{u} = \vec{v} - \vec{v}_{MY}  ***
***    with \vec{v} the 3-d velocity of a WIMP in the        ***
***    galactic frame, and \vec{v}_{MY} the projection on    ***
***    the frame you are considering                         ***
***                                                          ***
***    DF(u) = int dOmega DF(\vec{v}), where                 ***
***    DF(\vec{v}) is the halo local velocity distribution   ***
***    function in the galactic frame                        ***
***                                                          ***
*** Note: u*DF(u) is the same as f(u)/u, where f(u) is the   ***
*** one-dimensional distribution function as defined in e.g. ***
*** Gould, ApJ 321 (1987) 571.                               ***
***                                                          ***
*** Note: it is also the same as the one-dimensional         ***
*** distribution function g(u) as defined in, e.g.,          ***
*** Ullio & Kamionkowski, JHEP ....                          ***
***                                                          ***
*** dshmuDF is normalized such that                          ***
*** int_0^\infty u*dshmuDF du = int_0^\infty u^2 DF(u) du =  ***
*** int_0^\infty f(u) du = 1                                 ***
***                                                          ***
*** Input:  u = Speed in km/s                                ***
*** Output: u*DF(u) in (km/s)^(-2)                           ***
***                                                          ***
*** Calls other routines depending of choice of velocity     ***
*** distribution function (as set by veldf in the common     ***
*** blocks in dshmcom.h)                                     ***
*** Author: Joakim Edsjo                                     ***
*** Date: 2004-01-29                                         ***
****************************************************************

      real*8 function dshmuDF(u)
\end{verbatim}
 \end{routine}

%%%%% routine dshmudfearth.f %%%%%
\begin{routine}{dshmudfearth.f}
\begin{verbatim}
****************************************************************
*** Dark matter halo velocity profile as seen from the       ***
*** Earth. Compared to dshmuDF, this routine also includes   ***
*** the possibility to use distribution functions where      ***
*** solar system diffusion is included.                      ***
***                                                          ***
*** This routine gives back u*DF(u) in units of (km/s)^(-2)  ***
*** DF(u) = int dOmega DF(abs(v)) where DF(abs(v-vector)) is ***
*** the three-dimensional distribution function in the halo  ***
*** and v = v_us + u with u being the velocity relative us   ***
*** (Earth/Sun).                                             ***
*** Note: u*DF(u) is the same as f(u)/u, where f(u) is the   ***
*** one-dimensional distribution function as defined in e.g. ***
*** Gould, ApJ 321 (1987) 571.                               ***
***                                                          ***
*** dshmuDF is normalized such that                          ***
*** int_0^\infty u*dshmuDF du = int_0^\infty u^2 DF(u) du =  ***
*** int_0^\infty f(u) du = 1                                 ***
***                                                          ***
*** Input:  u = Speed in km/s                                ***
*** Output: u*DF(u) in (km/s)^(-2)                           ***
***                                                          ***
*** Calls other routines depending of choice of velocity     ***
*** distribution function (as set by veldfearth in the   ***
*** common blocks in dshmcom.h)                              ***
*** Author: Joakim Edsjo                                     ***
*** Date: 2004-01-29                                         ***
****************************************************************

      real*8 function dshmuDFearth(u)
\end{verbatim}
 \end{routine}

%%%%% routine dshmudfearthtab.f %%%%%
\begin{routine}{dshmudfearthtab.f}
\begin{verbatim}
***********************************************************************
*** dshmudfearthtab returns the halo velocity distribution
*** (same as dshmudfearth.f), but reads it from a file.
*** The file should have two header lines (with arbitrary content)
*** and then lines with two columns each  with u and u*DF(u).
*** u should be in units of km/s and u*DF(u) (or f(u)/u) in units
*** of (km/s)^(-2).
*** 
*** The file loaded is given by the option type.
*** Some possible types are velocity distributions as obtained from
*** numerical simulations of WIMP propagation in the solar system
*** including solar capture. 
*** 
*** For the simulations made by Johan Lundberg, see astro-ph/0401113,
*** available options are
***   type = 1, reads file <ds-root>/dat/vdfearth-sdbest.dat :
***      best estimate of distribution at Earth from numerical sims
***   type = 2, reads file <ds-root>/dat/vdfearth-sdconserv.dat :
***      conservative estimate, only including free orbits and
***      jupiter-crossing orbits
***   type = 3, reads file <ds-root>/dat/vdfearth-sdultraconserv.dat :
***      ultraconservative estimate, only including free orbits
***   type = 4, reads file <ds-root>/dat/vdfearth-sdgauss.dat :
***      as if Earth was in free space, i.e. gaussian approx.
***   Note: tot.txt is the best estimate of the distribution at Earth
***   and should be used as a default
***
*** There are also other options, like 
***   type = 5, read a user-supplied file with file name given
***     by udfearthfile in dshmcom.h. If you change the file or
***     for any other reason want to reload it here, you have to
***     set the flag udfearthload to true, in which case it will
***     be loaded here on next call.
***
*** Input: velocity relative to earth [ km s^-1 ]
*** Output: f(u) / u [ (km/s)^(-2) ]
*** Date: January 30, 2004
***********************************************************************

      real*8 function dshmuDFearthtab(u,type)
\end{verbatim}
 \end{routine}

%%%%% routine dshmudfgauss.f %%%%%
\begin{routine}{dshmudfgauss.f}
\begin{verbatim}
***********************************************************************
*** The halo velocity profile in the Maxwell-Boltzmann (Gaussian)
*** approximation.
*** input: velocity relative to earth [ km s^-1 ]
*** output: f(u) / u [ (km/ s)^(-2) ]
*** date: april 6, 1999
*** Modified: 2004-01-29
***********************************************************************

      real*8 function dshmuDFgauss(u)
\end{verbatim}
 \end{routine}

%%%%% routine dshmudfiso.f %%%%%
\begin{routine}{dshmudfiso.f}
\begin{verbatim}
****************************************************************
***                                                          ***
*** function which gives u*DF(u) where:                      ***
***                                                          ***
***    u is the modulus of \vec{u} = \vec{v} - \vec{v}_{ob}  ***
***    with \vec{v} the 3-d velocity of a WIMP in the        ***
***    galactic frame, and \vec{v}_{ob} the projection on    ***
***    the frame you are considering                         ***
***                                                          ***
***    DF(u) = int dOmega DF(\vec{v}), where                 ***
***    DF(\vec{v}) is the halo local velocity distribution   ***
***    function in the galactic frame                        ***
***                                                          ***
*** the function implemented here is valid for:              ***
***    a) an isothermal sphere profile                       ***
***    b) an isotropic profile, i.e.                         ***
***       DF(\vec{v}) = DF(|\vec{v}|)                        ***
*** condition b) implies that the integral is performed by   ***
*** setting |\vec{v}|^2 = u^2 + |\vec{v}_ob|^2               ***
***                      + 2*cos(alpha)*|\vec{v}_ob|*u       ***
*** and then integrating in d(cos(alpha))                    ***
***                                                          ***
*** u in km s**-1                                            ***
*** u*DF(u) in km**-2 s**2                                   ***
***                                                          ***
*** Author: Piero Ullio                                      ***
*** Date: 2004-01-29                                         ***
****************************************************************

      real*8 function dshmuDFiso(u)
\end{verbatim}
 \end{routine}

%%%%% routine dshmudfnum.f %%%%%
\begin{routine}{dshmudfnum.f}
\begin{verbatim}
****************************************************************
***                                                          ***
*** function which gives u*DF(u) where:                      ***
***                                                          ***
***    u is the modulus of \vec{u} = \vec{v} - \vec{v}_{ob}  ***
***    with \vec{v} the 3-d velocity of a WIMP in the        ***
***    galactic frame, and \vec{v}_{ob} the projection on    ***
***    the frame you are considering                         ***
***                                                          ***
***    DF(u) = int dOmega DF(\vec{v}), where                 ***
***    DF(\vec{v}) is the halo local velocity distribution   ***
***    function in the galactic frame                        ***
***                                                          ***
*** on first call the function loads from file a table of    ***
*** values and then interpolates between them.               ***
***                                                          ***
*** the file name is set by the udfnumfile variable          ***
*** it is assumed that this file has no header and u uDF(u)  ***
*** are given with the format 1000 below                     ***
***                                                          ***
*** to reload a (different) file the integer flag uDFnumset  ***
*** into the uDFnumsetcom common block has to be manually    ***
*** reset to 0                                               ***
***                                                          ***
*** u in km s**-1                                            ***
*** u*DF(u) in km**-2 s**2                                   ***
***                                                          ***
*** Author: Piero Ullio                                      ***
*** Date: 2004-01-30                                         ***
****************************************************************

      real*8 function dshmuDFnum(u)
\end{verbatim}
 \end{routine}

%%%%% routine dshmudfnumc.f %%%%%
\begin{routine}{dshmudfnumc.f}
\begin{verbatim}
****************************************************************
***                                                          ***
*** function which gives u*DF(u) where:                      ***
***                                                          ***
***    u is the modulus of \vec{u} = \vec{v} - \vec{v}_{ob}  ***
***    with \vec{v} the 3-d velocity of a WIMP in the        ***
***    galactic frame, and \vec{v}_{ob} the projection on    ***
***    the frame you are considering                         ***
***                                                          ***
***    DF(u) = int dOmega DF(\vec{v}), where                 ***
***    DF(\vec{v}) is the halo local velocity distribution   ***
***    function in the galactic frame                        ***
***                                                          ***
*** on first call the function tabulates uDF(u) and saves    ***
*** the tabulated values in the file whose name is set by    ***
*** the udfnumfile variable in dshmcom.h                     ***
*** interpolation between tabulated values are then used     ***
*** the tabulation has at least 200 points, and more points  ***
*** are added if there are jumps in u*DF which are more than ***
*** 10%; this can be adjusted by changing the reratio        ***
*** variable which is hard coded in the file                 ***
***                                                          ***
*** the implementation is valid only for an isotropic        ***
*** profile, i.e. for                                        ***
***       DF(\vec{v}) = DF(|\vec{v}|)                        ***
*** with the integral performed by setting                   *** 
***       |\vec{v}|^2 = u^2 + |\vec{v}_ob|^2                 ***
***                        + 2*cos(alpha)*|\vec{v}_ob|*u     ***
*** and then integrating in d(cos(alpha))                    ***
***                                                          ***
*** u in km s**-1                                            ***
*** u*DF(u) in km**-2 s**2                                   ***
***                                                          ***
*** Author: Piero Ullio                                      ***
*** Date: 2004-01-30                                         ***
****************************************************************


      real*8 function dshmuDFnumc(u)
\end{verbatim}
 \end{routine}

%%%%% routine dshmudftab.f %%%%%
\begin{routine}{dshmudftab.f}
\begin{verbatim}
***********************************************************************
*** dshmudftab returns the halo velocity distribution
*** (same as dshmudf.f), but reads it from a file.
*** The file should have two header lines (with arbitrary content)
*** and then lines with two columns each  with u and u*DF(u).
*** u should be in units of km/s and u*DF(u) (or f(u)/u) in units
*** of (km/s)^(-2).
*** 
*** The file loaded is given by the option type.
*** Some possible types are velocity distributions as obtained from
*** numerical simulations of WIMP propagation in the solar system
*** including solar capture. 
*** 
*** Available options
***   type = 1, read a user-supplied file with file name given
***     by udffile in dshmcom.h. If you change the file or
***     for any other reason want to reload it here, you have to
***     set the flag udfload to true, in which case it will
***     be loaded here on next call.
***
*** Input: velocity relative to earth [ km s^-1 ]
*** Output: f(u) / u [ (km/s)^(-2) ]
*** Date: January 30, 2004
***********************************************************************

      real*8 function dshmudftab(u,type)
\end{verbatim}
 \end{routine}

%%%%% routine dshmvelearth.f %%%%%
\begin{routine}{dshmvelearth.f}
\begin{verbatim}
      subroutine dshmvelearth(tdays)
No header found.
\end{verbatim}
 \end{routine}

\newpage
\chapter[hr: Halo rates from annihilation]{\codeb{src/hr}:\\ Halo rates from annihilation}
\label{ch:src-hr}

%%%%%%%%%%%%%%%%%%%%%%%%%%%%%%%%%%%%%%%%%%%%%%%%%%%%%%%%%%%%%%%%%%%%

\section{Gamma rays from the halo -- theory}

Among the yields of pair annihilations of halo dark matter particles,
the role played by gamma-rays could be a major one. Unlike the cases
involving charged particles, for gamma-rays it is straightforward
to relate the distribution of sources and the expected flux
at the earth. Most flux estimated can be obtained just by summing
over the contributions along 
lines of sight (or better, geodesics): gamma-rays have a low 
enough cross section on gas and dust and therefore the Galaxy is 
essentially transparent to them (except perhaps in the innermost part, 
very close to the region where a massive black hole is inferred); 
absorption by starlight and infrared background becomes effecient
only for very far away sources (redshift larger than about 1).

It follows that in case the gamma-ray signal is detectable, 
this might be the only chance for mapping the fine structure of a dark 
halo, with a much better resolution for inomogenities (clumps) with 
respect what is accevable through dynamical measurements or lensing effects. 
Turning the latter argument around, if the fine structure of the Galactic 
halo is clumpy, or if a large density enhancement is present towards the 
Galactic center, as seen in N-body simulations of dark matter halos,
this dark matter detection tecnique is much more promising than indicated 
by the earliest estimates in which smooth non-singular halo 
scenarios were considered (recall that the fluxes per unit volume are 
proportional to the square of the dark matter density locally in space).

A further reason to examine in details this detection methods is that
we are approaching what will probably be the golden age for
gamma-ray observations, with a several new experiments that are going 
to map the gamma-ray sky. These experiments will have unprecendented 
sensitivities and cover an energy range, namely 10 GeV -- few hundred GeV, 
in which very scarce data are available at the time being and which may 
turn out to be the most interesting for dark matter detection.
The hypothesis of a gamma-ray signal from neutralino annihilations 
will be tested for both by the upcoming space experiments 
(GLAST, AMS, AGILE) and by the new generation of ground-based 
air cherenkov telescopes (ACTs) being built (Magic, Hess, Veritas).

The bulk of the gamma-ray yield from neutralino annihilations arise in the
decay of neutral pions produced in the fragmetation processes initiated
by tree level final states~\cite{oldcontga,lpj,gahalo}
(analogously to the other halo signals,
in \ds\ we include all tree level final states and make use of a
Monte Carlo simulation for fragmentation and decay processes, see 
Section~\ref{sec:mcsim}). Unfortunately the $\pi^0$ 
intermediate state is common to other astrophysical processes, and this
may turn out to be a limiting factor to disentangle dark matter sources.
At the same time, however, a relevant gamma-ray contribution may arise 
directly (at one-loop level) in two body final states; although such 
photons are much fewer than those from $\pi^0$ decays they have a much 
better signature: neutralinos annihilating in the galactic halos move 
with a velocity of the order $v/c \sim$ 10$^{-3}$, hence these outgoing 
photons (as any particle in any of the allowed two body final states)
will then be nearly monochromatic, with energy of the order of the
neutralino mass\cite{charm,oldlines,jkline,lp,ub,lpj}. 
There is no other known astrophysical source with such a signature:
the detection of a line signal out of a spectrally smooth gamma-ray 
background would be a spectacular confirmation of the existence of dark 
matter in form of exotic massive particles.

If dark matter is in form of neutralinos, there are two processes
givin rise to line signals, the annihilation into two photons and into
one photon and a Z boson. Both of them are included in the \ds\ 
package, as well as the contribution with a continuum energy spectrum.
We review them briefly here, focussing first on annihilation rates
and giving then expressions for gamma-ray fluxes.


\subsection{$\chi\chi\to \gamma\gamma$}

In \ds\ the full expression for the annihilation cross section of
the process
\beq
\tilde{\chi}^{0}_{1} + \tilde{\chi}^{0}_{1} \rightarrow \gamma
+\gamma
\eeq
is computed at full one loop level, in the limit of vanishing relative 
velocity of the neutralino pair, i.e. the case of interest for neutralinos
in galactic halos; the outgoing photons have an energy equal to 
the mass of ${\chi}^{0}_{1}$:
\beq
E_{\gamma} = M_{\chi}.
\eeq
The neutralino pair must be in an S wave state with pseudoscalar quantum 
numbers; projecting out of the amplitude the $^{1}$S$_{0}$ state
simplifies the calculation, and a further simplification is obtained by
computing the amplitude in the non linear gauge defined in~\cite{fujikawa}, 
which is a slight variant of the usual linear R-gauge (or 't Hooft gauge). 

The amplitude of the annihilation process can be factorized in the
form
\beq
\mathcal{A} =\frac{e^2}{2 \sqrt{2} \; \pi^2}
  \epsilon\left(\epsilon_{1},\epsilon_{2},k_{1},k_{2} \right)
  \;\tilde{\mathcal{A}}
\eeq
where $\epsilon_{1}$, $\epsilon_{2}$ and $k_{1}$, $k_{2}$ are
respectively the polarization tensors and the momenta of the two outgoing 
photons. The cross section is then given, as a function of
$\tilde{\mathcal{A}}$, by the formula
\beq
  v\sigma_{2\gamma} = \frac{\alpha^2 M^2_{\chi}}{16 \pi^3} \left|\;
\tilde{\mathcal{A}}\; \right|^{2}\;\;\;.  \label{eq:sigmav2g}
\eeq

The total amplitude is implemented in \ds\ as the sum of the contributions 
obtained from four different classes of diagrams:
\begin{eqnarray*}
\tilde{\mathcal{A}}=\tilde{\mathcal{A}}_{f\tilde{f}}+
  \tilde{\mathcal{A}}_{H^+}+\tilde{\mathcal{A}}_{W}+\tilde{\mathcal{A}}_{G},
\end{eqnarray*}
where the indices label the particles in the internal loops, i.e.,
respectively, fermions and sfermions, charged Higgs and charginos, 
W-bosons and charginos, and, in the gauge we chose, charginos and Goldstone 
bosons. For every $\mathcal{A}$ term, real
and imaginary parts are splitted; the full set of analytic formulas are 
given in \cite{lp}, following the notation of \cite{linejk}, where some of 
the contributions were first computed. They are rather lengthy expressions
with non trivial dependences on various combinations of parameters in 
the MSSM. We reproduce here, as an example, the formulas for the diagrams
with W bosons and charginos, which, in  most cases, give the dominant 
contribution to the cross section as discovered in \cite{lp}. The sum over
$\chi^+_i$ includes the two chargino eigenstates:
\begin{eqnarray}
Re\,\tilde{\mathcal{A}}_{W} & = &
\sum_{\chi^+_i} \frac{1}{M^2_{\chi}} \; \left[ 2\;
\frac{\left(a-b\right)\;
  S_{\chi W}}{1+a-b} \;I_{1} \left( a,b \right)+\frac{S_{\chi W}-2
\sqrt{a}\;
  D_{\chi W}}{1-a-b} \;I_{1} \left( a,1 \right) \right. \nonumber \\
   &&\left. + \left( 2\;\frac{S_{\chi W}-2 \sqrt{a}\;D_{\chi
W}}{1-a-b}-
    \frac{3\,S_{\chi W}-4 \sqrt{a}\;D_{\chi W}}{1-b}  \right)\;I_{2}
    \left( a,b \right) \right. \nonumber \\
   && \left.+ \left( \frac{\left(2+b\right)\,S_{\chi W}-4 \sqrt{a}\;
    D_{\chi W}}{1-b} -2\;\frac{\left( 1-a+b \right)\; S_{\chi
W}}{1+a-b}
    \right)\;I_{3}\left( a,b \right) \right]
\end{eqnarray}

\begin{eqnarray}
Im\,\tilde{\mathcal{A}}_{W} & = &
  -\pi\;\sum_{\chi^+_i} \frac{1}{M^2_{\chi}} \; \left( 2\;
   \frac{\left(a-b\right)\;S_{\chi W}}{1+a-b} \right) \cdot \nonumber
\\
  &&\cdot \, \log \left( \frac{1+\sqrt{1-b/a}}{1-\sqrt{1-b/a}} \right)
   \theta \left(1-m^2_{W}\,/\,M^2_{\chi} \right) \label{imw}
\end{eqnarray}
where we defined:
\begin{eqnarray*}
a=\frac{M^2_{\chi^0_1}}{M^2_{\chi^+_i}} &&
b=\frac{m^2_{W}}{M^2_{\chi^+_i}}
\end{eqnarray*}
\begin{eqnarray*}
S_{\chi W}=\frac{1}{2}\;\left(g^L_{W1i}\;
g^{L\,\ast}_{W1i}+g^R_{W1i}\;
  g^{R\,\ast}_{W1i} \right) && D_{\chi W}=\frac{1}{2}\;\left(
g^L_{W1i}\;
  g^{R\,\ast}_{W1i}+g^R_{W1i}\; g^{L\,\ast}_{W1i} \right)\;\; ,
\end{eqnarray*}
and the functions $I_{1}\left( a,b \right)$,
$I_{2}\left( a,b \right)$ and $I_{3}\left( a,b \right)$, which arise
from the loop integrations, are given by:
\begin{eqnarray}
I_{1}\left( a,b \right) = \int_{0}^{1} \frac{d\,x}{x} \; \log \left(
\left|
\frac{4\,a\,x^2-4\;a\,x + b}{b} \right| \right)
\end{eqnarray}
\begin{eqnarray}
I_{2}\left( a,b \right) = \int_{0}^{1} \frac{d\,x}{x} \; \log \left(
\left|
\frac{-a\,x^2+(a+b-1)\,x + 1}{a\,x^2+(-a+b-1)\,x + 1} \right| \right)
\end{eqnarray}
\begin{eqnarray}
I_{3}\left( a,b \right) = \int_{0}^{1} \frac{d\,x}{x} \; \log \left(
\left|
\frac{-a\,x^2+(a+1-b)\,x + b}{a\,x^2+(-a+1-b)\,x + b} \right| \right) .
\end{eqnarray}
$I_{1}\left( a,b \right)$ is the well known three point function that
appears in triangle diagrams; it is an analytic function of
a/b. $I_{2}\left( a,b \right)$ and $I_{3}\left( a,b \right)$ may be
expressed in terms of dilogarithms. In \ds, they are computed in the
integral form as, for any physically interesting value of the
parameters a and b, the integrands are smooth functions of $x$.

The branching ratio for neutralino annihilations into $2\gamma$ is 
typically not larger than 1\%, with the largest values of
$v\sigma_{2\gamma}$, for neutralinos with a cosmologically significant
relic aboundance, in the range $10^{-29}$--$10^{-28}$~cm$^3$s$^{-1}$.
Such values may be large enough for the discovery of this signal
in upcoming measurements; at the same time it should be kept in mind
that very low values for the cross section are feasible as well.


%%%%%%%%%%%%%%%%%%%%%%%%%%%%%%
\subsection{$\chi\chi\to Z\gamma$}

The process of neutralino annihilation into a photon and a Z$^0$ boson
\cite{ub}
\beq
\tilde{\chi}^{0}_{1} + \tilde{\chi}^{0}_{1} \rightarrow \gamma
+ Z^0
\eeq
also gives a nearly
monochromatic line (with a small smearing caused by the finite
  width of the $Z^0$ boson), with  energy
\beq
E_{\gamma} = M_{\chi} - \frac{m_Z^2}{4\,M_{\chi}}.
\eeq

The steps followed in \ds\
to compute the cross section are essentially the same
as those described for the  $2\gamma$ case.
Again the total amplitude is obtained by summing the contribution
from four classes of diagrams and by splitting for each of them 
real and imaginary parts. The analytic formulas were derived in 
\cite{ub}, and are much less compact than those 
obtained for the process of neutralino annihilation into two photons.

The maximum value of $v\sigma_{Z\gamma}$, for neutralinos with a 
cosmologically significant relic aboundance, is around 
$2\cdot 10^{-28}$~cm$^3$s$^{-1}$ and occurs for a nearly
pure Higgsinos. In the heavy mass range, the value of $v\sigma_{Z\gamma}$
reaches a plateau of around $0.6 \cdot 10^{-28}$~cm$^3$s$^{-1}$. This
interesting effect of a non-diminishing cross section with higgsino mass
(which is due to a contribution to the real part of the amplitude)
is also valid for the $2\gamma$ final state in the corresponding limit, 
with a value of $1\cdot 10^{-28}$~cm$^3$s$^{-1}$ \cite{lp}.
Since the gamma-ray background drops rapidly with increasing photon
energy, these processes may be interesting for detecting dark
matter neutralinos near the upper range of allowed neutralino masses.

Whenever the lightest neutralino contain a significant Wino or Higgsino
component the value of $v\sigma_{Z\gamma}$ maybe as large as, or even larger 
than, twice the value of $v\sigma_{2\gamma}$. It is therefore usually
not a good approximation to neglect the $Z\gamma$ state compared to
$2\gamma$.


%%%%%%%%%%%%%%%%%%%%%%%%%%%%%%
\subsection{Gamma rays with continuum energy spectrum}

The advantage with the gamma-ray lines discussed in the previous Sections
is the distinctive spectral signature, which has no plausible astrophysical
counterpart. 

Compared to the monochromatic flux, the gamma-ray flux produced in 
$\pi^0$ decays is much larger but has less distinctive features.
The photon spectrum in the process of a pion decaying into $2\gamma$ is, 
independent of the pion energy, peaked at half of the $\pi^0$ mass, 
about 70~MeV, and symmetric with respect to this peak if plotted in 
logaritmic variables. Of course, this is true both for pions produced in 
neutralino annihilations and, e.g., for those generated by cosmic ray 
protons interacting with the interstellar medium.

When considered together with to the cosmic ray induced Galactic gamma-ray
background, the neutralino induced signal looks like a component analogous
to the secondary flux due to nucleon nucleon interactions: it is
drowned into the Bremsstrahlung component at low energy, while it may 
be the dominant contribution at energies above 1 GeV or so. 
In fact, if the exotic component is indeed significant
an option to disentangle it would be to search for a break in the
energy spectrum at about the neutralino mass, where the line feature
might be present as well: while the maximal energy for a photon emitted 
in neatralino pair annihilations is equal to the neutralino mass,
the component from cosmic ray protons extends to much higher energies,
essentially with the same spectral index as for the proton spectrum
(the role played by the third main background component, 
inverse Compton emission, has still to settled at the time being and
may worsen the problem of discrimitation against background).

Besides this (weak) spectral feature, another way to disentangle
the dark matter signal may be to exploit a directional signature:
data with a wide angular coverage should be analyzed to seach for 
a gamma-ray flux component following the shape and density profile 
of the dark halo, including eventual contributions from clumps.


\subsection{Sources and fluxes}

Given a density distribution of dark matter neutralinos along some line
of sight $l$, the monochromatic gamma-ray flux per unit solid
angle in that direction is:
\beq
{d\Phi_{\gamma}(\psi)\over d\Omega} 
= \frac{N_{\gamma}\;v\sigma_{X^0\gamma}}{4\pi M_\chi^2}
\int_{line\;of\;sight}\rho_{\chi}^2(l)\; d\,l(\psi)\;,
\label{eq:gaflux}
\eeq
where $\psi$ is an angle to label the direction of observation and where 
$N_{\gamma} = 2$ for $\chi\,\chi\rightarrow \gamma\,\gamma$, 
$N_{\gamma} = 1$ for $\chi\,\chi\rightarrow Z\,\gamma$. Analogously,
the gamma-ray flux with continuum energy spectrum is obtained by replacing
$N_{\gamma}\;v\sigma_{X^0\gamma}$ with 
$\sum_f dN_{\gamma}^f/dE\;v\sigma_{f}$, where the sum is over all tree
level final states. Separating the dependence on the dark matter distribution
from the part which is related to values of the cross section and the 
neutralino mass, we rewrite Eq.~(\ref{eq:gaflux}) as:
\beq
{d\Phi_{\gamma}(\psi)\over d\Omega} 
\simeq  1.87 \cdot 10^{-11}\left( \frac{N_{\gamma}\;v\sigma_{X^0\gamma}}
{10^{-29}\ {\rm cm}^3 {\rm s}^{-1}}\right)\left( \frac{10\,\rm{GeV}}
{M_\chi}\right)^2 \cdot 
J\left(\psi\right)\;\rm{cm}^{-2}\;\rm{s}^{-1}\;\rm{sr}^{-1}\;,
\eeq
where we have defined the dimensionless function
\beq
J\left(\psi\right) = \frac{1} {8.5\, \rm{kpc}}
\cdot \left(\frac{1}{0.3\,{\rm GeV}/{\rm cm}^3}\right)^2
\int_{line\;of\;sight}\rho_{\chi}^2(l)\; d\,l(\psi)\;.
\label{eq:jpsi}
\eeq
The relevant quantity for a measurement is, rather than $J\left(\psi\right)$,
the integral of $J\left(\psi\right)$ over the solid angle given by the
angular acceptance $\Delta\Omega$ of a detector which is pointing in
the direction $\psi$. Defining:
\begin{equation}
\langle\,J\left(\psi\right)\rangle_{\Delta\Omega}
= \frac{1}{\Delta\Omega} \int_{\Delta\Omega} d\Omega^{\prime}
J\left(\psi^{\prime}\right)\;, 
\label{eq:jave}
\end{equation}
the flux measured in a detector is:
\begin{equation}
\Phi_{\gamma}(\psi, \Delta\Omega) = 1.87 \cdot 10^{-11}
\left(\frac{N_{\gamma}\;v\sigma_{X^0\gamma}}
{10^{-29}\ {\rm cm}^3 {\rm s}^{-1}}\right)
\left( \frac{10\,\rm{GeV}}{M_\chi}\right)^2 
\langle\,J\left(\psi\right)\rangle_{\Delta\Omega}\times\Delta\Omega 
\;\rm{cm}^{-2}\;\rm{s}^{-1}\; \rm{sr}^{-1}\; .
\label{signal}
\end{equation}
Finally, the formalism we introduced can be used also to estimate
the flux in the simple case of a single source which, for the given 
detector, can be approximated as point-like (see examples below). 
If such 
source is in the direction $\psi$ at a distance $d$, Eq.~(\ref{eq:jave})
becomes:
\begin{equation}
\langle\,J\left(\psi\right)\rangle_{\Delta\Omega}
= \frac{1} {8.5\, \rm{kpc}}
\cdot \left(\frac{1}{0.3\,{\rm GeV}/{\rm cm}^3}\right)^2
\cdot \frac{1}{d^2} \cdot \frac{1}{\Delta\Omega}
\int d^3r \;\rho_{\chi}^2(\vec{r}) 
\end{equation}
where the integral is over the extention of the source (much smaller
than $d$).

Several targets have been discussed as sources of gamma-rays from the
annihilation of dark matter particles. An obvious source is the dark
halo of our own galaxy~\cite{galga} and in particular the Galactic center,
as the dark matter density profile is expected, in most models, to be 
picked towards it, possibly with huge enhancements close to te central 
black hole. The Galactic center is an ideal target for both ground-
and space-based gamma-ray telescopes. As satellite experiments have 
much wider field of views and will provide a full sky coverage,
they will test the hypothesis of gamma-rays emitted in clumps of dark 
matter which may be present in the 
halo~\cite{clumpyga,gahalo,clumpy,clumpybeg}. 
Another possibility which has been considered is the case of 
gamma-ray fluxes from external nearby galaxies~\cite{extergal}. 
Furthermore, it has 
been proposed to search for an extragalactic flux originated by all 
cosmological annihilations of dark matter 
particles~\cite{extragal,extragalbeu}.

\ds\ is suitable to compute the gamma-ray flux from all these (and possibly 
other) sources. Two cases are fully included in the package:
assuming neutralinos are smoothly distributed in the Galactic halo
with $\rho_{\chi}$ equal to the dark matter density profile, in 
\ds\ Eq.~\ref{eq:jave} is computed for a specified halo profile and 
any given $\psi$ and $\Delta\Omega$~\cite{lpj}. 
The second option deals with the
case of a portion of dark matter being in the form of clumps, each of
which is treated as a non-resolvable source in the detector, distributed
in the Galaxy according to a probability distribution which
follows the dark matter density profile (see \cite{clumpy}
for details). It is straightforward to extend this to all other 
astrophysical sources; in case of cosmological sources one has just 
to pay attention to include redshift effects and absoption on starlight 
and infrared background, see~\cite{extragalbeu}.



%%%%%%%%%%%%%%%%%%%%%%%%%%%%%%
\section{Neutrinos from halo -- theory}

Usually, the flux of neutrinos from annihilation of neutralinos in
the Milky Way halo is too small to be detectable, but for some clumpy
or cuspy models, it might be detectable. The calculation of the
neutrino-flux follows closely the calculation of the continuous gamma
ray flux, with the main addition that neutrino interactions close to
the detector are also included. Hence, both the neutrino flux and the
neutrino-induced muon flux can be obtained. The neutrino to muon
conversion rate in the Earth can also be obtained.
\section{Routine headers -- fortran files}

%%%%% routine dshaloyield.f %%%%%
\begin{routine}{dshaloyield.f}
\begin{verbatim}
*****************************************************************************
***   function dshaloyield gives the total yield of positrons, cont. gammas
***   or neutrinos coming from neutralino annihilation in the halo
***   the yields are given as number / annihilation. the energy egev
***   is the threshold for integrated yields and the energy for
***   differential yields. the yields are
***     yieldk =  51: integrated positron yields
***     yieldk =  52: integrated cont. gammas
***     yieldk =  53: integrated muon neutrinos
***     yieldk =  54: integrated antiproton yields
***     yieldk =  71: integrated neutrino yields (same as 53)
***     yieldk =  72: integrated muon yields at creation
***     yieldk =  73: integrated muon yields in ice
***     yieldk = above+100: differential in energy
*** the annihilation branching ratios and
*** higgs parameters are extracted from susy.h and by calling dsandwdcosnn
*** if istat=1 upon return,
*** some inaccesible parts the differential muon spectra has been wanted,
*** and the returned yield should then be treated as a lower bound.
*** if istat=2 energetically forbidden annihilation channels have been
*** wanted. if istat=3 both of these things has happened.
*** author: joakim edsjo  edsjo@physics.berkeley.edu
*** date: 98-01-29
*** modified: 98-04-15
*****************************************************************************

      real*8 function dshaloyield(egev,yieldk,istat)
\end{verbatim}
 \end{routine}

%%%%% routine dshaloyielddb.f %%%%%
\begin{routine}{dshaloyielddb.f}
\begin{verbatim}
*****************************************************************************
***   function dshaloyielddb is the version of dshaloyield appropriate
***   for antideuterons
***   yieldk = 159 - antideuteron differential yield
***   author: piero ullio, ullio@sissa.it
***   date: 03-01-14
*****************************************************************************

      real*8 function dshaloyielddb(egev,yieldk,istat)
\end{verbatim}
 \end{routine}

%%%%% routine dshrdbardiff.f %%%%%
\begin{routine}{dshrdbardiff.f}
\begin{verbatim}
      real*8 function dshrdbardiff(td,solarmod,how)

**********************************************************************
*** function dshrdbardiff calculates the differential flux of
*** antideuterons for the antideuteron kinetic energy per nucleon td 
*** as a result of neutralino annihilation in the halo.
*** compared to dshrdbdiff0, dshrdbardiff uses the rescaled local density
***   input:
***     td - antideuteron kinetic energy per nucleon in gev
***     solarmod - 0 no solar modulation
***                1 solar modulation a la perko
***     how - 1 calculate t_diff only for requested momentum
***           2 tabulate t_diff for first call and use table for
***             subsequent calls
***           3 as 2, but also write the table to disk as pbtd.dat
***             at the first call
***           4 read table from disk on first call, and use that for
***             subsequent calls
*** units: gev^-1 cm^-2 s^-1 sr^-1
*** author: 00-07-19 paolo gondolo
**********************************************************************
\end{verbatim}
 \end{routine}

%%%%% routine dshrdbdiff0.f %%%%%
\begin{routine}{dshrdbdiff0.f}
\begin{verbatim}
      real*8 function dshrdbdiff0(td,solarmod,how)

**********************************************************************
*** function dshrdbdiff0 calculates the differential flux of
*** antideuterons for the kinetic energy per nucleon td as a result of
*** neutralino annihilation in the halo.
*** dshrdbdiff0 uses the unrescaled local density
*** see dshrdbardiff for rescaling the local density
***   input:
***     td - antideuteron kinetic energy per nucleon in gev
***     solarmod - 0 no solar modulation
***     how - 1 calculate t_diff only for requested momentum
***           2 tabulate t_diff for first call and use table for
***             subsequent calls
***           3 as 2, but also write the table to disk as pbtd.dat
***             at the first call
***           4 read table from disk on first call, and use that for
***             subsequent calls
*** units: gev^-1 cm^-2 s^-1 sr^-1
*** author: joakim edsjo
*** date: 98-02-10
*** modified: joakim edsjo, edsjo@physto.se
*** modified: 98-07-13, 00-07-19 paolo gondolo
**********************************************************************

\end{verbatim}
 \end{routine}

%%%%% routine dshrgacdiffsusy.f %%%%%
\begin{routine}{dshrgacdiffsusy.f}
\begin{verbatim}
**********************************************************************
*** function dshrgacdiffsusy gives the susy dependent term in the
*** flux of gamma-rays with continuum energy spectrum per gev
*** at the energy egam (gev) from neutralino annihilation in the halo.
***
*** dshrgacdiffsusy in unit of gev^-1
*** 
*** the flux in a solid angle delta in the direction psi0 is given by:
***   cm^-2 s^-1 sr^-1 * dshrgacdiffsusy(egam,istat)
***   * dshmjave(cospsi0,delta) * delta (in sr)
***
*** the flux per solid angle in the direction psi0 is given by:
***   cm^-2 s^-1 sr^-1 * dshrgacdiffsusy(egam,istat)
***   * dshmj(cospsi0)
*** 
*** in case of a clumpy halo the factor fdelta has to be added 
***
*** author: joakim edsjo, edsjo@physto.se
*** modified: piero ullio (piero@tapir.caltech.edu) 00-07-13
***           Joakim Edsjo (edsjo@physto.se) 03-01-21, factor of 1/2
***           in annihilation rate added
**********************************************************************

      real*8 function dshrgacdiffsusy(egam,istat)
\end{verbatim}
 \end{routine}

%%%%% routine dshrgacont.f %%%%%
\begin{routine}{dshrgacont.f}
\begin{verbatim}
      real*8 function dshrgacont(egath,jpsi,istat)
**********************************************************************
***   function dshrgacont gives the flux of gamma-rays with continuum
***   energy spectrum above the threshold egath (gev) from neutralino
***   annihilation in the halo.
***   
***   jpsi is the value of the integral of rho^2 along the line of
***   sight, and can be obtained with a call to dshmj.  jpsi can also be
***   the averaged value of j over a solid angle delta, obtained with a
***   call to dshmjave.
***   
***   dshrgacont uses the rescaled local density, while j uses the
***   unrescaled local density
***   
***   dshrgacont in units of cm^-2 s^-1 sr^-1
***   
***   in case of a clumpy halo the factor fdelta has to be added
***   
***   author: paolo gondolo (gondolo@mppmu.mpg.de) 00-07-19
**********************************************************************
\end{verbatim}
 \end{routine}

%%%%% routine dshrgacontdiff.f %%%%%
\begin{routine}{dshrgacontdiff.f}
\begin{verbatim}
      real*8 function dshrgacontdiff(egam,jpsi,istat)
**********************************************************************
***   function dshrgacontdiff gives the flux of gamma-rays with continuum
***   energy spectrum per gev at the energy egam (gev) from neutralino
***   annihilation in the halo.
***   
***   jpsi is the value of the integral of rho^2 along the line of
***   sight, and can be obtained with a call to dshmj.  jpsi can also be
***   the averaged value of j over a solid angle delta, obtained with a
***   call to dshmjave.
***   
***   dshrgacontdiff uses the rescaled local density, while j uses the
***   unrescaled local density
***   
***   dshrgacontdiff in units of cm^-2 s^-1 sr^-1
***   
***   in case of a clumpy halo the factor fdelta has to be added
***   
***   author: paolo gondolo (gondolo@mppmu.mpg.de) 00-07-19
**********************************************************************
\end{verbatim}
 \end{routine}

%%%%% routine dshrgacsusy.f %%%%%
\begin{routine}{dshrgacsusy.f}
\begin{verbatim}
**********************************************************************
*** function dshrgacsusy gives the susy dependent term in the
*** flux of gamma-rays with continuum energy spectrum above the 
*** threshold egath (gev) from neutralino annihilation in the halo.
***
*** dshrgacsusy is dimensionless
*** 
*** the flux in a solid angle delta in the direction psi0 is given by:
***   cm^-2 s^-1 sr^-1 * dshrgacsusy(egath,istat)
***   * dshmjave(cospsi0,delta) * delta (in sr)
***
*** the flux per solid angle in the direction psi0 is given by:
***   cm^-2 s^-1 sr^-1 * dshrgacsusy(egath,istat)
***   * dshmj(cospsi0)
*** 
*** in case of a clumpy halo the factor fdelta has to be added 
***
*** author: joakim edsjo, edsjo@physto.se
*** modified: piero ullio (piero@tapir.caltech.edu) 00-07-13
***           Joakim Edsjo (edsjo@physto.se) 03-01-21, factor of 1/2
***           in annihilation rate added
**********************************************************************

      real*8 function dshrgacsusy(egath,istat)
\end{verbatim}
 \end{routine}

%%%%% routine dshrgaline.f %%%%%
\begin{routine}{dshrgaline.f}
\begin{verbatim}
      subroutine dshrgaline(jpsi,gaga,gaz)
**********************************************************************
***   subroutine dshrgaline gives the flux of monoenergetic gamma-rays
***   from neutralino annihilation in the halo.
***   
***   jpsi is the value of the integral of rho^2 along the line of
***   sight, and can be obtained with a call to dshmj.  jpsi can also be
***   the averaged value of j over a solid angle delta, obtained with a
***   call to dshmjave.
***   
***   dshrgaline uses the rescaled local density, while j uses the
***   unrescaled local density
***   
***   dshrgaline in units of cm^-2 s^-1 sr^-1
***   
***   in case of a clumpy halo the factor fdelta has to be added
***   
***   author: paolo gondolo (gondolo@mppmu.mpg.de) 00-07-19
**********************************************************************
\end{verbatim}
 \end{routine}

%%%%% routine dshrgalsusy.f %%%%%
\begin{routine}{dshrgalsusy.f}
\begin{verbatim}
**********************************************************************
*** subroutine dshrgalsusy gives the susy dependent term in the
*** flux of monoenergetic gamma-rays from neutralino annihilation 
*** in the halo.
***
*** dshrgalsusy is dimensionless
*** 
*** the flux in a solid angle delta in the direction psi0 is given by:
***   cm^-2 s^-1 sr^-1 * gagarate (or gazrate)
***   * dshmjave(cospsi0,delta) * delta (in sr)
***
*** the flux per solid angle in the direction psi0 is given by:
***   cm^-2 s^-1 sr^-1 * gagarate (or gazrate)
***   * dshmj(cospsi0)
*** 
*** in case of a clumpy halo the factor fdelta has to be added 
***
*** author: joakim edsjo, edsjo@physto.se
*** modified: piero ullio (piero@tapir.caltech.edu) 00-07-13
***           Joakim Edsjo (edsjo@physto.se) 03-01-21, factor of 1/2
***           in annihilation rate added
**********************************************************************

      subroutine dshrgalsusy(gagarate,gazrate)
\end{verbatim}
 \end{routine}

%%%%% routine dshrmudiff.f %%%%%
\begin{routine}{dshrmudiff.f}
\begin{verbatim}
**********************************************************************
*** function dshrmudiff calculates the flux of diffuse neutrino-
*** induced muons from neutralino annihilation in the halo.
*** the flux given, is the differential flux at the requested energy.
*** there are some approximations going on for the higgses assuming
*** de/dx for muons are constant to simplify the integration. the
*** errors for this shouldn't be too big.
*** units: km^-2 yr^-1 sr^-1 gev^-1 (if jpsi is given as 1st arg.)
*** units: km^-2 yr^-1 gev^-1       (if jpsi*delta is given as 1st arg.) 
*** dnsigma is also returned, which is 
***   dN_mu/dE_mu * (sigma v) / (10^-29 cm^3 s^-1)
*** in units of GeV^-1.
*** author: joakim edsjo, edsjo@physto.se
*** date: 98-05-07
*** modified: 00-09-03
**********************************************************************

      real*8 function dshrmudiff(jpsi,emu,dnsigma,istat)
\end{verbatim}
 \end{routine}

%%%%% routine dshrmuhalo.f %%%%%
\begin{routine}{dshrmuhalo.f}
\begin{verbatim}
**********************************************************************
*** function dshrmuhalo calculates the flux of diffuse neutrino-
*** induced muons from neutralino annihilation in the halo.
*** the flux given, is the total flux above a given threshold.
*** there are some approximations going on for the higgses assuming
*** de/dx for muons are constant to simplify the integration. the
*** errors for this shouldn't be too big.
*** units: km^-2 yr^-1 sr^-1 (if jpsi is given as 1st arg.)
***        km^-2 yr^-1       (if jpsi*delta is given as 1st arg.)
*** dnsigma is also returned, which is the dimensionless number
***   N_mu * (sigma v) / (10^-29 cm^3 s^-1)
*** author: joakim edsjo, edsjo@physto.se
*** date: 98-05-07
*** modified: 00-09-03
***           Joakim Edsjo (edsjo@physto.se) 03-01-21, factor of 1/2
***           in annihilation rate added
**********************************************************************

      real*8 function dshrmuhalo(jpsi,eth,dnsigma,istat)
\end{verbatim}
 \end{routine}

%%%%% routine dshrpbardiff.f %%%%%
\begin{routine}{dshrpbardiff.f}
\begin{verbatim}
      real*8 function dshrpbardiff(tp,solarmod,how)

**********************************************************************
*** function dshrpbardiff calculates the differential flux of
*** antiprotons for the antiproton kinetic energy tp as a result of
*** neutralino annihilation in the halo.
*** compared to dshrpbdiff0, dshrpbardiff uses the rescaled local density
***   input:
***     tp - antiproton kinetic energy in gev
***     solarmod - 0 no solar modulation
***                1 solar modulation a la perko
***     how - 1 calculate t_diff only for requested momentum
***           2 tabulate t_diff for first call and use table for
***             subsequent calls
***           3 as 2, but also write the table to disk as pbtd.dat
***             at the first call
***           4 read table from disk on first call, and use that for
***             subsequent calls
*** units: gev^-1 cm^-2 s^-1 sr^-1
*** author: 00-07-19 paolo gondolo
**********************************************************************
\end{verbatim}
 \end{routine}

%%%%% routine dshrpbdiff0.f %%%%%
\begin{routine}{dshrpbdiff0.f}
\begin{verbatim}
      real*8 function dshrpbdiff0(tp,solarmod,how)

**********************************************************************
*** function dshrpbdiff0 calculates the differential flux of
*** antiprotons for the antiproton kinetic energy tp as a result of
*** neutralino annihilation in the halo.
*** dshrpbdiff0 uses the unrescaled local density
*** see dshrpbardiff for rescaling the local density
***   input:
***     tp - antiproton kinetic energy in gev
***     solarmod - 0 no solar modulation
***                1 solar modulation a la perko
***     how - 1 calculate t_diff only for requested momentum
***           2 tabulate t_diff for first call and use table for
***             subsequent calls
***           3 as 2, but also write the table to disk as pbtd.dat
***             at the first call
***           4 read table from disk on first call, and use that for
***             subsequent calls
*** units: gev^-1 cm^-2 s^-1 sr^-1
*** author: joakim edsjo
*** date: 98-02-10
*** modified: joakim edsjo, edsjo@physto.se
*** modified: 98-07-13, 00-07-19 paolo gondolo
***           Joakim Edsjo (edsjo@physto.se) 03-01-21, factor of 1/2
***           in annihilation rate added
**********************************************************************

\end{verbatim}
 \end{routine}

%%%%% routine dsnsigvgacdiff.f %%%%%
\begin{routine}{dsnsigvgacdiff.f}
\begin{verbatim}
**********************************************************************
*** subroutine dsnsigvgacdiff gives the number of contiuous gammas
*** per GeV at the energy ega (in GeV) times the annihilation cross
*** section.
*** The result given is the number
***     N_gamma * (sigma * v) / (10-29 cm^3 s^-1)
*** in units of GeV^-1.
***
*** author: joakim edsjo, edsjo@physto.se
*** date: 00-09-03
**********************************************************************

      subroutine dsnsigvgacdiff(ega,nsigvgacdiff)
\end{verbatim}
 \end{routine}

%%%%% routine dsnsigvgacont.f %%%%%
\begin{routine}{dsnsigvgacont.f}
\begin{verbatim}
**********************************************************************
*** subroutine dsnsigvgacont gives the number of photons above
*** the threshold egath (in GeV) times the annihilation cross section
*** into continuous gammas.
*** The result given is the dimensionless number
***     N_gamma * (sigma * v) / (10-29 cm^3 s^-1)
***
*** author: joakim edsjo, edsjo@physto.se
*** date: 00-09-03
**********************************************************************

      subroutine dsnsigvgacont(egath,nsigvgacont)
\end{verbatim}
 \end{routine}

%%%%% routine dsnsigvgaline.f %%%%%
\begin{routine}{dsnsigvgaline.f}
\begin{verbatim}
**********************************************************************
*** subroutine dsnsigvgaline gives the number of photons times
*** the annihilation cross section into gamma gamma and Z gamma
*** respectively. The result given is the dimensionless number
***     N_gamma * (sigma * v) / (10-29 cm^3 s^-1)
***
*** author: joakim edsjo, edsjo@physto.se
*** date: 00-09-03
**********************************************************************

      subroutine dsnsigvgaline(nsigvgaga,nsigvgaz)
\end{verbatim}
 \end{routine}

\newpage
\chapter[ini: Initialization routines]{\codeb{src/ini}:\\ Initialization routines}
\label{ch:src-ini}

%%%%%%%%%%%%%%%%%%%%%%%%%%%%%%%%%%%%%%%%%%%%%%%%%%%%%%%%%%%%%%%%%%%%

\section{Initialization routines}

Before \ds\ is used for some calculations, it needs to be
initialized. This is done with a call to \codeb{dsinit}. This routine
makes sure that all standard parameters are defined, such as standard
model parameters and particle codes. It also calls the different
\codeb{ds*set} routines with the argument \code{default}. E.g., the
halo model is set to the default choice with a call to
\codeb{dshmset}\code{('default')}. Analogously, all other routines
with a \codeb{ds*set} routine is also called to set them up to the
default model/parameters.

This means that the call to \codeb{dsinit} should be the first call in
any program using \ds. Any calls the user makes to other routines,
either to calculte things or select a different model (e.g.\ a
different halo model) should come after the call to \codeb{dsinit}.
\section{Routine headers -- fortran files}

%%%%% routine dscval.f %%%%%
\begin{routine}{dscval.f}
\begin{verbatim}
      function dscval(a)
No header found.
\end{verbatim}
 \end{routine}

%%%%% routine dsfval.f %%%%%
\begin{routine}{dsfval.f}
\begin{verbatim}
      function dsfval(a)
No header found.
\end{verbatim}
 \end{routine}

%%%%% routine dsinit.f %%%%%
\begin{routine}{dsinit.f}
\begin{verbatim}
      subroutine dsinit
No header found.
\end{verbatim}
 \end{routine}

%%%%% routine dsival.f %%%%%
\begin{routine}{dsival.f}
\begin{verbatim}
      function dsival(a)
No header found.
\end{verbatim}
 \end{routine}

%%%%% routine dskillsp.f %%%%%
\begin{routine}{dskillsp.f}
\begin{verbatim}
      function dskillsp(a1,a2)
No header found.
\end{verbatim}
 \end{routine}

%%%%% routine dslowcase.f %%%%%
\begin{routine}{dslowcase.f}
\begin{verbatim}
      subroutine dslowcase(a)
No header found.
\end{verbatim}
 \end{routine}

%%%%% routine dslval.f %%%%%
\begin{routine}{dslval.f}
\begin{verbatim}
      function dslval(a)
No header found.
\end{verbatim}
 \end{routine}

%%%%% routine dsreadpar.f %%%%%
\begin{routine}{dsreadpar.f}
\begin{verbatim}
      subroutine dsreadpar(unit)
No header found.
\end{verbatim}
 \end{routine}

\newpage
\chapter[mu: Muon neutrino yields from annihilation in the Sun/Earth]{\codeb{src/mu}:\\ Muon neutrino yields from annihilation in the Sun/Earth}
\label{ch:src-mu}

%%%%%%%%%%%%%%%%%%%%%%%%%%%%%%
%%%%%%%%%%%%%%%%%%%%%%%%%%%%%%
\section{Muon yields from annihilation in the Earth/Sun -- theory}

We need to take into account all processes that yield muon neutrinos from
annihilation in the Earth/Sun.

%%%%%%%%%%%%%%%%%%%%
%%%%%%%%%%%%%%%%%%%%%%%%%%%%%%
\subsection{Monte Carlo simulations}
\label{sec:nt-mcsim}

We need to
evaluate the yield of different particles per neutralino annihilation.
The hadronization and/or decay of the annihilation products are
simulated with {\sc Pythia} \cite{pythia} 6.154
and we here describe how the simulations are done.
For annihilation in the Sun/Earth 
the simulations are done for a set of 18 neutralino
masses, $m_{\chi}$ = 10, 25, 50, 80.3, 91.2, 100, 150, 176, 200, 250,
350, 500, 750, 1000, 1500, 2000, 3000 and 5000 GeV\@. We tabulate the
yields and then interpolate these tables in \ds.

We are mainly interested in the flux of high energy muon neutrinos
     and neutrino-induced muons at a neutrino telescope.  We simulate 6
     `fundamental' annihilation channels, $c\bar{c}$, $b\bar{b}$,
     $t\bar{t}$, $\tau^+\tau^-$, $W^+W^-$ and $Z^0 Z^{0}$ for each mass
     (where kinematically allowed) above. The lighter leptons and
     quarks don't contribute significantly and can safely be
     neglected. \comment{Include $gg$?}  Pions and kaons get stopped
     before they decay and are thus made stable in the {\sc Pythia}
     simulations so that they don't produce any neutrinos.  For
     annihilation channels containing Higgs bosons, we can calculate
     the yield from these fundamental channels by letting the Higgs
     bosons decaying in flight (see below).  We also take into account
     the energy losses of $B$-mesons in the Sun and the Earth by
     following the approximate treatment of \cite{RS} but with updated
     $B$-meson interaction cross sections as given in
     \cite{joakimthesis}.  We also take neutrino-interactions on the
     way out of the Sun into account by considering the charged-current
     interaction as a neutrino-loss and the neutral current
     interactions are simulated with {\sc Pythia}.  The
     neutrino-nucleon charged current interactions close to the
     detector are also simulated with {\sc Pythia} and finally the
     multiple Coulomb scattering of the muon on its way to the detector
     is calculated using distributions from~\cite{PDG}. We have used the ???
     structure functions in these simulations.
     For more details on these simulations, see~\cite{Edpre,Angdist}.

     \comment{Structure functions?}

     For each annihilation channel and mass we simulate $1.25 \times
     10^{6}$ annihilations and tabulate the final results as a
     neutrino-yield, neutrino-to-muon conversion rate and a muon yield
     differential in energy and angle from the center of the Sun/Earth.
     We also tabulate the integrated yield above a given threshold and
     below an open-angle $\theta$. We assumed throughout that the
     surrounding medium is water with a density of 1.0 g/cm$^3$. Hence,
     the neutrino-to-muon conversion rates have to be multiplied by the
     density of the medium. In the muon fluxes, the density cancels out
     (to within a few percent).
     \comment{Neutrino-nucleon cross sections?}
     \comment{Simulate for rock?}


With these simulations, we can calculate the yield for any of these
particles for a given MSSM model.  For the Higgs bosons, which decay
in flight, an integration over the angle of the decay products with
respect to the direction of the Higgs boson is performed.  Given the
branching ratios for different annihilation channels it is then
straightforward to compute the muon flux above any given energy
threshold and within any angular region around the Sun or the center
of the Earth.

\section{Routine headers -- fortran files}

%%%%% routine dsmucom.f %%%%%
\begin{routine}{dsmucom.f}
\begin{verbatim}
No header found.
\end{verbatim}
 \end{routine}

%%%%% routine dsmudydth.f %%%%%
\begin{routine}{dsmudydth.f}
\begin{verbatim}
*****************************************************************************
*** function dsmudydth is the differential yield dyield/dtheta which
*** should be integrated (by the routine gadap).
*** units: 1.0e-15 m**-2 (annihilation)**-1
*****************************************************************************

      real*8 function dsmudydth(th)
\end{verbatim}
 \end{routine}

%%%%% routine dsmuemean.f %%%%%
\begin{routine}{dsmuemean.f}
\begin{verbatim}
******************************************************************************
*** function dsmuemean is used to calculate the mean energy of a decay product
*** when a moving particle decays. e0 and m0 are the energy and mass of
*** the moving particle and m1 and m2 are the masses of the decay products.
*** it is the mean energy of m1 that is returned. all energies and masses
*** should be given in gev.
******************************************************************************

      real*8 function dsmuemean(e0,m0,m1,m2)
\end{verbatim}
 \end{routine}

%%%%% routine dsmuifind.f %%%%%
\begin{routine}{dsmuifind.f}
\begin{verbatim}
***********************************************
*** routine to find the index of an entry   ***
*** the closest lowest hit is given          ***
***********************************************

      subroutine dsmuifind(value,array,ipl,ii,imin,imax)
\end{verbatim}
 \end{routine}

%%%%% routine dsmuinit.f %%%%%
\begin{routine}{dsmuinit.f}
\begin{verbatim}
*****************************************************************************
*** subroutine dsmuinit initializes and loads (from disk) the common
*** block variables needed by the other muon yield routines.
*** flxk is the yield type (1,2 or 3 (or 101, 102, 103)) for neutrino yields
*** muon distributions at creation or muon yields at a detector respectively.
*** flxk is used to check that the provided data file is of the correct type.
*** if flxk=1,2 or 3 integrated yields are loaded and if flxk=101, 102 or
*** 103, differential yields are loaded
*** author: joakim edsjo  edsjo@physto.se
*** date: 96-10-23
*** modified: 97-12-03
*****************************************************************************

      subroutine dsmuinit(flxk)
\end{verbatim}
 \end{routine}

%%%%% routine dsmuyield.f %%%%%
\begin{routine}{dsmuyield.f}
\begin{verbatim}
*****************************************************************************
*** function yield calculates the yield above threshold (flxk=1,2 or 3)
*** or the differential yield (flxk=101, 102 or 103) from a given
*** annihilation channel. channels ch=1-14 are supported.
*** Note. Gluons (channel 12) are not included yet, this channel
*** returns a zero yield. Channel 13 (mu+ mu-) never yields anything,
*** but is included for compatibility with the halo annihilation routines.
*** if flxk = 1 or 101 - neutrino yields are given
***           2 or 102 - muon distributions at creation are given
***           3 or 103 - muon yields at the detector are given
*** the units are 1e-30 m**-2 (annihilation)**-1 for 1 and 3, and
*** 1e-30 m**-3 (annihilation)**-1 for 2.
*** For the differential yields, the units are the same plus
*** gev**-1 degree**-1.
***
*** author: joakim edsjo (edsjo@physics.berkeley.edu)
*** date: 1995
*** modified: dec 03, 1997
*****************************************************************************

      real*8 function dsmuyield(mneu,emuthr,thmax,ch,wh,flxk,istat)
\end{verbatim}
 \end{routine}

%%%%% routine dsmuyield_int.f %%%%%
\begin{routine}{dsmuyield\_int.f}
\begin{verbatim}
      real*8 function dsmuyield_int(f,a,b)
c_______________________________________________________________________
c  integrate function f between a and b
c  input
c    integration limits a and b
c  called by dsmuyieldfth
c  author: joakim edsjo (edsjo@physto.se) 96-05-16
c  based on paolo gondolos wxint.f routine.
c=======================================================================
\end{verbatim}
 \end{routine}

%%%%% routine dsmuyieldf.f %%%%%
\begin{routine}{dsmuyieldf.f}
\begin{verbatim}
*****************************************************************************
*** function dsmuyieldf calculates the muon yield above threshold and within
*** the selected angular window (yieldk=1) or the differential yield
*** (yieldk=2) for channel ch.
*** yieldv=1 (nu flux), yieldv=2 (nu-to-mu conv.rate), yieldv=3 (mu flux)
*** only channels ch = 1-6 are supported.
*** units: 1.0e-30 m**-2 (annihilation)**-1  integrated
*** units: 1.0e-30 m**-2 gev**-1 (degree)**-1 (annihilation)**1 differential
*** author: joakim edsjo, edsjo@physics.berkeley.edu
*** date: 1995
*** modified: dec 03, 1997.
*****************************************************************************

      real*8 function dsmuyieldf(mneu,emuthr,thmax,ch,wh,yieldk,
     &  yieldv,istat)
\end{verbatim}
 \end{routine}

%%%%% routine dsmuyieldfth.f %%%%%
\begin{routine}{dsmuyieldfth.f}
\begin{verbatim}
*****************************************************************************
*** function phiith integrates dsmudydth over the angle theta.
*** it is the yield from particle 1 (which decays from m0)
*** that is calculated. particle one corresponds to channel ch.
*** units: 1.0e-30 m**-2 (annihilation)**-1
*****************************************************************************

      real*8 function dsmuyieldfth(e0,m0,mp1,mp2,emuthr,thmax,ch,wh,
     &  yieldk,yieldv,istat)
\end{verbatim}
 \end{routine}

%%%%% routine dsmuyieldh.f %%%%%
\begin{routine}{dsmuyieldh.f}
\begin{verbatim}
*****************************************************************************
*** function dsmuyieldh calculates the yield above threshold (yieldk=1) or the
*** differential yield (yieldk=2) from a given higgs
*** boson decaying in flight, the energy of the higgs boson should be given
*** in eh.
*** higgses hno = 1-4 are supported (h10, h20, h30 and h+/h-)
*** units: 1.0e-30 m**-2 (annihilation)**-1
*****************************************************************************

      real*8 function dsmuyieldh(eh,emuth,thmax,hno,wh,yieldk,
     &  yieldv,istat)
\end{verbatim}
 \end{routine}

%%%%% routine dsmuyieldh2.f %%%%%
\begin{routine}{dsmuyieldh2.f}
\begin{verbatim}
*****************************************************************************
*** function dsmuyieldh2 calculates the yield above threshold (yieldk=1) or the
*** differential yield (yieldk=2) from a given higgs
*** boson decaying in flight, the energy of the higgs boson should be given
*** in eh.
*** higgses hno = 1-4 are supported (h10, h20, h30 and h+/h-)
*** units: 1.0e-30 m**-2 (annihilation)**-1
*****************************************************************************

      real*8 function dsmuyieldh2(eh,emuth,thmax,hno,wh,
     &  yieldk,yieldv,istat)
\end{verbatim}
 \end{routine}

%%%%% routine dsmuyieldh3.f %%%%%
\begin{routine}{dsmuyieldh3.f}
\begin{verbatim}
*****************************************************************************
*** function dsmuyieldh3 calculates the yield above threshold (yieldk=1) or the
*** differntial yield (yieldk=2) from a given higgs
*** boson decaying in flight, the energy of the higgs boson should be given
*** in eh.
*** higgses hno = 1-4 are supported (h10, h20, h30 and h+/h-)
*** units: 1.0e-30 m**-2 (annihilation)**-1
*****************************************************************************

      real*8 function dsmuyieldh3(eh,emuth,thmax,hno,wh,
     &  yieldk,yieldv,istat)
\end{verbatim}
 \end{routine}

%%%%% routine dsmuyieldh4.f %%%%%
\begin{routine}{dsmuyieldh4.f}
\begin{verbatim}
*****************************************************************************
*** function dsmuyieldh4 calculates the yield above threshold (yieldk=1) or the
*** differential yield (yieldk=2) from a given higgs
*** boson decaying in flight, the energy of the higgs boson should be given
*** in eh.
*** higgses hno = 1-4 are supported (h10, h20, h30 and h+/h-)
*** units: 1.0e-30 m**-2 (annihilation)**-1
*****************************************************************************

      real*8 function dsmuyieldh4(eh,emuth,thmax,hno,wh,
     &  yieldk,yieldv,istat)
\end{verbatim}
 \end{routine}

\newpage
\chapter[nt: Neutrino and muon rates from annihilation in the Sun/Earth]{\codeb{src/nt}:\\ Neutrino and muon rates from annihilation in the Sun/Earth}
\label{ch:src-nt}

%%%%%%%%%%%%%%%%%%%%%%%%%%%%%%
%%%%%%%%%%%%%%%%%%%%%%%%%%%%%%
\section{Neutrinos from the Sun and Earth --  theory}

There are several indirect methods for detection of neutralinos.
One of the most promising \cite{neutrinos} is to make use of the
fact that scattering of halo neutralinos by the Sun and the planets,
in particular the Earth, during the several
billion years that the Solar system has existed, will have trapped
these neutralinos within these astrophysical bodies. Being trapped
within the Solar or terrestrial material, they will sink towards
the center, where a considerable enrichment and corresponding
increase of annihilation rate will occur.


Searches for neutralino annihilation into neutrinos
will be subject to  extensive experimental investigations in view
of the new neutrino telescopes (AMANDA, IceCube, Baikal, NESTOR, ANTARES)
planned or under construction \cite{halzen}. A high-energy
neutrino signal in the direction of the centre of the Sun or Earth
is an excellent experimental signature which may stand up against
the background of neutrinos generated by cosmic-ray interactions in the
Earth's atmosphere.

There are several different approximations one could do, or processes to
include when calculating the capture rates in the Earth/Sun and many of these
are coded into \ds. The default in \ds\ is always to use the best calculations
available, but more approximate (older) routines are also available,
as well as more speculative signals, like the Damour-Krauss signal
(not included by default). If you want to use something else than the
defaults, or want to call more internal rotuines (more internal than
\code{dsntrates} or \code{dsntdiffrates}), you should read the
following sections carefully.


%%%%%%%%%%%%%%%%%%%%
\subsection{Neutrino yield from annihilations}

The differential neutrino flux from neutralino annihilation is
\beq
\frac{dN_\nu}{dE_\nu} =
\frac{\Gamma_A}{4\pi D^2} \sum_{f}
B^{f}_{\chi}\frac{dN^f_\nu}{dE_\nu}
\eeq
where $\Gamma_A$ is the annihilation rate,
$D$ is the distance of the detector from the source (the
central region of the Earth or the Sun), $f$ is the neutralino pair
annihilation final states,
and $B^{f}_{\chi}$ are the branching ratios into the final state $f$.
  $dN^f_\nu/dE_{\nu}$ are the energy
distributions of  neutrinos generated by the final state $f$ and are
obtained from the {\sc Pythia} simulations described in section
\ref{sec:mcsim}.

In comparison with calculations using the results of \cite{RS}
(e.g.\ ~\cite{neuprod}), this
Monte Carlo treatment of the neutrino propagation through the Sun
does not need the simplifying assumptions previously made, namely neutral
currents are no more assumed to be much weaker than charged currents
and energy loss is no more considered continuous.

The neutrino-induced muon flux may be detected in a neutrino telescope
by measuring the muons that come from the direction of the centre
of the Sun or Earth. For a shallow detector, this usually has to
be done in the case of the Sun by looking (as always the case for
the Earth) at upward-going muons, since there is a huge background
of downward-going muons created by cosmic-ray interactions in the
atmosphere. There is always in addition a more isotropic
background coming from muon neutrinos created on the other side of
the Earth in such cosmic-ray events (and also from cosmic-ray
interactions in the outer regions of the Sun).
The flux of muons at the detector is  given by

\beq
\frac{d N_\mu}{d E_\mu}
= N_A \int^\infty_{E_\mu^{\rm th}} d E_\nu
\int_0^\infty d\lambda \int_{E_\mu}^{E_\nu}
d {E'_\mu }\,\,
P(E_\mu,E'_\mu; \lambda)\,\,
\frac{d \sigma_\nu (E_\nu,E'_\mu)}{d E'_\mu} \,\,
\frac{d N_\nu}{d E_\nu}\, ,
\label{eq:muflux}
\eeq
where $\lambda$ is the muon range in the medium (ice or water
for the large detectors in the ocean or at the South Pole,
or rock which surrounds the smaller underground detectors),
$d \sigma_\nu (E_\nu,E'_\mu) / d E'_\mu$ is
the weak interaction cross section for production of a muon of
energy $E'_\mu$ from a parent neutrino of energy $E_\nu$, and
$P(E_\mu,E'_\mu; \lambda)$ is the
probability for a muon of initial energy $E'_\mu$
to have a final energy $E_\mu$ after passing
  a path--length $\lambda$ inside the detector medium.
$E_\mu^{\rm th}$ is the detector threshold energy, which for
``small''
neutrino telescopes like Baksan, MACRO and Super-Kamiokande is
around 1 GeV.
Large area neutrino telescopes in the ocean  or in Antarctic ice
typically
have thresholds of the order of tens of GeV, which makes them
sensitive mainly to heavy neutralinos (above 100 GeV)
\cite{begnu2}.

The integrand in Eq.~(\ref{eq:muflux}) is weighted towards high
neutrino energies, both because the cross section $\sigma_\nu$
rises approximately linearly with energy and because the average
muon energy, and therefore the range $\lambda$, also grow
approximately linearly with $E_\nu$. Therefore, final states
which give a hard neutrino spectrum (such as heavy quarks, $\tau$
leptons and $W$ or $Z$ bosons) are usually more important
than the soft spectrum arising from light quarks and gluons.

%%%%%%%%%%%%%%%%%%%%
\subsection{Evolution of the number density in the Earth/Sun}

Neutralinos are steadily being trapped in the Sun or Earth by
scattering, whereas annihilations take them away.
Let $N(t)$ be the total number of neutralinos trapped, at time $t$, in the core
of, for example,  the Earth.
The annihilation rate of neutralino pairs can be written as
\begin{equation}
\Gamma_a (t) = \frac{1}{2} \ C_a \, N^2 (t) \, . \label{eqx.1}
\end{equation}

The evolution of $N(t)$ is the result of the competition between capture and
annihilation:
\begin{equation}
\frac{dN}{dt} = C_c (t) - C_a \, N^2 \label{eqx.2}
\end{equation}
The constant $C_c$ is the capture rate, and
$C_a$ entering equations (\ref{eqx.1}) and (\ref{eqx.2}) is linked to
the annihilation cross-section $\sigma_a$, and to some effective volumes $V_j$,
$j=1,2$, taking into account the quasi-thermal distribution of neutralinos in
the Earth core:
\begin{equation}
C_a = \langle \sigma_a \, v \rangle \, \frac{V_2 }{ V_1^2} \, , \label{eq6.3}
\end{equation}
\begin{equation}
V_j \simeq 2.3 \times 10^{25} \, \left(\frac{j \, m_X}{ 10 \, {\rm 
GeV}}\right)^{-3/2} \, {\rm
cm}^3 \, . \label{eq6.4}
\end{equation}

This has the solution for the annihilation rate implemented in \ds\
\beq
\Gamma_A={C_c\over 2} \tanh^2\left({t\over \tau}\right),\label{eq:tanh}
\eeq
where the equilibration time scale $\tau=1/\sqrt{C_cC_a}$.  In most
cases for the Sun, and in the cases of observable fluxes for the
Earth, $\tau$ is much smaller than a few billion years, and therefore
equilibrium is often a good approximation ($\dot N(t)=0$).
This means
that it is the capture rate which is the important quantity that
determines the neutrino flux. However, in the program we keep the exact
formula (\ref{eq:tanh}), with some
modifications discussed in Sec.~\ref{sec:dk}).

%%%%%%%%%%%%%%%%%%%%
\subsection{Approximate capture rate expressions}

The capture rate induced by scalar (spin-independent) interactions
between the neutralinos and the nuclei in the interior of the Earth or
Sun is the most difficult one to compute, since it depends sensitively
on the Higgs mass, form factors, and other poorly known quantities.
However, this spin-independent capture rate calculation is the same as
for direct detection treated in Section~\ref{subs:direct}.  Therefore,
there is a strong correlation between the neutrino flux expected from
the Earth (which is mainly composed of spin-less nuclei) and the
signal predicted in direct detection experiments \cite{begnu2,kamsad}.
It seems that even the large (kilometer-scale) neutrino telescopes
planned, when searching for neutralino annihilation in the Earth, will
not be competitive with the next generation of direct detection
experiments when it comes to detecting neutralino dark matter.
However, the situation concerning the Sun is more favourable.  Due to
the low counting rates for the spin-dependent interactions in
terrestrial detectors, high-energy neutrinos from the Sun constitute a
competitive and complementary neutralino dark matter search.  Of
course, even if a neutralino is found through direct detection, it
will be extremely important to confirm its identity and investigate
its properties through indirect detection.  In particular, the mass
can be determined with reasonable accuracy by looking at the angular
distribution of the detected muons \cite{EG,BEK}.

For the Sun, dominated by hydrogen, the axial (spin-dependent)
cross section is important and relatively easy to compute.  A
reasonably good
approximation is given by \cite{jkg}
\beq
      {C^{\rm sd}_\odot\over (1.3\cdot 10^{23}\, {\rm s}^{-1})
\left(270\ {\rm km\,s^{-1}}/ \bar v\right)}=
\left({\rho_\chi\over 0.3\ {\rm GeV}\,{\rm cm}^{-3}}\right)
      \left({100\,{\rm GeV}\over m_\chi}\right)
\left({\sigma_{p\chi}^{\rm sd}\over 10^{-40}\ {\rm
      cm}^2}\right)
\eeq
where $\sigma_{p\chi}^{\rm sd}$ is the cross section for
neutralino-proton elastic scattering via the axial-vector interaction,
$\bar v$ is the dark-matter velocity dispersion, and $\rho_\chi$ is
the local dark matter mass.  The capture rate in the Earth is
dominated by scalar interactions, where there may be kinematic and
other enhancements, in particular if the mass of the neutralino almost
matches one of the heavy elements in the Earth.  For this case, a more
detailed analysis is called for, which is available in \cite{Gould87}
with convenient approximations in \cite{jkg}.
   In fact, also for the
Sun the spin-independent contribution can be important, in particular
iron may contribute non-negligibly.
For the Sun, the
  approximation in \cite{jkg} is also available,
\beqa
      {C^{\rm si}_\odot\over (4.8\cdot 10^{22}\, {\rm s}^{-1})
\left(270\ {\rm km\,s^{-1}}/ \bar v\right)}=
\left({\rho_\chi\over 0.3\ {\rm GeV}\,{\rm cm}^{-3}}\right)
      \left({100\,{\rm GeV}\over m_\chi}\right)\times & &\nonumber\\
\sum_A\left({\sigma_{A}^{\rm si}\over 10^{-40}\ {\rm
      cm}^2}\right)F_A(m_\chi)f_A\phi_AS\left(m_\chi/m_{A}\right)/m_{A},
\eeqa
where $f_A$ is the mass fraction of element $A$ and $\phi_A$ is the
typical gravitational potential (relative to the surface) for that
element. I.e.\ an element that is concentrated in the core will have a
higher $\phi_A$ than an element at the surface. $A$ is the atomic
number of the element and $M_{A}$ is its mass. The factor $S$ is a
kinematical suppression factor \cite{jkg,kamion91}. In the next
subsection we will go through the compositions of the Earth/Sun that
we use.

The approximate capture rate expressions above are coded into the routines
\code{dsntcapsun} and \code{dsntcapearth}. More accurate expressions
will follow in the coming subsections.


%%%%%%%%%%%%%%%%%%%%%%%%%%%%%%
\subsection{Earth and Sun composition}

When the capture rates are calculated, we need to know the composition
and density of the Earth/Sun as a function of depth.  

In \cite{jkg} they used average mass fractions and potentials for the
location of the various elements in the Sun. We have updated these to
the BP2000 \cite{bp2000} values instead, as given in Table~\ref{tab:suncomp}

\begin{table}
  \centering
  \begin{tabular}{lccc}
   & & \multicolumn{2}{c}{Average parameters} \\ \cline{3-4}
   Element & Mass number ($A$) & $f_i$ & $\phi_i$ \\ \hline
   Hydrogen, H       &  1 & 0.670    & 3.15 \\
   Helium-4, $^4$He  &  4 & 0.311    & 3.40 \\
   Carbon, C         & 12 & 0.00237  & 2.85 \\
   Nitrogen, N       & 14 & 0.00188  & 3.83 \\
   Oxygen, O         & 16 & 0.00878  & 3.25 \\
   Neon, Ne          & 20 & 0.00193  & 3.22 \\
   Magnesium, Mg     & 24 & 0.000733 & 3.22 \\
   Silicon, Si       & 28 & 0.000798 & 3.22 \\
   Sulphur, S        & 32 & 0.000550 & 3.22 \\
   Iron, Fe          & 56 & 0.00142  & 3.22 \\ \hline
\end{tabular}
\caption{The composition of the Sun with average parameters to be used
  in the approximative relations given in \cite{jkg}. These values are
  updated with the solar model of \cite{bp2000} and differs slightly
  from the values used in \cite{jkg}.}
  \label{tab:suncomp}
\end{table}

\begin{table}
  \centering 
  \begin{tabular}{lcccccc}
& Mass & \multicolumn{2}{c}{Mass fraction} & & \multicolumn{2}{c}{Average parameters}
\\ \cline{3-4} \cline{6-7}
 Element & number ($A$) & Core & Mantle & & $f_i$ & $\phi_i$ \\ \hline
  Oxygen, O     & 16 & 0.0   & 0.440   & & 0.298   & 1.20  \\
  Silicon, Si   & 28 & 0.06  & 0.210   & & 0.162   & 1.24  \\
  Magnesium, Mg & 24 & 0.0   & 0.228   & & 0.154   & 1.20  \\
  Iron, Fe      & 56 & 0.855 & 0.0626  & & 0.319   & 1.546 \\
  Calcium, Ca   & 40 & 0.0   & 0.0253  & & 0.0171  & 1.20  \\
  Phosphor, P   & 30 & 0.002 & 0.00009 & & 0.00071 & 1.56  \\
  Sodium, Na    & 23 & 0.0   & 0.0027  & & 0.00183 & 1.20  \\
  Sulphur, S    & 32 & 0.019 & 0.00025 & & 0.0063  & 1.59  \\
  Nickel, Ni    & 59 & 0.052 & 0.00196 & & 0.0181  & 1.57  \\
  Aluminum, Al  & 27 & 0.0   & 0.0235  & & 0.0159  & 1.20  \\
  Chromium, Cr  & 52 & 0.009 & 0.0026  & & 0.0047  & 1.44  \\ \hline
\end{tabular}
  \caption{The composition of the Earth's core and mantle. The core
    mass fractions are from \cite{earthcomp}[Table 4] and the mantle
    mass fractions are from \cite{earthcomp}[Table 2]. The average
    mass fractions and potentials in the last two columns are weighted
    averages assuming a core mass of $1.93\cdot10^{24}$ kg and a
    mantle mass of $4.04\cdot10^{24}$ kg with average potentials
    (relative to the surface) of 1.6 in the core and 1.2 in the mantle
    \cite{Gould87}.} \label{tab:earthcomp} 
\end{table}

For the Earth, we have also implemented more accurate
density profiles and more up-to date chemical distributions within the
Earth. We use the estimates for the Earth composition given in
\cite{earthcomp}[Table 2 for the mantle and Table 4 for the core]. In
Table~\ref{tab:earthcomp} we list these values together with the
average parameters $f_i$ and $\phi_i$ that should be used in the
expressions for the approximate capture rates in the previous
section. Note that using these average parameters instead of
integrating over the full radius is equivalent to putting all the
elements of the give type at the gravitational potential $\phi_i$.

\begin{figure}
\centerline{\epsfig{file=fig/eadensvesc.eps,width=0.6\textwidth}}
\caption{In a) the density profile and in b) the escape velocity in the Earth is shown.}
\label{fig:eadensvesc}
\end{figure}

We also need the density profile of the Earth, and for this we use the
values in \cite{EncBrit}. Using this density profile, we can calculate
the gravitational potential, $\phi(r)$ inside the Earth and from this
one the escape velocity $v$ inside the Earth,
\begin{equation}
   \label{eq:vesc}
   v = 11.2 \sqrt{\frac{\phi(r)}{\phi(R_\oplus)}} \mbox{~km/s.}
\end{equation}
In Fig.~\ref{fig:eadensvesc} we show the density profile and escape
velocity inside the Earth. 

%%%%%%%%%%%%%%%%%%%%%%%%%%%%%%%%%%%%%%%%
\subsection{More accurate capture rate expressions}

Another complicating factor when calculating the capture rates is the
integration over the velocity 
distribution. In \cite{Gould87}, parts of the integrations are
performed analytically for a Gaussian veolocity distributions. These
expressions are also coded in \ds for the Earth and give a more
accurate calculation of the capture rate in the Earth than the
approximations given above. The routine \code{dsntcapearth2} performs
these calculations for the Earth.

%%%%%%%%%%%%%%%%%%%%%%%%%%%%%%
\subsection{Accurate capture rates in the Earth for general velocity distributions}


If one wants even more accurate and general expressions for the
capture rates in the Sun/Earth, we have also implemented the full
expressions in \cite{Gould87}, but without assuming that the velocity
distribution is a Gaussian (or Maxwell-Boltzmann). These routines are
now the default in \ds.

We will here outline how these expressions look
like for the Earth and how they can be used both for a Maxwell-Boltzmann
distribution and for a general velocity distribution. The expressions
will of course look analogously for the Sun. We start with
the general case and study the special case of a Maxwell-Boltzmann
distribution in the next section. 

We will divide the Earth into shells and calculate the capture from
element $i$ in each shell individually. At the end we will integrate
over all the shells and sum over all the elements in the Earth. The
capture rate from element $i$ per unit shell volume is given by
\cite{Gould87}[Eq.~(2.8)] 
\begin{equation}
\label{eq:dcdv}
    \frac{dC_i}{dV} = \int_0^{u_{max}} du \frac{f(u)}{u} w \Omega_{v,i}^-(w)
\end{equation}
where $f(u)$ is the velocity distribution (normalized such that
$\int_0^\infty\ f(u) = n_\chi$ where $n_\chi$ is the number density of
WIMPs. The expression $\Omega_{v,i}^-(w)$ is related to the
probability that we scatter to orbits below the escape velocity. $w$
is the velocity at the given shell and it is related to the velocity
at infinity $u$ and the escape velocity $v$ by 
\begin{equation}
\label{ }
   w = \sqrt{u^2 + v^2}.
\end{equation}
The upper limit of integration is a priori set to $u_{max} = \infty$,
but we will see below that due to kinematical reasons we can set it to
a lower value (Eq.~(\ref{eq:umax}) below). 
If we allow for a form factor suppression of the form \cite{Gould87}[Eq.~(A3)]
\begin{equation}
\label{ }
   |F(q^2)|^2 = \exp\left( - \frac{\Delta E}{E_0} \right)
\end{equation}
with \cite{Gould87}[Eq.~(A4)]
\begin{equation}
\label{ }
   E_0 = \frac{3 \hbar^2}{2m_\chi R^2}
\end{equation}
we can evaluate $w \Omega_{v,i}^-(w)$ and arrive at the expression
\cite{Gould87}[Eq.~(A6)] 
\begin{equation}
\label{eq:womega}
   w \Omega_{v,i}^- (w) = \sigma_i n_i \frac{\mu_+^2}{\mu}2 E_0 \left[
   e^{- \frac{m_\chi u^2}{2E_0}} - e^{-\frac{\mu}{\mu_+^2}m_\chi \frac{u^2+v^2}{2E_0}}
   \right] \Theta\left( \frac{\mu}{\mu_+^2} - \frac{u^2}{u^2+v^2} \right)
\end{equation}
where we have introduced
\begin{equation}
\label{ }
   \mu = \frac{m_\chi}{m_i} \quad ; \quad \mu_\pm = \frac{\mu \pm 1}{2}
\end{equation}
with $m_i$ the mass of element $i$. The Heaviside step function
$\Theta$ plays the role of only including WIMPs that can scatter to a
velocity lower then the escape velocity $v$. To simplify our
calculations we can drop this step function in Eq.~(\ref{eq:womega})
and instead set the upper limit of integration in Eq.~(\ref{eq:dcdv})
to 
\begin{equation}
\label{eq:umax}
  u_{max} = \sqrt{\frac{\mu}{\mu_-^2}} v
\end{equation}
We also need the scattering cross section on element $i$, which can be
written as 
\cite{jkg}[Eq.~(9-25)]
\begin{equation}
\label{eq:sigma_i}
   \sigma_i = \sigma_p A_i^2 \frac{(m_\chi m_i)^2}{(m_\chi+m_i)^2} 
   \frac{(m_\chi + m_p)^2}{(m_\chi m_p)^2}
\end{equation}
where $A_i$ is the atomic number of the element, $m_p$ is the proton
mass and $\sigma_p$ is the scattering cross section on protons. 

We now have what we need to calculate the capture rate. In
Eq.~(\ref{eq:dcdv}) we integrate over the velocity for our chosen
velocity distribution. We then integrate this equation over the radius
of the Earth and sum over all the different elements in the Earth, 
\begin{equation}
\label{ }
   C = \int_0^{R_\oplus} dr \sum_i \frac{dC_i}{dV} 4 \pi r^2
\end{equation}
Note that we have not assumed anything special about our velocity
distribution, it doesn't even have to be isotropic since the
distribution of elements evenly in the shells will make an anisotropic
distribution on average to behave as an isotropic one. 

The routines that calculate the capture rates with these general (and
accurate) expressions are \code{dsntcapearthnum} and
\code{dsntcapsunnum}. As these calculations are somewhat
time-consuming, we have also added a possibility to tabulate the
result and interpolate in these tables. To use (or create, if the
table files are missing) instead call \code{dsntcapearthtab} and
\code{dsntcapsuntab}. These last two routines are the default in
\ds. The velocity distribution used is determined by a switch when the
halo model is set (i.e.\ when \code{dshmset} is called).


%%%%%%%%%%
\subsection{Accurate capture rates for the Earth for a
  Maxwell-Boltzmann velocity distribution}

We will here give some more information on how the approximations
introduced in the beginning of this chapter are derived from the
general expressions in the preceeding section.

If the velocity distribution is of Maxwell-Boltzmann type we can
greatly simplify our expressions above as we can perform the
integration over velocity analytically. The integration over radius
can also be further simplified by using the average mass fractions
$f_i$ and potentials $\phi_i$ in Tables~\ref{tab:suncomp}--\ref{tab:earthcomp}. 

If the velocity distribution in the halo is Maxwell-Boltzmann, it looks like
\begin{equation}
\label{ }
   f_h(u) du = n_\chi \frac{4}{\sqrt{\pi}} \left(\frac{3}{2}\right)^{\frac{3}{2}}
   \frac{u^2}{\bar{v}^3}
   e^{-\frac{3}{2}\frac{u^2}{\bar{v}^2}} du
\end{equation}
where $\bar{v}$ is the three-dimensional velocity dispersion and
$n_\chi$ is the number density of WIMPs in the halo. However, the
solar system moves through the halo with a velocity $v_*$ and the
distribution on observer with this velocity through the halo will see
is 
\begin{equation}
\label{eq:fstaru}
  f_*(u) = f_h(u) e^{-\frac{3}{2}\frac{v_*^2}{\bar{v}^2}} 
  \frac{\sinh\left(\frac{3 u v_*}{\bar{v}^2}\right)}{\frac{3 u v_*}{\bar{v}^2}} =
  n_\chi \sqrt{\frac{3}{2\pi}} \frac{u}{\bar{v} v_*} \left[
  e^{-\frac{3}{2} \frac{( u-v_*)^2}{\bar{v}^2}} -
  e^{-\frac{3}{2} \frac{( u+v_*)^2}{\bar{v}^2}} \right]
\end{equation}
Now one would naively believe that this is not the distribution that
an observer at the Earth will see. First of all, the Earth is moving
with respect to the Sun and secondly, the WIMPs have gained speed by
the gravitational attraction of the Sun when they reach the
Earth. Both of these arguments are true and the distribution of WIMPs
in the halo will not look like Eq.~(\ref{eq:fstaru}) to an observer on
the Earth. However, Gould \cite{Gould91} showed that WIMPs from the
halo can diffuse into the solar system due to gravitational
interactions with the planets and this distribution of WIMPs will
roughly look like as if the Earth was in free space moving through the
halo with the velocity of the solar system,
i.e.\ Eq.~(\ref{eq:fstaru}). We will later scrutinize this statement,
as it turns out that it does not quite hold, but 
as a first guess it is a reasonable approximation. For the Sun,
though, the velocity distribution give above is the correct one for a
Maxwell-Boltzmann distribution.

With the distribution Eq.~(\ref{eq:fstaru}) we can analytically
perform the integration over velocity in Eq.~(\ref{eq:dcdv}). After
some algebra we arrive at \cite{Gould87}[Eq.~(A10)] 
\begin{eqnarray}
\frac{dC_i}{dV} & = & \left( \frac{8}{3\pi} \right)^{\frac{1}{2}} 
\frac{\sigma_i n_i n_\chi \bar{v}}{2b\eta} \nonumber \\
 & & \Bigg[ \frac{e^{-a\hat{\eta}^2}}{\sqrt{1+a}} \left[ 2 \widetilde{\rm erf}(\hat{\eta}) - \widetilde{\rm erf} (\hat{A}_+) + \widetilde{\rm erf} (\hat{A}_-) \right]
   \nonumber \\ 
& &  - \frac{e^{-b\check{\eta}^2}}{\sqrt{1+b}}
   e^{-(a-b)A^2} \left[ 2 \widetilde{\rm erf}(\check{\eta}) - \widetilde{\rm erf} (\check{A}_+) + \widetilde{\rm erf} (\check{A}_-)
   \right] \Bigg]
   \label{eq:dcdvmb}
\end{eqnarray}
where $\widetilde{\rm erf}$ is the modified error function,
\begin{equation}
\label{eq:moderf}
   \widetilde{\rm erf} (x) = \frac{\sqrt{\pi}}{2} {\rm erf} (x)
   \qquad ; \qquad {\rm erf}(x) = \frac{2}{\sqrt{\pi}} \int_0^x e^{-y^2} dy.
\end{equation}
Following Gould \cite{Gould87}, we have in Eq.~(\ref{eq:dcdvmb})
introduced the following shorthand notation: 
\begin{equation}
\begin{array}{ccccc}
\eta = \sqrt{\frac{3}{2}} \frac{v_*^2}{\bar{v}^2} & ; &
a = \frac{m_chi \bar{v}^2}{3E_0} & ; & b=\frac{\mu}{\mu_+^2} a \\
\hat{\eta} = \frac{\eta}{\sqrt{1+a}} & ; & \check{\eta} = \frac{\eta}{\sqrt{1+b}} \\
A^2 = \frac{3}{2} \frac{v^2}{\bar{v}^2} \frac{\mu}{\mu_-^2} & ; &
\hat{A} = A \sqrt{1+a} & ; & \check{A} = A \sqrt{1+b} \\
\hat{A}_\pm = \hat{A} \pm \hat{\eta} & ; & \check{A}_\pm = \check{A} \pm \check{\eta}
\end{array}
\end{equation}
If we wish, we can now integrate Eq.~(\ref{eq:dcdvmb}) over radius
just like in the previous section, but we can without loosing too much
accuracy, replace this integration with a sum over the elements in the
Earth with their respective typical location. I.e.\ we can write 
\begin{equation}
\label{ctotmb}
  C = \sum_i \frac{dC_i}{dV}\frac{1}{n_i} \frac{f_i M_\oplus}{m_i}  
\end{equation}
where we instead of the number density $n_i$ use the total number of
atoms of the given type $f_i M_\oplus/m_i$. Note that for each element
in the sum we should evaluate this expression at the given typical
gravitational potential $\phi_i$ of the element, i.e.\ with the escape
velocity given by Eq.~(\ref{eq:vesc}). The mass fractions $f_i$ and
typical potentials $\phi_i$ are listed in Table
\ref{tab:earthcomp} (and analogously in Table~\ref{tab:suncomp} for
the Sun). This approximation introduces an error of no more
than about 1--2\% for a Maxwell-Boltzmann distribution\footnote{Note
  that it is not advisable to use this approximation for general
  velocity distributions. If one e.g.\ has a lower limit on possible
  velocities, $u_{min}$, for heavy WIMPs capture will then only be
  possible very close to the central core. Replacing the actual
  distribution of potentials $\phi(r)$ with the typical value $\phi_i$
  may then introduce larger errors. We will encounter these kind of
  distributions shortly.} 

The capture rate evaluated with the expressions shown here are encoded
into the routine \code{dsntcapearth2}. Note that we have not coded the
corresponding approximate expressions for the Sun. Instead, as given
in the preceeding section, we now have more accurate expressions for
both the Sun and the Earth.

%%%%%%%%%%%%%%%%%%%%%%%%%%%%%%
\subsection{A possible new population of neutralinos}\label{sec:dk}

Recently, it has been shown that the scattering process in the Sun can
populate orbits which subsequently result in a bound Solar System
population of WIMPs~\cite{dk1,dk2} and which can be comparable in
spectral density, in the region of the Earth, to the Galactic halo
WIMP population.  This new population consists of WIMPs that have
scattered in the outer layers of the Sun and due to perturbations by
the other planets (mainly Venus and Jupiter) evolve into bound orbits
which do not cross the Sun but do cross the Earth's orbit.  This
population of WIMPs should have a completely different velocity
distribution than halo WIMPs and will thus have quite different
capture probabilities in the Earth.  The predicted WIMP abundance, and
spectrum, relevant for direct detection have been calculated in
\cite{dk1,dk2}, where it was shown that although the total rates may
not change by a large amount, there could be a striking directional
effect, which could be of impoertance once detectors with directional
sensitivity are built.  Also for capture in the Earth, and the
predicted indirect neutrino signature, there are poissibly large
effects, incorporated as an optional choice in \ds, coming from this
new population \cite{dkpop}.  (Other studies of solar system
populations of WIMPs can be found in \cite{otherpop,Gould91}.  See
also the comments in \cite{gouldnew} about the uncertainties involved
in estimating these effects.)

The enhancement caused by the new population is only important for
neutralino mass less than 150 - 170 GeV (the exact number depending on
details about the angular momentum distribution \cite{dkpop}).

Following the notation of~\cite{dk1,dk2} one can write the
contribution from the new population of neutralinos to the usual halo
neutralino density as
\begin{equation}
   \delta_E \equiv \frac{n (a_1)}{n_X} \equiv \frac{\hbox{(secondary) neutralino
   density at the Earth}}{\hbox{halo neutralino density at infinity}} \, ,
   \label{eq5.11}
\end{equation}
where
\begin{equation}
   \delta_E = \frac{5.44 \times 10^{36}}{(v_o / 220 \, {\rm km\,s}^{-1})}
   \times
   g_{\rm tot} \, {\rm GeV} \, {\rm cm}^{-2} = \frac{0.212}{(v_o / 220 \,
   {\rm km\,s}^{-1})} \, g_{\rm tot}^{(-10)} \, . \label{eq5.15}
\end{equation}
Here, $g_{\rm tot}^{(-10)} \equiv 10^{10} \, g_{\rm tot} ({\rm
GeV})^3$, and $g_{\rm tot} = { \sum_A} \, (f_A / m_A) \, \sigma_A \,
\phi_A^s$, where $f_A$ is the mass fraction of element $A$ in the Sun,
and $\phi_A^s$ is the surface value of the capture function on the
element of mass number $A$ in the Sun \cite{dk2}.

The scattering rate of neutralinos in the outer layers of the Sun (which
causes the fast halo neutralinos to lose enough energy to enter bound orbits
close to the Earth's orbit) is proportional to $\sigma_A \phi_A^s$, which
can be calculated once the parameters of the SUSY neutralino in question are
fixed. (For the elemental abundances in the Sun, we use the compilation
in~\cite{jnb}.)

The values of $g_{\rm tot}^{(-10)}$
can in some cases approach unity. The spread is
very large, however, and some models give orders of magnitude smaller
values. As would be expected, the models with the highest values of
$g_{\rm tot}^{(-10)}$ are the same models which give high scattering
rates in direct detection experiments.
As mentioned, the integrated effect of the
new population in direct experiments is not very prominent
(see refs.~\cite{dk1,dk2}). On
the other hand,  they can imply a large effect
on indirect detection neutrino rates.

The total capture rate  is computed according
to the formulas in \cite{dkpop}, which take into account that the annihilation
rates from the earth will, in general depend on time in a different
way than the simple result in Eq.~(\ref{eq:tanh}).

Due to this nonlinear nature of the capture rate, there is no simple
scaling of the computed detection rates with the local halo density.
Therefore, it is advisable that the user rescales the local halo
density (see section \ref{sec:rescale}) before calculating the rates
in neutrino telescopes.

The new population can cause an increase of the detection rates
by as much as a factor of 100 when the neutralino mass is less
than around 150 GeV\@.


%%%%%%%%%%%%%%%%%%%%%%%%%%%%%%
\subsection{Effects of WIMP diffusion in the solar system}

As the Earth has a rather low escape velocity, the Earth will only be
able to capture WIMPs that have a rather low velocity with respect to
the Earth. However, WIMPs from the halo have gained speed in the
gravitaional potential from the Sun and will essentially be impossible
to capture by the Earth. Hence, the Earth will only capture WIMPs that
have diffused around in the solar system (by gravitational
interactions with the other planets). Gould showed \cite{gould-diff}
that effectively this diffusion will lead to the same phase space
distribution at the Earth as if the Earth was in free space
(i.e.\ neglecting the solar potential). However, numerical simulations
of asteroids showed that they are thrown into the Sun due to
perturbations of the orbits by other planets, see
e.g.\ \cite{farinella}. These analyses led to worried that maybe the
population of WIMPs diffusing around in the solar system is not as big
as thought \cite{gould-conserv}. In \cite{earth-diff}, Lundberg and
Edsj\"o investigated this issue with detailed numerical simulations of
WIMP orbits in the solar system, showing that the annihilation rate in
the Earth is typically reduced by up to two orders of magnitude. In
\ds, we include these results for the neutrino rates from the Earth by
using the velocity distribution at the Earth (as obtained in
\cite{earth-diff}). This velocity distribution is then used as input
for our numerical capture rate routines instead of the usual
approximation of using the halo velocity distribution directly. Using
these new velocity distributions for the Earth is the default in \ds.
%%%%%%%%%%%%%%%%%%%%%%%%%%%%%%%%%%%%%%%%%%%%%%%%%%%%%%%%%%%%%%%%%%%%
\section{Neutrinos from Sun and Earth --  routines}

\comment{NOTE: This section is not up-to date with the current
  \ds\ release.}

This set of routines contain routines to calculate the
neutrino-induced muon flux from the Earth and the Sun in various
models. It also includes routines that calculate the neutrino-induced
muon flux from other sources, like the Sun's atmosphere, the Earth's
atmosphere \mcomment{(include these???)}.

There are three different methods of calculation available (determined
by \texttt{ntcalcmet} in \texttt{dsntcom.h}). Method 1 uses the
approximate formulae for the capture rates in the Earth/Sun from the
Jungman, Kamionkowski and Griest review \cite{jkg}. Method 2, uses the
same expression for the Sun, but the full expression from Gould
\cite{Gould321} for capture in the Earth (this is the default). Method
3, finally, is the same as 2, but it also includes capture in the
Earth from the Damour-Krauss population of WIMPs that have scattered
in the outscirts of the Sun. The easiest way to select method is by
calling \texttt{dsntset}, with the argument 'jkg' for method 1, 'gould'
or 'default' for method 2 and 'dk' for method 3. A call to 
\texttt{dsntset('default')} is made in \texttt{dsinit}, but can be
changed by the user by calling \texttt{dsntset} after \texttt{dsinit}.

To calculate the neutrino-induced muon flux from the Earth, you call

\begin{sub}{subroutine
\ftb{dsntrates}(emuth,thmax,rtype,rateea,ratesu,istat)}
  \itit{Purpose:} Calculate the rate of neutrinos or neutrino-induced
  muons in a neutrino telescope from neutralino annihilation in the
  Earth and the Sun.
  \itit{Input:}
  \itv{emuth}{r8} The neutrino or muon energy threshold in GeV.
  \itv{thmax}{r8} The half-aperture opening angle (in degrees) 
  towards the center
  of the Sun or the Earth (i.e.\ the flux will be summed in a cone
  towards the center of the Sun or the Earth, where the top-angle of
  the cone is 2*\ft{thmax}).
  \itv{rtype}{i}Type of flux to calculate:\\
  =1: muon neutrino-flux (neutrino and anti-neutrino summed) in units
  of km$^{-2}$ yr$^{-1}$.\\
  =2: neutrino-to-muon conversion rate (muons and anti-muons summed)
  in units of km$^{-3}$ yr$^{-1}$.\\
  =3: muon flux (muons and anti-muons summed)
  in units of km$^{-2}$ yr$^{-1}$.
  \itit{Output:}
  \itv{rateea}{r8} The rate from neutralino annihilation in the Earth
  in the above units.
  \itv{ratesu}{r8} The rate from neutralino annihilation in the Sun
  in the above units.
  \itv{istat}{i} =0: Everything went OK.\\
  $\neq0$: Some of the tables of neutrino or muon yields had to be
  used outside their tabulated regions. Extrapolations have been used.
\end{sub}

\begin{sub}{subroutine
\ftb{dsntdiffrates}(emu,theta,rtype,rateea,ratesu,istat)}
  \itit{Purpose:} Calculate the differential 
  rate of neutrinos or neutrino-induced
  muons in a neutrino telescope from neutralino annihilation in the
  Earth and the Sun.
  \itit{Input:}
  \itv{emu}{r8} The neutrino or muon energy in GeV.
  \itv{theta}{r8} The angle (in degrees) from the center of 
  the Sun or the Earth.
  \itv{rtype}{i}Type of flux to calculate:\\
  =1: muon neutrino-flux (neutrino and anti-neutrino summed) in units
  of km$^{-2}$ yr$^{-1}$ GeV$^{-1}$ degrees$^{-1}$.\\
  =2: neutrino-to-muon conversion rate (muons and anti-muons summed)
  in units of km$^{-3}$ yr$^{-1}$ GeV$^{-1}$ degrees$^{-1}$.\\
  =3: muon flux (muons and anti-muons summed)
  in units of km$^{-2}$ yr$^{-1}$ GeV$^{-1}$ degrees$^{-1}$.
  \itit{Output:}
  \itv{rateea}{r8} The rate from neutralino annihilation in the Earth
  in the above units.
  \itv{ratesu}{r8} The rate from neutralino annihilation in the Sun
  in the above units.
  \itv{istat}{i} =0: Everything went OK.\\
  $\neq0$: Some of the tables of neutrino or muon yields had to be
  used outside their tabulated regions. Extrapolations have been used.
\end{sub}
\section{Routine headers -- fortran files}

%%%%% routine dsai.f %%%%%
\begin{routine}{dsai.f}
\begin{verbatim}


      real*8 function dsai(x)
c   dsairy function
c   lb 990224
\end{verbatim}
 \end{routine}

%%%%% routine dsaip.f %%%%%
\begin{routine}{dsaip.f}
\begin{verbatim}


      real*8 function dsaip(x)
c   dsairy function derivative
c   lb 990224
\end{verbatim}
 \end{routine}

%%%%% routine dsatm_mu.f %%%%%
\begin{routine}{dsatm\_mu.f}
\begin{verbatim}
       real*8 function dsatm_mu(e_mu,c_th,flt)
c ***********************************************************************
c    gives muon flux from atmospheric neutrinos. uses dshonda.f and
c    dsgauss1.f.
c    based on the approximation in gaisser and stanev prd30 (1984) 985.
c    variables:
c     e_mu  muon energy in gev
c     c_th  cosine of zenith angle
c     fltype - 1 flux of muons in units of cm^-2 s^-1 sr^-1 gev^-1
c              2 cont. event rates in units of cm^-3 s^-1 sr^-1 gev^-1
c
c    output is dn/de_mu in muons per cm**2(3) per sec per sr per gev
c    l. bergstrom 1996-09-02
c    modified by j. edsjo (edsjo@physto.se)
c    date: jun-03-98
c
c ***********************************************************************
\end{verbatim}
 \end{routine}

%%%%% routine dsbi.f %%%%%
\begin{routine}{dsbi.f}
\begin{verbatim}


      real*8 function dsbi(x)
c   dsairy function
c   lb 990224
\end{verbatim}
 \end{routine}

%%%%% routine dsbip.f %%%%%
\begin{routine}{dsbip.f}
\begin{verbatim}


      real*8 function dsbip(x)
c   dsairy function derivative
c   lb 990224
\end{verbatim}
 \end{routine}

%%%%% routine dsff.f %%%%%
\begin{routine}{dsff.f}
\begin{verbatim}
       real*8 function dsff(x)
No header found.
\end{verbatim}
 \end{routine}

%%%%% routine dsff2.f %%%%%
\begin{routine}{dsff2.f}
\begin{verbatim}
      real*8 function dsff2(x)
No header found.
\end{verbatim}
 \end{routine}

%%%%% routine dsff3.f %%%%%
\begin{routine}{dsff3.f}
\begin{verbatim}
      real*8 function dsff3(x)
No header found.
\end{verbatim}
 \end{routine}

%%%%% routine dsfff2.f %%%%%
\begin{routine}{dsfff2.f}
\begin{verbatim}
      real*8 function dsfff2(x)
No header found.
\end{verbatim}
 \end{routine}

%%%%% routine dsfff3.f %%%%%
\begin{routine}{dsfff3.f}
\begin{verbatim}
      real*8 function dsfff3(x)
No header found.
\end{verbatim}
 \end{routine}

%%%%% routine dsgauss1.f %%%%%
\begin{routine}{dsgauss1.f}
\begin{verbatim}
      subroutine dsgauss1(f,a,b,result,eps,lambda)

\end{verbatim}
 \end{routine}

%%%%% routine dshiprecint.f %%%%%
\begin{routine}{dshiprecint.f}
\begin{verbatim}

      subroutine dshiprecint(fun,foveru,lowlim,upplim,result)
\end{verbatim}
 \end{routine}

%%%%% routine dshiprecint2.f %%%%%
\begin{routine}{dshiprecint2.f}
\begin{verbatim}

      subroutine dshiprecint2(fun,foveru,lowlim,upplim,result)
\end{verbatim}
 \end{routine}

%%%%% routine dshonda.f %%%%%
\begin{routine}{dshonda.f}
\begin{verbatim}
c ***********************************************************************
       real*8 function dshonda(nu_type,e_nu,c_th)
c ***********************************************************************
c
c      gives atmospheric neutrino flux according to m. dshonda et al.,
c      phys. rev. d52 (1995) 4985, by interpolating their tables iv
c      and v in log(e_nu) and cos_theta.
c
c      variables:
c                 nu_type - type of neutrino:
c
c                 nu_type=1 muon neutrino
c                 nu_type=2 muon antineutrino
c                 nu_type=3 electron antineutrino (not yet implemented)
c                 nu_type=4 electron antineutrino (not yet implemented)
c
c
c                 e_nu - neutrino energy in gev
c                 c_th - cosine of zenith angle
c      output: dshonda returns neutrino differential flux dn_nu/de in units
c              of cm^(-2)sec^(-1)sr^(-1)gev^(-1).
c
c      allowed energy range: 1 gev to 3.9 tev (above 3.1 tev extrapolation
c      is made)
c
c      lars bergstrom 1996-09-02
c
c
c ***********************************************************************
\end{verbatim}
 \end{routine}

%%%%% routine dslnff.f %%%%%
\begin{routine}{dslnff.f}
\begin{verbatim}
      real*8 function dslnff(x)
No header found.
\end{verbatim}
 \end{routine}

%%%%% routine dsntannrate.f %%%%%
\begin{routine}{dsntannrate.f}
\begin{verbatim}
      subroutine dsntannrate(mx,sigsip,sigsdp,sigma_v,arateea,
     &  aratesu)
c_______________________________________________________________________
c
c     wimp annihilation rate in the sun and in the earth
c     in units of 10^24 annihilations per year
c
c     also gives the capture rate and the annih/capt equilibration time
c
c    november, 1995
c    uses routines by p. gondolo and j. edsjo
c    modified by l. bergstrom and j. edsjo and p. gondolo
c    capture rate routines are written by l. bergstrom
c    input:  mx      - wimp mass
c            sigsip  - spin-indep wimp-proton cross section in cm^2
c            sigsdp  - spin-dep wimp-proton cross section in cm^2
c            sigma_v  - wimp self-annihilation cross section in cm^3/s
c            rescale - rescale factor for local density
c    output: arateea  - 10^24 annihilations per year, earth
c            aratesu  - 10^24 annihilations per year, sun
c    slightly modified by j. edsjo.
c    modified by j. edsjo 97-05-15 to match new inv. rate convention
c    modified by j. edsjo 97-12-03 to match muflux3.21 routines.
c    modified by p. gondolo 98-03-04 to detach it from susy routines.
c
c=======================================================================
\end{verbatim}
 \end{routine}

%%%%% routine dsntcapcom.f %%%%%
\begin{routine}{dsntcapcom.f}
\begin{verbatim}
No header found.
\end{verbatim}
 \end{routine}

%%%%% routine dsntcapearth.f %%%%%
\begin{routine}{dsntcapearth.f}
\begin{verbatim}
***********************************************************************
*** note. this routine assumes a maxwell-boltzmann velocity
*** distribution and uses approximations in the jkg review,
*** Jungman, Kamionkowski and Griest, Phys. Rep. 267 (1996) 195.
*** In particular, it is assumed that the Sun's velocity is
*** sqrt(2/3)*vd_3d.
*** for more accurate results, use dsntcapearthfull instead.
*** for an arbitrary velocity distribution, use dsntcapearthnumi instead.
***********************************************************************

       real*8 function dsntcapearth(mx,sigsi)
c----------------------------------------------------------------------
c         capture rate in the earth
c         based on jungman, kamionkowski, griest review
c       mx: neutralino mass
c       sigsi: spin independent cross section in units of cm^2
c       vobs: average halo velocity
c       lars bergstrom 1995-12-14
c----------------------------------------------------------------------
\end{verbatim}
 \end{routine}

%%%%% routine dsntcapearth2.f %%%%%
\begin{routine}{dsntcapearth2.f}
\begin{verbatim}
***********************************************************************
*** note. this routine uses the full expressions for the capture
*** rate in the earth from gould, apj 521 (1987) 571.
*** this routine replaces dsntcapearth which use the approximations
*** given in the jkg review.
***********************************************************************


       real*8 function dsntcapearth2(mx,sigsi)
c----------------------------------------------------------------------
c         capture rate in the earth
c       uses the full routines instead of jkg (as in dsntcapearth).
c *** full: use formulas by gould as reported in jkg
c
c       mx: neutralino mass
c       sigsi: spin independent cross section in units of cm^2
c       vbar: 3D WIMP velocity dispersion in the halo
c       vstar: Sun's velocity through the halo
c       lars bergstrom 1998-09-15
c----------------------------------------------------------------------
\end{verbatim}
 \end{routine}

%%%%% routine dsntcapearthfull.f %%%%%
\begin{routine}{dsntcapearthfull.f}
\begin{verbatim}
***********************************************************************
*** full capture rate routines for the earth.
*** this set of routines use the full expressions for the capture rate
*** in the earth as given in gould, apj 321 (1987) 571.
*** dsntcapearthfull thus replaces dsntcapearth which use the approximations
*** given in the jkg review. these routines assume a maxwell-
*** boltmann velocity distribution. for the damour-krauss population
*** of wimps, the routine dsntcapearthnumi should be used instead.
***********************************************************************

      real*8 function dsntcapearthfull(mx,sigsi,v_star,v_bar,rho_x)
c----------------------------------------------------------------------
c         capture rate in the earth
c *** full: use formulas by gould ap.j. 321 (1987) 571
c       mass fractions and phi_i from jkg review
c       mx: neutralino mass
c       sigsi: spin independent cross section in units of cm^2
c       v_star: solar system velocity through halo (220 km/s in standard case)
c       v_bar: 3D velocity dispersion of wimps (270 km/s in standard case)
c       rho_x: local wimp density (units of gev/cm**3)
c       lars bergstrom 1998-09-21
c       Modified by J. Edsjo, 2003-11-22
c References: gould: Gould ApJ 321 (1987) 571
c----------------------------------------------------------------------
\end{verbatim}
 \end{routine}

%%%%% routine dsntcapearthnum.f %%%%%
\begin{routine}{dsntcapearthnum.f}
\begin{verbatim}
***********************************************************************
*** dsntcapearthnum calculates the capture rate at present.
*** Intead of using the assumptions of Gould (i.e. capture as in
*** free space), a tabulted velocity distribution based on detailed
*** numerical simulations of Johan Lundberg is used.
*** A numerical intregration has to be performed instead of the
*** convenient expressions in jkg.
*** Input: mx = neutralino mass in GeV
***        sigsi = spin-independent scattering cross section in cm^2
***        type = type of velocity distribution
***          1 = best estimate of distribution at Earth form numerical sims
***          2 = conservative estimate, only including free orbits and
***              jupiter-crossing orbits
***          3 = ultraconservative estimate, only including free orbits
***          4 = as if Earth was in free space, i.e. with a Gaussian
***              (Gaussian provided by dsntsdfoveru)
***          5 = as if Earth was in free space, i.e. with a Gaussian
***              (Gaussian provided by dsntdkfoverugauss)
***          6 = Damour-Krauss population (this is per gtot10), i.e.
***              multiply with gtot10 to get the full capture rate
***          Note: 1 is the best estimate of the distribution at Earth
***          and should be used as a default
*** author: joakim edsjo (edsjo@physto.se)
*** date: July 10, 2003
***********************************************************************
      real*8 function dsntcapearthnum(mx,sigsi)
\end{verbatim}
 \end{routine}

%%%%% routine dsntcapearthnumi.f %%%%%
\begin{routine}{dsntcapearthnumi.f}
\begin{verbatim}
c  dsntcapearthnumi.f:
***********************************************************************
*** dsntcapearthnumi gives the capture rate of neutralinos in the earth
*** given a specified velocity distribution. the integrations over
*** the earth's radius and over the velocity distribution are
*** performed numerically
*** input: mx [ gev ]
***        sigsi [ cm^2 ]
***        foveru [ cm^-3 (cm s^-1)^-2 ] external function f(u)/u with
***          velocity u [ km s^-1 ] as argument.
***        vt (velocity type): 1=general type, 2=DK-type (i.e. only
***          include non-zero parts )
*** output: capture rate [ s^-1 ]
*** l.b. and j.e. 1999-04-06
*** Modified by Joakim Edsjo 2003-07-10 to allow for arbitrary external
*** velocity distributions foveru.
***********************************************************************

      real*8 function dsntcapearthnumi(mx,sigsi,foveru,vt)
\end{verbatim}
 \end{routine}

%%%%% routine dsntcapearthtab.f %%%%%
\begin{routine}{dsntcapearthtab.f}
\begin{verbatim}
***********************************************************************
*** This routine calculates the capture rates in the Earth.
*** It does the same thing as dsntcapearthnum (i.e. dsntcapearthnumi),
*** except that it uses tabulted versions of the results intead
*** of performing a numerical integration every time. 
*** Inputs: mx - neutralino mass in GeV
***         sigsi - spin-independent capture rate in cm^2
***         type - type of distribution (same as in dsntcapearthnum)
*** Author: Joakim Edsjo
*** Date: 2003-11-27
***********************************************************************

      real*8 function dsntcapearthtab(mx,sigsi)

\end{verbatim}
 \end{routine}

%%%%% routine dsntcapsun.f %%%%%
\begin{routine}{dsntcapsun.f}
\begin{verbatim}
       real*8 function dsntcapsun(mx,sigsi,sigsd)
c----------------------------------------------------------------------
c         capture rate in the sun
c         based on jungman, kamionkowski, griest review
c       mx: neutralino mass
c       sigsi: spin independent cross section in units of cm^2
c       sigsd: spin dependent cross section in units of cm^2
c       vobs: average halo velocity
c       output:
c         capture rate in s^-1
c       lars bergstrom 1995-12-12
c----------------------------------------------------------------------
\end{verbatim}
 \end{routine}

%%%%% routine dsntcapsunnum.f %%%%%
\begin{routine}{dsntcapsunnum.f}
\begin{verbatim}
***********************************************************************
*** dsntcapsunnum calculates the capture rate at present.
*** Instead of using the approximations in jkg, i.e. a gaussian
*** velocity distribution and approximating all elements as being 
*** at their typical radius, we here integrate numerically over
*** the actual velocity distribution and over the Sun's radius.
*** The velocity distribution used is the one set up by the
*** option veldf in dshmcom.h (see src/hm/dshmudf.f for details)
*** Input: mx = neutralino mass in GeV
***        sigsi = spin-independent scattering cross section (cm^2)
***        sgisd = spin-dependent scattering cross section on protons (cm^2)
*** author: joakim edsjo (edsjo@physto.se)
*** date: 2003-11-26
***********************************************************************
      real*8 function dsntcapsunnum(mx,sigsi,sigsd)
\end{verbatim}
 \end{routine}

%%%%% routine dsntcapsunnumi.f %%%%%
\begin{routine}{dsntcapsunnumi.f}
\begin{verbatim}
c  dsntcapsunnumi.f:
***********************************************************************
*** dsntcapsunnumi gives the capture rate of neutralinos in the sun
*** given a specified velocity distribution. the integrations over
*** the sun's radius and over the velocity distribution are
*** performed numerically
*** input: mx [ gev ]
***        sigsi [ cm^2 ]
***        sgisd [ cm^2 ]
***        foveru [ cm^-3 (cm s^-1)^-2 ] external function f(u)/u with
***          velocity u [ km s^-1 ] as argument.
*** Author: Joakim Edsjo
*** Date: 2003-11-26
***********************************************************************

      real*8 function dsntcapsunnumi(mx,sigsi,sigsd,foveru)
\end{verbatim}
 \end{routine}

%%%%% routine dsntcapsuntab.f %%%%%
\begin{routine}{dsntcapsuntab.f}
\begin{verbatim}
***********************************************************************
*** This routine calculates the capture rates in the Sun.
*** It does the same thing as dsntcapsunnum (i.e. dsntcapsunnumi),
*** except that it uses tabulted versions of the results intead
*** of performing a numerical integration every time. 
*** Inputs: mx - neutralino mass in GeV
***         sigsi - spin-independent capture rate in cm^2
***         type - type of distribution (same as in dsntcapsunnum)
*** Author: Joakim Edsjo
*** Date: 2003-11-27
***********************************************************************

      real*8 function dsntcapsuntab(mx,sigsi,sigsd)

\end{verbatim}
 \end{routine}

%%%%% routine dsntceint.f %%%%%
\begin{routine}{dsntceint.f}
\begin{verbatim}

***********************************************************************
*** l.b. and j.e. 1999-04-06
*** auxiliary function for r-integration
*** input: radius in centimeters
*** output: integrand in cm^-1 s^-1
***********************************************************************

      real*8 function dsntceint(r,foveru)
\end{verbatim}
 \end{routine}

%%%%% routine dsntceint2.f %%%%%
\begin{routine}{dsntceint2.f}
\begin{verbatim}

c...
c...auxiliary function for inner integrand
c...input: velocity relaitve to earth in km/s
c...output: integrand in cm^-4
c...We here follow the analysis in Gould, ApJ 321 (1987) 571 and more
c...specifically the more general expressions in appendix A.
      real*8 function dsntceint2(u,foveru)
\end{verbatim}
 \end{routine}

%%%%% routine dsntcsint.f %%%%%
\begin{routine}{dsntcsint.f}
\begin{verbatim}

***********************************************************************
*** l.b. and j.e. 1999-04-06
*** auxiliary function for r-integration
*** input: radius in centimeters
*** output: integrand in cm^-1 s^-1
*** Adapted for the Sun by J. Edsjo, 2003-11-26
***********************************************************************

      real*8 function dsntcsint(r,foveru)
\end{verbatim}
 \end{routine}

%%%%% routine dsntcsint2.f %%%%%
\begin{routine}{dsntcsint2.f}
\begin{verbatim}

c...
c...auxiliary function for inner integrand
c...input: velocity relaitve to sun in km/s
c...output: integrand in cm^-4
c...We here follow the analysis in Gould, ApJ 321 (1987) 571 and more
c...specifically the more general expressions in appendix A.
      real*8 function dsntcsint2(u,foveru)
\end{verbatim}
 \end{routine}

%%%%% routine dsntctabcreate.f %%%%%
\begin{routine}{dsntctabcreate.f}
\begin{verbatim}
      subroutine dsntctabcreate(wh,i)

***********************************************************************
*** Creates tabulated capture rates (apart from cross section)
*** Input: wh = 'su' or 'ea' for sun or earth
***        i = table number to store the results in
*** Author: Joakim Edsjo
*** Date: 2003-11-27
***********************************************************************

\end{verbatim}
 \end{routine}

%%%%% routine dsntctabget.f %%%%%
\begin{routine}{dsntctabget.f}
\begin{verbatim}
      real*8 function dsntctabget(wh,st,mx)

***********************************************************************
*** Interpolates in capture rate tables and returns the capture
*** rate (apart from the cross section)
*** Input: wh ('su' or 'ea' for sun or earth)
***        st, spin-type (1:spin-independent, 2:spin-dependent)
***        mx neutralino mass in GeV
*** Hidden input: velocity distribution model as given in
*** veldf (for the Sun) and veldfearth (for the Earth)
*** Author: Joakim Edsjo
*** Date: 2003-11-27
***********************************************************************

\end{verbatim}
 \end{routine}

%%%%% routine dsntctabread.f %%%%%
\begin{routine}{dsntctabread.f}
\begin{verbatim}
      subroutine dsntctabread(wh,i,file)

***********************************************************************
*** Reads in tabulated capture rates (apart from cross section)
*** Input: wh = 'su' or 'ea' for sun or earth
***        i = table number to read
***        file = file name to read
*** Author: Joakim Edsjo
*** Date: 2003-11-27
*** Modified: 2004-02-01
***********************************************************************

\end{verbatim}
 \end{routine}

%%%%% routine dsntctabwrite.f %%%%%
\begin{routine}{dsntctabwrite.f}
\begin{verbatim}
      subroutine dsntctabwrite(wh,i,file)

***********************************************************************
*** Writes out tabulated capture rates (apart from cross section)
*** Input: wh = 'su' or 'ea' for sun or earth
***        i = table number to write
***        file = file to write to
*** Author: Joakim Edsjo
*** Date: 2003-11-27
*** Modified: 2004-02-01
***********************************************************************

\end{verbatim}
 \end{routine}

%%%%% routine dsntdiffrates.f %%%%%
\begin{routine}{dsntdiffrates.f}
\begin{verbatim}
      subroutine dsntdiffrates(emu,theta,rtype,rateea,
     &  ratesu,istat)
c_______________________________________________________________________
c
c            n e u t r a l i n o   b r a n c h i n g   r a t i o s
c            a n d  c a p t u r e  r a t e  i n  t h e   s u n
c            m u o n   f l u x  c a l c u l a t e d
c    november, 1995
c    uses routines by p. gondolo and j. edsjo
c    modified by l. bergstrom and j. edsjo
c    capture rate routines are written by l. bergstrom
c    input:  emu   - muon (neutrino) energy in gev
c            theta - muon (neutrino) angle from the center of the earth/sun
c                    in degrees
c            rtype   - 1 = neutrino flux, km^-2 yr^-1 gev^-1 degree^-1
c                      2 = contained events km^-3 yr^-1 gev^-1 degree^-1
c                      3 = through-going events km^-2 yr^-1 gev^-1 degree^-1
c    hidden input: ntcalcmet - 1 use jkg approximations
c                              2 use jkg for sun, full gould for earth
c                              3 use jkg for sun, full gould+dk for earth
c                              4 use full numerical calculations for Sun, Earth
c    output: rateea  - events from earth ann. km^-2(-3) yr^-1 gev^-1 deg^-1 
c            ratesu  - events from sun ann. per km^-2(-3) yr^-1 gev^-1 deg^-1
c    slightly modified by j. edsjo.
c    modified by j. edsjo 97-05-15 to match new inv. rate convention
c    modified by j. edsjo 97-12-03 to match muflux3.21 routines.
c    modified by p. gondolo 98-03-04 to detach dsntannrate from susy
c    routines.
c    modified by j. edsjo 98-09-07 to fix istat bug.
c    modified by j. edsjo 98-09-23 to use damour-krauss distributions
c      and full earth formulas.
c    modified by j. edsjo 99-03-17 to include better damour-krauss
c      velocity distributions and numerical capture rate integrations
c      for these non-gaussian distributions
c
c=======================================================================
\end{verbatim}
 \end{routine}

%%%%% routine dsntdkannrate.f %%%%%
\begin{routine}{dsntdkannrate.f}
\begin{verbatim}
      subroutine dsntdkannrate(m_x,sigsip,sigsdp,sigma_v,arateea,
     &  aratesu)
c_______________________________________________________________________
c
c     wimp annihilation rate in the sun and in the earth
c     in units of 10^24 annihilations per year
c
c     also gives the capture rate and the annih/capt equilibration time
c
c    november, 1995
c    uses routines by p. gondolo and j. edsjo
c    modified by l. bergstrom and j. edsjo and p. gondolo
c    capture rate routines are written by l. bergstrom
c    input:  m_x      - wimp mass
c            sigsip  - spin-indep wimp-proton cross section in cm^2
c            sigsdp  - spin-dep wimp-proton cross section in cm^2
c            sigma_v  - wimp self-annihilation cross section in cm^3/s
c            rescale - rescale factor for local density
c    output: arateea  - 10^24 annihilations per year, earth
c            aratesu  - 10^24 annihilations per year, sun
c            aratedk  - 10^24 annihilations per year, earth including dk
c    slightly modified by j. edsjo.
c    modified by j. edsjo 97-05-15 to match new inv. rate convention
c    modified by j. edsjo 97-12-03 to match muflux3.21 routines.
c    modified by p. gondolo 98-03-04 to detach it from susy routines.
c    added damour-krauss population l.bergstrom 98-09-15
c    added better damour-krauss velocity distribution and numerical
c      integration of the capture rate for these non-gaussian
c      distributions. j. edsjo 99-03-17.
c=======================================================================
\end{verbatim}
 \end{routine}

%%%%% routine dsntdkcapea.f %%%%%
\begin{routine}{dsntdkcapea.f}
\begin{verbatim}
      real*8 function dsntdkcapea(mx,sigsi,sigsd)
c----------------------------------------------------------------------
c         capture rate in the earth
c         low-velocity population described by damour and krauss (1998)
c *** simple version: only change vobs -> v_dk \sim 3*v_esc_earth
c *** in a factor (9.22) in jkg
c         based on jungman, kamionkowski, griest review
c       mx: neutralino mass
c       sigsi: spin independent cross section in units of cm^2
c       vobs: average halo velocity
c       lars bergstrom 1998-09-15
c----------------------------------------------------------------------
\end{verbatim}
 \end{routine}

%%%%% routine dsntdkcapeafull.f %%%%%
\begin{routine}{dsntdkcapeafull.f}
\begin{verbatim}

       real*8 function dsntdkcapeafull(mx,sigsi,sigsd)
c----------------------------------------------------------------------
c         capture rate in the earth
c         low-velocity population described by damour and krauss (1998)
c *** full: use formulas by gould as reported in jkg
c
c       mx: neutralino mass
c       sigsi: spin independent cross section in units of cm^2
c       vobs: average halo velocity
c       lars bergstrom 1998-09-15
c----------------------------------------------------------------------
\end{verbatim}
 \end{routine}

%%%%% routine dsntdkcapearth.f %%%%%
\begin{routine}{dsntdkcapearth.f}
\begin{verbatim}


***********************************************************************
*** dsntdkcapearth calculates the capture rate at present from the
*** damour-krauss distribution of wimps. a numerical integration
*** has to be performed instead of the convenient expressions in jkg.
*** author: joakim edsjo (edsjo@physto.se)
*** date: march 18, 1999
***********************************************************************
      real*8 function dsntdkcapearth(mx,sigsi,sigsd)
\end{verbatim}
 \end{routine}

%%%%% routine dsntdkcom.f %%%%%
\begin{routine}{dsntdkcom.f}
\begin{verbatim}
No header found.
\end{verbatim}
 \end{routine}

%%%%% routine dsntdkfbigu.f %%%%%
\begin{routine}{dsntdkfbigu.f}
\begin{verbatim}


**********************************************************************
*** dsntdkfbigu is the velocity distribution in terms of big u=(u/v_e)^2.
c damour's velocity distribution
c l. bergstrom 99-03-12
c u = (u/v_e)^2
c lambda = j_z^{now}/j_z^0
**********************************************************************
      real*8 function dsntdkfbigu(bigu)
\end{verbatim}
 \end{routine}

%%%%% routine dsntdkfoveru.f %%%%%
\begin{routine}{dsntdkfoveru.f}
\begin{verbatim}
***********************************************************************
*** dsntdkfoveru is the velocity distribution for the damour-krauss
*** distribution of wimps divided by u, i.e. f(u)/u.
*** the normalization is such that int_0^infinity f(u) du = 1/gtot10
*** where gtot10 is a normalization factor that depends on how effectively
*** the WIMPs scatter off the Sun into these orbits.
*** Multiplying by (rhox/mx) and gtot10 gives the full f(u)/u.
*** The first factor is added in dsntfoveru.f and the second in
*** dsntdkcapearth.f
*** author: j. edsjo (edsjo@physto.se)
*** input: velocity relative to earth [ km s^-1 ]
*** output: f(u) / u [ (km/s)^2 ]
*** date: april 6, 1999
*** modified: 2004-01-30, gtot10 normalization taken out
***********************************************************************

      real*8 function dsntdkfoveru(u)
\end{verbatim}
 \end{routine}

%%%%% routine dsntdkgtot10.f %%%%%
\begin{routine}{dsntdkgtot10.f}
\begin{verbatim}
      real*8 function dsntdkgtot10(mx,sigsi,sigsd)
c----------------------------------------------------------------------
c       gtot used in damour-krauss calculation
c       units of 10**(-10) gev**(-2)
c       mx: neutralino mass
c       sigsi: spin-indep cross section in cm**2
c       sigsd: spin-dep   cross section in cm**2
c       lars bergstrom 1998-09-15
c----------------------------------------------------------------------
\end{verbatim}
 \end{routine}

%%%%% routine dsntdkka.f %%%%%
\begin{routine}{dsntdkka.f}
\begin{verbatim}

       real*8 function dsntdkka(aa,mx)
c--------------------------------------------------------------------------
c   solar fringe capture rate constant k_a^s for given mass number aa
c   for low-velocity population described by damour and krauss (1998)
c
c   lars bergstrom 1995-12-12
c--------------------------------------------------------------------------
\end{verbatim}
 \end{routine}

%%%%% routine dsntdkyf.f %%%%%
\begin{routine}{dsntdkyf.f}
\begin{verbatim}
      real*8 function dsntdkyf(xi,xf)
c y_f according to damour (gamma^{tot}_a = \half\alpha y_f^2 )
c lb 1998-02-24
\end{verbatim}
 \end{routine}

%%%%% routine dsntdqagse.f %%%%%
\begin{routine}{dsntdqagse.f}
\begin{verbatim}
* ======================================================================
* nist guide to available math software.
* fullsource for module dqagse from package cmlib.
* retrieved from camsun on wed oct  8 08:26:30 1997.
* ======================================================================
      subroutine dsntdqagse(f,foveru,
     &   a,b,epsabs,epsrel,limit,result,abserr,neval,
     1   ier,alist,blist,rlist,elist,iord,last)
c***begin prologue  dqagse
c***date written   800101   (yymmdd)
c***revision date  830518   (yymmdd)
c***category no.  h2a1a1
c***keywords  (end point) singularities,automatic integrator,
c             extrapolation,general-purpose,globally adaptive
c***author  piessens, robert, applied math. and progr. div. -
c             k. u. leuven
c           de doncker, elise, applied math. and progr. div. -
c             k. u. leuven
c***purpose  the routine calculates an approximation result to a given
c            definite integral i = integral of f over (a,b),
c            hopefully satisfying following claim for accuracy
c            abs(i-result).le.max(epsabs,epsrel*abs(i)).
c***description
c
c        computation of a definite integral
c        standard fortran subroutine
c        real*8 version
c
c        parameters
c         on entry
c            f      - real*8
c                     function subprogram defining the integrand
c                     function f(x). the actual name for f needs to be
c                     declared e x t e r n a l in the driver program.
c
c            a      - real*8
c                     lower limit of integration
c
c            b      - real*8
c                     upper limit of integration
c
c            epsabs - real*8
c                     absolute accuracy requested
c            epsrel - real*8
c                     relative accuracy requested
c                     if  epsabs.le.0
c                     and epsrel.lt.max(50*rel.mach.acc.,0.5d-28),
c                     the routine will end with ier = 6.
c
c            limit  - integer
c                     gives an upperbound on the number of subintervals
c                     in the partition of (a,b)
c
c         on return
c            result - real*8
c                     approximation to the integral
c
c            abserr - real*8
c                     estimate of the modulus of the absolute error,
c                     which should equal or exceed abs(i-result)
c
c            neval  - integer
c                     number of integrand evaluations
c
c            ier    - integer
c                     ier = 0 normal and reliable termination of the
c                             routine. it is assumed that the requested
c                             accuracy has been achieved.
c                     ier.gt.0 abnormal termination of the routine
c                             the estimates for integral and error are
c                             less reliable. it is assumed that the
c                             requested accuracy has not been achieved.
c            error messages
c                         = 1 maximum number of subdivisions allowed
c                             has been achieved. one can allow more sub-
c                             divisions by increasing the value of limit
c                             (and taking the according dimension
c                             adjustments into account). however, if
c                             this yields no improvement it is advised
c                             to analyze the integrand in order to
c                             determine the integration difficulties. if
c                             the position of a local difficulty can be
c                             determined (e.g. singularity,
c                             discontinuity within the interval) one
c                             will probably gain from splitting up the
c                             interval at this point and calling the
c                             integrator on the subranges. if possible,
c                             an appropriate special-purpose integrator
c                             should be used, which is designed for
c                             handling the type of difficulty involved.
c                         = 2 the occurrence of roundoff error is detec-
c                             ted, which prevents the requested
c                             tolerance from being achieved.
c                             the error may be under-estimated.
c                         = 3 extremely bad integrand behaviour
c                             occurs at some points of the integration
c                             interval.
c                         = 4 the algorithm does not converge.
c                             roundoff error is detected in the
c                             extrapolation table.
c                             it is presumed that the requested
c                             tolerance cannot be achieved, and that the
c                             returned result is the best which can be
c                             obtained.
c                         = 5 the integral is probably divergent, or
c                             slowly convergent. it must be noted that
c                             divergence can occur with any other value
c                             of ier.
c                         = 6 the input is invalid, because
c                             epsabs.le.0 and
c                             epsrel.lt.max(50*rel.mach.acc.,0.5d-28).
c                             result, abserr, neval, last, rlist(1),
c                             iord(1) and elist(1) are set to zero.
c                             alist(1) and blist(1) are set to a and b
c                             respectively.
c
c            alist  - real*8
c                     vector of dimension at least limit, the first
c                      last  elements of which are the left end points
c                     of the subintervals in the partition of the
c                     given integration range (a,b)
c
c            blist  - real*8
c                     vector of dimension at least limit, the first
c                      last  elements of which are the right end points
c                     of the subintervals in the partition of the given
c                     integration range (a,b)
c
c            rlist  - real*8
c                     vector of dimension at least limit, the first
c                      last  elements of which are the integral
c                     approximations on the subintervals
c
c            elist  - real*8
c                     vector of dimension at least limit, the first
c                      last  elements of which are the moduli of the
c                     absolute error estimates on the subintervals
c
c            iord   - integer
c                     vector of dimension at least limit, the first k
c                     elements of which are pointers to the
c                     error estimates over the subintervals,
c                     such that elist(iord(1)), ..., elist(iord(k))
c                     form a decreasing sequence, with k = last
c                     if last.le.(limit/2+2), and k = limit+1-last
c                     otherwise
c
c            last   - integer
c                     number of subintervals actually produced in the
c                     subdivision process
c***references  (none)
c***routines called  d1mach,dqelg,dqk21,dqpsrt
c***end prologue  dqagse
c
\end{verbatim}
 \end{routine}

%%%%% routine dsntdqagseb.f %%%%%
\begin{routine}{dsntdqagseb.f}
\begin{verbatim}
* ======================================================================
* nist guide to available math software.
* fullsource for module dqagse from package cmlib.
* retrieved from camsun on wed oct  8 08:26:30 1997.
* ======================================================================
      subroutine dsntdqagseb(f,foveru,
     &   a,b,epsabs,epsrel,limit,result,abserr,neval,
     1   ier,alist,blist,rlist,elist,iord,last)
c***begin prologue  dqagse
c***date written   800101   (yymmdd)
c***revision date  830518   (yymmdd)
c***category no.  h2a1a1
c***keywords  (end point) singularities,automatic integrator,
c             extrapolation,general-purpose,globally adaptive
c***author  piessens, robert, applied math. and progr. div. -
c             k. u. leuven
c           de doncker, elise, applied math. and progr. div. -
c             k. u. leuven
c***purpose  the routine calculates an approximation result to a given
c            definite integral i = integral of f over (a,b),
c            hopefully satisfying following claim for accuracy
c            abs(i-result).le.max(epsabs,epsrel*abs(i)).
c***description
c
c        computation of a definite integral
c        standard fortran subroutine
c        real*8 version
c
c        parameters
c         on entry
c            f      - real*8
c                     function subprogram defining the integrand
c                     function f(x). the actual name for f needs to be
c                     declared e x t e r n a l in the driver program.
c
c            a      - real*8
c                     lower limit of integration
c
c            b      - real*8
c                     upper limit of integration
c
c            epsabs - real*8
c                     absolute accuracy requested
c            epsrel - real*8
c                     relative accuracy requested
c                     if  epsabs.le.0
c                     and epsrel.lt.max(50*rel.mach.acc.,0.5d-28),
c                     the routine will end with ier = 6.
c
c            limit  - integer
c                     gives an upperbound on the number of subintervals
c                     in the partition of (a,b)
c
c         on return
c            result - real*8
c                     approximation to the integral
c
c            abserr - real*8
c                     estimate of the modulus of the absolute error,
c                     which should equal or exceed abs(i-result)
c
c            neval  - integer
c                     number of integrand evaluations
c
c            ier    - integer
c                     ier = 0 normal and reliable termination of the
c                             routine. it is assumed that the requested
c                             accuracy has been achieved.
c                     ier.gt.0 abnormal termination of the routine
c                             the estimates for integral and error are
c                             less reliable. it is assumed that the
c                             requested accuracy has not been achieved.
c            error messages
c                         = 1 maximum number of subdivisions allowed
c                             has been achieved. one can allow more sub-
c                             divisions by increasing the value of limit
c                             (and taking the according dimension
c                             adjustments into account). however, if
c                             this yields no improvement it is advised
c                             to analyze the integrand in order to
c                             determine the integration difficulties. if
c                             the position of a local difficulty can be
c                             determined (e.g. singularity,
c                             discontinuity within the interval) one
c                             will probably gain from splitting up the
c                             interval at this point and calling the
c                             integrator on the subranges. if possible,
c                             an appropriate special-purpose integrator
c                             should be used, which is designed for
c                             handling the type of difficulty involved.
c                         = 2 the occurrence of roundoff error is detec-
c                             ted, which prevents the requested
c                             tolerance from being achieved.
c                             the error may be under-estimated.
c                         = 3 extremely bad integrand behaviour
c                             occurs at some points of the integration
c                             interval.
c                         = 4 the algorithm does not converge.
c                             roundoff error is detected in the
c                             extrapolation table.
c                             it is presumed that the requested
c                             tolerance cannot be achieved, and that the
c                             returned result is the best which can be
c                             obtained.
c                         = 5 the integral is probably divergent, or
c                             slowly convergent. it must be noted that
c                             divergence can occur with any other value
c                             of ier.
c                         = 6 the input is invalid, because
c                             epsabs.le.0 and
c                             epsrel.lt.max(50*rel.mach.acc.,0.5d-28).
c                             result, abserr, neval, last, rlist(1),
c                             iord(1) and elist(1) are set to zero.
c                             alist(1) and blist(1) are set to a and b
c                             respectively.
c
c            alist  - real*8
c                     vector of dimension at least limit, the first
c                      last  elements of which are the left end points
c                     of the subintervals in the partition of the
c                     given integration range (a,b)
c
c            blist  - real*8
c                     vector of dimension at least limit, the first
c                      last  elements of which are the right end points
c                     of the subintervals in the partition of the given
c                     integration range (a,b)
c
c            rlist  - real*8
c                     vector of dimension at least limit, the first
c                      last  elements of which are the integral
c                     approximations on the subintervals
c
c            elist  - real*8
c                     vector of dimension at least limit, the first
c                      last  elements of which are the moduli of the
c                     absolute error estimates on the subintervals
c
c            iord   - integer
c                     vector of dimension at least limit, the first k
c                     elements of which are pointers to the
c                     error estimates over the subintervals,
c                     such that elist(iord(1)), ..., elist(iord(k))
c                     form a decreasing sequence, with k = last
c                     if last.le.(limit/2+2), and k = limit+1-last
c                     otherwise
c
c            last   - integer
c                     number of subintervals actually produced in the
c                     subdivision process
c***references  (none)
c***routines called  d1mach,dqelg,dqk21,dqpsrt
c***end prologue  dqagse
c
\end{verbatim}
 \end{routine}

%%%%% routine dsntdqk21.f %%%%%
\begin{routine}{dsntdqk21.f}
\begin{verbatim}
      subroutine dsntdqk21(f,foveru,a,b,result,abserr,resabs,resasc)
c***begin prologue  dqk21
c***date written   800101   (yymmdd)
c***revision date  830518   (yymmdd)
c***category no.  h2a1a2
c***keywords  21-point gauss-kronrod rules
c***author  piessens, robert, applied math. and progr. div. -
c             k. u. leuven
c           de doncker, elise, applied math. and progr. div. -
c             k. u. leuven
c***purpose  to compute i = integral of f over (a,b), with error
c                           estimate
c                       j = integral of abs(f) over (a,b)
c***description
c
c           integration rules
c           standard fortran subroutine
c           real*8 version
c
c           parameters
c            on entry
c              f      - real*8
c                       function subprogram defining the integrand
c                       function f(x). the actual name for f needs to be
c                       declared e x t e r n a l in the driver program.
c
c              a      - real*8
c                       lower limit of integration
c
c              b      - real*8
c                       upper limit of integration
c
c            on return
c              result - real*8
c                       approximation to the integral i
c                       result is computed by applying the 21-point
c                       kronrod rule (resk) obtained by optimal addition
c                       of abscissae to the 10-point gauss rule (resg).
c
c              abserr - real*8
c                       estimate of the modulus of the absolute error,
c                       which should not exceed abs(i-result)
c
c              resabs - real*8
c                       approximation to the integral j
c
c              resasc - real*8
c                       approximation to the integral of abs(f-i/(b-a))
c                       over (a,b)
c***references  (none)
c***routines called  d1mach
c***end prologue  dqk21
c
\end{verbatim}
 \end{routine}

%%%%% routine dsntdqk21b.f %%%%%
\begin{routine}{dsntdqk21b.f}
\begin{verbatim}
      subroutine dsntdqk21b(f,foveru,a,b,result,abserr,resabs,resasc)
c***begin prologue  dqk21
c***date written   800101   (yymmdd)
c***revision date  830518   (yymmdd)
c***category no.  h2a1a2
c***keywords  21-point gauss-kronrod rules
c***author  piessens, robert, applied math. and progr. div. -
c             k. u. leuven
c           de doncker, elise, applied math. and progr. div. -
c             k. u. leuven
c***purpose  to compute i = integral of f over (a,b), with error
c                           estimate
c                       j = integral of abs(f) over (a,b)
c***description
c
c           integration rules
c           standard fortran subroutine
c           real*8 version
c
c           parameters
c            on entry
c              f      - real*8
c                       function subprogram defining the integrand
c                       function f(x). the actual name for f needs to be
c                       declared e x t e r n a l in the driver program.
c
c              a      - real*8
c                       lower limit of integration
c
c              b      - real*8
c                       upper limit of integration
c
c            on return
c              result - real*8
c                       approximation to the integral i
c                       result is computed by applying the 21-point
c                       kronrod rule (resk) obtained by optimal addition
c                       of abscissae to the 10-point gauss rule (resg).
c
c              abserr - real*8
c                       estimate of the modulus of the absolute error,
c                       which should not exceed abs(i-result)
c
c              resabs - real*8
c                       approximation to the integral j
c
c              resasc - real*8
c                       approximation to the integral of abs(f-i/(b-a))
c                       over (a,b)
c***references  (none)
c***routines called  d1mach
c***end prologue  dqk21
c
\end{verbatim}
 \end{routine}

%%%%% routine dsntearthdens.f %%%%%
\begin{routine}{dsntearthdens.f}
\begin{verbatim}
c      program test
c      implicit none
c      integer i
c      real*8 radius,dsntearthdens,dsntearthmassint,dsntearthmass,
c     &  dsntearthpotint,dsntearthpot,tmp,dsntearthvesc,
c     &  dsntearthdenscomp
c      real*8 depth(42)
c      data depth/0.0d0,3.0d0,15.0d0,24.0d0,80.0d0,
c     &  219.99d0,220.0d0,399.99d0,400.0d0,500.0d0,
c     &  600.0d0,669.99d0,670.0d0,770.0d0,1000.0d0,
c     &  1250.0d0,1500.0d0,1750.0d0,2000.0d0,2250.0d0,
c     &  2500.0d0,2750.0d0,2899.99d0,2900.0d0,3000.0d0,
c     &  3250.0d0,3500.0d0,3750.0d0,4000.0d0,4250.0d0,
c     &  4500.0d0,4750.0d0,5000.0d0,5149.99d0,5150.0d0,
c     &  5250.0d0,5500.0d0,5750.0d0,6000.0d0,6250.0d0,
c     &  6371.0d0,6379.0d0/   ! in km
c
cc      do i=42,1,-1
cc        radius=max((6378.140-depth(i))*1.0d3,0.0d0)
cc        write(*,*) radius,dsntearthpotint(radius)
cc      enddo
c
c      do i=0,1100
c        radius=dble(i)/dble(1000.0d0)*6378.14d0*1.0d3
c        write(*,'(10(x,e12.6))') radius,dsntearthdens(radius),
c     &    dsntearthmassint(radius)/1.0d24,
c     &    dsntearthmass(radius)/1.0d24,
c     &    dsntearthpotint(radius),dsntearthpot(radius),
c     &    dsntearthvesc(radius)
c
c        write(*,'(10(x,e12.6))') radius,
c     &    dsntearthdenscomp(radius,16),
c     &    dsntearthdenscomp(radius,24),
c     &    dsntearthdenscomp(radius,28),
c     &    dsntearthdenscomp(radius,56)
c      enddo
c
c      end



***********************************************************************
*** dsntearthdens gives the density in the earth as a function of radius
*** the radius should be given in m and the density is returned in
*** g/cm^3
*** author: joakim edsjo
*** date: march 19, 1999
***********************************************************************

      real*8 function dsntearthdens(r)
\end{verbatim}
 \end{routine}

%%%%% routine dsntearthdenscomp.f %%%%%
\begin{routine}{dsntearthdenscomp.f}
\begin{verbatim}


***********************************************************************
*** dsntearthdenscomp gives the number density of nucleons of mass
*** number a per cm^3
*** input: radius - in meters
***        mass number - 16 for o
***                      24 for mg
***                      28 for si
***                      56 for fe JE UPDATE!
*** the radius should be given in m and the density is returned in
*** g/cm^3
*** author: joakim edsjo
*** date: april 6, 1999
*** Updated with new values by J. Edsjo, 2003-11-21
***********************************************************************

      real*8 function dsntearthdenscomp(r,a)
\end{verbatim}
 \end{routine}

%%%%% routine dsntearthmass.f %%%%%
\begin{routine}{dsntearthmass.f}
\begin{verbatim}



**********************************************************************
*** dsntearthmass gives the mass of the earth in units of kg within
*** a sphere with of radius r meters.
*** author: joakim edsjo (edsjo@physto.se)
*** date: april 1, 1999
**********************************************************************

      real*8 function dsntearthmass(r)
\end{verbatim}
 \end{routine}

%%%%% routine dsntearthmassint.f %%%%%
\begin{routine}{dsntearthmassint.f}
\begin{verbatim}


**********************************************************************
*** dsntearthmassint gives the mass of the earth in units of kg within
*** a sphere with of radius r meters.
*** in this routine, the actual integration is performed. for speed,
*** use dsntearthmass instead which uses a tabulation of this result.
*** author: joakim edsjo (edsjo@physto.se)
*** date: april 1, 1999
**********************************************************************

      real*8 function dsntearthmassint(r)
\end{verbatim}
 \end{routine}

%%%%% routine dsntearthne.f %%%%%
\begin{routine}{dsntearthne.f}
\begin{verbatim}
***********************************************************************
*** dsntearthne gives the number density of electrons as a function
*** of the Earth's radius.
*** Input: Earth radius [m]
*** Output: n_e [cm^-3]
*** Author: Joakim Edsjo, edsjo@physto.se
*** Date: 2006-04-21
***********************************************************************

      real*8 function dsntearthne(r)
\end{verbatim}
 \end{routine}

%%%%% routine dsntearthpot.f %%%%%
\begin{routine}{dsntearthpot.f}
\begin{verbatim}



**********************************************************************
*** dsntearthpot  gives the gravitational potential inside and outside
*** of the earth as a function of the radius r (in meters).
*** input: radius in meters
*** output units: s^-1
*** author: joakim edsjo (edsjo@physto.se)
*** date: april 1, 1999
**********************************************************************

      real*8 function dsntearthpot(r)
\end{verbatim}
 \end{routine}

%%%%% routine dsntearthpotint.f %%%%%
\begin{routine}{dsntearthpotint.f}
\begin{verbatim}
**********************************************************************
*** dsntearthpotint gives the gravitational potential inside and outside
*** of the earth as a function of the radius r (in meters).
*** in this routine, the actual integration is performed. for speed,
*** use dsntearthpot instead which uses a tabulation of this result.
*** author: joakim edsjo (edsjo@physto.se)
*** date: april 1, 1999
**********************************************************************

      real*8 function dsntearthpotint(r)
\end{verbatim}
 \end{routine}

%%%%% routine dsntearthvesc.f %%%%%
\begin{routine}{dsntearthvesc.f}
\begin{verbatim}


**********************************************************************
*** dsntearthvesc gives the escape velocity in km/s as a function of
*** the radius r (in meters) from the earth's core.
*** author: joakim edsjo (edsjo@physto.se)
*** input: radius in m
*** output escape velocity in km/s
*** date: april 1, 1999
**********************************************************************

      real*8 function dsntearthvesc(r)
\end{verbatim}
 \end{routine}

%%%%% routine dsntedfunc.f %%%%%
\begin{routine}{dsntedfunc.f}
\begin{verbatim}

*************************
      real*8 function dsntedfunc(r)
\end{verbatim}
 \end{routine}

%%%%% routine dsntepfunc.f %%%%%
\begin{routine}{dsntepfunc.f}
\begin{verbatim}

*************************
      real*8 function dsntepfunc(r)
\end{verbatim}
 \end{routine}

%%%%% routine dsntfoveru.f %%%%%
\begin{routine}{dsntfoveru.f}
\begin{verbatim}
***********************************************************************
*** input: velocity relative to Sun [ km s^-1 ]
*** output: f(u) / u [ cm^-3 (cm/s)^(-2) ]
*** Date: 2004-01-28
***********************************************************************

      real*8 function dsntfoveru(u)
\end{verbatim}
 \end{routine}

%%%%% routine dsntfoveruearth.f %%%%%
\begin{routine}{dsntfoveruearth.f}
\begin{verbatim}
***********************************************************************
*** input: velocity relative to Earth [ km s^-1 ]
*** output: f(u) / u [ cm^-3 (cm/s)^(-2) ]
*** Date: 2004-01-28
***********************************************************************

      real*8 function dsntfoveruearth(u)
\end{verbatim}
 \end{routine}

%%%%% routine dsntismbkg.f %%%%%
\begin{routine}{dsntismbkg.f}
\begin{verbatim}
***********************************************************************
*** dsntismbkg calculats the differential background of muons cosmic
*** ray interactions with the interstellar medium.
*** the muon neutrino fluxes are from
*** g. ingelman and m. thunman, hep-ph/9604286.
***   input:
***     emu - muon energy in gev
***     fltype = 1 - flux of muons
***              2 - contained event rate
***     rdelta = column density in units of nucleons / cm^2 kpc/cm
***   output:
***     muon flux in units of gev^-1 km^-2(3) yr^-1 sr^-1
*** partly based on routines by l. bergstrom.
*** author: j. edsjo (edsjo@physto.se)
*** date: 1998-09-20
***********************************************************************

      real*8 function dsntismbkg(emu,flt,rdelta)
\end{verbatim}
 \end{routine}

%%%%% routine dsntismrd.f %%%%%
\begin{routine}{dsntismrd.f}
\begin{verbatim}
***********************************************************************
*** function dsntismrd gives the column density of interstellar matter
*** along the line of sight. the model is from ingelman & thunman with
*** rho=rho_0 exp(z/z_0) with rho_0=1.0 nucleon/cm^3 and z_0=0.26 kpc
*** input: b   = galactic latitude (degrees)
***        psi = angle (in the plane) from the galactic centre (degrees)
*** output: column density in units of nucleons / cm^2 kpc/cm.
*** author: joakim edsjo (edsjo@physto.se)
*** date: 1998-09-20
**********************************************************************

      real*8 function dsntismrd(b,psi)
\end{verbatim}
 \end{routine}

%%%%% routine dsntlitlf_e.f %%%%%
\begin{routine}{dsntlitlf\_e.f}
\begin{verbatim}
       real*8 function dsntlitlf_e(mx,vbar)
c_______________________________________________________________________
c
c    dsntlitlf_e is used by capearth to calculate the capture rate
c    in the earth.
c    mx is the neutralino mass in gev
c    written by l. bergstrom 1995-12-12
c
c=======================================================================

\end{verbatim}
 \end{routine}

%%%%% routine dsntlitlf_s.f %%%%%
\begin{routine}{dsntlitlf\_s.f}
\begin{verbatim}
      real*8 function dsntlitlf_s(mx,vbar)
c----------------------------------------------------------------------
c       dsntlitlf_s used by capsun
c       mx: neutralino mass
c       lars bergstrom 1995-12-12
c----------------------------------------------------------------------
\end{verbatim}
 \end{routine}

%%%%% routine dsntmoderf.f %%%%%
\begin{routine}{dsntmoderf.f}
\begin{verbatim}
      real*8 function dsntmoderf(x)
c =================================================================
c error function 
c modified by l. bergstrom 98-09-15
c modified by p. gondolo 2000-07-19
c see test output below
c used for damour-krauss calculations
c =================================================================
\end{verbatim}
 \end{routine}

%%%%% routine dsntmuonyield.f %%%%%
\begin{routine}{dsntmuonyield.f}
\begin{verbatim}
*****************************************************************************
*** function dsntmuonyield gives the total yield of muons above threshold for
*** a given neutralino mass or the differential muon yield for a given
*** energy and a given angle. put yieldk=3 for integrated yield above
*** given thresholds and put yieldk=103 for differntial yield.
*** the annihilation branching ratios and
*** higgs parameters are extracted from susy.h and by calling dsandwdcosnn
*** wh='su' corresponds to annihilation in the sun and wh='ea' corresponds
*** to annihilation in the earth. if istat=1 upon return,
*** some inaccesible parts the differential muon spectra has been wanted,
*** and the returned yield should then be treated as a lower bound.
*** if istat=2 energetically forbidden annihilation channels have been
*** wanted. if istat=3 both of these things has happened.
*** units: 1.0e-30 m**-2 (annihilation)**-1 for integrated yield.
***        1.0e-30 m**-2 gev**-1 (degree)^-1 (annihilation)**-1 for
***        differential yield.
*** author: joakim edsjo  edsjo@physto.se
*** date: 96-03-19
*** modified: 96-09-03 to include new index order
*** modified: 97-12-03 to include new muyield routines (v3.21)
*****************************************************************************

      real*8 function dsntmuonyield(emuth0,thmax,wh,yieldk,istat)
\end{verbatim}
 \end{routine}

%%%%% routine dsntnuism.f %%%%%
\begin{routine}{dsntnuism.f}
\begin{verbatim}
      real*8 function dsntnuism(enu)
c...  with enu in gev, the flux is returned in units o
c...  gev^-1 cm^-2 s^-1 sr^-1
\end{verbatim}
 \end{routine}

%%%%% routine dsntnusun.f %%%%%
\begin{routine}{dsntnusun.f}
\begin{verbatim}
      real*8 function dsntnusun(enu)
c... with enu in gev, the flux is returned in units of gev^-1 cm^-2 s^-1
\end{verbatim}
 \end{routine}

%%%%% routine dsntrates.f %%%%%
\begin{routine}{dsntrates.f}
\begin{verbatim}
      subroutine dsntrates(emuth0,thmax0,rtype,rateea,
     &  ratesu,istat)
c_______________________________________________________________________
c
c            n e u t r a l i n o   b r a n c h i n g   r a t i o s
c            a n d  c a p t u r e  r a t e  i n  t h e   s u n
c            m u o n   f l u x  c a l c u l a t e d
c    november, 1995
c    uses routines by p. gondolo and j. edsjo
c    modified by l. bergstrom and j. edsjo
c    capture rate routines are written by l. bergstrom
c    input:  emuth0  - muon energy threshold in gev
c            thmax0  - muon angel cut in degrees
c            rtype   - 2 = contained events km^-3 yr^-1
c                      3 = through-going events km^-2 yr^-1
c    hidden input: ntcalcmet - 1 use jkg approximations
c                              2 use jkg for sun, full gould for earth
c                              3 use jkg for sun, full gould+dk for earth
c                              4 use full numerical calculations for Sun, Earth
c    output: rateea  - events from earth ann. per km^2(3) per yr
c            ratesu  - events from sun ann. per km^2(3) per yr
c    slightly modified by j. edsjo.
c    modified by j. edsjo 97-05-15 to match new inv. rate convention
c    modified by j. edsjo 97-12-03 to match muflux3.21 routines.
c    modified by p. gondolo 98-03-04 to detach dsntannrate from susy
c    routines.
c    modified by j. edsjo 98-09-07 to fix istat bug.
c    modified by j. edsjo 98-09-23 to use damour-krauss distributions
c      and full earth formulas.
c    modified by j. edsjo 99-03-17 to include better damour-krauss
c      velocity distributions and numerical capture rate integrations
c      for these non-gaussian distributions
c
c=======================================================================
\end{verbatim}
 \end{routine}

%%%%% routine dsntse.f %%%%%
\begin{routine}{dsntse.f}
\begin{verbatim}
       real*8 function dsntse(x,vbar)
c----------------------------------------------------------------------
c      dsntse
c      x: mx/m(i)
c      vbar: three-dimensional velocity dispersion of WIMPs in the halo
c      lars bergstrom 1995-12-12
c----------------------------------------------------------------------
\end{verbatim}
 \end{routine}

%%%%% routine dsntsefull.f %%%%%
\begin{routine}{dsntsefull.f}
\begin{verbatim}
       real*8 function dsntsefull(mx,m_a,v_star,v_bar,phi)
c--------------------------------------------------------------------------
c the gould function for capture in the earth, as given by the expression
c (a10) in gould ap.j. 321 (1987) 571
c mx: neutralino mass in gev
c m_a: nuclear mass
c v_star: velocity of earth with respect to wimps
c v_bar: velocity dispersion of wimps
c output in natural units (power of gev)
c l. bergstrom 1998-09-21
c Modified by J. Edsjo, 2003-11-22
c References:
c    dk: Damour and Krauss, Phys. Rev. D59 (1999) 063509.
c   jkg: Jungman, Kamionkowski and Griest, Phys. Rep. 267 (1996) 195.
c gould: Gould, ApJ 321 (1987) 571.
c--------------------------------------------------------------------------
\end{verbatim}
 \end{routine}

%%%%% routine dsntset.f %%%%%
\begin{routine}{dsntset.f}
\begin{verbatim}
      subroutine dsntset(c)
c...set parameters for neutrino telescope routines
c...  c - character string specifying choice to be made
c...author: joakim edsjo, 2000-08-16
\end{verbatim}
 \end{routine}

%%%%% routine dsntspfunc.f %%%%%
\begin{routine}{dsntspfunc.f}
\begin{verbatim}

*************************
      real*8 function dsntspfunc(r)
\end{verbatim}
 \end{routine}

%%%%% routine dsntss.f %%%%%
\begin{routine}{dsntss.f}
\begin{verbatim}
       real*8 function dsntss(x,vbar)
c----------------------------------------------------------------------
c       dsntss used by capsun and litlf_s
c       x=mx/m(i)
c       lars bergstrom 1995-12-12
c----------------------------------------------------------------------

\end{verbatim}
 \end{routine}

%%%%% routine dsntsunbkg.f %%%%%
\begin{routine}{dsntsunbkg.f}
\begin{verbatim}
***********************************************************************
*** dsntsunbkg calculats the differential background of muons cosmic
*** ray interactions in the sun's corona. the muon neutrino fluxes are from
*** g. ingelman and m. thunman, prd 54 (1996) 4385.
***   input:
***     emu - muon energy in gev
***     fltype = 1 - flux of muons
***              2 - contained event rate
***   output:
***     muon flux in units of gev^-1 km^-2(3) yr^-1
*** partly based on routines by l. bergstrom.
*** author: j. edsjo (edsjo@physto.se)
*** date: 1998-06-03
***********************************************************************

      real*8 function dsntsunbkg(emu,flt)
\end{verbatim}
 \end{routine}

%%%%% routine dsntsuncdens.f %%%%%
\begin{routine}{dsntsuncdens.f}
\begin{verbatim}
***********************************************************************
*** This routine uses a derived column density from the BP2000 model
*** The data in sdcens() is calculated by dsntsunread.f. 
***********************************************************************

***********************************************************************
*** dsntsuncdens gives the column density in the Sun from the
*** centre out the tha given radius r (in meters).
*** The radius should be given in m and the column density is returned in
*** g/cm^2
*** if type = 'N', the total column density (up to that r) is calculated
***         = 'p', the column density on protons is calculated
***         = 'n', the column density on neutrons is calculated
***
*** Author: Joakim Edsjo, edsjo@physto.se
*** Date: 2005-11-25
***********************************************************************

      real*8 function dsntsuncdens(r,type)
\end{verbatim}
 \end{routine}

%%%%% routine dsntsuncdensint.f %%%%%
\begin{routine}{dsntsuncdensint.f}
\begin{verbatim}
**********************************************************************
*** dsntsuncdensint gives the column density in the Sun from the
*** centre out the tha given radius r (in meters)
*** if type = 'N', the total column density (up to that r) is calculated
***         = 'p', the column density on protons is calculated
***         = 'n', the column density on neutrons is calculated
*** in this routine, the actual integration is performed. for speed,
*** use dsntsuncdens instead which uses a tabulation of this result.
*** Author: joakim edsjo (edsjo@physto.se)
*** Date: November 24, 2005
**********************************************************************

      real*8 function dsntsuncdensint(r,type)
\end{verbatim}
 \end{routine}

%%%%% routine dsntsuncdfunc.f %%%%%
\begin{routine}{dsntsuncdfunc.f}
\begin{verbatim}
***********************************************************************
*** dsntsuncdfunc returns the density of protons, neutrons or the total
*** density depending on the common block variable cdt. If
***   cdt='N': the total density is returned
***   cdt='p': the density in protons is returned
***   cdt='n': the density in neutrons is returned
*** the radius should be given in m and the density is returned in
*** g/cm^3.
*** This routine is used by dsntsuncdensint to calculate the column
*** density in the Sun.
***
*** Author: Joakim Edsjo, edsjo@physto.se
*** Date: 2005-11-24
***********************************************************************

      real*8 function dsntsuncdfunc(r)
\end{verbatim}
 \end{routine}

%%%%% routine dsntsundens.f %%%%%
\begin{routine}{dsntsundens.f}
\begin{verbatim}
***********************************************************************
*** dsntsundens gives the density in the Sun as a function of radius
*** the radius should be given in m and the density is returned in
*** g/cm^3
*** Density and element mass fractions up to O16 are from the standard
*** solar model BP2000 of Bahcall, Pinsonneault and Basu,
*** ApJ 555 (2001) 990.
*** The mass fractions for heavier elements are from N. Grevesse and
*** A.J. Sauval, Space Science Reviews 85 (1998) 161 normalized such that
*** their total mass fractions matches that of the heavier elements in 
*** the BP2000 model.
***
*** Author: Joakim Edsjo, edsjo@physto.se
*** Date: 2003-11-25
***********************************************************************

      real*8 function dsntsundens(r)
\end{verbatim}
 \end{routine}

%%%%% routine dsntsundenscomp.f %%%%%
\begin{routine}{dsntsundenscomp.f}
\begin{verbatim}
***********************************************************************
*** dsntsundenscomp gives the number density of nucleons of atomic
*** number Z per cm^3
*** input: radius - in meters
***        itype: internal element number (def. in dsntsunread.f)
*** Author: joakim edsjo
*** Date: 2003-11-26
*** Modified: 2006-03-21 (atomic mass unit fix (was off by 6%)) JE
***********************************************************************

      real*8 function dsntsundenscomp(r,itype)
\end{verbatim}
 \end{routine}

%%%%% routine dsntsunmass.f %%%%%
\begin{routine}{dsntsunmass.f}
\begin{verbatim}
***********************************************************************
*** dsntsunmass gives the mass of the Sun as a function of radius
*** the radius should be given in m and the mass is given in kg
*** up to the specified radius.
*** Density and element mass fractions up to O16 are from the standard
*** solar model BP2000 of Bahcall, Pinsonneault and Basu,
*** ApJ 555 (2001) 990.
*** The mass fractions for heavier elements are from N. Grevesse and
*** A.J. Sauval, Space Science Reviews 85 (1998) 161 normalized such that
*** their total mass fractions matches that of the heavier elements in 
*** the BP2000 model.
***
*** Author: Joakim Edsjo, edsjo@physto.se
*** Date: 2003-11-25
***********************************************************************

      real*8 function dsntsunmass(r)
\end{verbatim}
 \end{routine}

%%%%% routine dsntsunmfrac.f %%%%%
\begin{routine}{dsntsunmfrac.f}
\begin{verbatim}
***********************************************************************
*** dsntsunmfrac gives the mass fraction of element i (see dsntsunread.f
*** for definition of i) as a function of the solar radius r.
*** the radius should be given in m and returned is the mass fraction.
***
*** Element mass fractions up to O16 are from the standard
*** solar model BP2000 of Bahcall, Pinsonneault and Basu,
*** ApJ 555 (2001) 990.
*** The mass fractions for heavier elements are from N. Grevesse and
*** A.J. Sauval, Space Science Reviews 85 (1998) 161 normalized such that
*** their total mass fractions matches that of the heavier elements in 
*** the BP2000 model.
***
*** Author: Joakim Edsjo, edsjo@physto.se
*** Date: 2003-11-26
***********************************************************************

      real*8 function dsntsunmfrac(r,itype)
\end{verbatim}
 \end{routine}

%%%%% routine dsntsunne.f %%%%%
\begin{routine}{dsntsunne.f}
\begin{verbatim}
***********************************************************************
*** dsntsunne gives the number density of electrons as a function
*** of the Sun's radius.
*** Input: solar radius [m]
*** Output: n_e [cm^-3]
*** See dsntsunread for information about which solar model is used.
*** Author: Joakim Edsjo, edsjo@physto.se
*** Date: 2006-03-27
***********************************************************************

      real*8 function dsntsunne(r)
\end{verbatim}
 \end{routine}

%%%%% routine dsntsunne2x.f %%%%%
\begin{routine}{dsntsunne2x.f}
\begin{verbatim}
***********************************************************************
*** dsntsunne2x takes an input number density of electrons and
*** converts this to a fractional solar radius, x.
*** Input: n_e [cm^-3]
*** Output: x = r/r_sun [0,1]
*** See dsntsunread for information about which solar model is used.
*** Author: Joakim Edsjo, edsjo@physto.se
*** Date: 2006-03-27
***********************************************************************

      real*8 function dsntsunne2x(ne)
\end{verbatim}
 \end{routine}

%%%%% routine dsntsunpot.f %%%%%
\begin{routine}{dsntsunpot.f}
\begin{verbatim}
***********************************************************************
*** This routine uses a derived potential from the BP2000 model
*** The data in sdphi() is calculated by dsntsunread.f. 
***********************************************************************

***********************************************************************
*** dsntsunpot gives the potential in the Sun as a function of radius
*** the radius should be given in m and the potential is returned in
*** m^2 s^-2
*** Density and element mass fractions up to O16 are from the standard
*** solar model BP2000 of Bahcall, Pinsonneault and Basu,
*** ApJ 555 (2001) 990.
*** The mass fractions for heavier elements are from N. Grevesse and
*** A.J. Sauval, Space Science Reviews 85 (1998) 161 normalized such that
*** their total mass fractions matches that of the heavier elements in 
*** the BP2000 model.
***
*** Author: Joakim Edsjo, edsjo@physto.se
*** Date: 2003-11-26
***********************************************************************

      real*8 function dsntsunpot(r)
\end{verbatim}
 \end{routine}

%%%%% routine dsntsunpotint.f %%%%%
\begin{routine}{dsntsunpotint.f}
\begin{verbatim}
**********************************************************************
*** dsntsunpotint gives the gravitational potential inside and outside
*** of the sun as a function of the radius r (in meters).
*** in this routine, the actual integration is performed. for speed,
*** use dsntsunpot instead which uses a tabulation of this result.
*** author: joakim edsjo (edsjo@physto.se)
*** date: april 1, 1999
**********************************************************************

      real*8 function dsntsunpotint(r)
\end{verbatim}
 \end{routine}

%%%%% routine dsntsunread.f %%%%%
\begin{routine}{dsntsunread.f}
\begin{verbatim}
      subroutine dsntsunread

***********************************************************************
*** Reads in data about the solar model used and stores it in a
*** common block (as described in dssun.h).
*** Author: Joakim Edsjo
*** Date: 2003-11-25
*** Modified: 2004-01-28 (calculates potential instead of reading file)
***********************************************************************

\end{verbatim}
 \end{routine}

%%%%% routine dsntsunvesc.f %%%%%
\begin{routine}{dsntsunvesc.f}
\begin{verbatim}


**********************************************************************
*** dsntsunvesc gives the escape velocity in km/s as a function of
*** the radius r (in meters) from the sun's core.
*** author: joakim edsjo (edsjo@physto.se)
*** input: radius in m
*** output escape velocity in km/s
*** date: 2003-11-26
**********************************************************************

      real*8 function dsntsunvesc(r)
\end{verbatim}
 \end{routine}

%%%%% routine dsntsunx2z.f %%%%%
\begin{routine}{dsntsunx2z.f}
\begin{verbatim}
***********************************************************************
*** The Sun routines uses different variables to describe position
*** in the Sun:
***    r:  radius (in meters)   [0,r_sun]
***    x:  radius in units of r_sun  [0,1]
***    z:  fraction of total column density traversed  [0,1]
***        the column density is either of p or n or the total
***        and the totals are stored in cd_sun
***
*** This routine converts from x to z (in p, n or total)
***
*** Inputs
***       x = radius in units of r_sun [0,1]
***    type = 'N', the total column density (up to that r) is calculated
***         = 'p', the column density on protons is calculated
***         = 'n', the column density on neutrons is calculated
***
*** Outputs
***       z = fraction of total column density (for chosen type) that
***           has been traversed from the centre of the Sun out to x.
***
*** Author: Joakim Edsjo, edsjo@physto.se
*** Date: 2005-11-25
***********************************************************************

      real*8 function dsntsunx2z(x,type)
\end{verbatim}
 \end{routine}

%%%%% routine dsntsunz2x.f %%%%%
\begin{routine}{dsntsunz2x.f}
\begin{verbatim}
***********************************************************************
*** The Sun routines uses different variables to describe position
*** in the Sun:
***    r:  radius (in meters)   [0,r_sun]
***    x:  radius in units of r_sun  [0,1]
***    z:  fraction of total column density traversed  [0,1]
***        the column density is either of p or n or the total
***        and the totals are stored in cd_sun
***
*** This routine converts from z (in p, n or total) to x 
***
*** Inputs
***       z =  = fraction of total column density (for chosen type) that
***              has been traversed from the centre of the Sun
***    type = 'N', the total column density (up to that r) is calculated
***         = 'p', the column density on protons is calculated
***         = 'n', the column density on neutrons is calculated
***
*** Outputs
***       x = radius in units of r_sun [0,1] that corresponds to the
***           supplied z value
***
*** Author: Joakim Edsjo, edsjo@physto.se
*** Date: 2005-11-25
***********************************************************************

      real*8 function dsntsunz2x(z,type)
\end{verbatim}
 \end{routine}

\newpage
\chapter[pb: Antiproton fluxes from the halo]{\codeb{src/pb}:\\ Antiproton fluxes from the halo}
\label{ch:src-pb}

%%%%%%%%%%%%%%%%%%%%%%%%%%%%%%%%%%%%%%%%%%%%%%%%%%%%%%%%%%%%%%%%%%%%
%%%%%%%%%%%%%%%%%%%%%%%%%%%%%%
\section{Antiprotons -- theory}

Neutralinos can annihilate each other in the halo producing leptons,
quarks, gluons, gauge bosons and Higgs bosons. The quarks, gauge
bosons and Higgs bosons will decay and/or form jets that will give
rise to antiprotons (and antineutrons which decay shortly to
antiprotons). Since antiprotons are not very abundant in the Universe,
this could in principle be a good signature for supersymmetric dark
matter. However, the cosmic rays (mainly protons) may produce
secondary antiprotons in collisions with the interstellar medium,
giving an important background. It was hoped that the difference in
kinematics between such secondary antiprotons and the primary ones
generated in neutralino annihilations would give an unambiguous
signature at low antiproton energy. However, recent calculations
indicate that other effects spoil this picture to a large degree
\cite{pbar,gaisserpbar}. It still remains true, however, that present
measurements and upper limits to the antiproton flux may be used as a
constraint to rule out some MSSM configurations with large rates.

Unfortunately, there is a larger uncertainty in
limits thus obtained than, for example, for the signal from neutrinos
from the Earth and Sun. This is due to the severe astrophysical
uncertainties about the phase space structure of the dark matter
halo, in particular the density profile towards the Galactic center.
This uncertainty will plague all indirect detection signals from
the halo: antiprotons, positrons and gamma-rays. Therefore, the
limits that can be put generally involve a combination of
MSSM and halo model parameters, and are therefore of limited use
constraining the MSSM alone.

At tree level the relevant final states for $\bar{p}$ production
are $q \bar{q}$, $\ell \bar{\ell}$, $W^{+} W^{-}$, $Z^{0} Z^{0}$,
$W^{+} H^{-}$, $Z H_{1}^{0}$, $Z H_{2}^{0}$, $H_{1}^{0} H_{3}^{0}$ and
$H_{2}^{0} H_{3}^{0}$. We have included in \ds\ all the heavier quarks ($c$,
$b$ and $t$), gauge bosons and Higgs boson final states.
  In addition, we have included the $Z \gamma$ (\cite{ub}) and
the 2 gluon (\cite{bua}; \cite{lp}) final states which occur at one
loop-level.


The hadronization and/or decay of all final states (including) gluons
is simulated with {\sc Pythia} as described in section \ref{sec:mcsim}.
A word of caution should be raised, however, that
antiproton data is not very abundant, in particular not at the
lowest antiproton lab energies which tend to dominate the signal.
Therefore an uncertainty in normalization, probably of the order of a
factor 2, cannot be excluded at least in the low energy region.

%%%%%%%%%%%%%%%
\subsection{The Antiproton Source Function}

The source function $Q_{\bar{p}}^{\chi}$ gives the number of antiprotons
per unit time, energy and volume element produced in annihilation
of neutralinos locally in space. It is given by
\begin{equation}
   Q_{\bar{p}}^{\chi}(T,\vec{x}\,)=
   (\sigma_{\rm ann}v)
   \left(\frac{\rho_{\chi}(\vec{x}\,)}{m_{\chi}}\right)^{2}
   \sum_{f}^{}\frac{dN^{f}}{dT}B^{f}
   \label{sourcefkn}
\end{equation}
where $T$ is the $\bar{p}$ kinetic energy. For a given annihilation
channel $f$, $B^{f}$ and $dN^{f} / dT$ are, respectively, the branching ratio
and the fragmentation function, and $(\sigma_{\rm ann}v)$ is the
annihilation rate at $v=0$ (which is very good approximation since the
velocity of the neutralinos in the halo is so low).
As dark matter neutralinos annihilate in pairs, the source function is
proportional to the square of the neutralino number density
$n_{\chi} = \rho_{\chi} / m_{\chi}$.
Assuming that most of the dark matter in the Galaxy is made up of neutralinos
and that these are smoothly distributed in the halo, one can directly relate
the neutralino number density to the dark matter density profile in the
galactic halo $\rho$.
%
Although what is implemented is a smooth distribution of dark
matter particles in the halo, an extension to a clumpy distribution is
potentially interesting as well (\cite{clumpy}; \cite{pieroclumpy}).

%%%%%%%%%%%%%%%%%%%%%%%%%%%%%%
\subsection{Propagation model}
\label{sec:prop}

In the absence of a well established theory to describe the
interactions of charged particles with the magnetic field of the
Galaxy and the interstellar medium, the propagation of cosmic rays has
generally been treated by postulating a semiempirical model and
fitting the necessary set of unknown parameters to available data.  A
common approach is to use a diffusion approximation defined by a
transport equation and an appropriate choice of boundary
conditions~(see e.g.\ \cite{Berezinskii}; \cite{Gaisserbook} and
references therein).

We have chosen to compute the propagation of cosmic rays in the Galaxy by
means of a
  transport equation of the diffusion type
  (see \cite{Berezinskii}; \cite{Gaisserbook}).
In the case of a stationary solution, the number density $N$ of a
stable cosmic ray
species whose distribution of sources is defined by the function of
energy and space $Q(E,\vec{x})$, is given by:
\begin{equation}
\frac{\partial{N(E,\vec{x})}}{\partial{t}} = 0 = \nabla \cdot
\left(D(R,\vec{x})\,\nabla N(E,\vec{x})\right)
- \nabla \cdot \left( \vec{u}(\vec{x})\,N(E,\vec{x}) \right)
- p(E,\vec{x})\,N(E,\vec{x}) + Q(E,\vec{x}) \;\;.
\label{eq:diff}
\end{equation}

On the right hand side
of Eq.~(\ref{eq:diff}) the first term implements the diffusion approximation
for a given diffusion coefficient $D$, generally assumed to be a function of
rigidity $R$, while the second term describes a large-scale convective
motion of velocity $\vec{u}$. The third term is added to take into account
losses due to to collisions with the interstellar matter.
It is a very good approximation to include in this term only the interactions
with interstellar hydrogen, in this case
$p$ is given by:
\begin{equation}
p(E,\vec{x}) = n^H(\vec{x}) \, v(E) \, \sigma^{\rm in}_{cr\,p}(E)
\end{equation}
where $n^H$ is the hydrogen number density in the Galaxy, $v$ is the
velocity of the cosmic ray particle considered `$cr$', while
$\sigma^{\rm in}_{cr\,p}$ is the inelastic cross section for $cr$-proton
collisions.


The propagation region is assumed to have a cylindrical symmetry: the
Galaxy is split into two parts, a disk of radius $R_h$ and height
$2\cdot h_g$, where most of the interstellar gas is confined, and a
halo of height $2\cdot h_h$ and the same radius. We assume that the
diffusion coefficient is isotropic with possibly two different values
in the disk and in the halo, reflecting the fact that in the disk
there may be a larger random component of the magnetic fields.
The spatial dependence is then:
\begin{equation}
D(\vec{x}) = D(z) = D_g \, \theta(h_g - |z|) + D_h \, \theta(|z| - h_g)\;\;.
\end{equation}
Regarding the rigidity dependence,
we consider the same functional form as in
\cite{Chardonnet} and \cite{bottinolast}:
\begin{equation}
D_l(R) = D_l^0 \left(1+\frac{R}{R_0}\right)^{0.6}
\label{eq:diffco}
\end{equation}
where $l=g,h$.

The convective term has been introduced in Eq.~(\ref{eq:diff}) to describe the
effect of particle motion against the wind of cosmic rays leaving the disk,
assuming a galactic wind of velocity
\begin{equation}
\vec{u}(\vec{x}) = \left(0, 0, u(z)\right)
\end{equation}
where
\begin{equation}
u(z)= {\rm sign}(z) \, u_h \, \theta(|z| - h_g)\;\;.
\end{equation}
An analytic solution is possible also in the case of a linearly increasing
wind~(\cite{pieroclumpy}).
The distribution of gas in the Galaxy is for convenience assumed
to have the very simple $z$ dependence
\begin{equation}
n^H(\vec{x}) = n^H(z)
= n_g^H \, \theta(h_g - |z|) + n_h^H \, \theta(|z| - h_g)
\end{equation}
where $n_h \ll n_g$ (in practice, $n_h=0$ is taken) and an average
in the radial direction is performed.

As boundary condition, it is usually assumed that cosmic rays can escape
freely at the border of the propagation region, i.e.\
\begin{equation}
N(R_h,z) = N(r,h_h) = N(r,-h_h) = 0
\label{eq:bound}
\end{equation}
as the density of cosmic rays is assumed to be negligibly small in the
intergalactic space.

The cylindrical symmetry and the free escape at the
boundaries makes it possible
to solve in \ds\
the transport equation expanding the number density distribution $N$
in a Fourier-Bessel series:
\begin{equation}
N(r,z,\theta) = \sum_{k=0}^{\infty} \,\sum_{s=1}^{\infty} \;
J_k \left(\nu_s^k \frac{r}{R_h}\right) \cdot
\left[ M_s^k(z) \cos(k\theta) + \tilde{M}_s^k(z) \sin(k\theta)\right] \\
\end{equation}
which automatically satisfies the boundary condition at $r=R_h$,
$\nu_s^k$ being the $s$-th zero of $J_k$ (the Bessel function of the
first kind  and of order $k$).
In the same way the source function can be expanded as:
\begin{equation}
Q(r,z,\theta) = \sum_{k=0}^{\infty} \,\sum_{s=1}^{\infty} \;
J_k \left(\nu_s^k \frac{r}{R_h}\right) \cdot
\left[ Q_s^k(z) \cos(k\theta) + \tilde{Q}_s^k(z) \sin(k\theta)\right]
\end{equation}
where
\begin{equation}
Q_s^k(z) = \frac{2}{{R_h}^2\,{J_{k+1}}^2(\nu_s^k)}
\int\limits_0^{R_h} dr^{\prime} \;r^{\prime}
J_k \left(\nu_s^k \frac{r^{\prime}}{R_h}\right)
\frac{1}{\alpha_k\,\pi}
\int\limits_{-\pi}^{\;\pi} d\theta^{\prime} \; \cos(k\theta^{\prime})
\,Q(r^{\prime},z,\theta^{\prime})\;\;.
\end{equation}
The equation relevant for the propagation in the $z$ direction is \cite{pbar}:
\begin{equation}
\frac{\partial}{\partial{z}} D(z) \frac{\partial}{\partial{z}} M_s^k(z)
- D(z) \left(\frac{\nu_s^k}{R_h}\right)^2 M_s^k(z)
- \frac{\partial}{\partial{z}} \left(u(z)\,M_s^k(z) \right)
- p(z) M_s^k(z) + Q_s^k(z) =0 \;\;.
\end{equation}
For $-h_g \leq z \leq h_g$ the solution is given by:
\begin{equation}
M_s^k(z) = M_s^k(0) \cosh(\lambda_g^{ks} z) - \frac{1}{D_g\, \lambda_g^{ks}}
\int\limits_{0}^{z} \;dz^{\prime}
\sinh\left(\lambda_g^{ks} (z-z^{\prime})\right) Q_s^k(z^{\prime})
\label{eq:zdep}
\end{equation}
where
\begin{eqnarray}
M_s^k(0) \ & = & \ \frac{1}{\cosh(\lambda_g^{ks} h_g)}
\left\{\frac{I_H}{\sinh\left(\lambda_h^{ks} (h_h-h_g)\right)}+
\frac{D_h\,I_{GS}}{D_g\, \lambda_g^{ks}} \,\left[\gamma_h +
\lambda_h^{ks} \coth\left(\lambda_h^{ks} (h_h-h_g)\right)\right]
+I_{GC}\right\} \nonumber \\
& & \times \left[D_g\,\lambda_g^{ks}\,\tanh\left(\lambda_g^{ks} h_g\right)
+ D_h\,\gamma_h + D_h\,\lambda_h^{ks}\,
\coth\left(\lambda_h^{ks} (h_h-h_g)\right) \right]^{-1}
\label{eq:halodiff}
\end{eqnarray}
with
\begin{eqnarray}
\lambda_g^{ks} = \sqrt{\left({\nu_s^k\over R_h}\right)^2 +
\frac{n_g^H v \sigma^{\rm in}_{cr\,p}}{D_g}} \;,\;\;\;\;\
\lambda_h^{ks} = \sqrt{\left({\nu_s^k\over R_h}\right)^2 +
\frac{n_h^H v \sigma^{\rm in}_{cr\,p}}{D_h} + {\gamma_h}^2}\;,\;\;\;\; \
\gamma_h = \frac{u_h}{2\,D_h}
\label{eq:halodiff2}
\end{eqnarray}
and
\begin{eqnarray}
I_H & = & \int\limits_{h_g}^{h_h} \;dz^{\prime} \,
\sinh\left(\lambda_h^{ks} (h_h-z^{\prime})\right) \,
\exp\left(\gamma_h (h_g-z^{\prime})\right)
\cdot \frac{Q_s^k(z^{\prime}) + Q_s^k(-z^{\prime})}{2} \nonumber \\
I_{GS} & = & \int\limits_{0}^{h_g} \;dz^{\prime} \,
\sinh\left(\lambda_g^{ks} (h_g-z^{\prime})\right)
\cdot \frac{Q_s^k(z^{\prime}) + Q_s^k(-z^{\prime})}{2} \nonumber \\
I_{GC} & = & \int\limits_{0}^{h_g} \;dz^{\prime} \,
\cosh\left(\lambda_g^{ks} (h_g-z^{\prime})\right)
\cdot \frac{Q_s^k(z^{\prime}) + Q_s^k(-z^{\prime})}{2} \;.
\label{eq:int}
\end{eqnarray}

In \ds\ we have also as an option included the propagation models by
Chardonnay et al.\ \cite{chardonnay} and Bottino et al.\
\cite{bottinopbar}.

%%%%%%%%%%%%%%%%%%%%%%%%%%%%%%
\subsection{Solar Modulation}

A complication when comparing predictions of a theoretical
model with data on cosmic rays taken at Earth is given by the solar
modulation effect. During their propagation from the interstellar
medium through the solar system, charged particles are affected by the
solar wind and tend to lose energy. The net result of the modulation
is a shift in energy between the interstellar spectrum and the
spectrum at the Earth and a substantial depletion of particles with
non-relativistic energies.

The simplest way to describe the phenomenon is the analytical force-field
approximation by Gleeson \& Axford \cite{GleesonAxford} for a spherically
symmetric model. The prescription of
this effective treatment is that, given an interstellar flux at the
heliospheric boundary, $d\Phi_{\rm b}/dT_{\rm b}$, the flux at the Earth
is related to this by
\begin{equation}
   \frac{d\Phi_{\oplus}}{dT_\oplus}(T_\oplus) = \frac{p_\oplus^2}{p_{\rm
   b}^2} \frac{d\Phi_{\rm b}}{dT_{\rm b}} (T_{\rm b})
\end{equation}
where the energy at the heliospheric boundary is given by
\begin{equation}
   E_{\rm b} = E_\oplus + |Ze|\phi_F
\end{equation}
and $p_{\otimes}$ and $p_{\rm b}$ are the momenta at the Earth and
the heliospheric boundary respectively.
Here $e$ is the absolute value of the electron charge and $Z$ the
particle charge in units of $e$ (e.g.\ $Z=-1$ for antiprotons).

An alternative approach is to solve numerically the propagation equation
of the spherically symmetric model~(\cite{fisk}): the solar modulation
parameter one has to introduce with this method roughly corresponds to
$\phi_F$ as given above. When computing solar modulated antiproton fluxes,
the two treatments seem not to be completely equivalent in the low energy
regime. Keeping this in mind, we have
anyway
implemented the force field approximation in \ds\, avoiding the
CPU time-consuming problem of
having to solve a partial differential equation for each
supersymmetric model.


%%%%%%%%%%%%%%%%%%%%%%%%%%%%%%%%%%%%%%%%%%%%%%%%%%%%%%%%%%%%%%%%%%%%

\section{Antiprotons from the halo --  routines}

................

\section{Routine headers -- fortran files}

%%%%% routine dspbaddterm.f %%%%%
\begin{routine}{dspbaddterm.f}
\begin{verbatim}
**********************************************************************
*** auxiliary function needed in dspbtd15beucl
***
*** diffusion constant in units of 10^27 cm^2 s^-1
*** axec in mb*10^10 cm s^-1
*** lambdag, lambdah in 10^-21 cm^-1
*** addterm in 1/(10^27 cm^2 s^-1 * 10^-21 cm^-1) 
***           = 1/10^6 s/cm
**********************************************************************

      real*8 function dspbaddterm(k,nusk,Jklocal,Jkplus1squared)
\end{verbatim}
 \end{routine}

%%%%% routine dspbbeupargc.f %%%%%
\begin{routine}{dspbbeupargc.f}
\begin{verbatim}
**********************************************************************
*** function called in dspbbeuparm
*** it is integrated in the cylindrical coordinate z from 0 to
*** pbhg/pbhh (linear change of variables such that z=1 => z=pbhh
*** (half height of the diffusion box)) - part associated with cosh
*** version valid in case of constant galactic wind in the z direction
*** 
*** author: piero ullio (piero@tapir.caltech.edu)
*** date: 00-07-13
*** modified: 04-01-22 (pu)
**********************************************************************

      real*8 function dspbbeupargc(z)
\end{verbatim}
 \end{routine}

%%%%% routine dspbbeupargs.f %%%%%
\begin{routine}{dspbbeupargs.f}
\begin{verbatim}
**********************************************************************
*** function called in dspbbeuparm
*** it is integrated in the cylindrical coordinate z from 0 to
*** pbhg/pbhh (linear change of variables such that z=1 => z=pbhh
*** (half height of the diffusion box)) - part associated with sinh
*** version valid in case of constant galactic wind in the z direction
*** 
*** author: piero ullio (piero@tapir.caltech.edu)
*** date: 00-07-13
*** modified: 04-01-22 (pu)
**********************************************************************

      real*8 function dspbbeupargs(z)
\end{verbatim}
 \end{routine}

%%%%% routine dspbbeuparh.f %%%%%
\begin{routine}{dspbbeuparh.f}
\begin{verbatim}
**********************************************************************
*** function called in dspbbeuparm
*** it is integrated in the cylindrical coordinate z from 
*** pbhg/pbhh to 1 (linear change of variables such that z=1 => z=pbhh
*** (half height of the diffusion box))
*** version valid in case of constant galactic wind in the z direction
*** 
*** author: piero ullio (piero@tapir.caltech.edu)
*** date: 00-07-13
*** modified: 04-01-22 (pu)
**********************************************************************

      real*8 function dspbbeuparh(z)
\end{verbatim}
 \end{routine}

%%%%% routine dspbbeuparm.f %%%%%
\begin{routine}{dspbbeuparm.f}
\begin{verbatim}
**********************************************************************
*** function called in dspbtd15beum
*** it is integrated in the cylindrical coordinate r from 0 to 1 
*** (linear change of variables such that r=1 corresponds to r=pbrh 
*** (radial extent of the diffusion box))
*** 
*** author: piero ullio (piero@tapir.caltech.edu)
*** date: 00-07-13
*** modified: 04-01-22 (pu)
**********************************************************************

      real*8 function dspbbeuparm(r)
\end{verbatim}
 \end{routine}

%%%%% routine dspbcharpar1.f %%%%%
\begin{routine}{dspbcharpar1.f}
\begin{verbatim}
**********************************************************************
*** function called in dspbtd15char
*** 
*** author: piero ullio (piero@tapir.caltech.edu)
*** date: 00-07-13
**********************************************************************

      real*8 function dspbcharpar1(x)
\end{verbatim}
 \end{routine}

%%%%% routine dspbcharpar2.f %%%%%
\begin{routine}{dspbcharpar2.f}
\begin{verbatim}
**********************************************************************
*** function called in dspbcharpar1
*** 
*** author: piero ullio (piero@tapir.caltech.edu)
*** date: 00-07-13
**********************************************************************

      real*8 function dspbcharpar2(y)
\end{verbatim}
 \end{routine}

%%%%% routine dspbgalpropdiff.f %%%%%
\begin{routine}{dspbgalpropdiff.f}
\begin{verbatim}
**********************************************************************
*** function dspbgalpropdiff calculates the differential flux of
*** antiprotons for the energy egev as a result of
*** neutralino annihilation in the halo.
*** units: gev^-1 cm^-2 sec^-1 sr^-1
*** author: edward baltz (eabaltz@alum.mit.edu), joakim edsjo
*** date: 4/28/2006
**********************************************************************

      real*8 function dspbgalpropdiff(egev)
\end{verbatim}
 \end{routine}

%%%%% routine dspbgalpropig.f %%%%%
\begin{routine}{dspbgalpropig.f}
\begin{verbatim}
      real*8 function dspbgalpropig(eep)
No header found.
\end{verbatim}
 \end{routine}

%%%%% routine dspbgalpropig2.f %%%%%
\begin{routine}{dspbgalpropig2.f}
\begin{verbatim}
      real*8 function dspbgalpropig2(eep)
No header found.
\end{verbatim}
 \end{routine}

%%%%% routine dspbkdiff.f %%%%%
\begin{routine}{dspbkdiff.f}
\begin{verbatim}
**********************************************************************
*** diffusion constant in units of 10^27 cm^2 s^-1
*** n=1 value in the halo,  n=2 value in the gas disk
***
*** author: piero ullio (piero@tapir.caltech.edu)
*** date: 00-07-13
**********************************************************************

      real*8 function dspbkdiff(rig,n)
\end{verbatim}
 \end{routine}

%%%%% routine dspbkdiffm.f %%%%%
\begin{routine}{dspbkdiffm.f}
\begin{verbatim}
**********************************************************************
*** diffusion constant in units of 10^27 cm^2 s^-1
*** n=1 value in the halo,  n=2 value in the gas disk
*** form needed for routine dspbtd15beum
***
*** author: piero ullio (piero@tapir.caltech.edu)
*** date: 00-07-13
**********************************************************************

      real*8 function dspbkdiffm(beta,rig,n)
\end{verbatim}
 \end{routine}

%%%%% routine dspbset.f %%%%%
\begin{routine}{dspbset.f}
\begin{verbatim}
      subroutine dspbset(c)
c...set parameters for antiproton routines
c...  c - character string specifying choice to be made
c...author: paolo gondolo 1999-07-14
\end{verbatim}
 \end{routine}

%%%%% routine dspbsigmavpbar.f %%%%%
\begin{routine}{dspbsigmavpbar.f}
\begin{verbatim}
      real*8 function dspbsigmavpbar(en)
c total inelastic cross section pbar + h
c tan and ng, j.phys.g 9 (1983) 227. formula 3.7
\end{verbatim}
 \end{routine}

%%%%% routine dspbtd15.f %%%%%
\begin{routine}{dspbtd15.f}
\begin{verbatim}
      real*8 function dspbtd15(tp,howinp)

**********************************************************************
*** function dspbtd15 is the containment time in 10^15 sec
***   input:
***     tp - antiproton kinetic energy in gev
***     how - 1 calculate t_diff only for requested momentum
***           2 tabulate t_diff for first call and use table for
***             subsequent calls
***           3 as 2, but also write the table to disk as 
***             pbtd-<mode>-<haloid>.dat
***           4 read table from disk on first call, and use that for
***             subsequent calls
***   output:
***     t_diff in units of 10^15 sec
*** calls dspbtd15x for the actual calculation.
*** author: joakim edsjo (edsjo@physto.se)
*** uses piero ullios propagation routines.
*** date: dec 16, 1998
*** modified: 98-07-13 paolo gondolo
**********************************************************************

\end{verbatim}
 \end{routine}

%%%%% routine dspbtd15beu.f %%%%%
\begin{routine}{dspbtd15beu.f}
\begin{verbatim}
**********************************************************************
*** function called in dspbtd15x
*** it gives the antiproton diffusion time in units of 10^15 sec
*** it assumes the diffusion model in:
***   bergstrom, edsjo & ullio, ajp 526 (1999) 215
*** inputs:
***     tp - antiproton kinetic energy (gev)
*** 
*** author: piero ullio (piero@tapir.caltech.edu)
*** date: 00-07-13
*** modified: 04-01-22 (pu)
**********************************************************************

      real*8 function dspbtd15beu(tp)
\end{verbatim}
 \end{routine}

%%%%% routine dspbtd15beucl.f %%%%%
\begin{routine}{dspbtd15beucl.f}
\begin{verbatim}
**********************************************************************
*** function that gives the antiproton diffusion time per unit volume
*** (units of 10^15 sec kpc^-3) for an antiproton point source located
*** at rcl, zcl, thetacl (in the cylidrical framework with the sun 
*** located at r=r_0, z=0 theta=0) and some small "angular width"
*** deltathetacl which makes the routine converge much faster
*** rcl, zcl, thetacl and deltathetacl are in the dspbcom.h common 
*** blocks and must be before calling this routine. rcl and zcl are in
*** kpc, thetacl and deltathetacl in rad.
*** numerical convergence gets slower for rcl->0 or zcl->0
***
*** it assumes the diffusion model in:
***   bergstrom, edsjo & ullio, ajp 526 (1999) 215
*** inputs:
***     tp - antiproton kinetic energy (gev)
***
*** the conversion from this source function to the local antiproton flux 
*** is the same as for dspbtd15beu(tp), except that dspbtd15beucl(tp)
*** must be multiplied by: 
***     int dV (rho_cl(\vec{x}_cl)/rho0)**2
***   where the integral is over the volume of the clump,
***   rho_cl(\vec{x}_cl) is the density profile in the clump
***   and the local halo density rho0 is the normalization scale used 
***   everywhere 
*** 
*** author: piero ullio (ullio@sissa.it)
*** date: 04-01-22
**********************************************************************

      real*8 function dspbtd15beucl(tp)
\end{verbatim}
 \end{routine}

%%%%% routine dspbtd15beuclsp.f %%%%%
\begin{routine}{dspbtd15beuclsp.f}
\begin{verbatim}
**********************************************************************
*** function which makes a tabulation of dspbtd15beucl as function
*** the distance between source and observer L, and neglecting the
*** weak dependence of dspbtd15beucl over the vertical coordinate for 
*** the source zcl
*** 
*** for every tp dspbtd15beuclsp is tabulated on first call in L, with        
*** L between: 
***    Lmin=0.9d0*(r_0-pbrcy) and 
***    Lmax=1.1d0*dsqrt((r_0+pbrcy)**2+pbzcy**2)
*** and stored in spline tables. 
***
*** pbrcy and pbzcy in kpc are passed through a common block in 
*** dspbcom.h and should be set before the calling this routine.
***
*** there is no internal check to verify whether between to consecutive 
*** calls, with the same tp, pbrcy and pbzcy, or halo parameters, or  
*** propagation parameters are changed. If this is done make sure, 
*** before calling this function, to reinitialize to zero the integer 
*** parameter clspset in the common block:
***
***      real*8 tpsetup
***      integer clspset
***      common/clspsetcom/tpsetup,clspset
*** 
*** input: L in kpc, tp in GeV
*** output in 10^15 s kpc^-3
***
*** author: piero ullio (ullio@sissa.it)
*** date: 04-01-22
**********************************************************************


      real*8 function dspbtd15beuclsp(L,tp)
\end{verbatim}
 \end{routine}

%%%%% routine dspbtd15beum.f %%%%%
\begin{routine}{dspbtd15beum.f}
\begin{verbatim}
**********************************************************************
*** function called in dspbtd15x
*** it gives the antiproton diffusion time in units of 10^15 sec
*** it assumes the diffusion model in:
***   bergstrom, edsjo & ullio, ajp 526 (1999) 215
***   but with the DC-like setup as in moskalenko et al.
***      ApJ 565 (2002) 280
*** inputs:
***     tp - antiproton kinetic energy (gev)
*** 
*** author: piero ullio (piero@tapir.caltech.edu)
*** date: 00-07-13
**********************************************************************

      real*8 function dspbtd15beum(tp)
\end{verbatim}
 \end{routine}

%%%%% routine dspbtd15char.f %%%%%
\begin{routine}{dspbtd15char.f}
\begin{verbatim}
**********************************************************************
*** function called in dspbtd15x
*** it gives the antiproton diffusion time in units of 10^15 sec
*** it assumes the diffusion model in:
***   chardonnet et al., phys. lett. b384 (1996) 161
***   bottino et al.,  phys. rev. d58 (1998) 123503
*** inputs:
***     tp - antiproton kinetic energy (gev)
*** 
*** author: piero ullio (piero@tapir.caltech.edu)
*** date: 00-07-13
**********************************************************************

      real*8 function dspbtd15char(tp)
\end{verbatim}
 \end{routine}

%%%%% routine dspbtd15comp.f %%%%%
\begin{routine}{dspbtd15comp.f}
\begin{verbatim}
**********************************************************************
*** function which computes the pbar diffusion time term corresponding 
*** to the axisymmetric diffuse source within a cylinder of radius
*** pbrcy and height 2* pbzcy. 
*** This routine assumes also that the Green function of
*** the diffusion equation dspbtd15beuclsp(L,tp) does depend just
*** on kintic energy tp and distance from the observer L, neglecting
*** a weak dependence on the cylindrical coordinate z.
*** For every tp, dspbtd15beuclsp is tabulated on first call in L and
*** stored in spline tables. 
*** In this function and in dspbtd15beuclsp, pbrcy and pbzcy in kpc are 
*** passed through a common block in dspbcom.h. There is no check 
*** in dspbtd15beuclsp on whether, pbrcy and pbzcy which define the
*** interval of tabulation are changed. Check header dspbtd15beuclsp
*** for more details on this and other warnings, and how to get the 
*** right implementation is such parameters are changed while running 
*** our own code
*** After the tabulation, the following integral is performed:
***
***  2 int_0^{pbzcy} int_0^{pbrcy} dr r  int_0^{2\pi} dphi
***     (dshmaxirho(r,zint)/rho0)^2 * dspbtd15beuclsp(L(z,r,theta),tp)
***
*** The triple integral is splitted into a double integral on r and 
*** theta, this result is tabulated in z and then this integral is 
*** performed. The tabulation in z has at least 100 points on a 
*** regular grid between 0 and pbzcy (this is set by the parameter
*** incompnpoints in the dspbcompint1 function), however points are 
*** added as long as the values of the function in two nearest 
*** neighbour points differs more than 10% (this is set by the 
*** parameter reratio in the dspbcompint1 function)
***
*** input: scale in kpc, tp in GeV
*** output in 10^15 s
*** 
*** author: piero ullio (ullio@sissa.it)
*** date: 04-01-22
**********************************************************************

      real*8 function dspbtd15comp(tp)
\end{verbatim}
 \end{routine}

%%%%% routine dspbtd15point.f %%%%%
\begin{routine}{dspbtd15point.f}
\begin{verbatim}
**********************************************************************
*** function which approximates the function dspbtd15comp by 
*** estimating that diffusion time term supposing to have a point 
*** source located at the galactic center but then weighting it with 
*** the emission over a whole cylinder of radius scale and 
*** height 2*scale, i.e. rho2int (to be given in kpc^3). 
*** The goodness of the approximation should be checked by comparing
***    dspbtd15point(rho2int,tp) with
***    dspbtd15comp(tp) for different value of tp and scale, 
*** and depending on the halo profile chosen and level of precision 
*** required. The comparison has to be performed but setting 
*** rho2int=dshmrho2cylint(scale,scale) and each 
*** pbrcy and pbzcy pair equal to scalebefore calling dspbtd15comp, 
*** possibly resetting the parameter clspset as well, see the header
*** of the function dspbtd15beuclsp
*** 
*** input: rho2int=dshmrho2cylint(scale,scale) in kpc^3, tp in GeV
*** output in 10^15 s
***
*** author: piero ullio (ullio@sissa.it)
*** date: 04-01-22
**********************************************************************

      real*8 function dspbtd15point(rho2int,tp)
\end{verbatim}
 \end{routine}

%%%%% routine dspbtd15x.f %%%%%
\begin{routine}{dspbtd15x.f}
\begin{verbatim}
      real*8 function dspbtd15x(tp)
**********************************************************************
*** antiproton propagation according to various models
*** dspbtd15x is containment time in 10^15 sec
*** inputs:
***     tp - antiproton kinetic energy (gev)
*** from common blocks
***     pbpropmodel - 0 leaky box with energy dependent esc. time
***                   1 chardonnet et al diffusion
***                   2 bergstrom,edsjo,ullio diffusion
***                   3 bergstrom,edsjo,ullio diffusion
***                       but with the DC-like setup as in moskalenko 
***                       et al. ApJ 565 (2002) 280
*** author: paolo gondolo 99-07-13
*** modified: piero ullio 00-07-13
*** modified: piero ullio 04-01-22
**********************************************************************
\end{verbatim}
 \end{routine}

%%%%% routine dspbtpb.f %%%%%
\begin{routine}{dspbtpb.f}
\begin{verbatim}
**********************************************************************
*** function dspbtpb gives the antiproton kinetic energy at the helio-
*** sphere as a function of the kinetic energy at the earth.
***   input:
***     tp - antiproton kinetic energy in gev
*** date: 98-02-10
**********************************************************************

      real*8 function dspbtpb(tp)
\end{verbatim}
 \end{routine}

\newpage
\chapter[rd: Relic density routines (general)]{\codeb{src/rd}:\\ Relic density routines (general)}
\label{ch:src-rd}

%%%%%%%%%%%%%%%%%%%%%%%%%%%%%%%%%%%%%%%%%%%%%%%%%%%%%%%%%%%%%%%%%%%%
\section{Relic density -- theoretical background}

\subsection{The Boltzmann equation and thermal averaging}
\label{sec:Boltzmann}

Griest and Seckel \cite{GriestSeckel} have worked out the Boltzmann
equation when coannihilations are included. We start by reviewing
their expressions and then continue by rewriting them into a more
convenient form that resembles the familiar case without
coannihilations. This allows us to use similar expressions for
calculating thermal averages and solving the Boltzmann equation
whether coannihilations are included or not. The implementation in
\ds\ is based upon the work done in \cite{edsjo97}. We will later in this
chapter, for the sake of clarification, assume that we work with
supersymmetric dark matter with the lightest neutralino being the
LSP. The routines here are completely general though and the interface
between supersymmetry and the relic density routines is handled by the
routines in \codeb{src/rn}.


\subsection{Review of the Boltzmann equation with coannihilations}

Consider annihilation of $N$ supersymmetric particles $\chi_i$
($i=1,\ldots,N$) with masses $m_i$ and internal degrees of freedom
(statistical weights) $g_i$.  Also assume that $m_1 \leq m_2 \leq
\cdots \leq m_{N-1} \leq m_N$ and that $R$-parity is conserved. Note
that for the mass of the lightest neutralino we will use the
notation $m_{\chi}$ and $m_{1}$ interchangeably.

The evolution of the number density $n_i$ of particle $i$ is
\begin{eqnarray} \label{eq:Boltzmann}
  \frac{dn_{i}}{dt} 
  &=& 
  -3 H n_{i} 
  - \sum_{j=1}^N \langle \sigma_{ij} v_{ij} \rangle 
    \left( n_{i} n_{j} - n_{i}^{\rm{eq}} n_{j}^{\rm{eq}} \right) 
  \nonumber \\ 
  & & 
  - \sum_{j\ne i} 
  \big[ \langle \sigma'_{Xij} v_{ij} \rangle 
        \left( n_i n_X - n_{i}^{\rm{eq}} n_{X}^{\rm{eq}} \right)
      - \langle \sigma'_{Xji} v_{ij} \rangle
        \left( n_j n_X - n_{j}^{\rm{eq}} n_{X}^{\rm{eq}} \right)
  \big]
  \nonumber \\ 
  & &
  - \sum_{j\ne i} 
  \big[ \Gamma_{ij} 
        \left( n_i - n_i^{\rm{eq}} \right) 
      - \Gamma_{ji} 
        \left( n_j - n_j^{\rm{eq}} \right) 
  \big] .
\end{eqnarray}
The first term on the right-hand side is the dilution due to the
expansion of the Universe. $H$ is the Hubble parameter. The second
term describes $\chi_i\chi_j$ annihilations, whose total
annihilation cross section is 
\begin{eqnarray}
  \sigma_{ij}  & = & \sum_X \sigma (\chi_i \chi_j \rightarrow X).
\end{eqnarray}
The third term describes $\chi_i \to \chi_j$ conversions by
scattering off the cosmic thermal background,
\begin{eqnarray}
  \sigma'_{Xij} & = & \sum_Y \sigma (\chi_i X \rightarrow \chi_j Y)
\end{eqnarray}
being the inclusive scattering cross section. The last term accounts
for $\chi_i$ decays, with inclusive decay rates 
\begin{eqnarray}
  \Gamma_{ij}  & = & \sum_X \Gamma (\chi_i \rightarrow \chi_j X).
\end{eqnarray}
In the previous expressions, $X$ and $Y$
are (sets of) standard model particles involved in the
interactions, $v_{ij}$ is the `relative velocity' defined by
\begin{equation}
  v_{ij} = \frac{\sqrt{(p_{i} \cdot p_{j})^2-m_{i}^2 m_{j}^2}}{E_{i} E_{j}}
\end{equation}
with $p_{i}$ and $E_{i}$ being the four-momentum and energy of 
particle $i$, and finally $n_{i}^{\rm{eq}}$ is the equilibrium number
density of particle $\chi_i$,
\begin{equation}
  n_{i}^{\rm{eq}} = \frac{g_{i}}{(2\pi)^3} \int d^3{\bf p}_{i}f_{i}
\end{equation}
where ${\bf p}_i$ is the three-momentum of particle $i$, and
 $f_i$ is its equilibrium distribution function. 
In the Maxwell-Boltzmann approximation it is given by
\begin{equation}
  f_{i} = e^{-E_{i}/T}.
\end{equation}
The thermal average $\langle\sigma_{ij}v_{ij}\rangle$ is defined
with equilibrium distributions and is given by
\begin{equation}
  \langle \sigma_{ij}v_{ij} \rangle = \frac{\int d^3{\bf
      p}_{i}d^3{\bf p}_{j} 
  f_{i}f_{j}\sigma_{ij}v_{ij}}
  {\int d^3{\bf p}_{i}d^3{\bf p}_{j}f_{i}f_{j}}
\end{equation}

Normally, the decay rate of supersymmetric particles $\chi_i$ other
than the lightest which is stable is much faster than the age of the
universe. Since we have assumed $R$-parity conservation, all of these
particles decay into the lightest one. So its final abundance is
simply described by the sum of the density of all supersymmetric
particles,
\begin{equation}
  n= \sum_{i=1}^N n_{i}.
\end{equation}
For $n$ we get the following evolution equation
\begin{equation}
  \frac{dn}{dt} = -3Hn - \sum_{i,j=1}^N \langle \sigma_{ij} v_{ij} \rangle 
  \left( n_{i}n_{j} - n_{i}^{\rm{eq}}n_{j}^{\rm{eq}} \right)
\end{equation}
where the terms on the second and third lines in
Eq.~(\ref{eq:Boltzmann}) cancel in the sum. 

The scattering rate of supersymmetric particles off particles in the
thermal background is much faster than their annihilation rate,
because the scattering cross sections $\sigma'_{Xij}$ are of the same
order of magnitude as the annihilation cross sections $\sigma_{ij}$
but the background particle density $n_X$ is much larger than each of
the supersymmetric particle densities $n_i$ when the former are
relativistic and the latter are non-relativistic, and so suppressed by
a Boltzmann factor. In this case, the $\chi_i$ distributions remain in
thermal equilibrium, and in particular their ratios are equal to the
equilibrium values,
\begin{equation}
  \frac{n_{i}}{n} \simeq \frac{n_{i}^{\rm{eq}}}{n^{\rm{eq}}}.
\end{equation}
We then get
\begin{equation} \label{eq:Boltzmann2}
  \frac{dn}{dt} =
  -3Hn - \langle \sigma_{\rm{eff}} v \rangle 
  \left( n^2 - n_{\rm{eq}}^2 \right)
\end{equation}
where
\begin{equation} \label{eq:sigmaveffdef}
  \langle \sigma_{\rm{eff}} v \rangle = \sum_{ij} \langle
  \sigma_{ij}v_{ij} \rangle \frac{n_{i}^{\rm{eq}}}{n^{\rm{eq}}}
  \frac{n_{j}^{\rm{eq}}}{n^{\rm{eq}}}.
\end{equation}

%%%%%%%%%%%%%%%%%

\subsection{Thermal averaging}
\label{sec:thermav}

So far the reviewing. Now let's continue
by reformulating the thermal averages into
more convenient expressions. 

We rewrite Eq.~(\ref{eq:sigmaveffdef}) as
\begin{equation} \label{eq:sigmaveff}
  \langle \sigma_{\rm{eff}} v \rangle = \frac{ \sum_{ij} \langle
  \sigma_{ij}v_{ij} \rangle n_{i}^{\rm{eq}} n_{j}^{\rm{eq}}}{n^2_{\rm{eq}}}
  = 
  \frac{A}{n_{\rm{eq}}^2} \, .
\end{equation}

For the denominator we obtain, 
using Boltzmann statistics for $f_i$,
\begin{equation} \label{eq:neq}
  n^{\rm eq} = \sum_i n^{\rm eq}_i = 
  \sum_i \frac{g_i}{(2\pi)^3} \int d^3 p_i 
  e^{-E_{i}/T} = 
  \frac{T}{2\pi^2} \sum_i g_i m_{i}^2
  K_{2} \left( \frac{m_{i}}{T}\right)
\end{equation}
where $K_{2}$ is the modified Bessel function of the second kind of 
order 2.

The numerator is the total annihilation rate per unit volume
at temperature $T$,
\begin{equation} 
  A = \sum_{ij} \langle \sigma_{ij} v_{ij} \rangle n_i^{\rm eq}
  n_j^{\rm eq} = \sum_{ij} \frac{g_{i}g_{j}}{(2\pi)^6} \int d^3 {\bf p}_{i}
  d^3{\bf p}_{j} f_{i}f_{j} \sigma_{ij} v_{ij}
\end{equation}
It is convenient
to cast it in a covariant form,
\begin{equation} 
  A = \sum_{ij} 
  \int W_{ij} \frac{g_i f_i d^3{\bf p}_i}{(2\pi)^3 2E_i}
  \frac{g_j f_j d^3{\bf p}_j}{(2\pi)^3 2E_j} .
\label{eq:Aij2}
\end{equation}
$W_{ij}$ is the (unpolarized) annihilation rate per unit volume
corresponding to the covariant normalization of $2E$ colliding
particles per unit volume. $W_{ij}$ is a dimensionless Lorentz
invariant, related to the (unpolarized) cross section
through\footnote{The quantity $w_{ij}$ in Ref.\ \protect\cite{SWO}
  is $W_{ij}/4$.}
\begin{equation} \label{eq:Wijcross}
  W_{ij} = 4 p_{ij} \sqrt{s} \sigma_{ij} = 4 \sigma_{ij} \sqrt{(p_i
\cdot p_j)^2 - m_i^2 m_j^2} = 4 E_{i} E_{j} \sigma_{ij} v_{ij} .
\end{equation}
Here
\begin{equation}
  p_{ij} =
\frac{\left[s-(m_i+m_j)^2\right]^{1/2}
\left[s-(m_i-m_j)^2\right]^{1/2}}{2\sqrt{s}}
\end{equation}
is the momentum of particle $\chi_i$ (or $\chi_j$) in the
center-of-mass frame of the pair $\chi_i\chi_j$.

Averaging over initial and summing over final internal states, the
contribution to $W_{ij}$ of a general $n$-body final state is
\begin{equation}
  W^{n\rm{-body}}_{ij} = 
  \frac{1}{g_i g_j S_f} \sum_{\rm{internal~d.o.f.}} 
  \int  \left| {\cal M} \right|^2 (2\pi)^4 
\delta^4(p_i+p_j-{\textstyle \sum_f}p_f) \prod_f
   \frac{d^3{\bf p}_f}{(2\pi)^3 2E_f} , 
\end{equation}
where $S_f$ is a symmetry factor accounting for identical final state
particles (if there are $K$ sets of $N_k$ identical particles,
$k=1,\dots,K$, then $S_f = \prod_{k=1}^{K} N_k!$).  In particular, 
the contribution
of a two-body final state can be written as
\begin{equation}
  W^{\rm{2-body}}_{ij\to kl} = \frac{p_{kl}}{16\pi^2 g_i g_j S_{kl} \sqrt{s}}
  \sum_{\rm{internal~d.o.f.}} \int \left| {\cal M}(ij\to kl) \right|^2
  d\Omega ,
\end{equation}
where $p_{kl}$ is the final center-of-mass momentum, $S_{kl}$ is a
symmetry factor equal to 2 for identical final particles and to 1
otherwise, and the integration is over the outgoing directions of
one of the final particles.  As usual, an average over initial
internal degrees of freedom is performed.

We now reduce the integral in the covariant expression for $A$,
Eq.~(\ref{eq:Aij2}), from 6 dimensions to 1.
Using Boltzmann statistics for $f_i$ (a good approximation for
$T\lsim m$)
\begin{equation} \label{eq:Aij2b}
  A =
  \sum_{ij} \int g_i g_j W_{ij} e^{-E_{i}/T} e^{-E_{j}/T} 
\frac{d^3{\bf p}_i}{(2\pi)^3 2E_i}
  \frac{d^3{\bf p}_j}{(2\pi)^3 2E_j} ,
\end{equation}
where ${\bf p}_{i}$ and ${\bf p}_{j}$ are the three-momenta and
$E_{i}$ and $E_{j}$ are the energies of the colliding particles.
Following the procedure in Ref.~\cite{GondoloGelmini} we then rewrite
the momentum volume element as
\begin{equation}
  d^3 {\bf p}_{i} d^3 {\bf p}_{j} = 4 \pi |{\bf p}_{i}| E_i dE_{i}
  \, 4 \pi |{\bf p}_{j}| E_j dE_{j} \, \frac{1}{2} d \cos \theta
\end{equation}
where $\theta$ is the angle between ${\bf p}_{i}$ and 
${\bf p}_{j}$. Then we change integration variables from 
$E_{i}$, $E_{j}$, $\theta$ to $E_{+}$, $E_{-}$ and $s$, given by
\begin{equation}
  \left\{ \begin{array}{lcl}
  E_{+} & = & E_{i}+E_{j} \\
  E_{-} & = & E_{i}-E_{j} \\
  s & = & m_{i}^2+m_{j}^2 + 2E_{i}E_{j}-2 |{\bf p}_{i}| |{\bf
    p}_{j}| \cos \theta,
  \end{array} \right.
\end{equation}
whence the volume element becomes
\begin{equation}
  \frac{d^3{\bf p}_i}{(2\pi)^3 2E_i} \frac{d^3{\bf p}_j}{(2\pi)^3 2E_j} =
  \frac{1}{(2\pi)^4} \frac{dE_{+}dE_{-}ds}{8},
\end{equation}
and the integration region $\{ E_i \geq m_i, E_j \geq m_j, |\cos \theta| 
\leq
1\}$ transforms into 
\begin{eqnarray}
  && s \geq (m_i+m_j)^2, \\ && E_{+} \geq \sqrt{s} , \\ && \left\vert
  E_{-} - E_{+} \frac{m_j^2-m_i^2}{s} \right\vert \leq 2 p_{ij}
  \sqrt{\frac{E_{+}^2-s}{s}}.
\end{eqnarray}

Notice now that the product of the equilibrium distribution
functions depends only on $E_{+}$ and not $E_{-}$ due to the
Maxwell-Boltzmann approximation, and that the invariant rate
$W_{ij}$ depends only on $s$ due to the neglect of final state
statistical factors. Hence we can immediately integrate over
$E_{-}$,
\begin{equation}
  \int dE_{-} = 4p_{ij} \sqrt{\frac{E_{+}^2-s}{s}}.
\end{equation}
The volume element is now
\begin{equation}
  \frac{d^3{\bf p}_i}{(2\pi)^3 2E_i} \frac{d^3{\bf p}_j}{(2\pi)^3 2E_j} = 
  \frac{1}{(2\pi)^4} \frac{p_{ij}}{2} \sqrt{\frac{E_{+}^2-s}{s}} 
dE_{+} ds 
\end{equation}

We now perform the $E_{+}$ integration. We obtain
\begin{equation}
\label{eq:As}
  A = \frac{T}{32 \pi^4} \sum_{ij} \int_{(m_i+m_j)^2}^\infty ds
  g_ig_jp_{ij} W_{ij} K_{1} \left( \frac{\sqrt{s}}{T}\right)
\end{equation}
where $K_{1}$ is the modified Bessel function of the second kind of 
order 1.

We can take the sum inside the integral and define an effective
annihilation rate $W_{\rm eff}$ through
\begin{equation}
  \sum_{ij} g_i g_j p_{ij} W_{ij} = g_1^2 p_{\rm{eff}} W_{\rm{eff}}
\end{equation}
with
\begin{equation}
\label{eq:peff}
  p_{\rm{eff}} = p_{11} = \frac{1}{2} \sqrt{s-4m_{1}^2} .
\end{equation}
In other words
\begin{equation} \label{eq:weff}
  W_{\rm{eff}} = \sum_{ij}\frac{p_{ij}}{p_{11}}
  \frac{g_ig_j}{g_1^2} W_{ij} = 
  \sum_{ij} \sqrt{\frac{[s-(m_{i}-m_{j})^2][s-(m_{i}+m_{j})^2]}
  {s(s-4m_1^2)}} \frac{g_ig_j}{g_1^2} W_{ij}.
\end{equation}
Because $W_{ij}(s) = 0 $ for $s \le (m_i+m_j)^2$, the radicand is  
never negative.

In terms of cross sections, this is equivalent to the definition
\begin{equation}
\sigma_{\rm eff} = \sum_{ij} \frac{p^2_{ij}}{p^2_{11}}
  \frac{g_ig_j}{g_1^2} \sigma_{ij}.
\end{equation}  

Eq.~(\ref{eq:As}) then reads
\begin{equation}
  A = \frac{g_1^2 T}{32 \pi^4} \int_{4m_1^2}^\infty ds
  p_{\rm eff} W_{\rm eff} K_{1} \left( \frac{\sqrt{s}}{T}\right)
\end{equation}
This can be written in a form more suitable
for numerical integration by using $p_{\rm{eff}}$ instead of $s$ as
integration variable.  From Eq.~(\ref{eq:peff}), we have 
 $ ds = 8 p_{\rm{eff}} dp_{\rm{eff}} $, and 
\begin{equation}
\label{eq:Apeff}
  A = \frac{g_1^2 T}{4 \pi^4} \int_{0}^\infty dp_{\rm eff}
  p^2_{\rm eff} W_{\rm eff} K_{1} 
  \left( \frac{\sqrt{s}}{T}\right)
\end{equation}
with
\begin{equation}
  s = 4p_{\rm{eff}}^2 + 4m_1^2
\end{equation}
So we have succeeded in rewriting $A$ as a 1-dimensional integral.

{}From Eqs.~(\ref{eq:Apeff}) and~(\ref{eq:neq}), the thermal average of
the effective cross section results
\begin{equation} \label{eq:sigmavefffin2}
  \langle \sigma_{\rm{eff}}v \rangle = \frac{\int_0^\infty
  dp_{\rm{eff}} p_{\rm{eff}}^2 W_{\rm{eff}} K_1 \left(
  \frac{\sqrt{s}}{T} \right) } { m_1^4 T \left[ \sum_i \frac{g_i}{g_1}
  \frac{m_i^2}{m_1^2} K_2 \left(\frac{m_i}{T}\right) \right]^2}.
\end{equation}
This expression is very similar to the case without coannihilations,
the differences being the denominator and the replacement of the
annihilation rate with the effective annihilation rate. 
In the absence of coannihilations, this expression
correctly reduces to the formula in Gondolo and
Gelmini~\cite{GondoloGelmini}.

The definition of an effective annihilation rate independent of
temperature is a remarkable calculational advantage. As in the case
without coannihilations, the effective annihilation rate can in fact
be tabulated in advance, before taking the thermal average and
solving the Boltzmann equation.

\begin{figure}
  \centerline{\epsfig{file=fig/rateex.eps,width=0.75\textwidth}}
  \caption{The effective invariant annihiliation rate $W_{\rm eff}$
    as a function of $p_{\rm eff}$ for an example model. 
    The final state threshold for
    annihilation into $W^+ W^-$ and the coannihilation thresholds, as
    given by Eq.~(\protect\ref{eq:weff}), are indicated.  
    The $\chi_2^0 \chi_2^0$ coannihilation threshold is too small to
    be seen.}
  \label{fig:effrate}
\end{figure}

In the effective annihilation rate, coannihilations appear
as thresholds at $\sqrt{s}$ equal to the sum of the masses of the
coannihilating particles.  We show an example in
Fig.~\ref{fig:effrate} where it is clearly seen that the
coannihilation thresholds appear in the effective invariant rate
just as final state thresholds do.  For the same example,
Fig.~\ref{fig:k1effrate} shows the differential annihilation rate
per unit volume $dA/dp_{\rm eff}$, the integrand in
Eq.~(\ref{eq:Apeff}), as a function of $p_{\rm eff}$. We have
chosen a temperature $T=m_{\chi}/20$, a typical freeze-out
temperature. The Boltzmann suppression contained in the exponential
decay of $K_{1}$ at high $p_{\rm eff}$ is clearly visible.  At
higher temperatures the peak shifts to the right and at lower
temperatures to the left.  For the particular model shown in
Figs.~\ref{fig:effrate}--\ref{fig:k1effrate}, the relic density
results $\Omega_\chi h^2=0.030$ when coannihilations are included
and $\Omega_\chi h^2=0.18$ when they are not. Coannihilations
have lowered $\Omega_\chi h^2$ by a factor of 6.

\begin{figure}
  \centerline{\epsfig{file=fig/k1rateex.eps,width=0.75\textwidth}}
  \caption{Total differential annihilation rate per unit volume 
    $dA/dp_{\rm eff}$ for the same model as in
    Fig.~\protect\ref{fig:effrate}, evaluated at a temperature
    $T=m_\chi/20$, typical of freeze-out. Notice the Boltzmann
    suppression at high $p_{\rm eff}$.}
  \label{fig:k1effrate}
\end{figure}

%%%%%%%%%%%%%%%%%%%%%%%%%%%%%%%%%%%%%%%%%%%%%%%%%%%%%%%%%%%%
\subsection{Internal degrees of freedom}
\label{sec:dof}

If we look at Eqs.~(\ref{eq:weff}) and (\ref{eq:sigmavefffin2}) we see
that we have a freedom on how to treat particles degenerate in mass,
e.g.\ a chargino can be treated either 
\begin{enumerate}
\item[a)]
  as two separate species
  $\chi_{i}^+$ and $\chi_{i}^-$, each with internal degrees of freedom
  $g_{\chi^+}=g_{\chi^-}=2$, or,
\item[b)]
  as a single species
  $\chi_{i}^\pm$ with $g_{\chi_{i}^\pm}=4$ internal degrees of freedom. 
\end{enumerate}
Of course the two views are equivalent, we just have to be careful 
including the $g_{i}$'s consistently whichever view we take.
In a), we have the advantage that all the $W_{ij}$ that enter into 
Eq.~(\ref{eq:weff}) enter as they are, i.e.\ without any correction 
factors for the degrees of freedom. On the other hand we get many 
terms in the sum that are identical and we need some book-keeping 
machinery to avoid calculating identical terms more than once. On the 
other hand, with option b), the sum over $W_{ij}$ in Eq.~(\ref{eq:weff}) 
is much simpler only containing terms that are not identical (except 
for the trivial identity $W_{ij}=W_{ji}$ which is easily taken care of). 
However, the individual $W_{ij}$ will be some linear combinations of 
the more basic $W_{ij}$ entering in option a), where the coefficients 
have to be calculated for each specific type of initial condition. 

Below we will perform this calculation to show how the $W_{ij}$ look 
like in option b) for different initial states. We will use a prime on 
the $W_{ij}$ when they refer to these combined states to indicate the 
difference.

%%%%%
\subsubsection{Neutralino-chargino annihilation}

The starting point is Eq.~(\ref{eq:weff}) which we will use to define 
the $W_{ij}$ in option b) such that $W_{\rm eff}$ is the same as in 
option a). Eq.~(\ref{eq:sigmavefffin2}) is then guaranteed to be the 
same in both cases since the sum in the denominator is linear in $g_{i}$.

Now consider annihilation between $\chi_{i}^0$ and $\chi_{c}^+$ or 
$\chi_{c}^-$. The corresponding terms in Eq.~(\ref{eq:weff}) does for 
option a) read
\begin{eqnarray}
    W_{\rm eff} & = & \sum_{ij}\frac{p_{ij}}{p_{11}} 
    \frac{g_{i}g_{j}}{g_{1}^2} W_{ij}
    =
    \frac{p_{ic}}{p_{11}} \frac{2 \cdot 2}{2^2}
    \bigg[ 
    W_{\chi_{i}^0 \chi_{c}^+} +
    W_{\chi_{i}^0 \chi_{c}^-} +
    \underbrace{W_{\chi_{c}^+ \chi_{i}^0}}_{W_{\chi_{i}^0 \chi_{c}^+}} +
    \underbrace{W_{\chi_{c}^- \chi_{i}^0}}_{W_{\chi_{i}^0 \chi_{c}^-}}
    \bigg] \nonumber \\
    & = & 2 \frac{p_{ic}}{p_{11}} 
    \bigg[
    W_{\chi_{i}^0 \chi_{c}^+} +
    \underbrace{W_{\chi_{i}^0 \chi_{c}^-}}_{W_{\chi_{i}^0 \chi_{c}^+}}
    \bigg]
    = 4 \frac{p_{ic}}{p_{11}} 
    W_{\chi_{i}^0 \chi_{c}^+}
    \label{eq:weffneucha-a}
\end{eqnarray}

For option b), we instead get
\begin{equation}
    W_{\rm eff} = \sum_{ij}\frac{p_{ij}}{p_{11}} 
    \frac{g_{i}g_{j}}{g_{1}^2} W_{ij}
    =
    \frac{p_{ic}}{p_{11}} \frac{2 \cdot 4}{2^2}
    \bigg[ 
    W'_{\chi_{i}^0 \chi_{c}^\pm} +
    \underbrace{W'_{\chi_{c}^\pm \chi_{i}^0}}_{W'_{\chi_{i}^0 \chi_{c}^\pm}}
    \bigg]
    =
    4 \frac{p_{ic}}{p_{11}} W'_{\chi_{i}^0 \chi_{c}^\pm}
    \label{eq:weffneucha-b}
\end{equation}
Comparing Eq.~(\ref{eq:weffneucha-b}) and Eq.~(\ref{eq:weffneucha-a}) 
we see that they are indentical if we make the identification
\begin{equation}
    W'_{\chi_{i}^0 \chi_{c}^\pm} \equiv W_{\chi_{i}^0 \chi_{c}^+}
\end{equation}

%%%%%
\subsubsection{Chargino-chargino annihilation}

First consider the case where we include the terms in the sum for 
which we have annihilation between $\chi_{c}^+$ or $\chi_{c}^-$ and 
$\chi_{d}^+$ or $\chi_{d}^-$ with $c \ne d$.

In option a), the corresponding terms in Eq.~(\ref{eq:weff}) reads
\begin{eqnarray}
    W_{\rm eff} & = & \sum_{ij}\frac{p_{ij}}{p_{11}} 
    \frac{g_{i}g_{j}}{g_{1}^2} W_{ij} \nonumber \\
    & = &
    \frac{p_{cd}}{p_{11}} \frac{2 \cdot 2}{2^2}
    \bigg[ 
    W_{\chi_{c}^+ \chi_{d}^+} +
    W_{\chi_{c}^+ \chi_{d}^-} +
    W_{\chi_{c}^- \chi_{d}^+} +
    W_{\chi_{c}^- \chi_{d}^-} \nonumber \\
    & & +
    \underbrace{W_{\chi_{d}^+ \chi_{c}^+}}_{W_{\chi_{c}^+ \chi_{d}^+}} +
    \underbrace{W_{\chi_{d}^+ \chi_{c}^-}}_{W_{\chi_{c}^- \chi_{d}^+}} +
    \underbrace{W_{\chi_{d}^- \chi_{c}^+}}_{W_{\chi_{c}^+ \chi_{d}^-}} +
    \underbrace{W_{\chi_{d}^- \chi_{c}^-}}_{W_{\chi_{c}^- \chi_{d}^-}}
    \bigg] \nonumber \\
    & = & 
    2 \frac{p_{cd}}{p_{11}}
    \bigg[
    W_{\chi_{c}^+ \chi_{d}^+} +
    W_{\chi_{c}^+ \chi_{d}^-} +
    \underbrace{W_{\chi_{c}^- \chi_{d}^+}}_{W_{\chi_{c}^+ \chi_{d}^-}} +
    \underbrace{W_{\chi_{c}^- \chi_{d}^-}}_{W_{\chi_{c}^+ \chi_{d}^+}}
    \bigg] \nonumber \\
    & = & 
    4 \frac{p_{cd}}{p_{11}}
    \bigg[
    W_{\chi_{c}^+ \chi_{d}^+} +
    W_{\chi_{c}^+ \chi_{d}^-}
    \bigg]
    \label{eq:weffchacha-a}
\end{eqnarray}
In option b), the corresponding terms would instead read
\begin{equation}
    W_{\rm eff} = \sum_{ij}\frac{p_{ij}}{p_{11}} 
    \frac{g_{i}g_{j}}{g_{1}^2} W'_{ij} =
    \frac{p_{cd}}{p_{11}} \frac{4 \cdot 4}{2^2}
    \bigg[ 
    W'_{\chi_{c}^\pm \chi_{d}^\pm} +
    \underbrace{W'_{\chi_{d}^\pm \chi_{c}^\pm}}_{W'_{\chi_{c}^\pm \chi_{d}^\pm}}
    \bigg]
     = 8 \frac{p_{cd}}{p_{11}} W'_{\chi_{c}^\pm \chi_{d}^\pm}
    \label{eq:weffchacha-b}
\end{equation}
Comparing Eq.~(\ref{eq:weffchacha-a}) and Eq.~(\ref{eq:weffchacha-b}) 
we see that they are identical if we make the following identifcation
\begin{equation}
    W'_{\chi_{c}^\pm \chi_{d}^\pm} \equiv \frac{1}{2} 
        \bigg[
    W_{\chi_{c}^+ \chi_{d}^+} +
    W_{\chi_{c}^+ \chi_{d}^-}
    \bigg]
\end{equation} 

For clarity, let's also consider the case where $c=d$.
In option a), the terms in $W_{\rm eff}$ are
\begin{eqnarray}
    W_{\rm eff} & = & \sum_{ij}\frac{p_{ij}}{p_{11}} 
    \frac{g_{i}g_{j}}{g_{1}^2} W_{ij}
    =
    \frac{p_{cc}}{p_{11}} \frac{2 \cdot 2}{2^2}
    \bigg[ 
    W_{\chi_{c}^+ \chi_{c}^+} +
    W_{\chi_{c}^+ \chi_{c}^-} +
    \underbrace{W_{\chi_{c}^- \chi_{c}^+}}_{W_{\chi_{c}^+ \chi_{c}^-}} +
    \underbrace{W_{\chi_{c}^- \chi_{c}^-}}_{W_{\chi_{c}^+ \chi_{c}^+}}
    \bigg] 
    \nonumber \\
    & = &
    2 \frac{p_{cc}}{p_{11}}
    \bigg[
    W_{\chi_{c}^+ \chi_{c}^+} +
    W_{\chi_{c}^+ \chi_{c}^-}
    \bigg]
    \label{eq:weffchacha-a-ident}
\end{eqnarray}
In option b), the corresponding term would instead read
\begin{equation}
    W_{\rm eff} = \sum_{ij}\frac{p_{ij}}{p_{11}} 
    \frac{g_{i}g_{j}}{g_{1}^2} W'_{ij} =
    \frac{p_{cc}}{p_{11}} \frac{4 \cdot 4}{2^2}
    W'_{\chi_{c}^\pm \chi_{c}^\pm} +
     = 4 \frac{p_{cc}}{p_{11}} W'_{\chi_{c}^\pm \chi_{c}^\pm}
    \label{eq:weffchacha-b-ident}
\end{equation}
Comparing Eq.~(\ref{eq:weffchacha-a-ident}) and 
Eq.~(\ref{eq:weffchacha-b-ident}) 
we see that they are identical if we make the following identifcation
\begin{equation}
    W'_{\chi_{c}^\pm \chi_{c}^\pm} \equiv \frac{1}{2} 
        \bigg[
    W_{\chi_{c}^+ \chi_{c}^+} +
    W_{\chi_{c}^+ \chi_{c}^-}
    \bigg]
\end{equation} 
i.e.\ the same identification as in the case $c \ne d$.

%%%%%
\subsubsection{Neutralino-sfermion annihilation}

For each sfermion we have in total four different states,
$\tilde{f}_{1}$, $\tilde{f}_{2}$, $\tilde{f}_{1}^{*}$ and
$\tilde{f}_{2}^{*}$.  Of these, the $\tilde{f}_{1}$ and
$\tilde{f}_{2}$ in general have different masses and have to be treated
separately.  Considering only one mass eigenstate $\tilde{f}_{k}$,
option a) then means that we treat $\tilde{f}_{k}$ and
$\tilde{f}_{k}^{*}$ as two separate species with $g_{i}=1$ degree of
freedom each, whereas option b) means that we treat them as one
species $\tilde{f}'_{k}$ with $g_{i}=2$ degrees of freedom.  As
before, the prime indicates that we mean both the particle and the
antiparticle state.

Note, that for squarks we also have the number of colours $N_c=3$ to take into account.
In option a) we should choose to treat even colour state differently, i.e.\ $g_i=1$, whereas $g_i=6$ in case b). 
The expressions would be the same as above except that both the expression in a) and b) would be multiplied by the colour factor $N_c=3$. The expression relating case a) and case b) is thus unaffected by this colour factor. Note however, that in option b) we take the average over the squark colours (or in this case calculate it only for one colour. See sections \ref{sec:sqsq} and \ref{sec:sfsq} below for more details.

For option a), Eq.~(\ref{eq:weff}) then reads
\begin{eqnarray}
    W_{\rm eff} & = & \sum_{ij}\frac{p_{ij}}{p_{11}} 
    \frac{g_{i}g_{j}}{g_{1}^2} W_{ij}
    =
    \frac{p_{ik}}{p_{11}} \frac{2 \cdot 1}{2^2}
    \bigg[ 
    W_{\chi_{i}^0 \tilde{f}_{k}} +
    W_{\chi_{i}^0 \tilde{f}_{k}^*} +
    \underbrace{W_{\tilde{f}_{k} \chi_{i}^0}}_{W_{\chi_{i}^0 \tilde{f}_{k}}} +
    \underbrace{W_{\tilde{f}_{k}^{*} \chi_{i}^0}}_{W_{\chi_{i}^0 \tilde{f}_{k}^*}}
    \bigg] 
    \nonumber \\
    & = &
    \frac{p_{ik}}{p_{11}}
    \bigg[
    W_{\chi_{i}^0 \tilde{f}_{k}} +
    \underbrace{W_{\chi_{i}^0 \tilde{f}_{k}^*}}_{W_{\chi_{i}^0 \tilde{f}_{k}}}
    \bigg]
    =
    2 \frac{p_{ik}}{p_{11}}
    W_{\chi_{i}^0 \tilde{f}_{k}}
    \label{eq:weffneusf-a}
\end{eqnarray}
whereas for option b), Eq.~(\ref{eq:weff}) reads
\begin{equation}
    W_{\rm eff} = \sum_{ij}\frac{p_{ij}}{p_{11}} 
    \frac{g_{i}g_{j}}{g_{1}^2} W'_{ij}
    =
    \frac{p_{ik}}{p_{11}} \frac{2 \cdot 2}{2^2}
    \bigg[ 
    W'_{\chi_{i}^0 \tilde{f'}_{k}} +
    \underbrace{W'_{\tilde{f'}_{k} \chi_{i}^0}}_{W'_{\chi_{i}^0 \tilde{f'}_{k}}}
    \bigg] 
    =
    2 \frac{p_{ik}}{p_{11}}
    W'_{\chi_{i}^0 \tilde{f'}_{k}}
    \label{eq:weffneusf-b}
\end{equation}
Comparing Eq.~(\ref{eq:weffneusf-b}) and Eq.~(\ref{eq:weffneusf-a}) we 
see that they are indentical if we make the identification
\begin{equation}
    W'_{\chi_{i}^0 \tilde{f'}_{k}} \equiv W_{\chi_{i}^0 \tilde{f}_{k}}
\end{equation}

For clarity, for squarks the corresponding expression would be
\begin{equation}
    W'_{\chi_{i}^0 \tilde{q'}_{k}} \equiv
   \frac{1}{3}\sum_{a=1}^3 W_{\chi_{i}^0 \tilde{q}_{k}^a}
\end{equation}
where $a$ is a colour index.

%%%%%
\subsubsection{Chargino-sfermion annihilation}

In option a) the chargino has $g_i=2$ and the sfermion has $g_i=1$ degrees of freedom, whereas in option b), the chargino has $g_i=4$ and the sfermion has $g_i=2$ degrees of freedom

For option a), Eq.~(\ref{eq:weff}) then reads
\begin{eqnarray}
    W_{\rm eff} & = & \sum_{ij}\frac{p_{ij}}{p_{11}} 
    \frac{g_{i}g_{j}}{g_{1}^2} W_{ij} \nonumber \\
    & = &
    \frac{p_{ck}}{p_{11}} \frac{2 \cdot 1}{2^2}
    \bigg[ 
    W_{\chi_{c}^+ \tilde{f}_{k}} +
    W_{\chi_{c}^+ \tilde{f}_{k}^*} +
    W_{\chi_{c}^- \tilde{f}_{k}} +
    W_{\chi_{c}^- \tilde{f}_{k}^*} \nonumber \\
    & & +
    \underbrace{W_{\tilde{f}_{k} \chi_{c}^+}}_
      {W_{\chi_{c}^+ \tilde{f}_{k}}} +
    \underbrace{W_{\tilde{f}_{k}^* \chi_{c}^+}}_
      {W_{\chi_{c}^+ \tilde{f}_{k}^*}} +
    \underbrace{W_{\tilde{f}_{k} \chi_{c}^-}}_
      {W_{\chi_{c}^- \tilde{f}_{k}}} +
    \underbrace{W_{\tilde{f}_{k}^* \chi_{c}^-}}_
      {W_{\chi_{c}^- \tilde{f}_{k}^*}}
    \bigg] \nonumber \\
    & = &
    \frac{p_{ck}}{p_{11}}
    \bigg[ 
    W_{\chi_{c}^+ \tilde{f}_{k}} +
    W_{\chi_{c}^+ \tilde{f}_{k}^*} +
    \underbrace{W_{\chi_{c}^- \tilde{f}_{k}}}_
      {W_{\chi_{c}^+ \tilde{f}_{k}^*}} +
    \underbrace{W_{\chi_{c}^- \tilde{f}_{k}^*}}_
      {W_{\chi_{c}^+ \tilde{f}_{k}}}
    \bigg]
    = 
    2 \frac{p_{ck}}{p_{11}}
    \bigg[ 
    W_{\chi_{c}^+ \tilde{f}_{k}} +
    W_{\chi_{c}^+ \tilde{f}_{k}^*} \bigg]
    \label{eq:weffchasf-a}
\end{eqnarray}

In option b), Eq.~(\ref{eq:weff}) reads
\begin{eqnarray}
    W_{\rm eff} & = & \sum_{ij}\frac{p_{ij}}{p_{11}} 
    \frac{g_{i}g_{j}}{g_{1}^2} W'_{ij}
    =
    \frac{p_{ck}}{p_{11}} \frac{4 \cdot 2}{2^2}
    \bigg[ 
    W'_{\chi_{c}^\pm \tilde{f'}_{k}} +
    \underbrace{W'_{\tilde{f'}_{k} \chi_{c}^\pm}}_
      {W'_{\chi_{c}^\pm \tilde{f'}_{k}}} \bigg]
    = 4 \frac{p_{ck}}{p_{11}} W'_{\chi_{c}^\pm \tilde{f'}_{k}}
    \label{eq:weffchasf-b}
\end{eqnarray}

Comparing Eq.~(\ref{eq:weffchasf-b}) and Eq.~(\ref{eq:weffchasf-a}) we 
see that they are indentical if we make the identification
\begin{equation}
    W'_{\chi_{c}^\pm \tilde{f'}_{k}} \equiv
    \frac{1}{2} \bigg[ 
    W_{\chi_{c}^+ \tilde{f}_{k}} + 
    W_{\chi_{c}^+ \tilde{f}_{k}^*}
    \bigg] 
\end{equation}

For clarity, for squarks the corresponding expression would be
\begin{equation}
    W'_{\chi_{c}^\pm \tilde{q'}_{k}} \equiv
    \frac{1}{2} \frac{1}{3}\sum_{a=1}^3 \bigg[ 
    W_{\chi_{c}^+ \tilde{q}_{k}^a} + 
    W_{\chi_{c}^+ \tilde{q}_{k}^{a*}}
    \bigg] 
\end{equation}
where $a$ is a colour index.


%%%%%
\subsubsection{Sfermion-sfermion annihilation}

First consider the case where we have annihilation between sfmerions 
of different types, i.e.\ annihilation between $\tilde{f}_{k}$ or 
$\tilde{f}_{k}^{*}$ and $\tilde{f}_{l}$ or $\tilde{f}_{l}^{*}$.

For option a), Eq.~(\ref{eq:weff}) then reads
\begin{eqnarray}
    W_{\rm eff} & = & \sum_{ij}\frac{p_{ij}}{p_{11}} 
    \frac{g_{i}g_{j}}{g_{1}^2} W_{ij}
    \nonumber \\
    & = & \frac{p_{kl}}{p_{11}} \frac{1 \cdot 1}{2^2}
    \bigg[ 
    W_{\tilde{f}_{k} \tilde{f}_{l}} +
    W_{\tilde{f}_{k} \tilde{f}_{l}^{*}} +
    W_{\tilde{f}_{k}^{*} \tilde{f}_{l}} +
    W_{\tilde{f}_{k}^{*} \tilde{f}_{l}^{*}} \nonumber \\
    & & +
    \underbrace{W_{\tilde{f}_{l} \tilde{f}_{k}}}_
       {W_{\tilde{f}_{k} \tilde{f}_{l}}} +
    \underbrace{W_{\tilde{f}_{l} \tilde{f}_{k}^{*}}}_
       {W_{\tilde{f}_{k}^{*} \tilde{f}_{l}}} +
    \underbrace{W_{\tilde{f}_{l}^{*} \tilde{f}_{k}}}_
       {W_{\tilde{f}_{k} \tilde{f}_{l}^{*}}} +
    \underbrace{W_{\tilde{f}_{l}^{*} \tilde{f}_{k}^{*}}}_
       {W_{\tilde{f}_{k}^{*} \tilde{f}_{l}^{*}}}
    \bigg]  \nonumber \\
    & = &
    \frac{1}{2} \frac{p_{kl}}{p_{11}}
    \bigg[ 
    W_{\tilde{f}_{k} \tilde{f}_{l}} +
    W_{\tilde{f}_{k} \tilde{f}_{l}^{*}} +
    \underbrace{W_{\tilde{f}_{k}^{*} \tilde{f}_{l}}}_
        {W_{\tilde{f}_{k} \tilde{f}_{l}^{*}}} +
    \underbrace{W_{\tilde{f}_{k}^{*} \tilde{f}_{l}^{*}}}_
        {W_{\tilde{f}_{k} \tilde{f}_{l}}}
    \bigg] 
    = \frac{p_{kl}}{p_{11}}
    \bigg[ 
    W_{\tilde{f}_{k} \tilde{f}_{l}} +
    W_{\tilde{f}_{k} \tilde{f}_{l}^{*}}
    \bigg] \label{eq:weffsfsf-a}
\end{eqnarray}    

In option b) we would get 
\begin{eqnarray}
    W_{\rm eff} & = & \sum_{ij}\frac{p_{ij}}{p_{11}} 
    \frac{g_{i}g_{j}}{g_{1}^2} W'_{ij}
    \nonumber \\
    & = & \frac{p_{kl}}{p_{11}} \frac{2 \cdot 2}{2^2}
    \bigg[ 
    W'_{\tilde{f'}_{k} \tilde{f'}_{l}} +
    \underbrace{W'_{\tilde{f'}_{l} \tilde{f'}_{k}}}_
       {W'_{\tilde{f'}_{k} \tilde{f'}_{l}}} \bigg]
    = 2 \frac{p_{kl}}{p_{11}} \bigg[
    W'_{\tilde{f'}_{k} \tilde{f'}_{l}}\bigg]
    \label{eq:weffsfsf-b}
\end{eqnarray}
Comparing Eq.~(\ref{eq:weffsfsf-b}) and Eq.~(\ref{eq:weffsfsf-a}) we 
see that they are indentical if we make the identification
\begin{equation}
    W'_{\tilde{f'}_{k} \tilde{f'}_{l}} \equiv 
    \frac{1}{2} \bigg[
    W_{\tilde{f}_{k} \tilde{f}_{l}} + 
    W_{\tilde{f}_{k} \tilde{f}_{l}^*} \bigg] 
\end{equation}
It is easy to show that this relation holds true even if $k=l$.

%%%%%
\subsubsection{Squark-squark annihilation}
\label{sec:sqsq}

Even though we treated sfermion-sfermion annihilation in the previous subsection, squarks have colour which can complicate things, so let's for clarity consider squarks separately.

Let's denote the squarks $\tilde{q}_k^a$ where $a$ is now a colour index. In option a) we will let each colour be a seprate species, which means that $g_i=1$ in this case. In option b) we will instead have $g_i=6$.

In option a) we would have 
\begin{eqnarray}
    W_{\rm eff} & = & \sum_{ij}\frac{p_{ij}}{p_{11}} 
    \frac{g_{i}g_{j}}{g_{1}^2} W_{ij}
    \nonumber \\
    & = & \frac{p_{kl}}{p_{11}} \frac{1 \cdot 1}{2^2}
    \sum_{a,b=1}^3 \bigg[ 
    W_{\tilde{q}_{k}^a \tilde{q}_{l}^b} +
    W_{\tilde{q}_{k}^a \tilde{q}_{l}^{b*}} +
    W_{\tilde{q}_{k}^{a*} \tilde{q}_{l}^b} +
    W_{\tilde{q}_{k}^{a*} \tilde{q}_{l}^{b*}} \nonumber \\
    & & +
    \underbrace{W_{\tilde{q}_{l}^a \tilde{q}_{k}^b}}_
       {W_{\tilde{q}_{k}^a \tilde{q}_{l}^b}} +
    \underbrace{W_{\tilde{q}_{l}^a \tilde{q}_{k}^{b*}}}_
       {W_{\tilde{q}_{k}^{a*} \tilde{q}_{l}^b}} +
    \underbrace{W_{\tilde{q}_{l}^{a*} \tilde{q}_{k}^b}}_
       {W_{\tilde{q}_{k}^a \tilde{q}_{l}^{b*}}} +
    \underbrace{W_{\tilde{q}_{l}^{a*} \tilde{q}_{k}^{b*}}}_
       {W_{\tilde{q}_{k}^{a*} \tilde{q}_{l}^{b*}}}
    \bigg]  \nonumber \\
    & = &
    \frac{1}{2} \frac{p_{kl}}{p_{11}}
    \sum_{a,b=1}^3 \bigg[ 
    W_{\tilde{q}_{k}^a \tilde{f}_{l}^b} +
    W_{\tilde{q}_{k}^a \tilde{f}_{l}^{b*}} +
    \underbrace{W_{\tilde{q}_{k}^{a*} \tilde{q}_{l}^b}}_
        {W_{\tilde{q}_{k}^a \tilde{q}_{l}^{b*}}} +
    \underbrace{W_{\tilde{q}_{k}^{a*} \tilde{q}_{l}^{b*}}}_
        {W_{\tilde{q}_{k}^a \tilde{q}_{l}^b}}
    \bigg] 
    = \frac{p_{kl}}{p_{11}}
    \sum_{a,b=1}^3 \bigg[ 
    W_{\tilde{q}_{k}^a \tilde{q}_{l}^b} +
    W_{\tilde{q}_{k}^a \tilde{q}_{l}^{b*}}
    \bigg] \label{eq:weffsqsq-a}
\end{eqnarray}    

In option b) we would get 
\begin{eqnarray}
    W_{\rm eff} & = & \sum_{ij}\frac{p_{ij}}{p_{11}} 
    \frac{g_{i}g_{j}}{g_{1}^2} W'_{ij}
    \nonumber \\
    & = & \frac{p_{kl}}{p_{11}} \frac{6 \cdot 6}{2^2}
    \bigg[ 
    W'_{\tilde{q'}_{k} \tilde{q'}_{l}} +
    \underbrace{W'_{\tilde{q'}_{l} \tilde{q'}_{k}}}_
       {W'_{\tilde{q'}_{k} \tilde{q'}_{l}}} \bigg]
    = 18 \frac{p_{kl}}{p_{11}} \bigg[
    W'_{\tilde{q'}_{k} \tilde{q'}_{l}}\bigg]
    \label{eq:weffsqsq-b}
\end{eqnarray}
Comparing Eq.~(\ref{eq:weffsqsq-b}) and Eq.~(\ref{eq:weffsqsq-a}) we 
see that they are indentical if we make the identification
\begin{equation}
    W'_{\tilde{q'}_{k} \tilde{q'}_{l}} \equiv 
    \frac{1}{2} \frac{1}{9}\sum_{a,b=1}^3 \bigg[
    W_{\tilde{q}_{k}^a \tilde{q}_{l}^b} + 
    W_{\tilde{q}_{k}^a \tilde{q}_{l}^{b*}} \bigg] 
\end{equation}
i.e. we get the same relation as for other sfermions, the only difference being that
we in option b) should also take the average over the colour states. 

%%%%%
\subsubsection{Sfermion-squark annihilation}
\label{sec:sfsq}

For clarity, if we have annihilation between a non-coloured sfermion and a squark, we would in the same way as in the previous subsection get
\begin{equation}
    W'_{\tilde{f'}_{k} \tilde{q'}_{l}} \equiv 
    \frac{1}{2} \frac{1}{3}\sum_{b=1}^3 \bigg[
    W_{\tilde{f}_{k} \tilde{q}_{l}^b} + 
    W_{\tilde{f}_{k} \tilde{q}_{l}^{b*}} \bigg] 
\end{equation}


%%%%%
\subsubsection{Summary of degrees of freedom}

We have found above the following relations between option b) and option a),
\begin{equation}
  \left\{ \begin{array}{lcl}
% neutralino-chargino
  W'_{\chi_{i}^0 \chi_{j}^\pm} & \equiv & W_{\chi_{i}^0 \chi_{j}^+} = 
    W_{\chi_{i}^0 \chi_{j}^-} \quad , \quad \forall\ i=1,\ldots,4,\ 
    j=1,2 \anl
% chargino-chargino
  W'_{\chi_{i}^\pm \chi_{j}^\pm} & \equiv & \frac{1}{2} 
  \left[ W_{\chi_{i}^+ \chi_{j}^+} +  
  W_{\chi_{i}^+ \chi_{j}^-}\right] = 
  \frac{1}{2} \left[ W_{\chi_{i}^- \chi_{j}^-} +  
  W_{\chi_{i}^- \chi_{j}^+}\right] \quad , \quad \forall\ i=1,2,\ j=1,2 \anl
% neutralino-sfermion
  W'_{\chi_{i}^0 \tilde{f'}_{k}} & \equiv & W_{\chi_{i}^0 \tilde{f}_{k}}
  \quad , \quad \forall i=1,\ldots 4,\ k=1,2 \anl
% chargino-sfermion
  W'_{\chi_{c}^\pm \tilde{f'}_{k}} & \equiv &
    \frac{1}{2} \left[ 
    W_{\chi_{c}^+ \tilde{f}_{k}} + 
    W_{\chi_{c}^+ \tilde{f}_{k}^*}
    \right]
  \quad , \quad \forall c=1,2,\ k=1,2 \anl
% sfermion-sfmerion
  W'_{\tilde{f'}_{k} \tilde{f'}_{l}} & \equiv & 
    \frac{1}{2} \left[
    W_{\tilde{f}_{k} \tilde{f}_{l}} + 
    W_{\tilde{f}_{k} \tilde{f}_{l}^*} \right] 
  \quad , \quad \forall k=1,2,\ l=1,2 \anl
% squark-squark
  W'_{\tilde{q'}_{k} \tilde{q'}_{l}} & \equiv & 
    \frac{1}{2} \frac{1}{9}\sum_{a,b=1}^3 \left[
    W_{\tilde{q}_{k}^a \tilde{q}_{l}^b} + 
    W_{\tilde{q}_{k}^a \tilde{q}_{l}^{b*}} \right] 
  \quad , \quad \forall k=1,2,\ l=1,2
  \end{array} \right.
\end{equation}
We don't list all the possible cases with squarks explicitly, the principle being that we in option b) should take the \emph{average} over the squark colour states (see the squark-squark entry in the list above).

We will choose option b) and the code (\code{dsandwdcoscn}, \code{dsandwdcoscn}, \code{dsasdwdcossfsf} and \code{dsasdwdcossfchi}) should thus return $W'$ as defined above. Note again that squarks are assumed to have $g_i=6$ degrees of freedom in this convention and the summing over colours should also be taken into account in the code.


%%%%%%%%%%%%%%%%%%%%%%%%%%%%%%%%%%%%%%
\subsection{Reformulation of the Boltzmann equation}

We now follow Gondolo and Gelmini \cite{GondoloGelmini} to 
put Eq.~(\ref{eq:Boltzmann2}) in a more convenient form by
considering the ratio of the number density to the entropy density,
\begin{equation} \label{eq:ydef}
  Y = \frac{n}{s}.
\end{equation}
Consider
\begin{equation}
  \frac{dY}{dt} = \frac{d}{dt} \left( \frac{n}{s} \right) = 
  \frac{\dot{n}}{s}-\frac{n}{s^2}\dot{s}
\end{equation}
where dot means time derivative. In absence
of entropy production, $S=R^3s$ is constant ($R$ is the scale factor).
Differentiating with respect to time we see 
that
\begin{equation}
  \dot{s} = -3\frac{\dot{R}}{R} s = -3Hs
\label{eq:entropycons}
\end{equation}
which yields
\begin{equation}
  \dot{Y} = \frac{\dot{n}}{s} + 3H \frac{n}{s}.
\end{equation}
Hence we can rewrite Eq.~(\ref{eq:Boltzmann2}) as
\begin{equation} \label{eq:Boltzmann3}
  \dot{Y} = -s  \langle \sigma_{\rm{eff}} v \rangle 
  \left( Y^2 - Y_{\rm{eq}}^2 \right).
\end{equation}

The right-hand side depends only on temperature, and it is therefore
convenient to use temperature $T$ instead of time $t$ as independent
variable. Defining $x=m_1/T$ we have
\begin{equation}
  \frac{dY}{dx} = - \frac{m_{1}}{x^2} \frac{1}{3H} \frac{ds}{dT}
  \langle \sigma_{\rm{eff}} v \rangle \left( Y^2 -
  Y_{\rm{eq}}^2 \right).
\label{eq:Boltzmann3bis}
\end{equation}
where we have used
\begin{equation}
  \frac{1}{\dot{T}} = \frac{1}{\dot{s}} \frac{ds}{dT} = -
  \frac{1}{3Hs} \frac{ds}{dT} 
\end{equation} 
which follows from Eq.~(\ref{eq:entropycons}). 
With the Friedmann equation in a radiation dominated universe
\begin{equation}
  H^2 = \frac{8\pi G \rho}{3} ,
\end{equation}
where $G$ is the gravitational constant, and the
usual parameterization of the energy and entropy densities
in terms of the effective degrees of freedom $g_{\rm{eff}}$ and
$h_{\rm{eff}}$, \begin{equation} \label{eq:geffheff}
  \rho = g_{\rm{eff}}(T) \frac{\pi^2}{30} T^4
  , \quad 
  s = h_{\rm{eff}}(T) \frac{2\pi^2}{45} T^3 ,
\end{equation}
we can cast Eq.~(\ref{eq:Boltzmann3bis})
into the form \cite{GondoloGelmini}
\begin{equation} \label{eq:Boltzmann4}
  \frac{dY}{dx} = - \sqrt{\frac{\pi}{45G}} \frac{g_{*}^{1/2}m_1}{x^2}
  \langle \sigma_{\rm{eff}} v \rangle \left( Y^2 -
  Y_{\rm{eq}}^2 \right) 
\end{equation}
where $Y_{\rm eq}$ can be written as
\begin{equation}
  Y_{\rm{eq}} = \frac{n_{\rm{eq}}}{s} = 
  \frac{45 x^2}{4 \pi^4 h_{\rm{eff}}(T)} \sum_i g_i
  \left( \frac{m_i}{m_1} \right)^2 K_{2} \left( x 
\frac{m_{i}}{m_1}\right),
\end{equation}
using Eqs.~(\ref{eq:neq}), (\ref{eq:ydef}) and
(\ref{eq:geffheff}).

The parameter $g_{*}^{1/2}$ is defined as
\begin{equation}
  g_{*}^{1/2} = \frac{h_{\rm{eff}}}{\sqrt{g_{\rm{eff}}}}
  \left( 1+\frac{T}{3h_{\rm{eff}}} \frac{d h_{\rm{eff}}}{dT}
  \right)
\end{equation}

For $g_{\rm eff}$, $h_{\rm eff}$ and $g_*^{1/2}$ we use the values
in Ref.~\cite{GondoloGelmini} with a QCD phase-transition
temperature $T_{QCD} = 150 $ MeV. Our results are insensitive to the
value of $T_{QCD}$, because due to a lower limit on the neutralino
mass the neutralino freeze-out temperature is always much larger
than $T_{QCD}$.

To obtain the relic density we integrate Eq.~(\ref{eq:Boltzmann4})
from $x=0$ to $x_0=m_\chi/T_0$ where $T_0$ is the photon temperature
of the Universe today. The relic density today in units of the
critical density is then given by
\begin{equation}
  \Omega_\chi = \rho_\chi^0/\rho_{\rm
  crit}=m_\chi s_0 Y_0/\rho_{\rm crit}
\end{equation}
where $\rho_{\rm crit}=3 H^2/8 \pi G$ is the critical density, $s_{0}$ 
is the entropy density today and $Y_{0}$ is the result of the 
integration of Eq.~(\ref{eq:Boltzmann4}). With a
background radiation temperature of $T_0=2.726$ K we finally obtain
\begin{equation} \label{eq:omegah2}
  \Omega_\chi h^2 = 2.755\times 10^8 \frac{m_\chi}{\mbox{GeV}} Y_0.
\end{equation}

%%%%% APPENDIX: Relic density - numerical details %%%%%

\section{Relic density -- numerical integration of the density equation}
\label{sec:numdens}

Let us write the evolution equation for the density,
\begin{equation} \label{eq:Boltzmann4}
  \frac{dY}{dx} = - \sqrt{\frac{\pi}{45G}} \frac{g_{*}^{1/2}m_1}{x^2}
  \langle \sigma_{\rm{eff}} v \rangle \left( Y^2 -
  Y_{\rm{eq}}^2 \right) 
\end{equation}
as
\begin{equation} 
  \frac{dY}{dx} = \lambda (Y^2 - q^2),
  \label{eq:evol}
\end{equation}
where $\lambda$ contains the annihilation rate and $q$ represents the
thermal-equilibrium density.

This equation is stiff and an explicit method, like Euler or Runge-Kutta, fails
to converge. To obtain a numerical solution, we use an adaptive implicit
trapezoidal method which we explain in the following.  Basically we discretize
the equation first with a trapezoidal then with an Euler method, and adapt the
step size according to the difference in the updated function values.

For simplicity we denote the right hand wide of eq.~(\ref{eq:evol}) as $f(x)$.
We further write $f_{i} = f(x_i)$ and similarly for the other functions
$\lambda(x)$ and $q(x)$. Given $Y_{i} = Y(x_i)$ we find $Y_{i+1} = Y(x_{i+1}) $
with $x_{i+1} = x_{i} + h$ as follows.

First we discretize the evolution equation as
\begin{equation}
  Y_{i+1} - Y_{i} = h \frac{ f_i + f_{i+1}}{2} .
\end{equation}
We insert
\begin{eqnarray}
  f_{i} &=& \lambda_{i} \left( Y_{i}^2 - q_{i}^2 \right), \\
  f_{i+1} &=& \lambda_{i+1} \left( Y_{i+1}^2 - q_{i+1}^2 \right) ,
\end{eqnarray}
and solve the resulting quadratic equation for $Y_{i+1}$ to obtain
\begin{equation}
  \label{eq:y_one}
  Y_{i+1} = \frac{c}{1+\sqrt{1+uc}} ,
\end{equation}
where
\begin{eqnarray}
c &=& 2 Y_{i} + u \left[ (q^2_{i+1} + \rho q^2_{i}) - \rho Y^2_{i} \right] , \\
u &=& h \lambda_{i+1} ,\\
\rho &=& \lambda_i / \lambda_{i+1} .
\end{eqnarray}
In the expression for $c$ we have explicitly indicated the order of evaluation
which we found avoids round-off errors. If in eq.~(\ref{eq:y_one}) $1+uc$ is
negative, we simply reduce the step $h$ to $h/2$ and try again.


Secondly we discretize the evolution equation as
\begin{equation}
  Y_{i+1} - Y_{i} = h f_{i+1} .
\end{equation}
We insert the expression for $f_{i+1}$ 
and solve the quadratic equation for $Y_{i+1}$ to obtain
\begin{equation}
  \label{eq:y_two}
  Y'_{i+1} = \frac{1}{2} \, \frac{c'}{1+\sqrt{1+uc'}} ,
\end{equation}
where
\begin{equation}
  c' = 4 \left( Y_{i} + u q^2_{i+1} \right) .
\end{equation}
Again if in eq.~(\ref{eq:y_two}) $1+uc'<0$, we reduce the step $h$ to $h/2$ and
try again.

We then adapt the step size according to the relative difference of $Y_{i+1} $
and $Y'_{i+1}$,
\begin{equation}
d = \left| \frac{ Y_{i+1} - Y'_{i+1} }{ Y_{i+1} } \right| .
\end{equation}
If the difference is larger than a prefixed $\epsilon$, set at 0.01, we reduce
the step size $h$ to $hs/\sqrt{\epsilon}$ but never to less than $h/10$.  $s$
is a safety factor set to 0.9. If $d<\epsilon$, we increase the step size by a
factor $s/\sqrt{\epsilon}$ but never by more than a factor of 5.  We do not
allow the step size to become smaller than $h_{\rm min} = 10^{-9}$. Error code
5 is reported if this happens.  Error code 4 occurs when $x_{i+1}$ is
numerically equal to $x_{i}$ because of round-off. Error code 6 occurs when the
number of steps exceeds a maximum of 100000.  Finally the initial step size is
taken to be 0.01.

%%%%%%%%%%%%%%%%%%%%%%%%%%%%%%%%%%%%%%%%%%%%%%%%%%%%%%%%%%%%%%%%%%%%
\section{Relic density -- routines}

In \ft{src/rd}, the general relic density routines are found. These
routines can be used for any dark matter candidate and the interface
to neutralino dark matter is in \ft{src/rn}. We will first discuss how
the routines for neutralino relic density are used and then how the
general routines work.

\subsection{Neutralino relic density}

\begin{sub}{function \ftb{dsrdomega}(coann,fast,xf,ierr,iwar,nfc)
\hfill r8}
  \itit{Purpose:} Calculate the relic density of the lightest
  neutralino, possibly including coannihilations between different
  neutralinos, neutralinos and charginos and between charginos.
  \itit{Input:}
  \itv{coann}{i} 
  =1: include coannihilations between
    neutralino--neutralino, neutralino--chargino and
    chargino--chargino.\\
  =2: do not include coannihilations.
  \itv{fast}{i}
  =1: Do a faster calculation, with slightly less accuracy in the
  numerical integrations and only including coannihilations (if
  \ft{coann=1}) with other particles up to 1.3 times heavier than the
  lightest neutralino.\\
  =2: Do a more accurate calculation, with higher accuracy in the
  numerical integrations and including coannihilations (if
  \ft{coann=1}) with other particles up to 2.1 times heavier than the
  lightest neutralino.
  \itit{Output:}
  \itv{xf}{r8} $x$ is defined as $x=m_\chi/T$ and \ft{xf} is the $x$
  at which freeze-out occurs (defined as the temperature at which the
  number density is a factor of two higher than the equilibrium
  density). \comment{Check!}
  \itv{ierr}{i} =0: Calculation went OK.\\
  $\neq0$: Somethig went wrong. \comment{Describe!}
  \itv{iwar}{i} =0: Calculation went OK.\\
  $\neq0$: A slight inaccuracy may have occured at a resonance or
  threshold for numerical reasons. Usually, this doesn't affect the
  result, but one should keep it in mind in case the returned relic
  density seems strange.
  \itv{nfc}{i} The number of points (in $p_{\rm eff}$) at which the
  cross section was evaluated.
\end{sub}

\begin{sub}{subroutine \ftb{dsrdwrate}(unit1,unit2,ich)}
  \itit{Purpose:} Writes a table of the partial
    annihilation rates $W_F(p,\cos\theta)$ into each final channel $F$ as a
    function of the center-of-mass momentum $p$ and at $\cos\theta=0.1$ to 
    \ft{unit2}. 
  \itit{Inputs:}
  \itv{unit1}{i} What is this?
  \itv{unit2}{i} Unit number to write output to.
  \itv{ich}{i} What initial state channel to use:\\
   =1: neutralino--neutralino annihilation\\
   =2: neutralino--chargino coannihilation\\
   =3: chargino--chargino coannihilations.
  \itit{Comment:} Only annihilation between the \emph{lightest}
    neutralinos and charginos are included.
\end{sub}

%%%%%%%%%%%%%%%%%%%%%%%%%%%%%%%%%%%%%%%%%%%%%%%%%%
\subsection{General relic density routines}

The routine that performs the actual relic density calculation is
\begin{sub}{subroutines
\ftb{dsrdens}(wx,ncoann,mcoann,dof,nrs,rm,rw,nt,tm,oh2,tf,ierr,iwar)}
  \itit{Purpose:} Calculate the relic density of a dark matter
  candidte.
  \itit{Input:}
  \itv{wx}{r8} User-defined function that returns the effective
    invariant annihilation rate, $W_{\rm eff}$, as a function of the
    effective momentum $p_{\rm eff}$. The function has to be declared
    \ft{external} in the calling routine.
  \itv{ncoann}{i} Number of particles that coannihilate.
  \itv{mcoann}{r8} An array with the masses (in GeV) that can
    coannihilate.
  \itv{dof}{r8} Number of internal degrees of freedom for the
  coannihilating particles.
  \itv{nrs}{i} Number of resonances.
  \itv{rm}{r8} An array with the masses of the resonances (in GeV).
  \itv{rw}{r8} An array with the widths of the resonances (in GeV).
  \itv{nt}{i} Number of thresholds.
  \itv{tm}{r8} An array with the $\sqrt{s}$ (in GeV) at which the
    thresholds occur.
  \itit{Output:}
  \itv{oh2}{r8} The relic density, $\Omega h^2$ where $h$ is the Hubble
  constant in units of 100 km s$^{-1}$ Mpc$^{-1}$.
  \itv{tf}{r8} The temperature (in GeV) at which the freeze-out
  occured. Freeze-out is defined to occur when the number density is 2
  times the equlibrium density.
  \itv{ierr}{i} =0: Calculation went OK.\\
    $\neq0$: Somethig went wrong. \comment{Describe!}
  \itv{iwar}{i} =0: Calculation went OK.\\
    $\neq0$: A slight inaccuracy may have occured at a resonance or
    threshold for numerical reasons. Usually, this doesn't affect the
    result, but one should keep it in mind in case the returned relic
    density seems strange.
\end{sub}
It is up to the user to prepare the input function and arrays
accordingly before calling the routine.

All internal settings of the relic density routines are set in common
blocks in \ft{dsrdcom.h}. The most important parameters that can be
changed by the user are

\begin{sub}{Important parameters in \ftb{dsrdcom.h}}
  \itit{Purpose:} Provide a set of parameters, with which the internal
  behaviour of the relic density routines can be changed.
  \itit{Parameters}
  \itv{tharsi}{i} Size of the coannihilation, resonance and threshold
    arrays (default=50). Increase this size if you have more than 50
    coannihilating particles, more than 50 resonances or more than 50
    thresholds.
  \itv{rdluerr}{i} Logical unit number where error messages are
    printed.
  \itv{rdtag}{c*12} Idtag that is printed in case of errors.
  \itv{cosmin}{r8} \ldots
  \itv{waccd}{r8} \ldots
  \itv{dpminr}{r8} \ldots
  \itv{dpthr}{r8} \ldots
  \itv{wdiffr}{r8} \ldots
  \itv{wdifft}{r8} \ldots
  \itv{hstep}{r8} \ldots
\end{sub}

When the relic density has been calculated, the integer variable \code{copart} in \code{dsandwcom.h} is set to indicate which coannihilating particles that have been included in the calculation. In Table~\ref{tab:copart}, the meaning if this variable is shown.

\begin{table}[!h]
\centering
\begin{tabular}{rrrcrrl} \hline
\multicolumn{3}{c}{\code{copart}} & & \multicolumn{2}{c}{PAW variables} \\ \cline{1-3} \cline{5-6}
Bit set & Octal value & Decimal value && \code{cop1} bit & \code{cop2} bit & Particle \\ \hline
 0 &             1 &            1 &&   0 &  -- & $\tilde{\chi}_1^0$ \\
 1 &             2 &            2 &&   1 &  -- & $\tilde{\chi}_2^0$ \\
 2 &             4 &            4 &&   2 &  -- & $\tilde{\chi}_3^0$ \\
 3 &            10 &            8 &&   3 &  -- & $\tilde{\chi}_4^0$ \\ \hline
 4 &            20 &           16 &&   4 &  -- & $\tilde{\chi}_1^\pm$ \\
 5 &            40 &           32 &&   5 &  -- & $\tilde{\chi}_2^\pm$ \\ \hline
 6 &           100 &           64 &&   6 &  -- & $\tilde{e}_1$ \\
 7 &           200 &          128 &&   7 &  -- & $\tilde{\mu}_1$ \\
 8 &           400 &          256 &&   8 &  -- & $\tilde{\tau}_1$ \\
 9 &        1\,000 &          512 &&   9 &  -- & $\tilde{e}_2$ \\
10 &        2\,000 &       1\,024 &&  10 &  -- & $\tilde{\mu}_2$ \\
11 &        4\,000 &       2\,048 &&  11 &  -- & $\tilde{\tau}_2$ \\ \hline
12 &       10\,000 &       4\,096 &&  12 &  -- & $\tilde{\nu}_e$ \\
13 &       20\,000 &       8\,192 &&  13 &  -- & $\tilde{\nu}_\mu$ \\
14 &       40\,000 &      16\,384 &&  14 &  -- & $\tilde{\nu}_\tau$ \\ \hline
15 &      100\,000 &      32\,768 &&  -- &   0 & $\tilde{u}_1$ \\
16 &      200\,000 &      65\,536 &&  -- &   1 & $\tilde{c}_1$ \\
17 &      400\,000 &     131\,072 &&  -- &   2 & $\tilde{t}_1$ \\
18 &   1\,000\,000 &     262\,144 &&  -- &   3 & $\tilde{u}_2$ \\
19 &   2\,000\,000 &     524\,288 &&  -- &   4 & $\tilde{c}_2$ \\
20 &   4\,000\,000 &  1\,048\,576 &&  -- &   5 & $\tilde{t}_2$ \\ \hline
21 &  10\,000\,000 &  2\,097\,152 &&  -- &   6 & $\tilde{d}_1$ \\
22 &  20\,000\,000 &  4\,197\,304 &&  -- &   7 & $\tilde{s}_1$ \\
23 &  40\,000\,000 &  8\,388\,608 &&  -- &   8 & $\tilde{b}_1$ \\
24 & 100\,000\,000 & 16\,777\,216 &&  -- &   9 & $\tilde{d}_2$ \\
25 & 200\,000\,000 & 33\,554\,432 &&  -- &  10 & $\tilde{s}_2$ \\
26 & 400\,000\,000 & 67\,108\,864 &&  -- &  11 & $\tilde{b}_2$ \\ \hline
\end{tabular}
\caption{The bits of \code{copart} are set to indicate which initial states that
are included in the coannihilation calculation. In the output file \code{*.omegaco}, the value of \code{copart} is written in octal format. In PAW \code{cop1} and \code{cop2} are available. Check if a bit is set with \code{btest(cop1,bit)}.}
\label{tab:copart}
\end{table}

%%%%%%%%%%%%%%%%%%%%%%%%%%%%%%%%%%%%%%%%%%%%%%%%%%
\subsection{Brief description of the internal routines}

Below, the remaining routines related to the relic density calculation
are briefly mentioned. For more details, we refer to the routines
themselves.

\begin{brief-subs}
\bsub{dsrdaddpt}
  To add one point in the $W_{\rm eff}$-$p_{\rm eff}$ table.
\bsub{dsrdcom}
  To initialize parameters in the common blocks in \ft{dsrdcom.h}. If
  you want to change these parameters yourself, include \ft{dsrdcom.h}
  in your code and change the parameters you want.
\bsub{dsrddof150}
  To prepare a table of the degrees of freedom as a
  function of the temperature in the early Universe.
\bsub{dsrddpmin}
  To return the allowed minimal distance in $p_{\rm
  eff}$ between two points in the $W_{\rm eff}$-$p_{\rm eff}$ plane.
  The returned value depends on if there is a resonance present or not
  at the given $p_{\rm eff}$.
\bsub{dsrdeqn}
  To solve the relic density equation by means of an
  implicit trapezoidal method with adaptive stepsize and termination.
\bsub{dsrdfunc}
  To return the invariant annihilation rate times the
  thermal distribution.
\bsub{dsrdfuncs}
  To provide \ft{dsrdfunc} in a form suitable for
  numerical integration.
\bsub{dsrdlny}
  To return $\ln(W_{\rm eff}$ for a given $p_{\rm
  eff}$.
\bsub{dsrdnormlz}
  To return a unit vector in a given direction.
\bsub{dsrdqad}
  To calculate the relic density with a
  quick-and-dirty method. It uses the approximative expressions in
  Kolb \& Turner with the cross section expaned in $v$.
\bsub{dsrdqrkck}
  To numerically integrate a function with a
  Runge-Kutta method
\bsub{dsrdrhs}
  To calculate terms on the right-hand side in the
  Boltzmann equation.
\bsub{dsrdspline}
  To set up the table $W_{\rm eff}$-$p_{\rm eff}$ for
  spline interpolation.
\bsub{dsrdstart}
  To sort and store information about coannihilations,
  resonances and thresholds in common blocks.
\bsub{dsrdtab}
  To set up the table $W_{\rm eff}$-$p_{\rm eff}$.
\bsub{dsrdthav}
  To calculate the thermally averaged annihilation
  cross section at a given temperature.
\bsub{dsrdthclose}
  ???\comment{What does this function do?}
\bsub{dsrdthlim}
  To determine the end-points for the thermal
  average integration.
\bsub{dsrdthtest}
  To check if a given entry in the $W_{\rm
  eff}$-$p_{\rm eff}$ table is at a threshold.
\bsub{dsrdwdwdcos}
  To write out a table of $dW_{\rm eff}/d\cos \theta$
  as a function of $\cos \theta$ for a given $p_{\rm eff}$.
\bsub{dsrdwfunc}
  To write out \ft{dsrdfunc} for a given $x=m_\chi/T$.
\bsub{dsrdwintp}
  To return the invariant rate $W_{\rm eff}$ for any
  given $p_{\rm eff}$ by performing a spline interpolation in the
  $W_{\rm eff}$-$p_{\rm eff}$ table.
\bsub{dsrdwintpch}
  To check the spline interpolation in the $W_{\rm
  eff}$-$p_{\rm eff}$ table and compare with a linear interpolation.
\bsub{dsrdwintrp}
  To write out a table of the invariant rate $W_{\rm
  eff}$ and some internal integration variables and expressions.
\bsub{dsrdwres}
  To write out the table $W_{\rm eff}$-$p_{\rm eff}$.
\end{brief-subs}

\bigskip

Below are brief descriptions of routines in \ft{src/rn} not mentioned above

\begin{brief-subs}
\bsub{dsrdres}
  To prepare the array of resonances needed before the
  call to \ft{dsrdens}.
\bsub{dsrdthr}
  To prepare the array of thresholds needed before the
  call to \ft{dsrdens}.
\end{brief-subs}

\section{Routine headers -- fortran files}

%%%%% routine dsrdaddpt.f %%%%%
\begin{routine}{dsrdaddpt.f}
\begin{verbatim}
      subroutine dsrdaddpt(wrate,pres,deltap)
c_______________________________________________________________________
c  add a point in rdrate table
c  input:
c    wrate - invariant annihilation rate (real, external)
c    pres - momentum of the point to add
c    deltap - scaling factor used in dsrdtab
c    pmax - maximum p used in dsrdtab (from common block)
c  common:
c    'dsrdcom.h' - included common blocks
c  used by dsrdtab
c  author: joakim edsjo (edsjo@physto.se)
c  modified: 01-01-31 paolo gondolo (paolo@mamma-mia.phys.cwru.edu) 
c=======================================================================
\end{verbatim}
 \end{routine}

%%%%% routine dsrdcom.f %%%%%
\begin{routine}{dsrdcom.f}
\begin{verbatim}
No header found.
\end{verbatim}
 \end{routine}

%%%%% routine dsrddof150.f %%%%%
\begin{routine}{dsrddof150.f}
\begin{verbatim}
      subroutine dsrddof150
c_______________________________________________________________________
c     table of effective degrees of freedom in the early universe
c
c     t [gev]    g_{\star}^{1/2}   g_{entropy}   for  t_{qcd} = 150 mev
c
c     common:
c       'dsrdcom.h' - included common blocks
c
c  author: paolo gondolo (gondolo@lpthe.jussieu.fr) 1994
c=======================================================================
\end{verbatim}
 \end{routine}

%%%%% routine dsrddpmin.f %%%%%
\begin{routine}{dsrddpmin.f}
\begin{verbatim}
      real*8 function dsrddpmin(p,dpmin)
c_______________________________________________________________________
c  routine to determine if there is a narrow resonance present which
c  jusifies changing dpmin to some fraction of lambda
c  author: joakim edsjo, edsjo@physto.se
c  date: april 30, 1998
c  modified: april 30, 1998.
c=======================================================================
\end{verbatim}
 \end{routine}

%%%%% routine dsrdens.f %%%%%
\begin{routine}{dsrdens.f}
\begin{verbatim}
      subroutine dsrdens(wrate,npart,mgev,dof,nrs,rm,rw,
     &  nt,tm,oh2,tf,ierr,iwar)
c_______________________________________________________________________
c  present density in units of the critical density times the
c    hubble constant squared.
c  input:
c    wrate - invariant annihilation rate (real, external)
c    npart - number of particles coannihilating
c    mgev  - relic and coannihilating mass in gev
c    dof   - internal degrees of freedom of the particles
c    nrs   - number of resonances to take special care of
c    rm    - mass of resonances in gev
c    rw    - width of resonances in gev
c    nt    - number of thresholds to take special care of
c            do not include coannihilation thresholds (that's automatic)
c    tm    - sqrt(s) of the thresholds in gev
c  output:
c    oh2   - relic density parameter times h**2 (real*8)
c    tf    - freeze-out temperature in gev (real*8)
c    ierr  - error code (integer)
c      dsbit 0 (1) = array capacity exceeded. increase nrmax in dsrdcom.h
c          1 (2) = a zero vector is given to dsrdnormlz.
c          2 (4) = step size underflow in dsrdeqn
c          3 (8) = stepsize smaller than minimum hmin in dsrdeqn
c          4 (16) = too many steps in dsrdeqn
c          5 (32) = step size underflow in dsrdqrkck
c          6 (64) = step size smaller than miminum in dsrdqrkck
c          7 (128) = too many steps in dsrdqrkck
c          8 (256) = gpindp integration failed in dsrdthav
c          9 (512) = threshold array too small. increase tharsi in dsrdcom.h
c    iwar  - warning code (integer)
c      dsbit 0 (1) = a difference of >5waccd in the ratio of w_spline
c                  and w_linear is obtained due to delta_p<dpmin.
c          1 (2) = a difference of >10waccd in the ratio of w_spline
c                  and w_linear is obtained due to delta_p<dpmin.
c          2 (4) = a difference of >15waccd in the ratio of w_spline
c                  and w_linear is obtained due to delta_p<dpmin.
c          3 (8) = wimp too heavy, d.o.f. table needs to be
c                  extended to higher temperatures. now the solution
c                  is started at a higher x than xinit (=2).
c          4 (16) = spline interpolated value too high (overflow) during
c                   check of interpolation accuracty (dsrdwintpch)
c  common:
c    'dsrdcom.h' - included common blocks
c  uses dsrdtab, dsrdeqn.
c  authors: paolo gondolo (gondolo@lpthe.jussieu.fr) 1994-1996 and
c           joakim edsjo (edsjo@physto.se) 30-april-98
c=======================================================================
\end{verbatim}
 \end{routine}

%%%%% routine dsrdeqn.f %%%%%
\begin{routine}{dsrdeqn.f}
\begin{verbatim}
      subroutine dsrdeqn(wrate,x0,x1,y1,xf,nfcn)
c_______________________________________________________________________
c  solve the relic density evolution equation by means of an implicit
c    trapezoidal method with adaptive stepsize and termination.
c  input:
c    wrate - invariant annihilation rate (real, external)
c    x0 - initial mass/temperature (real)
c    x1 - final mass/temperature (real)
c    y1 - final number/entropy densities (real)
c    nfcn - number of calls to wrate (integer)
c  common:
c    'dsrdcom.h' - included common blocks
c  uses dsrdrhs.
c  called by dsrdens.
c  author: paolo gondolo (gondolo@lpthe.jussieu.fr) 1994-1996
c  modified: joakim edsjo (edsjo@physto.se) 961212
c            Paolo Gondolo, factor added 2003
c=======================================================================
\end{verbatim}
 \end{routine}

%%%%% routine dsrdfunc.f %%%%%
\begin{routine}{dsrdfunc.f}
\begin{verbatim}
      function dsrdfunc(u,x,wrate)
c_______________________________________________________________________
c  invariant annihilation rate times thermal distribution.
c  when integrated over u, the effective thermal average times
c  m_chi^2 is obtained.
c  input:
c    u - integration variable (real)
c    x - mass/temperature (real)
c    wrate - invariant annihilation rate (real, external)
c  common:
c    'dsrdcom.h' - included common blocks
c  called by dsrdrhs, wirate, dsrdwintrp.
c  author: paolo gondolo (gondolo@lpthe.jussieu.fr) 1994-1996
c  modified: joakim edsjo (edsjo@physto.se) 98-04-29
c=======================================================================
\end{verbatim}
 \end{routine}

%%%%% routine dsrdfuncs.f %%%%%
\begin{routine}{dsrdfuncs.f}
\begin{verbatim}
      function dsrdfuncs(u)
c_______________________________________________________________________
c  10^15 * dsrdfunc.
c  input:
c    u - integration variable
c  uses dsrdfunc
c  used for gaussian integration with gadap.f
c  author: joakim edsjo (edsjo@physto.se)
c  date: 97-01-17
c=======================================================================
\end{verbatim}
 \end{routine}

%%%%% routine dsrdlny.f %%%%%
\begin{routine}{dsrdlny.f}
\begin{verbatim}
      function dsrdlny(p,wrate)
c_______________________________________________________________________
c  logarithm of the invariant rate.
c  input:
c    p - initial cm momentum (real)
c    wrate - invariant annihilation rate (real, external)
c  called by dsrdtab.
c  author: paolo gondolo (gondolo@lpthe.jussieu.fr) 1994
c=======================================================================
\end{verbatim}
 \end{routine}

%%%%% routine dsrdnormlz.f %%%%%
\begin{routine}{dsrdnormlz.f}
\begin{verbatim}
      subroutine dsrdnormlz(x,y,nx,ny)
c_______________________________________________________________________
c  find the unit vector (nx,ny) in the same direction as (x,y).
c  input:
c    x,y - coordinates of the vector (real)
c  output:
c    nx,ny - coordinates of the versor (real)
c  common:
c    'dsrdcom.h' - included common blocks
c  called by dsrdtab.
c  author: paolo gondolo (gondolo@lpthe.jussieu.fr) 1994
c=======================================================================
\end{verbatim}
 \end{routine}

%%%%% routine dsrdqad.f %%%%%
\begin{routine}{dsrdqad.f}
\begin{verbatim}
      subroutine dsrdqad(wrate,mgev,oh2,ierr)
c_______________________________________________________________________
c  present density in units of the critical density times the
c    hubble constant squared. quick and dirty method
c  input:
c    wrate - invariant annihilation rate (real, external)
c    mgev - relic and coannihilating mass in gev
c  output:
c    oh2 - relic density parameter times h**2 (real)
c    ierr - error code (integer)
c  common:
c    'dsrdcom.h' - included common blocks
c  uses dsrdtab, dsrdeqn.
c  author: paolo gondolo (gondolo@lpthe.jussieu.fr) 1994-1996
c  modified: joakim edsjo (edsjo@physto.se) 97-05-12
c=======================================================================
\end{verbatim}
 \end{routine}

%%%%% routine dsrdqrkck.f %%%%%
\begin{routine}{dsrdqrkck.f}
\begin{verbatim}
      subroutine dsrdqrkck(f,p,wrate,x1,x2,s)
c_______________________________________________________________________
c  numerical integration with runge-kutta method.
c  input:
c    f - integrand (real,external)
c    p - parameter mass/temperature (real)
c    wrate - invariant rate (real,external)
c    x1 - lower limit (real)
c    x2 - upper limit (real)
c  output:
c    s - integral (real)
c  common:
c    'dsrdcom.h' - included common blocks
c  author: paolo gondolo (gondolo@lpthe.jussieu.fr) 1994
c=======================================================================
\end{verbatim}
 \end{routine}

%%%%% routine dsrdrhs.f %%%%%
\begin{routine}{dsrdrhs.f}
\begin{verbatim}
      subroutine dsrdrhs(x,wrate,lambda,yeq,nfcn)
c_______________________________________________________________________
c  adimensional annihilation rate lambda in the boltzmann equation
c    y' = -lambda (y**2-yeq**2) and equilibrium dm density in units
c    of the entropy density.
c  input:
c    x - mass/temperature (real)
c    wrate - invariant annihilation rate (real)
c  output:
c    lambda - adimensional parameter in the evolution equation (real)
c    yeq - equilibrium number/entropy densities (real)
c    nfcn - number of calls to wrate (integer)
c  common:
c    'dsrdcom.h' - included common blocks
c  uses qrkck or dgadap.
c  called by dsrdeqn.
c  author: paolo gondolo (gondolo@lpthe.jussieu.fr) 1994-1996
c  modified: joakim edsjo (edsjo@physto.se) 98-04-28
c    bug fix 98-04-28: y_eq was too small by a factor of 2 (je)
c=======================================================================
\end{verbatim}
 \end{routine}

%%%%% routine dsrdspline.f %%%%%
\begin{routine}{dsrdspline.f}
\begin{verbatim}
      subroutine dsrdspline
c_______________________________________________________________________
c  set up 2nd derivatives for cubic dsrdspline interpolation.
c  common:
c    'dsrdcom.h' - included common blocks
c  called by dsrdtab.
c  author: paolo gondolo (gondolo@lpthe.jussieu.fr) 1994
c  modified by joakim edsjo, edsjo@physto.se, to split spline
c  at thresholds.
c  modified: april 30, 1998.
c=======================================================================
\end{verbatim}
 \end{routine}

%%%%% routine dsrdstart.f %%%%%
\begin{routine}{dsrdstart.f}
\begin{verbatim}
************************************************************************
*** dsrdstart stores coannihilation, resonance and threshold information
*** in common blocks and sort them
*** author: joakim edsjo (edsjo@physto.se)
*** date: 03-march-98
*** modifed: 08-may-98
***   08-may-98: bug with mdof not being sorted correctly fixed.
***   27-feb-02: bug with allocation of threshold array fixed.
***              increased number of possible coannihilations
************************************************************************


      subroutine dsrdstart(npart,mgev,dof,nrs,rm,rw,nt,tm)
\end{verbatim}
 \end{routine}

%%%%% routine dsrdtab.f %%%%%
\begin{routine}{dsrdtab.f}
\begin{verbatim}
      subroutine dsrdtab(wrate,xmin)
c_______________________________________________________________________
c  tabulate the invariant annihilation rate as a function of p.
c  input:
c    wrate - invariant annihilation rate (real*8, external)
c    xmin - minimum mass/temperature needed (real*8)
c  common:
c    'dsrdcom.h' - included common blocks
c  uses dsrdnormlz, dsrdlny, dsrdspline.
c  authors: paolo gondolo (gondolo@lpthe.jussieu.fr) 1994 and
c           joakim edsjo (edsjo@physto.se) 06-march-98
c=======================================================================
\end{verbatim}
 \end{routine}

%%%%% routine dsrdthav.f %%%%%
\begin{routine}{dsrdthav.f}
\begin{verbatim}
      real*8 function dsrdthav(x,wrate)
c_______________________________________________________________________
c  the thermal average of the effective annihilation cross section.
c  input:
c    x - mass/temperature (real)
c    wrate - invariant annihilation rate (real)
c  output:
c    dsrdthav - thermal averged cross section
c  common:
c    'dsrdcom.h' - included common blocks
c  uses qrkck or dgadap.
c  called by dsrdrhs
c  author: joakim edsjo (edsjo@physto.se) 98-05-01
c=======================================================================
\end{verbatim}
 \end{routine}

%%%%% routine dsrdthclose.f %%%%%
\begin{routine}{dsrdthclose.f}
\begin{verbatim}
      real*8 function dsrdthclose(p)
c_______________________________________________________________________
c  returns
c  author: joakim edsjo, edsjo@physto.se
c  date: april 30, 1998
c  modified: april 30, 1998.
c=======================================================================
\end{verbatim}
 \end{routine}

%%%%% routine dsrdthlim.f %%%%%
\begin{routine}{dsrdthlim.f}
\begin{verbatim}
      subroutine dsrdthlim
c_______________________________________________________________________
c  determine which limits in p_eff (or rather u) to use when
c  integrating for the thermal average
c  author: joakim edsjo (edsjo@physto.se)
c  date: 98-04-30
c=======================================================================
\end{verbatim}
 \end{routine}

%%%%% routine dsrdthtest.f %%%%%
\begin{routine}{dsrdthtest.f}
\begin{verbatim}
      logical function dsrdthtest(i)
c_______________________________________________________________________
c  routine to check if the momentum p with index i is below a
c     threshold and p with index i+1 is above it. if that is the case,
c     .true. is returned, otherwise .false.
c  author: joakim edsjo, edsjo@physto.se
c  date: april 30, 1998
c  modified: april 30, 1998.
c=======================================================================
\end{verbatim}
 \end{routine}

%%%%% routine dsrdwdwdcos.f %%%%%
\begin{routine}{dsrdwdwdcos.f}
\begin{verbatim}
      subroutine dsrdwdwdcos(p,n)

****************************************************************
** write out a table of dsandwdcos as a function of costheta for  **
** the given p and with n number of steps (n+1 points)        **
****************************************************************

\end{verbatim}
 \end{routine}

%%%%% routine dsrdwfunc.f %%%%%
\begin{routine}{dsrdwfunc.f}
\begin{verbatim}
************************************************************************
                      subroutine dsrdwfunc(x,wrate)
c_______________________________________________________________________
c  write out dsrdfunc for the given x = mass/temperature
c  common:
c    'dsrdcom.h' - included common blocks
c  uses dsrdfunc
c  author: joakim edsjo (edsjo@physto.se)
c  date: 97-01-20
c=======================================================================
\end{verbatim}
 \end{routine}

%%%%% routine dsrdwintp.f %%%%%
\begin{routine}{dsrdwintp.f}
\begin{verbatim}
      function dsrdwintp(p)
c_______________________________________________________________________
c  interpolation of tabulated invariant rate.
c  input:
c    p - initial cm momentum (real)
c  common:
c    'dsrdcom.h' - included common blocks
c  called by dsrdfunc.
c  author: paolo gondolo (gondolo@lpthe.jussieu.fr) 1994
c=======================================================================
\end{verbatim}
 \end{routine}

%%%%% routine dsrdwintpch.f %%%%%
\begin{routine}{dsrdwintpch.f}
\begin{verbatim}
      subroutine dsrdwintpch(p,wspline,wlin)
c_______________________________________________________________________
c  check of interpolation of tabulated invariant rate.
c  input:
c    p - initial cm momentum (real)
c  common:
c    'dsrdcom.h' - included common blocks
c  called by dsrdtab.
c  author: joakim edsjo 96-04-10
c  based on wintp.f by p. gondolo but wlin is also calculated
c=======================================================================
\end{verbatim}
 \end{routine}

%%%%% routine dsrdwintprint.f %%%%%
\begin{routine}{dsrdwintprint.f}
\begin{verbatim}
      subroutine dsrdwintprint(unit)
***********************************************************************
*** Print out a the table of invariant rate with the points
*** that are tabulated. The output is printed to unit.
***
*** Author: Joakim Edsjo, edsjo@physto.se
*** Date: 2005-10-31
***********************************************************************


\end{verbatim}
 \end{routine}

%%%%% routine dsrdwintrp.f %%%%%
\begin{routine}{dsrdwintrp.f}
\begin{verbatim}
************************************************************************
                      subroutine dsrdwintrp(wrate,unit)
c_______________________________________________________________________
c  write out a table of
c    initial cm momentum p
c    invariant annihilation rate w
c    integration variable u
c    integrand f
c    interpolated integrand g
c    interpolation relative error f/g-1
c  input:
c    unit - logical unit to write on (integer)
c    wrate - invariant annihilation rate (real, external)
c  common:
c    'dssusy.h' - file with susy common blocks
c    'dsrdcom.h' - included common blocks
c  uses dsrdtab,dsrdfunc
c  author: paolo gondolo (gondolo@lpthe.jussieu.fr) 1994
c=======================================================================
\end{verbatim}
 \end{routine}

%%%%% routine dsrdwprint.f %%%%%
\begin{routine}{dsrdwprint.f}
\begin{verbatim}
      subroutine dsrdwprint(unit,np,wrate,p_min,p_max)
***********************************************************************
*** Print out a the table of invariant rate starting at p_min, ending
*** at p_max and with np+1 number of points. The rate routine called is
*** wrate. The output is printed to unit.
***
*** Author: Joakim Edsjo, edsjo@physto.se
*** Date: 2005-10-31
***********************************************************************


\end{verbatim}
 \end{routine}

%%%%% routine dsrdwres.f %%%%%
\begin{routine}{dsrdwres.f}
\begin{verbatim}
************************************************************************
                      subroutine dsrdwres
c_______________________________________________________________________
c  write out dsrdtab and check the interpolation routine
c  common:
c    'dsrdcom.h' - included common blocks
c  uses dsrdtab,dsrdwintp.f
c  author: joakim edsjo (edsjo@physto.se)
c  date: 96-03-26
c=======================================================================
\end{verbatim}
 \end{routine}

\newpage
\chapter[rge: mSUGRA interface (Isasugra) to DarkSUSY]{\codeb{src/rge}:\\ mSUGRA interface (Isasugra) to DarkSUSY}
\label{ch:src-rge}

%%%%%%%%%%%%%%%%%%%%%%%%%%%%%%%%%%%%%%%%%%%%%%%%%%%%%%%%%%%%%%%%%%%%

\section{mSUGRA (ISASUGRA) interface to DarkSUSY}

If \code{Isasugra} is available, \ds\ can use \code{Isasugra} to
generate mSUGRA models. In \codeb{src/rge/}, routines are avaible to
transfer the mSUGRA parameters from \ds\ to \code{Isasugra}, call
\code{Isasugra} and then transfer back the results to \ds. The
philosophy of this interface is that whenever a user uses
\code{Isasugra}, we should use all the results of \code{Isasugra} also
in \ds. That means that instead of calculating the mass spectrum from
the low-energy parameters obtained from \code{Isasugra}, we extract
the masses and mixings from \code{Isasugra}.
\section{Routine headers -- fortran files}

%%%%% routine dsgive_model_isasugra.f %%%%%
\begin{routine}{dsgive\_model\_isasugra.f}
\begin{verbatim}
      subroutine dsgive_model_isasugra(m0,mhf,a0,sgnmu,tgbeta)
c----------------------------------------------------------------------
c
c     To specify the supersymmetric parameters of a model.
c     Inputs:
c        m0 - m0 parameter (GeV)
c        mhf - m_{1/2} parameter (GeV)
c        a0 - trilinear term (GeV)
c        sgnmu - sign of mu (+1.0d0 or -1.0d0)
c        tgbeta - ratio of Higgs vacuum expecation values, tan(beta)
c     Outputs:
c        The common blocks are set corresponding to the values above
c     Author: Joakim Edsjo, edsjo@physto.se
c        2002-03-12
c----------------------------------------------------------------------

\end{verbatim}
 \end{routine}

%%%%% routine dsisasugra_check.f %%%%%
\begin{routine}{dsisasugra\_check.f}
\begin{verbatim}
      subroutine dsisasugra_check(valid)
c=======================================================================
c  This routine checks that the neutralino, chargino and neutralino
c  mixing matrices extracted from ISASUGRA are consistent with the
c  DarkSUSY convention. It does this by checking that they really
c  do diagonlize the mass matrices. If any inconsistencies (apart
c  from small numerical differences are found), an error message
c  is written.
c  Author: J. Edsjo and M. Schelke, 2002-12-03
c=======================================================================
\end{verbatim}
 \end{routine}

%%%%% routine dsisasugra_darksusy.f %%%%%
\begin{routine}{dsisasugra\_darksusy.f}
\begin{verbatim}
      subroutine dsisasugra_darksusy(valid)
c=======================================================================
c  interface between ISASUGRA and DarkSUSY common blocks
c  author: E.A.Baltz, 2001 eabaltz@alum.mit.edu
c  modified by J. Edsjo, 2002-03-19 to set alph3
c  updated to Isajet 7.74 by J. Edsjo and E.A. Baltz, 2006-02-20
c  modified by P. Ullio 02-07-10, 02-11-21
c=======================================================================
\end{verbatim}
 \end{routine}

%%%%% routine dsmodelsetup_isasugra.f %%%%%
\begin{routine}{dsmodelsetup\_isasugra.f}
\begin{verbatim}
      subroutine dsmodelsetup_isasugra
c=======================================================================
c  replacement for dsmodelsetup for using ISASUGRA
c  author: E.A.Baltz, 2001 eabaltz@alum.mit.edu
c=======================================================================
\end{verbatim}
 \end{routine}

%%%%% routine dsrge_isasugra.f %%%%%
\begin{routine}{dsrge\_isasugra.f}
\begin{verbatim}
      subroutine dsrge_isasugra(unphys,valid)
c=======================================================================
c  interface to ISASUGRA (ISAJET 7.74) routines for SUSY spectra
c  author: E.A.Baltz, 2001 eabaltz@alum.mit.edu
c
c  if valid is non-zero, the model is no good
c  the valid flag is equal to the isasugra nogood flag:
c  valid  reason for model being bad
c  -----  --------------------------
c      1  TACHYONIC PARTICLES
c      2  NO EW SYMMETRY BREAKING
c      3  M(H_P)^2<0
c      4  YUKAWA>10
c      5  Z1SS NOT LSP
c      7  XT EWSB IS BAD
c      8  MHL^2<0
c      9  if in our check any Higgs mass is NaN
c  The following are not set, but can be set by uncommenting
c  the appropriate lines in dsisasugra_check.f
c  10-14  dsisasugra_check has reported a possible error in the interface
c         while checking that chargino, neutralino and sfermion mass
c         matrices are diagonlized by isasugra
c Updated to ISAJET 7.74 by J. Edsjo and E.A. Baltz, 2006-02-20
c=======================================================================
\end{verbatim}
 \end{routine}

%%%%% routine dssusy_isasugra.f %%%%%
\begin{routine}{dssusy\_isasugra.f}
\begin{verbatim}
      subroutine dssusy_isasugra(unphys,valid)
c--------------------------------------------------------------------
c     replacement for dssusy for using ISASUGRA RGE evolution
c     author: E.A. Baltz, 2001 eabaltz@alum.mit.edu
c====================================================================
\end{verbatim}
 \end{routine}

\newpage
\chapter[rn: Relic density of neutralinos (wrapper for rd routines)]{\codeb{src/rn}:\\ Relic density of neutralinos (wrapper for rd routines)}
\label{ch:src-rn}

%%%%%%%%%%%%%%%%%%%%%%%%%%%%%%%%%%%%%%%%%%%%%%%%%%%%%%%%%%%%%%%%%%%%

\section{Relic density of neutralinos}

The relic density routines in \codeb{src/rd} solve the Boltzmann
equation for any cold dark matter particle and it is up to us to tell
it what kind of particles that can participate in coannihilations and
what the effective annihilation rate is. This set-up for neutralino
dark matter is done in \codeb{dsrdomega}. This routine is therefor the
main routine the user should call, when the relic density of
neutralinos is wanted. 

What it does internally is the following:
\begin{itemize}
  \item It determines which particles that can coannihilate (based on
    their mass differences) and puts these particles into a common
    block for the annihlation rate routines (\codeb{dsanwx}) and an
    array for the relic density routines. The relic density routines
    need to know their masses and internal degrees of freedom.
  \item It checks where we have resonances and thresholds and adds
    these to an array, which is passed to the relic density
    routines. The relic density routines then use this knowledge to
    make sure the tabulation of the cross section and the integrations
    are performed correctly at these difficult points.
  \item It then calls the relic density routines to calculate the
    relic density.
\end{itemize}

The returned value is $\Omega_\chi h^2$.
\section{Routine headers -- fortran files}

%%%%% routine dsrdomega.f %%%%%
\begin{routine}{dsrdomega.f}
\begin{verbatim}
      real*8 function dsrdomega(omtype,fast,xf,ierr,iwar,nfc)

**********************************************************************
*** function dsrdomega calculates omega h^2 for the mssm neutralino
*** uses the mssm routines and the relic density routines
*** input:
***   omtype = 0 - no coann
***            1 - include all relevant coannihilations (charginos, 
***                neutralinos and sleptons)
***            2 - include only coannihilations betweeen charginos
***                and neutralinos
***            3 - include only coannihilations between sfermions
***                and the lightest neutralino
***   fast =   0 - standard accurate calculation (accuracy better than 1%)
***            1 - faster calculation: (recommended unless extreme accuracy
***                is needed).
***                * requires less accuracy in tabulation of w_eff
***            2 - quick and dirty method, i.e. expand the annihilation
***                cross section in x (not recommended)
*** output:
***   ierr = error from dsrdens or dsrdqad
*** authors: joakim edsjo, paolo gondolo
*** date: 98-03-03
*** modified: 98-03-03
***           99-07-30 pg
***           02-02-27 joakim edsjo: including sfermion coanns
***           06-02-22 paolo gondolo: streamlined inclusion of coanns
**********************************************************************

\end{verbatim}
 \end{routine}

%%%%% routine dsrdres.f %%%%%
\begin{routine}{dsrdres.f}
\begin{verbatim}
***********************************************************************
*** subroutine dsrdres sets up the resonances before calling
*** dsrdens.
*** author: joakim edsjo, (edsjo@physto.se)
*** date: 98-03-03
***********************************************************************

      subroutine dsrdres(npart,mgev,nres,rgev,rwid)
\end{verbatim}
 \end{routine}

%%%%% routine dsrdthr.f %%%%%
\begin{routine}{dsrdthr.f}
\begin{verbatim}
***********************************************************************
*** subroutine dsrdthr sets up the thresholds before calling
*** dsrdens.
*** author: joakim edsjo, (edsjo@physto.se)
*** date: 98-03-03
***********************************************************************

      subroutine dsrdthr(npart,mgev,nth,thgev)
\end{verbatim}
 \end{routine}

%%%%% routine dsrdwrate.f %%%%%
\begin{routine}{dsrdwrate.f}
\begin{verbatim}
************************************************************************
      subroutine dsrdwrate(unit1,unit2,ich)
c_______________________________________________________________________
c  write out a table of
c    initial cm momentum p
c    invariant annihilation rate w
c  input:
c    unit1 - logical unit to write total rate to (integer)
c    unit2 - logical unit to write partial differ. rates to (integer)
c    ich   - what initial channel to look at:
c              ich=1 nn-ann.  ich=2 cn-ann.  ich=3  cc-ann
c  common:
c    'dssusy.h' - file with susy common blocks
c  author: paolo gondolo (gondolo@lpthe.jussieu.fr) 1994
c  changes by je to include prtial in partials and coannihilation routines
c=======================================================================
\end{verbatim}
 \end{routine}

\newpage
\chapter[su: General SUSY model setup: masses, vertices etc]{\codeb{src/su}:\\ General SUSY model setup: masses, vertices etc}
\label{ch:src-su}

%%%%%%%%%%%%%%%%%%%%%%%%%%%%%%%%%%%%%%%%%%%%%%%%%%%%%%%%%%%%%%%%%%%%
\section{Supersymmetric model}

We will here review the definition of the MSSM as given in \cite{dspaper}.

\subsection{Parameters}

In our notation, the superpotential and the soft supersymmetry-breaking scalar
potential minimal supersymmetric standard model (MSSM) with R-parity
conservation \cite{mssm} read respectively
\begin{eqnarray}
  W &=& \epsilon_{ij} \left(
  - {\bf \hat{e}}_{R}^{*} {\bf h}_E {\bf \hat{l}}^i_{L} {\hat H}^j_1 
  - {\bf \hat{d}}_{R}^{*} {\bf h}_D {\bf \hat{q}}^i_{L} {\hat H}^j_1 
  + {\bf \hat{u}}_{R}^{*} {\bf h}_U {\bf \hat{q}}^i_{L} {\hat H}^j_2 
  - \mu {\hat H}^i_1 {\hat H}^j_2 
  \right),
\\
  V_{{\rm soft}} & = & 
  \epsilon_{ij} \left(
    -{\bf \tilde{e}}_{R}^{*} {\bf A}_E {\bf h}_E {\bf \tilde{l}}^i_{L} H^j_1 
  - {\bf \tilde{d}}_{R}^{*} {\bf A}_D {\bf h}_D {\bf \tilde{q}}^i_{L} H^j_1 
  + {\bf \tilde{u}}_{R}^{*} {\bf A}_U {\bf h}_U {\bf \tilde{q}}^i_{L} H^j_2
  - B \mu H^i_1 H^j_2 
  \right. \nonumber \\ && \phantom{\epsilon_{ij} \Bigl(} \left.
  + {\rm h.c.} 
  \right) \nonumber \\ &&
  + H^{i*}_1 m_1^2 H^i_1 + H^{i*}_2 m_2^2 H^i_2
  \nonumber \\ && +
  {\bf \tilde{q}}_{L}^{i*} {\bf M}_{Q}^{2} {\bf \tilde{q}}^i_{L} + 
  {\bf \tilde{l}}_{L}^{i*} {\bf M}_{L}^{2} {\bf \tilde{l}}^i_{L} + 
  {\bf \tilde{u}}_{R}^{*} {\bf M}_{U}^{2} {\bf \tilde{u}}_{R} + 
  {\bf \tilde{d}}_{R}^{*} {\bf M}_{D}^{2} {\bf \tilde{d}}_{R} + 
  {\bf \tilde{e}}_{R}^{*} {\bf M}_{E}^{2} {\bf \tilde{e}}_{R} .
\end{eqnarray}
Here $i$ and $j$ are SU(2) indices ($\epsilon_{12} = +1$), ${\bf h}$'s, ${\bf
  A}$'s and ${\bf M}$'s are $3\times3$ matrices in generation space,
and the other boldface letters are vectors in generation space.
  
The current version of \ds\ uses only a restricted set of parameters.
Namely the number of free parameters (a grand total of 124 \cite{dimopoulos95})
is reduced by setting the off-diagonal elements of the ${\bf A}$'s and ${\bf
  M}$'s to zero and imposing CP conservation (except in the CKM matrix). 

\subsection{Mass spectrum}

For easy reference, we now give the particle mass matrices, together with our
convention for the mixing matrices.

Concerning the Higgs sector, we choose as independent parameters $\tan\beta$
and the mass $m_A$ of the CP-odd Higgs boson. The code provides six options
for the calculation of the Higgs masses: 
\ft{higloop}=0:  tree level formulas;
\ft{higloop}=1: the effective potential approach in
\cite{ellis91,brignole91} (correcting the sign of $\mu$ in eq.~(4) of
\cite{brignole91}); 
\ft{higloop}=2: the effective potential approach in \cite{drees92}
  with addition of D-terms and correction of some signs and numerical factors;
\ft{higloop}=3: the analytical approximations to the
  RGE-improved effective potential in \cite{carena95};
\ft{higloop}=4: the pole mass calculation in \cite{carena96};
\ft{higloop}=5: FeynHiggs (requires FeynHiggs to be
installed) \cite{feynhiggs};
\ft{higloop}=6: FeynHiggsFast (default) \cite{feynhiggsfast}.

The  masses of the Higgs bosons are obtained from
\begin{eqnarray}
&&  {\cal M}^2_{H} = 
  \left( \matrix{ 
      {m_Z^2 \cos^2\beta + m_A^2 \sin^2\beta + \Delta_{11} } &
      {-\sin\beta\cos\beta(m_Z^2+m_A^2) + \Delta_{12} }
      \cr
      {-\sin\beta\cos\beta(m_Z^2+m_A^2) + \Delta_{21} } &
      {m_Z^2 \sin^2\beta + m_A^2 \cos^2\beta + \Delta_{22} }
      } \right) 
  \\
&&  m^2_{H^{\pm}} = m_{A}^2+m_W^2 + \Delta_{\pm}.
\end{eqnarray}
The quantities $\Delta_{ij}$ and $\Delta_{\pm}$ are the one-loop
radiative corrections, calculated acording to the value of \ft{higloop} as
described above. Diagonalization of $ {\cal M}^2_{H} $ gives the two CP-even
Higgs boson masses, $ m_{H_{1,2}} $, and their mixing angle $\alpha$ ($ -\pi/2
< \alpha < 0$). For \ft{higloop}=4, the pole masses are then obtained solving
$ m^{2\rm pole}_{H_i} = m^2_{H_i} + \Pi_{ii}(m^{2\rm pole}_{H_i}) - \Pi_{ii}(0)
$, where $\Pi_{ii}(p^2)$ is $H_iH_i$ the self-energy. In this case, $m_{H_3}$
is the pole mass and $m_A$ is the running mass.

\begin{table}
\centering
\begin{tabular}{lllll} \hline
 & \multicolumn{4}{c}{Higgs boson} \\ \cline{2-5}
Channel & $H_1^0$ & $H_2^0$ & $H_3^0$ & $H^+$ \\
\code{i=} & \code{j=1} & \code{j=2} & \code{j=3} & \code{j=4} \\ \hline
1  & $c\bar{c}$          & $c\bar{c}$          & $c\bar{c}$          & $u \bar{d}$ \\
2  & $b \bar{b}$         & $b \bar{b}$         & $b \bar{b}$         & $u \bar{s}$ \\
3  & $t\bar{t}$          & $t\bar{t}$          & $t\bar{t}$          & $u \bar{b}$ \\
4  & $\tau^+ \tau^-$     & $\tau^+ \tau^-$     & $\tau^+ \tau^-$     & $c \bar{d}$ \\
5  & $W^+W^-$            & $W^+W^-$            & --                  & $c \bar{s}$ \\ \hline
6  & $Z^0 Z^0$           & $Z^0 Z^0$           & --                  & $c \bar{b}$ \\
7  & --                  & $H_1^0 H_1^0$       & --                  & $t \bar{d}$ \\
8  & $H_2^0 H_2^0$       & --                  & --                  & $t \bar{s}$ \\
9  & $H_3^0 H_3^0$       & $H_3^0 H_3^0$       & --                  & $t \bar{b}$ \\
10 & $H^+ H^-$           & $H^+ H^-$           & --                  & $\nu_e e^+$ \\ \hline
11 & --                  & --                  & $Z H_1^0$           & $\nu_\mu \mu^+$ \\       
12 & --                  & --                  & $Z H_2^0$           & $\nu_\tau\tau^+$\\
13 & $Z H_3^0$           & $Z H_3^0$           & --                  & $W^+ H_1^0$ \\
14 & $W^+ H^- / W^- H^+$ & $W^+ H^- / W^- H^+$ & $W^+ H^- / W^- H^+$ & $W^+ H_2^0$ \\
15 & $\mu^+ \mu^-$       & $\mu^+ \mu^-$       & $\mu^+ \mu^-$       & $W^+ H_3^0$ \\ \hline
16 & $s\bar{s}$          & $s\bar{s}$          & $s\bar{s}$          & --\\
17 & $gg$                & $gg$                & $gg$                & --\\
18 & $\gamma \gamma$     & $\gamma \gamma$     & $\gamma \gamma$     & --\\
19 & $Z^0 \gamma$        & $Z^0 \gamma$        & $Z^0 \gamma$        & --\\
20 & $\tilde{f} \tilde{f}'$ & $\tilde{f} \tilde{f}'$ & $\tilde{f} \tilde{f}'$ & $\tilde{f} \tilde{f}'$\\ \hline
\end{tabular}
\caption{Higgs partial widths \code{hdwidth(i,j)}. Index \code{i} refers to the decay channel and index \code{j} to the Higgs boson. All widths are given in GeV\@. Note that typically we have that $m_{H_2}< m_{H_3}<m_{H^+}<m_{H_1}$ so many of these decay channels are not kinematically allowed, but included for completeness. If the \code{HDECAY} interface is used, the channels where $m_{H_2}< m_{H_3}<m_{H^+}<m_{H_1}$ is not satisfied are not included. Channels 16--19 are only included if HDECAY is used.}
\label{tab:hwidth}
\end{table}

The Higgs widths are calculated at tree level, but with QCD corrections \cite{higgsqcd}. The decays to supersymmetric particles are also included in the total width, so the sum of the partial widths in Table \ref{tab:hwidth} does not necessarily sum up to the total width given in \code{width(k)}. The loop corrections are also available via an interface to \code{HDECAY}.

  
The neutralinos $ \tilde{\chi}^0_i$ are linear combinations of the neutral
gauginos ${\tilde B}$, ${\tilde W_3}$ and of the neutral higgsinos ${\tilde
  H_1^0}$, ${\tilde H_2^0}$.  In this basis, we write their mass matrix as
\begin{eqnarray}
  {{\cal M}}_{\tilde \chi^0_{1,2,3,4}} = 
  \left( \matrix{
  {M_1} & 0 & -m_Z s_W c_\beta & +m_Z s_W s_\beta \cr
  0 & {M_2} & +m_Z c_W c_\beta & -m_Z c_W s_\beta \cr
  -m_Z s_W c_\beta & +m_Z c_W c_\beta & \delta_{33} & -\mu \cr
  +m_Z s_W s_\beta & -m_Z c_W s_\beta & -\mu & \delta_{44} \cr
  } \right) ,
\end{eqnarray}
with $c_W=\cos\theta_W$, $s_W=\sin\theta_W$, $c_\beta=\cos\beta$, and
$s_\beta=\sin\beta$.  Here $\delta_{33}$ and $\delta_{44}$ are radiative
corrections important when two higgsinos are close in mass. Their explicit
expressions are from ref.~\cite{drees97}. To neglect these radiative
corrections set \ft{neuloop}=0 instead of \ft{neuloop}=1 (default). The
neutralino mass eigenstates are written as
\begin{equation}
  \tilde{\chi}^0_i = 
  N_{i1} \tilde{B} + N_{i2} \tilde{W}^3 + 
  N_{i3} \tilde{H}^0_1 + N_{i4} \tilde{H}^0_2 .
\end{equation}
The phases of $N_{ij}$ are chosen so that the neutralino masses
$m_{\tilde{\chi}^0_i} \ge 0$.


The charginos are linear combinations of the charged gauge bosons ${\tilde
  W^\pm}$ and of the charged higgsinos ${\tilde H_1^-}$, ${\tilde H_2^+}$.
Their mass matrix,
\begin{eqnarray}
  {\cal M}_{\tilde{\chi}^\pm} = 
  \left( \matrix{
  {M_2} & \sqrt{2} m_W \sin\beta \cr 
  \sqrt{2} m_W \cos\beta & \mu 
  } \right) ,
\end{eqnarray}
is diagonalized by the following linear combinations
\begin{eqnarray}
  \tilde{\chi}^-_i & = & U_{i1} \tilde{W}^- + U_{i2} \tilde{H}_1^- , \\
  \tilde{\chi}^+_i & = & V_{i1} \tilde{W}^+ + V_{i2} \tilde{H}_1^+ .
\end{eqnarray}
We choose ${\rm det}(U)=1$ and $U^* {\cal M}_{\tilde{\chi}^\pm}
V^\dagger = {\rm diag} ( m_{\tilde{\chi}^\pm_1},
m_{\tilde{\chi}^\pm_2} )$ with non-negative chargino masses $
m_{\tilde{\chi}^\pm_i} \ge 0$.

When discussing the squark mass matrix including mixing, it is
convenient to choose a basis where the squarks are rotated in the same
way as the corresponding quarks in the standard model.  We follow the
conventions of the particle data group \cite{pdg99} and put the mixing
in the left-handed $d$-quark fields, so that the definition of the
Cabibbo-Kobayashi-Maskawa matrix is $\mbox{\bf K}= \mbox{\bf V}_1
\mbox{\bf V}_2^\dagger$, where $\mbox{\bf V}_1$ ($\mbox{\bf V}_2$)
rotates the interaction left-handed $u$-quark ($d$-quark) fields to
mass eigenstates.  For sleptons we choose an analogous basis, but due
to the masslessness of neutrinos no analog of the CKM matrix appears.
 
We then obtain the general $6\times6$ $\tilde{u}$- and
$\tilde{d}$-squark mass matrices:
\begin{equation}
  {\cal M}_{\tilde u}^2 = \left( \matrix{
      \mbox{\bf M}_Q^2 + \mbox{\bf m}_u^\dagger \mbox{\bf m}_u +
      D_{LL}^{u} \mbox{\bf 1} &
      \mbox{\bf m}_u^\dagger 
      ( {\bf A}_U^\dagger - \mu^* \cot\beta ) \cr
      ( {\bf A}_U - \mu \cot\beta ) \mbox{\bf m}_u &
      \mbox{\bf M}_U^2 + \mbox{\bf m}_u \mbox{\bf m}_u^\dagger +
      D_{RR}^{u} \mbox{\bf 1} \cr
      } \right),
  \label{mutilde}
\end{equation}
\begin{equation}
  {\cal M}_{\tilde d}^2=\left( {\matrix{
        {\mbox{\bf K}^\dagger \mbox{\bf M}_Q^2 \mbox{\bf K}+
          \mbox{\bf m}_d\mbox{\bf m}_d^\dagger+D_{LL}^{d}\mbox{\bf 1}}&
        {\mbox{\bf m}_d^\dagger ( {\bf A}_D^\dagger-\mu^*\tan\beta )}\cr
        {( {\bf A}_D-\mu\tan\beta ) \mbox{\bf m}_d}&
        {\mbox{\bf M}_D^2+\mbox{\bf m}_d^\dagger\mbox{\bf m}_d+
          D_{RR}^{d}\mbox{\bf 1}}\cr
        }} \right),
  \label{mdtilde}
\end{equation}
and the general sneutrino and charged slepton mass matrices
\begin{equation}
  {\cal M}^2_{\tilde\nu} = \mbox{\bf M}_L^2 + D^\nu_{LL} \mbox{\bf 1}
\end{equation}
\begin{equation}
  {\cal M}^2_{\tilde e} =\left( {\matrix{
        {\mbox{\bf M}_L^2+\mbox{\bf m}_e\mbox{\bf m}_e^\dagger+
          D_{LL}^{e}\mbox{\bf 1}}&
        {\mbox{\bf m}_e^\dagger ( {\bf A}_E^\dagger-\mu^*\tan\beta )}\cr
        {( {\bf A}_E-\mu\tan\beta ) \mbox{\bf m}_e}&
        {\mbox{\bf M}_E^2+\mbox{\bf m}_e^\dagger\mbox{\bf m}_e+
          D_{RR}^{e}\mbox{\bf 1}}\cr
        }} \right).
  \label{metilde}
\end{equation}
Here
\begin{equation}
  D^f_{LL}=m_Z^2\cos 2\beta(T_{3f}-e_f\sin^2\theta_w),
\end{equation}
\begin{equation}
  D^f_{RR}=m_Z^2\cos 2\beta e_f\sin^2\theta_w.
\end{equation}
In the chosen basis, $\mbox{\bf m}_u$ = $\mbox{\rm diag} ( m_{\rm
u}, m_{\rm c}, m_{\rm t} )$, $\mbox{\bf m}_d $ = $\mbox{\rm diag}
(m_{\rm d}, m_{\rm s}, m_{\rm b} )$ and $\mbox{\bf m}_e $ = $
\mbox{\rm diag} (m_e,$ $ m_\mu, m_\tau )$.

The slepton and squark mass eigenstates $\tilde{f}_k$ ($\tilde{\nu}_k$
with $k=1,2,3$ and $\tilde{e}_k$, $\tilde{u}_k$ and $\tilde{d}_k$ with
$k=1,\dots,6$) diagonalize the previous mass matrices and are related
to the current sfermion eigenstates ${\bf \tilde{f}}_{L}$ and ${\bf
\tilde{f}}_{R}$ via ($a=1,2,3$)
\begin{eqnarray}
  \tilde{f}_{La} & = & \sum_{k=1}^6 \tilde{f}_k {\bf \Gamma}_{FL}^{*ka} , \\
  \tilde{f}_{Ra} & = & \sum_{k=1}^6 \tilde{f}_k {\bf \Gamma}_{FR}^{*ka} .
\end{eqnarray} 
The squark and charged slepton mixing matrices ${\bf \Gamma}_{UL,R}$,
${\bf \Gamma}_{DL,R}$ and ${\bf \Gamma}_{EL,R}$ have dimension
$6\times 3$, while the sneutrino mixing matrix ${\bf \Gamma}_{\nu L}$
has dimension $3\times3$.

This version of \ds\ allows only for diagonal matrices ${\bf A}_U$,
${\bf A}_D$, ${\bf A}_E$, ${\bf M}_Q$, ${\bf M}_U$, ${\bf M}_D$, ${\bf M}_E$,
and ${\bf M}_L$. This ansatz, while not being the most general one, implies the
absence of tree-level flavor changing neutral currents in all sectors of the
model. In this case, the squark mass matrices can be diagonalized analytically.
For example, for the top squark one has, in terms of the top squark mixing
angle $\theta_{\tilde{t}}$,
\begin{equation}
  \Gamma_{UL}^{\tilde{t}_1\tilde{t}} =
  \Gamma_{UR}^{\tilde{t}_2\tilde{t}} = \cos \theta_{\tilde{t}} ,
  \qquad
  \Gamma_{UL}^{\tilde{t}_2\tilde{t}} =
  - \Gamma_{UR}^{\tilde{t}_1\tilde{t}} = \sin \theta_{\tilde{t}} .
\end{equation}

Special values of the sfermion masses can be set with the parameters {\tt
  msquarks}, and \ft{msleptons}.  If \ft{msquarks}=\ft{msleptons}=0, the
sfermion masses are obtained with the diagonalization described above. If {\tt
  msquarks}$>$0 (or \ft{msleptons}$>$0), all squark masses are set to {\tt
  msquarks} (or all slepton masses to \ft{msleptons}).  Finally, if {\tt
  msquarks}$<$0 (or \ft{msleptons}$<$0), the squark (or slepton) masses are
set equal to the neutralino mass but never less than $|\ft{msquarks}|$ (or
$|\ft{msleptons}|$).  This is to provide the lightest possible sfermions
compatible with a neutralino LSP. In all of these cases, there is no mixing
  between sfermions.


The particle masses are available in an array \ft{mass($p$)}, where $p$ is the
particle code from table \ref{tab:2}. Similarly, particle decay width are
available as \ft{width($p$)}, but currently only the width of the Higgs bosons
are calculated, the other particles having fictitious widths of 1 or 5 GeV (for
the sole purpose of regularizing annihilation amplitudes close to poles).

\begin{table*}
\label{tab:2}
\caption{Particle codes (synonyms are separated by commas). }
\begin{center}
\begin{tabular}{| l l | l l | l l | l l |}
\hline
$\nu_e$ & \ft{knue,knu(1)} &
$\gamma$ & \ft{kgamma} &
$\tilde{\chi}^0_i$ & \ft{kn($i$)~$i=1\ldots4$} &
$\tilde{u}_1$ & \ft{ksu(1),ksqu(1)}
\\
$e$ & \ft{ke,kl(1)} &
$W^\pm$ & \ft{kw} &
$\tilde{\chi}^\pm_k$ & \ft{kcha($k$)~$k=1,2$} &
$\tilde{u}_2$ & \ft{ksu(2),ksqu(4)}
\\
$\nu_\mu$ & \ft{knumu,knu(2)} &
$Z^0$ & \ft{kz} &
$\tilde{g}$ & \ft{kgluin} &
$\tilde{d}_1$ & \ft{ksd(1),ksqd(1)}
\\
$\mu$ & \ft{kmu,kl(2)} &
$g$ & \ft{kgluon} &
$\tilde{\nu}_e$ & \ft{ksnue,ksnu(1)} &
$\tilde{d}_2$ & \ft{ksd(2),ksqd(4)}
\\
$\nu_\tau$ & \ft{knutau,knu(3)} &
&&
$\tilde{e}_1$ & \ft{kse(1),ksl(1)} & 
$\tilde{c}_1$ &\ft{ksc(1),ksqu(2)}
\\
$\tau$ & \ft{ktau,kl(3)} &
&&
$\tilde{e}_2$ & \ft{kse(2),ksl(4)} & 
$\tilde{c}_2$ &\ft{ksc(2),ksqu(5)}
\\
$u$ & \ft{ku,kqu(1)} &
$H^0$ & \ft{kh1} &
$\tilde{\nu}_\mu$ &\ft{ksnumu,ksnu(2)} &
$\tilde{s}_1$ &\ft{kss(1),ksqd(2)}
\\
$d$ & \ft{kd,kqd(1)} & 
$h^0$ & \ft{kh2} & 
$\tilde{\mu}_1$ &\ft{ksmu(1),ksl(2)} &
$\tilde{s}_2$ &\ft{kss(2),ksqd(5)}
\\
$c$ &\ft{kc,kqu(2)} &
$A^0$ & \ft{kh3} &
$\tilde{\mu}_2$ &\ft{ksmu(2),ksl(5)} &
$\tilde{b}_1$ &\ft{ksb(1),ksqd(3)}
\\
$s$ &\ft{ks,kqd(2)} &
$H^\pm$ & \ft{khc} &
$\tilde{\nu}_\tau$ &\ft{ksnuta,ksnu(3)} &
$\tilde{b}_2$ &\ft{ksb(2),ksqd(6)}
\\
$b$ &\ft{kb,kqd(3)} &
$G^0$ & \ft{kgold0} &
$\tilde{\tau}_1$ & \ft{kstau(1),ksl(3)} &
$\tilde{t}_1$ & \ft{kst(1),ksqu(3)}
\\
$t$ &\ft{kt,kqu(3)} &
$G^\pm$ & \ft{kgoldc} &
$\tilde{\tau}_2$ & \ft{kstau(2),ksl(6)} &
$\tilde{t}_2$ & \ft{kst(2),ksqu(6)}
\\
\hline
\end{tabular}
\end{center}
\end{table*}

\subsection{Three-particle vertices}

We define three-particle vertices \ft{gl($i$,$j$,$k$)}$=g^L_{ijk}$ and
\ft{gr($i$,$j$,$k$)}$=g^R_{ijk}$ as follows. We adopt the convention
that the order of the particles in the indices is the order in which
they appear in the corresponding lagrangian term, so the last particle
is always entering. If there are charged particles in the vertex, they
are both assumed positively charged, and the particle that exits the
vertex is indexed before the particle that enters.
\begin{itemize}
\item Three scalar bosons:
\begin{equation}
{\cal L}_{\rm int} = g_{\phi_i\phi_j\phi_k} m_W \phi_i \phi_j \phi_k
\end{equation}
where $\phi_i$ is a Higgs or a Goldstone boson. In this case, \ft{gl}={\tt
  gr}=$g$. Available vertices are $\phi_i\phi_j\phi_k$ = $H^0_iH^0_jH^0_k$,
$H^0_iH^-H^+$, $H^0_iA^0A^0$, $H^0_iG^0G^0$, $H^0_iG^-G^+$, $H^0_iG^-H^+$,
$H^0_iG^-G^+$, $A^0G^-H^+$, $A^0G^0H^0_i$, and permutations.

\item Two scalar and one vector bosons:
\begin{equation}
{\cal L}_{\rm int} = g_{V\phi_1\phi_2} V^\mu \phi_1 i \lrpartial_\mu \phi_2 .
\end{equation}
Available vertices are $V\phi_1\phi_2$ = $Z^0H^0_iA^0$, $Z^0H^-H^+$, $\gamma
H^-H^+$, $W^-H^+A^0$, $W^-H^+H^0_i$, and permutations.

\item One scalar and two vector bosons:
\begin{equation}
{\cal L}_{\rm int} = g_{\phi V_1 V_2} m_W g_{\mu\nu} \phi V^\mu_1 V^\nu_2
\end{equation}
Available vertices are $\phi V_1V_2$ = $H^0_iW^-W^+$, $H^0_iZ^0Z^0$.

\item Three vector bosons:
%\begin{equation}
%{\cal L}_{\rm int} = - g_{V_1V_2V_3} \epsilon_{abc} V_1^{\mu a} V_2^{\nu b} 
%\partial_\mu V_{3\nu}^c \hbox{\bf to be checked!}
%\end{equation}
%which corresponds to the vertex
\begin{equation}
i g_{V_1V_2V_3} \left[ (k_1-k_3)_\nu g_{\mu\lambda} + (k_3-k_2)_\mu
  g_{\lambda\nu} + (k_2-k_1) g_{\mu\nu} \right]
\end{equation}
with all momenta incoming and assigned as $V_1^\mu(k_1)$, $V_2^\nu(k_2)$ and
$V_3^\lambda(k_3)$. 
Available vertices are $Z^0W^-W^+$ and $\gamma W^-W^+$.

\item One scalar boson and two Dirac fermions:
\begin{equation}
{\cal L}_{\rm int} = \phi \overline{\psi}_1 ( g^L_{\phi\psi_1\psi_2} P_L +
g^R_{\phi\psi_1\psi_2} P_R ) \psi_2
\end{equation}
Available vertices are $\phi\psi_1\psi_2$ = 

\item One vector boson and two Dirac fermions:
\begin{equation}
{\cal L}_{\rm int} = V_\mu \overline{\psi}_1 \gamma^\mu
( g^L_{V\psi_1\psi_2} P_L + g^R_{V\psi_1\psi_2} P_R ) \psi_2
\end{equation}
Available vertices are $V\psi_1\psi_2$ = 

\item One scalar boson, one Dirac and one Majorana fermion:
\begin{equation}
  {\cal L}_{\rm int} = \phi \overline{\psi} 
  ( g^L_{\phi\psi\chi} P_L + g^R_{\phi\psi\chi} P_R ) \chi
\end{equation}
Available vertices are $ \phi\psi\chi$ = 

\item One vector boson, one Dirac and one Majorana fermion:
\begin{equation}
  {\cal L}_{\rm int} = V_\mu \overline{\psi} \gamma^\mu 
  ( g^L_{V\psi\chi} P_L + g^R_{V\psi\chi} P_R ) \chi
\end{equation}
Available vertices are $V\psi\chi$ = 

\item One scalar boson and two Majorana fermions:
\begin{equation}
{\cal L}_{\rm int} = 
\end{equation}
Available vertices are\ldots

\item One vector boson and two Majorana fermions:
\begin{equation}
{\cal L}_{\rm int} = 
\end{equation}
\end{itemize}

%\begin{eqnarray}
%{\cal L}_{\rm int} & = & 
%V_\mu \bar{\chi} \gamma^\mu (g^L_{V\chi\psi} P_L + g^R_{V\chi\psi} P_R) \psi +
%\phi^* \bar{\chi} (g^L_{\phi\chi\psi} P_L + g^R_{\phi\chi\psi} P_R) \psi 
%\nonumber \\
%&& \, + g^A_{Z\psi\psi'} Z_\mu \bar{\psi} \gamma^\mu \gamma_5 \psi'
%+ g^P_{\phi\psi\psi'} \phi^* \bar{\psi} \gamma_5 \psi' 
%+ {\rm h.c.} 
%\nonumber\\
%&&\, + {\textstyle{1\over2}}
% g^A_{Z\chi\chi'} Z_\mu \bar{\chi}  \gamma^\mu \gamma_5 \chi' +
%{\textstyle{1\over2}} 
%g^P_{\phi\chi\chi'} \phi^* \bar{\chi} \gamma^\mu \gamma_5 \chi 
%+ {\rm h.c.} 
%\end{eqnarray}
%Here $V_\mu$ is a vector boson, $\chi,\chi'$ are Majorana fermions,
%$\psi,\psi'$ are Dirac fermions, and $\phi$ can be a neutral ($\phi^*=\phi$) or
%a charged scalar.

Explicit expressions for the coupling constants $g_{ijk}$ can be obtained
in~\cite{mssm}, with radiative corrections to trilinear scalar couplings in
\cite{haber97}. We have rederived from the superpotential all vertices we have
implemented.

Implemented vertices:  those listed above plus 
$Z^0W^\pm W^\mp $, $Z^0H^0_iH^0_i$, $W^\pm H^\mp A^0$,
$W^\pm H^\mp H^0_i$, $H^0_iW^\pm W^\mp $, $H^0_iZ^0Z^0$, $Z^0A^0H$,
$H^0_iA^0A^0$, $A^0ff$, $H^0_iff$, $Z^0ff$, $Z^0\tilde{\chi}^0\tilde{\chi}^0$,
$H^0_i\tilde{\chi}^0\tilde{\chi}^0$, $Z^0\tilde{\chi}^0\tilde{\chi}^0$,
$W^\mp\tilde{\chi}^0\tilde{\chi}^\pm$, $H^\mp\tilde{\chi}^0\tilde{\chi}^\pm$, $
\tilde{q}\tilde{g}q$, $\tilde{f}\tilde{\chi}^0 f$, $H^0_i\tilde{\chi}^\pm
\tilde{\chi}^\mp$, $A^0\tilde{\chi}^\pm \tilde{\chi}^\mp$ , $W^\pm f f'$,
$H^\pm f f'$, $\gamma W^\pm W^\mp$, $ \gamma H^\pm H^\mp$, $Z^0
\tilde{\chi}^\pm \tilde{\chi}^\mp$, $\gamma\tilde{\chi}^\pm \tilde{\chi}^\mp$,
$\gamma f f$, $GHH$, $GGH$, $G^\mp\tilde{\chi^0}\tilde{\chi}^\pm$.

In appendix \ref{app:feyn}, most of the Feynman rules and the explicit
expressions for the $g$'s are found.

\subsection{Accelerator bounds}

Accelerator bounds can be checked by a call to \ft{dsacbnd(p)}, where
\ft{p}=0 checks all implemented bounds, \ft{p}=1 leaves out the
bound from $b\to s\gamma$, and \ft{p}=2 checks only $b\to
s\gamma$. The accelerator bounds implemented in version 3.10.2 (July
1999) are listed in table \ref{tab:1}. The branching ratio ${\rm
BR}(b\to s\gamma)$ is calculated to 1-loop using the expressions in
ref.~\cite{bertolini91}, including or not including 1-loop QCD
corrections according to the switch \ft{bsgqcd} (=0 without, =1 with
[default]).

\begin{table*}[h]
\caption{Accelerator bounds implemented in version 3.10.2 (July 1999)
\comment{Update? Include? Throw away?}}
\label{tab:1}
\begin{center}
\begin{tabular}{|l l|}
\hline \hfil Bound \hfill & \hfil Ref. \hfill \\ \hline $
m_{H^\pm} > 59.5{\rm GeV} 
$ & \cite{abbiendi99} \\ $
m_{h} > [ 82.5 + 10.5 \sin^2(\beta-\alpha) ] {\rm GeV}
$ & \cite{gao99} \\ \hline $
m_{\tilde{\chi}^+_2} > 91 {\rm GeV}
  \;\hbox{if}\; m_{\tilde{\chi}^0_1}-m_{\tilde{\chi}^+_2} > 4 {\rm GeV} 
$ & \cite{carr98} \\ $
m_{\tilde{\chi}^+_2} > 64 {\rm GeV} 
    $ \hbox{if} $  m_{\tilde{\chi}^0_1} > 43 {\rm GeV}
    $ \hbox{and} $  m_{\tilde{\chi}^+_2} > m_{\tilde{\chi}^0_2}
$ & \cite{acciarri96} \\ $ 
m_{\tilde{\chi}^+_2} > 47 {\rm GeV} 
    $ \hbox{if} $  m_{\tilde{\chi}^0_1} > 41 {\rm GeV} 
$ & \cite{decamp92} \\ $ 
m_{\tilde{\chi}^+_1} > 99 {\rm GeV}
$ & \cite{hidaka91} \\ \hline $
m_{\tilde{\chi}^0_1} > 23{\rm GeV}  $ \hbox{if} $  \tan\beta>3
$ & \cite{acciarri95} \\ $
m_{\tilde{\chi}^0_1} > 20{\rm GeV}  $ \hbox{if} $  \tan\beta>2
$ & \cite{acciarri95} \\ $
m_{\tilde{\chi}^0_1} > 12.8{\rm GeV}  $ \hbox{if} $  m_{\tilde{\nu}}<200
  {\rm GeV} 
$ & \cite{buskulic96} \\ $
m_{\tilde{\chi}^0_1} > 10.9{\rm GeV} 
$ & \cite{acciarri98} \\ $
m_{\tilde{\chi}^0_2} > 44{\rm GeV} 
$ & \cite{abbiendi99b} \\ $
m_{\tilde{\chi}^0_3} > 102{\rm GeV} 
$ & \cite{abbiendi99b} \\ $
m_{\tilde{\chi}^0_4} > 127{\rm GeV} 
$ & \cite{acciarri95} \\ \hline $ 
m_{\tilde{g}} > 212{\rm GeV}  $ \hbox{if} $  m_{\tilde{q}_k} <
               m_{\tilde{g}}
$ & \cite{abachi95} \\ $
m_{\tilde{g}} > 162{\rm GeV}
$ & \cite{abe97} \\ \hline $
m_{\tilde{q}_k} > 90 {\rm GeV}  $ \hbox{if} $  m_{\tilde{g}} <410{\rm GeV}
$ & \cite{abe92} \\ $ 
m_{\tilde{q}_k} > 176 {\rm GeV}  $ \hbox{if} $  m_{\tilde{g}} <300{\rm GeV}
$ & \cite{abachi95} \\ $
m_{\tilde{q}_k} > 224 {\rm GeV}  $ \hbox{if} $  m_{\tilde{g}}>m_{\tilde{g}}
$ & \cite{abe96} \\ $
m_{\tilde{e}} > 78{\rm GeV}  $ \hbox{if} $  m_{\tilde{\chi}^0_1}<73{\rm
  GeV} 
$ & \cite{barate98} \\ $
m_{\tilde{\mu}} > 71{\rm GeV}  $ \hbox{if} $  m_{\tilde{\chi}^0_1}<66{\rm
  GeV} 
$ & \cite{barate98} \\ $
m_{\tilde{\tau}} > 65{\rm GeV}  $ \hbox{if} $  m_{\tilde{\chi}^0_1}<55{\rm
  GeV} 
$ & \cite{barate98} \\ $
m_{\tilde{\nu}} > 44.4{\rm GeV}
$ & \cite{pdg99} \\ \hline $
1 \times 10^{-4} < {\rm BR}(b\to s\gamma) < 4 \times 10^{-4}
$ & \cite{pdg99} \\ $
\Gamma_Z^{\rm inv} < 502.4 {\rm MeV}
$ & \cite{pdg99} \\ \hline
\end{tabular}
\end{center}
\end{table*}

%%%%%%%%%%%%%%%%%%%%%%%%%%%%%%%%%%%%%%%%%%%%%%%%%%%%%%%%%%%%
\section{General supersymmetry -- routines}

Input parameters, options, results, etc.\ are contained in common blocks in the
file \ft{dssusy.h}, which the user has to include. The input parameters
are ($a=1,2,3$) \newline\begin{tabular}{llll} \ft{ma}$=m_A$, & \ft{tanbe}$=\tan\beta$, & \ft{mu}$=\mu$, &
  \ft{m1}$=M_1$, \\
  \ft{m2}$=M_2$, & \ft{m3}$=M_3$, & \ft{asofte($a$)}$=A_{Eaa}$, &
  {asoftu($a$)}$=A_{Uaa}$, \\
  \ft{asoftd($a$)}$=A_{Daa}$, & \ft{mass2q($a$)}$=M^2_{Qaa}$, & {\tt
    mass2l($a$)}$=M^2_{Laa}$, &
  \ft{mass2u($a$)}$=M^2_{Uaa}$, \\
  \ft{mass2d($a$)}$=M^2_{Daa}$, &
  \ft{mass2e($a$)}$=M^2_{Eaa}$. & & \\
\end{tabular}\newline
The options are (see previous subsections for a description)
\newline
\ft{higloop} choice of tree-level or radiatively corrected Higgs boson
  masses; 
\newline 
\ft{neuloop} choice of tree-level or radiatively corrected neutralino
  masses; 
\newline
\ft{msquarks,msleptons} choice of squark and slepton masses.

To initialize \ds\ for a new model, you should call
\begin{sub}{subroutine \textbf{dssusy}(unphys,hwarning)}
  \itit{Purpose:} To calculate the particle spectrum, widths and
    couplings.
  \itit{Output:} 
  \itv{unphys}{i} non-zero if the model is unphysical
  \itv{hwarning}{i} non-zero if the Higgs code has issued a warning.
\end{sub}
which calculates couplings, masses and some basic cross sections.

The following subroutines specify the values of the model parameters,
and read/write them to a file. The user should create his own versions
by editing a copy of them. Please call them with a different name.

\begin{sub}{subroutine
  \textbf{dsgive\_model}(mu,m2,ma,tanbe,msq,atm,abm)}
  \itit{Purpose:} Set the MSSM parameters as specified by the
  arguments.
  \itit{Inputs:}
  \itv{mu}{r8} The $\mu$ parameter in GeV.
  \itv{m2}{r8} The $M_2$ parameter in GeV.
  \itv{ma}{r8} The mass of the CP-odd Higgs boson, $m_A$ in
  GeV.
  \itv{tanbe}{r8} $\tan \beta$.
  \itv{msq}{r8} Sets ${\bf M}^Q$, etc.\ to a common mass
  scale $m_0$ in GeV.
  \itv{atm}{r8} Sets $A_t$ in units of $m_0$ (range: -3 --- 3).
  \itv{abm}{r8} Sets $A_b$ in units of $m_0$ (range: -3 --- 3).
\end{sub}

\begin{sub}{subroutine \textbf{dsrndm\_model}(mftyp)}
  \itit{Purpose:} Sets the susy parameters in a random way.
    Parameter ranges and probability distributions are set inside.
  \itit{Inputs:}
  \itv{mftyp}{i}
     =1: $M_1$ is related to $M_2$ through GUT relations.\\
     =2: $M_1$ and $M_2$ are generated independently.
\end{sub}

\begin{sub}{function \textbf{rnduni}(iseed,a,b) \hfill r8}
  \itit{Purpose:} To give a random number uniformly
  distributed between \ft{a} and \ft{b}.
  \itit{Inputs:}
  \itv{iseed}{i} Seed for the random number generator. Must be a
  negative number at the first call and should not be changed from
  call to call.
  \itv{a}{r8} Lower limit of returned number.
  \itv{b}{r8} Upper limit of returned number.
\end{sub}

\begin{sub}{function \textbf{rndlog}(iseed,a,b)\hfill r8}
  \itit{Purpose:} To give a random number logarithmically
  distributed between \ft{a} and \ft{b}.
  \itit{Inputs:}
  \itv{iseed}{i} Seed for the random number generator. Must be a
  negative number at the first call and should not be changed from
  call to call.
  \itv{a}{r8} Lower limit of returned number.
  \itv{b}{r8} Upper limit of returned number.
\end{sub}

\begin{sub}{function \textbf{rndsgn}(iseed)\hfill r8}
  \itit{Purpose:} Returns $\pm1$ with equal probability.
  \itit{Inputs:}
  \itv{iseed}{i} Seed for the random number generator. Must be a
  negative number at the first call and should not be changed from
  call to call.
\end{sub}

\begin{sub}{subroutine \ftb{write\_model}(lunit,mftyp)}
  \itit{Purpose:} Writes out the model parameters
  to the file opened as unit \ft{lunit} (formatted). 
  \itit{Inputs:}
  \itv{lunit}{i} Unit number to write output to.
  \itv{mftyp}{i}
     =1: Only $M_2$ is written since $M_1$ is related to $M_2$ through
         GUT relations.\\ 
     =2: Both $M_1$ and $M_2$ are written.
\end{sub}

\begin{sub}{subroutine \ftb{read\_model}(lunit,nmodel,mftyp)}
  \itit{Purpose:} Reads in the model parameters from the file opened
  as unit \ft{unit} (formatted).  
  \itit{Inputs:}
  \itv{lunit}{i} Unit number to read from.
  \itv{nmodel}{i}
     =0: The next model is read.\\
     =n: Only the n:th model is read.
  \itv{mftyp}{i}
     =1: Only $M_2$ is read since $M_1$ is related to $M_2$ through
         GUT relations.\\ 
     =2: Both $M_1$ and $M_2$ are read.
\end{sub}

The following subroutines are useful in the analysis.

\begin{sub}{subroutine \ftb{widtag}(unit)}
  \itit{Purpose:} Write the model identification tag to unit \ft{unit}.
  \itit{Inputs:}
  \itv{unit}{i} Unit number to write to.
\end{sub}

\begin{sub}{subroutine \ftb{wspctm}(unit)}
  \itit{Purpose:} Write the particle mass spectrum and mixing matrices
  to unit \ft{unit}.
  \itit{Inputs:}
  \itv{unit}{i} Unit number to write to.
\end{sub}

\begin{sub}{subroutine \ftb{wvertx}(unit)}
  \itit{Purpose:} Write all non-vanishing three-particle vertices to
  unit \ft{unit}.
  \itit{Inputs:}
  \itv{unit}{i} Unit number to write to.
\end{sub}

\begin{sub}{subroutine \ftb{wunph}(unit)}
  \itit{Purpose:} Write the reason for which the model is not
  physically acceptable (tachyons, etc.) to unit \ft{unit}.
  \itit{Inputs:}
  \itv{unit}{i} Unit number to write to.
\end{sub}

\begin{sub}{subroutine \ftb{wexcl}(unit)}
  \itit{Purpose:} Write the reason(s) for which the model is
  experimentally excluded  to unit \t{unit}.
  \itit{Inputs:}
  \itv{unit}{i} Unit number to write to.
\end{sub}  

\begin{sub}{subroutine \ftb{dswhwar}(unit)}
  \itit{Purpose:} Write the reason(s) for which the Higgs
  calculation issued warnings to unit \ft{unit}.
  \itit{Inputs:}
  \itv{unit}{i} Unit number to write to.
\end{sub}  


\section{Routine headers -- fortran files}

%%%%% routine dsb0loop.f %%%%%
\begin{routine}{dsb0loop.f}
\begin{verbatim}
      complex*16 function dsb0loop(q,m1,m2)
c_______________________________________________________________________
c  the two-point function b0(q,m1,m1).
c     uses two-point function b_0 from m. drees, k. hagiwara and a.
c     yamada, phys. rev. d45 (1992) 1725.
c  author: joakim edsjo (edsjo@physto.se) 97-02-11
c=======================================================================
\end{verbatim}
 \end{routine}

%%%%% routine dschasct.f %%%%%
\begin{routine}{dschasct.f}
\begin{verbatim}
      subroutine dschasct
c_______________________________________________________________________
c  chargino masses and mixings.
c  common:
c    'dssusy.h' - file with susy common blocks
c  called by susyin or mrkin.
c  author: paolo gondolo (gondolo@lpthe.jussieu.fr) 1994,1995
c     940407 correction to chargino mixing
c     990724 drop v1,v2 (pg)
c=======================================================================
\end{verbatim}
 \end{routine}

%%%%% routine dsfeynhiggsfast.f %%%%%
\begin{routine}{dsfeynhiggsfast.f}
\begin{verbatim}
       subroutine dsfeynhiggsfast(hwar,HM,mh,hc,halpha,drho,
     &  ATop,ABot,inmt,inmb,my,M2,mqtl,mqtr,mqbl,mqbr,
     &  mgluino,mh3,tanb)
c
c -----------------------------------------------------------------       
c Implementation of FeynHiggsFast in DarkSUSY by J. Edsjo 2000-04-25
c
c All names of subroutines and functions that clashed with the
c full FeynHiggs names have _fast appended to them. In most cases,
c they are probably the same routines though, but to be on the
c safe side, the names were changed.
c Output: mh is lighter scalar Higgs mass, HM is heavier Higgs mass 
c -----------------------------------------------------------------
c
c     FeynHiggsFast
c     =============
c      
c       Calculation of the masses of the neutral CP-even
c       Higgs bosons in the MSSM
c       
c       Author: Sven Heinemeyer
c
c       Based on hep-ph/9903404
c       by S. Heinemeyer, W. Hollik, G. Weiglein
c      
c       In case of problems or questions,
c       contact Sven Heinemeyer :-)
c       email: Sven.Heinemeyer@desy.de
c       
c       FeynHiggs homepage:
c       http://www-itp.physik.uni-karlsruhe.de/feynhiggs/
c
c --------------------------------------------------------------
c
c Warnings implemented by J. Edsjo, 2000-04-25
c
c                   Bit  Value
c  Bits of hwar:    0 -  1:   Potential numerical problems at 1-loop
c                   1 -  2:   Potential numerical problems at 2-loop
c                   2 -  4:   Error with not used H2 mass expression 1-loop
c                   3 -  8:   Error with not used H2 mass expression 2-loop
c                   4 -  16:  Error with not used H2 mass expression 2-loop
c                   5 -  32:  1-loop Higgs sector not OK
c                   6 -  64:  2-loop Higgs sector not OK
c                   7 -  128: Stop or sbottom masses not OK
c---------------------------------------------------------------

\end{verbatim}
 \end{routine}

%%%%% routine dsfindmtmt.f %%%%%
\begin{routine}{dsfindmtmt.f}
\begin{verbatim}
      subroutine dsfindmtmt
No header found.
\end{verbatim}
 \end{routine}

%%%%% routine dsg0loop.f %%%%%
\begin{routine}{dsg0loop.f}
\begin{verbatim}
      function dsg0loop(qsq,m1sq,m2sq)
c     a loop function
\end{verbatim}
 \end{routine}

%%%%% routine dsg4set.f %%%%%
\begin{routine}{dsg4set.f}
\begin{verbatim}
      subroutine dsg4set(kp1,kp2,kp3,kp4,rvrtx,ivrtx)
c_______________________________________________________________________
c  auxiliary subroutine to dsvertx for quartic couplings 
c  set the value of the 4-particle vertex to vrtx=rcrtx+I*ivrtx
c  author: paolo gondolo (pxg26@po.cwru.edu) 2001
c=======================================================================
\end{verbatim}
 \end{routine}

%%%%% routine dsg4set12.f %%%%%
\begin{routine}{dsg4set12.f}
\begin{verbatim}
      subroutine dsg4set12(kp1,kp2,kp3,kp4,rvrtx,ivrtx)
c_______________________________________________________________________
c  auxiliary subroutine to dsvertx for quartic couplings 
c  set the value of the 4-particle vertex to vrtx=rcrtx+I*ivrtx
c  case of two neutral particles (1 and 2)
c  author: paolo gondolo (pxg26@po.cwru.edu) 2001
c=======================================================================
\end{verbatim}
 \end{routine}

%%%%% routine dsg4set1234.f %%%%%
\begin{routine}{dsg4set1234.f}
\begin{verbatim}
      subroutine dsg4set1234(kp1,kp2,kp3,kp4,rvrtx,ivrtx)
c_______________________________________________________________________
c  auxiliary subroutine to dsvertx for quartic couplings 
c  set the value of the 4-particle vertex to vrtx=rcrtx+I*ivrtx
c  case of four neutral particles
c  author: paolo gondolo (pxg26@po.cwru.edu) 2001
c=======================================================================
\end{verbatim}
 \end{routine}

%%%%% routine dsg4set13.f %%%%%
\begin{routine}{dsg4set13.f}
\begin{verbatim}
      subroutine dsg4set13(kp1,kp2,kp3,kp4,rvrtx,ivrtx)
c_______________________________________________________________________
c  auxiliary subroutine to dsvertx for quartic couplings 
c  set the value of the 4-particle vertex to vrtx=rcrtx+I*ivrtx
c  case of two neutral particles (1 and 3)
c  author: paolo gondolo (pxg26@po.cwru.edu) 2001
c=======================================================================
\end{verbatim}
 \end{routine}

%%%%% routine dsg4set23.f %%%%%
\begin{routine}{dsg4set23.f}
\begin{verbatim}
      subroutine dsg4set23(kp1,kp2,kp3,kp4,rvrtx,ivrtx)
c_______________________________________________________________________
c  auxiliary subroutine to dsvertx for quartic couplings 
c  set the value of the 4-particle vertex to vrtx=rcrtx+I*ivrtx
c  case of two neutral particles (2 and 3)
c  author: paolo gondolo (pxg26@po.cwru.edu) 2001
c=======================================================================
\end{verbatim}
 \end{routine}

%%%%% routine dsg4set34.f %%%%%
\begin{routine}{dsg4set34.f}
\begin{verbatim}
      subroutine dsg4set34(kp1,kp2,kp3,kp4,rvrtx,ivrtx)
c_______________________________________________________________________
c  auxiliary subroutine to dsvertx for quartic couplings 
c  set the value of the 4-particle vertex to vrtx=rcrtx+I*ivrtx
c  case of two neutral particles (3 and 4)
c  author: paolo gondolo (pxg26@po.cwru.edu) 2001
c=======================================================================
\end{verbatim}
 \end{routine}

%%%%% routine dsg4setc.f %%%%%
\begin{routine}{dsg4setc.f}
\begin{verbatim}
      subroutine dsg4setc(kp1,kp2,kp3,kp4,vrtx)
c_______________________________________________________________________
c  auxiliary subroutine to dsvertx for quartic couplings 
c  set the value of the 4-particle vertex to vrtx
c  author: paolo gondolo (pxg26@po.cwru.edu) 2001
c=======================================================================
\end{verbatim}
 \end{routine}

%%%%% routine dsg4setc12.f %%%%%
\begin{routine}{dsg4setc12.f}
\begin{verbatim}
      subroutine dsg4setc12(kp1,kp2,kp3,kp4,vrtx)
c_______________________________________________________________________
c  auxiliary subroutine to dsvertx for quartic couplings 
c  set the value of the 4-particle vertex to vrtx
c  case of two neutral particles (1 and 2)
c  author: paolo gondolo (pxg26@po.cwru.edu) 2001
c=======================================================================
\end{verbatim}
 \end{routine}

%%%%% routine dsg4setc1234.f %%%%%
\begin{routine}{dsg4setc1234.f}
\begin{verbatim}
      subroutine dsg4setc1234(kp1,kp2,kp3,kp4,vrtx)
c_______________________________________________________________________
c  auxiliary subroutine to dsvertx for quartic couplings 
c  set the value of the 4-particle vertex to vrtx
c  case of four neutral particles
c  author: paolo gondolo (pxg26@po.cwru.edu) 2001
c=======================================================================
\end{verbatim}
 \end{routine}

%%%%% routine dsg4setc13.f %%%%%
\begin{routine}{dsg4setc13.f}
\begin{verbatim}
      subroutine dsg4setc13(kp1,kp2,kp3,kp4,vrtx)
c_______________________________________________________________________
c  auxiliary subroutine to dsvertx for quartic couplings 
c  set the value of the 4-particle vertex to vrtx
c  case of two neutral particles (1 and 3)
c  author: paolo gondolo (pxg26@po.cwru.edu) 2001
c=======================================================================
\end{verbatim}
 \end{routine}

%%%%% routine dsg4setc23.f %%%%%
\begin{routine}{dsg4setc23.f}
\begin{verbatim}
      subroutine dsg4setc23(kp1,kp2,kp3,kp4,vrtx)
c_______________________________________________________________________
c  auxiliary subroutine to dsvertx for quartic couplings 
c  set the value of the 4-particle vertex to vrtx
c  case of two neutral particles (2 and 3)
c  author: paolo gondolo (pxg26@po.cwru.edu) 2001
c=======================================================================
\end{verbatim}
 \end{routine}

%%%%% routine dsg4setc34.f %%%%%
\begin{routine}{dsg4setc34.f}
\begin{verbatim}
      subroutine dsg4setc34(kp1,kp2,kp3,kp4,vrtx)
c_______________________________________________________________________
c  auxiliary subroutine to dsvertx for quartic couplings 
c  set the value of the 4-particle vertex to vrtx
c  case of two neutral particles (3 and 4)
c  author: paolo gondolo (pxg26@po.cwru.edu) 2001
c=======================================================================
\end{verbatim}
 \end{routine}

%%%%% routine dsgive_model.f %%%%%
\begin{routine}{dsgive\_model.f}
\begin{verbatim}
      subroutine dsgive_model(amu,am2,ama,atanbe,amsq,atm,abm)
c----------------------------------------------------------------------
c
c     To specify the supersymmetric parameters of a model.
c     Inputs:
c        amu - mu parameter (GeV)
c        am2 - M2 parameter (GeV)
c        ama - Mass of the CP-odd Higgs boson A (or H3)
c        atanbe - ratio of Higgs vacuum expecation values, tan(beta)
c        amsq - common sfermion mass scale, M_sq_tilde (GeV)
c        atm - trilinear term in units of amsq, top sector 
c        atb - trilinear term in units of amsq, bottom sector
c     Outputs:
c        The common blocks are set corresponding to the values above
c     Author: Paolo Gondolo, gondolo@mppmu.mpg.de
c     Date: 2000
c     Modified: Joakim Edsjo, edsjo@physto.se
c        2001-02-13 - setting of idtag taken away
c----------------------------------------------------------------------

\end{verbatim}
 \end{routine}

%%%%% routine dshgfu.f %%%%%
\begin{routine}{dshgfu.f}
\begin{verbatim}
      subroutine dshgfu(ma,tanb,mq,mur,md,mtop,at,ab,mu,vh,
     &     stop1,stop2,v,mz,alpha1,alpha2,alpha3z)
c
c carena, quiros, wagner gfun -- adapted to darksusy by gondolo
c
\end{verbatim}
 \end{routine}

%%%%% routine dshigferqcd.f %%%%%
\begin{routine}{dshigferqcd.f}
\begin{verbatim}
      subroutine dshigferqcd
c_______________________________________________________________________
c  THIS ROUTINE IS OBSOLETE. USE DSHIGWID INSTEAD.
c  qcd corrections to the widths of the decays H^0_i --> c cbar, b bbar,
c  t tbar and to the corresponding vertices
c
c    'dssusy.h' - file with susy common blocks
c
c  typed in from formulas in Djouadi, Spira and Zerwas, astro-ph/9511344
c  and Spira, astro-ph/9705337
c
c  author: piero ullio, ullio@sissa.it, 02-09-13
c=======================================================================
\end{verbatim}
 \end{routine}

%%%%% routine dshigsct.f %%%%%
\begin{routine}{dshigsct.f}
\begin{verbatim}
      subroutine dshigsct(unphys,hwarning)
c_______________________________________________________________________
c  higgs bosons masses and mixings
c  common:
c    'dssusy.h' - file with susy common blocks
c  called by dssusy.
c  needs dssfesct.
c  higloop =  0  tree-level
c             1  brignole-ellis-ridolfi-zwirner eff. pot.
c             2  drees-nojiri eff. pot.
c             3  carena-espinosa-quiros-wagner rg-impr. eff. pot.
c                    (uses subh.f.)
c             4  carena-quiros-wagner impr. eff. pot.
c                    (uses subhpole2.f)
c             5  use FeynHiggs by Heinemeyer, Hollik and Weiglein
c                requires full FeynHiggs to be installed (see below)
c             6  use FeynHiggsFast by Heinemeyer, Hollik and Weiglein
c  author: paolo gondolo (gondolo@lpthe.jussieu.fr) 1994,1995
c  modified by: joakim edsjo, edsjo@physto.se, 2000-09-01
c  modified by: paolo gondolo, 1999-2000
c=======================================================================
\end{verbatim}
 \end{routine}

%%%%% routine dshigwid.f %%%%%
\begin{routine}{dshigwid.f}
\begin{verbatim}
      subroutine dshigwid
c_______________________________________________________________________
c  common:
c    'dssusy.h' - file with susy common blocks
c  needs chasct, neusct, sfesct, higsct, vertx.
c  merging of dshwidths and dshigferqcd
c  author: Piero Ullio (ullio@sissa.it) 020917
c          partly based on dshwidth by P. Gondolo and J. Edsjo
c       formulas from higgs hunters guide, 
c       Djouadi, Spira and Zerwas, hep-ph/9511344
c       and Spira, hep-ph/9705337
c=======================================================================
\end{verbatim}
 \end{routine}

%%%%% routine dshlf2.f %%%%%
\begin{routine}{dshlf2.f}
\begin{verbatim}
      function dshlf2(x,y)
c paolo gondolo
\end{verbatim}
 \end{routine}

%%%%% routine dshlf3.f %%%%%
\begin{routine}{dshlf3.f}
\begin{verbatim}
      function dshlf3(p2,y1,y2)
c paolo gondolo
\end{verbatim}
 \end{routine}

%%%%% routine dsmodelsetup.f %%%%%
\begin{routine}{dsmodelsetup.f}
\begin{verbatim}
      subroutine dsmodelsetup(unphys,hwarning)
c_______________________________________________________________________
c  set up global variables for the supersymmetric model routines.
c  If rate routines are going to be called afterwards, dsprep should
c  be called after this routine, like it is done in dssusy.
c  You should only call this routine directly yourself if you know
c  what you are doing. If you are the least unsure, call dssusy instead.
c  common:
c    'dssusy.h' - file with susy common blocks
c  uses sconst, sfesct, chasct, higsct,
c    neusct, vertx, hwidths.
c  author: paolo gondolo (gondolo@lpthe.jussieu.fr) 1994,1995
c=======================================================================
\end{verbatim}
 \end{routine}

%%%%% routine dsmqpole1loop.f %%%%%
\begin{routine}{dsmqpole1loop.f}
\begin{verbatim}
      real*8 function dsmqpole1loop(mqmq)
No header found.
\end{verbatim}
 \end{routine}

%%%%% routine dsneusct.f %%%%%
\begin{routine}{dsneusct.f}
\begin{verbatim}
      subroutine dsneusct
c_______________________________________________________________________
c  neutralino masses and mixings. base is b-ino, w3-ino, h1-ino, h2-ino.
c  common:
c    'dssusy.h' - file with susy common blocks
c  uses quartic.
c  called by susyin or mrkin.
c  author: paolo gondolo (gondolo@lpthe.jussieu.fr) 1994,1995
c  history:
c     940528 readability improvement (pg)
c     950316 order by increasing mass (pg)
c     951110 positive mass convention (pg)
c     970211 loop corrections via switch neuloop (joakim edsjo)
c     990724 drop v1,v2 and change mass scale to max(mz,m1,m2,mu) (pg)
c=======================================================================
\end{verbatim}
 \end{routine}

%%%%% routine dspole.f %%%%%
\begin{routine}{dspole.f}
\begin{verbatim}
      subroutine dspole(mchi,ma,tanb,mq,mur,mdr,
     &     mtop,at,ab,mu,mh,mhp,hm,hmp,amp,sa,ca,
     &     v,mz,alpha1,alpha2,alpha3z,sint,lambda,prtlevel,ierr)
c
c carena, quiros, wagner -- adapted to darksusy by gondolo
c
\end{verbatim}
 \end{routine}

%%%%% routine dsprep.f %%%%%
\begin{routine}{dsprep.f}
\begin{verbatim}
*****************************************************************************
***   subroutine dsprep calculates different frequently used cross
***   sections, annihilation branching ratios and sets different common
***   block variable when a new model has been defined. this routine
***   hasto be called before the relic density or any rates are
***   calculated and is called from dssusy.
***   
*** author: joakim edsjo  edsjo@physics.berkeley.edu
*** date: 98-02-27
*** modified: 99-02-23 pg, 00-08-09 je
*****************************************************************************

      subroutine dsprep
\end{verbatim}
 \end{routine}

%%%%% routine dsqindx.f %%%%%
\begin{routine}{dsqindx.f}
\begin{verbatim}
      function dsqindx(kp1,kp2,kp3,kp4)
c_______________________________________________________________________
c  auxiliary function to dsvertx for quartic couplings 
c  author: paolo gondolo (pxg26@po.cwru.edu) 2001
c=======================================================================
\end{verbatim}
 \end{routine}

%%%%% routine dsralph3.f %%%%%
\begin{routine}{dsralph3.f}
\begin{verbatim}
      real*8 function dsralph3(mscale)
No header found.
\end{verbatim}
 \end{routine}

%%%%% routine dsralph31loop.f %%%%%
\begin{routine}{dsralph31loop.f}
\begin{verbatim}
      real*8 function dsralph31loop(mscale)
No header found.
\end{verbatim}
 \end{routine}

%%%%% routine dsrghm.f %%%%%
\begin{routine}{dsrghm.f}
\begin{verbatim}
      subroutine dsrghm(mchi,ma,tanb,mq,mur,md,mtop,au,ad,mu,
     &     mhp,hmp,sa,ca,tanba,v,mz,alpha1,alpha2,alpha3z,
     &     lambda,prtlevel)
c
c carena, quiros, wagner -- adapted to darksusy by gondolo
c
\end{verbatim}
 \end{routine}

%%%%% routine dsrmq.f %%%%%
\begin{routine}{dsrmq.f}
\begin{verbatim}
      real*8 function dsrmq(mscale,kpart)
No header found.
\end{verbatim}
 \end{routine}

%%%%% routine dsrmq1loop.f %%%%%
\begin{routine}{dsrmq1loop.f}
\begin{verbatim}
      real*8 function dsrmq1loop(mscale,kpart)
No header found.
\end{verbatim}
 \end{routine}

%%%%% routine dssettopmass.f %%%%%
\begin{routine}{dssettopmass.f}
\begin{verbatim}
      subroutine dssettopmass(mtoppole)
No header found.
\end{verbatim}
 \end{routine}

%%%%% routine dssfesct.f %%%%%
\begin{routine}{dssfesct.f}
\begin{verbatim}
      subroutine dssfesct(unphys)
c_______________________________________________________________________
c  sfermion masses and mixings.
c  options:
c    msquarks=0  : squark masses and mixing from mass matrix
c    msquarks>0  : all squark masses equal to msquarks, no mixing
c    msquarks<0  : all squark masses = max(lsp,-msquarks), no mixing
c    msleptons=0  : squark masses and mixing from mass matrix
c    msleptons>0  : all squark masses equal to msleptons, no mixing
c    msleptons<0  : all squark masses = max(lsp,-msleptons), no mixing
c  common:
c    'dssusy.h' - file with susy common blocks
c  called by susyin.
c  needs neusct.
c  author: paolo gondolo 1994-1999
c     981000 correction to diagonalization of mass matrices
c     941100 addition of generation-mixing for squarks
c     950100 correction to charged slepton mass matrix
c     990715 pg options msquarks and msleptons added
c     990724 pg drop chankowski constants
c     020219 pg better reporting of unphys
c     020405 pg matrix diagonalization rewritten to handle degenerate case
c=======================================================================
\end{verbatim}
 \end{routine}

%%%%% routine dsspectrum.f %%%%%
\begin{routine}{dsspectrum.f}
\begin{verbatim}
      subroutine dsspectrum(unphys,hwarning)
c_______________________________________________________________________
c  particle spectrum and mixing matrices
c  common:
c    'dssusy.h' - file with susy common blocks
c  uses sconst, sfesct, chasct, higsct, neusct
c  author: paolo gondolo (gondolo@lpthe.jussieu.fr) 1994-1999
c=======================================================================
\end{verbatim}
 \end{routine}

%%%%% routine dssuconst.f %%%%%
\begin{routine}{dssuconst.f}
\begin{verbatim}
      subroutine dssuconst
c_______________________________________________________________________
c  useful constants
c  common:
c    'dssusy.h' - file with susy common blocks
c  author: paolo gondolo 1994-1999
c  modified: 031105 neutrino's yukawa fixed (pg)
c=======================================================================
\end{verbatim}
 \end{routine}

%%%%% routine dssusy.f %%%%%
\begin{routine}{dssusy.f}
\begin{verbatim}
      subroutine dssusy(unphys,hwarning)
c_______________________________________________________________________
c  set up global variables for the supersymmetric model routines
c  and prepares the rate routines for a new model.
c  author: Joakim Edsjo, edsjo@physto.se, 2001
c=======================================================================
\end{verbatim}
 \end{routine}

%%%%% routine dsvertx.f %%%%%
\begin{routine}{dsvertx.f}
\begin{verbatim}
      subroutine dsvertx
c_______________________________________________________________________
c  some couplings used in DarkSUSY
c  author: paolo gondolo (gondolo@lpthe.jussieu.fr) 1994--
c  history:
c    951110 complex vertex constants
c    970213 joakim edsjo
c    990724 paolo gondolo trilinear higgs and goldstone couplings
c=======================================================================
c
c  vertices included:
c     see individual routines dsvertx1 and dsvertx3
c
\end{verbatim}
 \end{routine}

%%%%% routine dsvertx1.f %%%%%
\begin{routine}{dsvertx1.f}
\begin{verbatim}
      subroutine dsvertx1
c_______________________________________________________________________
c  some couplings used in neutralino-neutralino, neutralino-chargino
c  and chargino-chargino annihilation.
c  common:
c    'dssusy.h' - file with susy common blocks
c  called by susyin.
c  needs neusct, chasct, sfesct, higsct.
c  author: paolo gondolo (gondolo@lpthe.jussieu.fr) 1994-1999
c  history:
c    951110 complex vertex constants
c    970213 joakim edsjo
c    990724 paolo gondolo trilinear higgs and goldstone couplings
c    020710 Joakim Edsjo, chargino-(up-squark)-(down-quark) sign-change
c    020903 Mia Schelke, Higgs-squark-squark, A-terms sign-change
c=======================================================================
c
c  vertices included:
c     z-w-w, z-h-h, w-h-a, w-h-h, h-w-w, h-z-z, z-a-h, h-h-h, h-a-a,
c     h-h-h, a-f-f, h-f-f, z-f-f, a-n-n, h-n-n, z-n-n, w-n-c, h-n-c,
c     squark-gluino-quark, sf-n-f, h-c-c, a-c-c, squark-squark-higgs,
c     w-f-f', h-f-f', gamma-w-w, gamma-h-h, z-c-c, gamma-c-c
c     gamma-f-f
c     gld-h-h, gld-gld-h, gld-n-c
c     Z-f~-f~, gamma-f~-f~, gluon-f~-f~
c
\end{verbatim}
 \end{routine}

%%%%% routine dsvertx3.f %%%%%
\begin{routine}{dsvertx3.f}
\begin{verbatim}
      subroutine dsvertx3
c_______________________________________________________________________
c  some couplings used in sfermion coannihilations
c  common:
c    'dssusy.h' - file with susy common blocks
c  author: paolo gondolo (pxg26@po.cwru.edu) 2001
c  history:
c    0110XX-020618 paolo gondolo
c    020903 Mia Schelke, Higgs-sfermion-sfermion, A-terms sign-change
c=======================================================================
c
c  vertices included:
c     Z-f~-f~, gamma-f~-f~, gluon-f~-f~, W-f~-F~, h-f~-f~
c  quartic vertices included:
c     Z-Z-h-h, W-W-h-h, Z-W-h-h, W-W-f~-f~, 
c     gamma-gamma-f~-f~, Z-Z-f~-f~, gamma-Z-f~-f~, gamma-W-f~-F~,
c     gluon-gluon-f~-f~, gluon-W-f~-F~, gluon-gamma-f~-f~, gluon-Z-f~-f~,
c     h-h-f~-f~, h-goldstone-f~-f~, goldstone-goldstone-f~-f~,
c     h-h-h-h, h-h-h-goldstone, h-h-goldstone-goldstone, 
c     h-goldstone-goldstone-goldstone, 4 x goldstone
c
\end{verbatim}
 \end{routine}

%%%%% routine dswhwarn.f %%%%%
\begin{routine}{dswhwarn.f}
\begin{verbatim}
      subroutine dswhwarn(unit,hwarning)
c_______________________________________________________________________
c  write reasons for unphys<>0 to specified unit.
c  input:
c    unit - logical unit to write on (integer)
c  author: paolo gondolo (gondolo@lpthe.jussieu.fr) 1994
c=======================================================================
\end{verbatim}
 \end{routine}

%%%%% routine dswspectrum.f %%%%%
\begin{routine}{dswspectrum.f}
\begin{verbatim}
************************************************************************
      subroutine dswspectrum(unit)
c_______________________________________________________________________
c  write out a table of the mass spectrum.
c  input:
c    unit - logical unit to write on (integer)
c  common:
c    'dssusy.h' - file with susy common blocks
c  author: paolo gondolo (gondolo@lpthe.jussieu.fr) 1994
c=======================================================================
\end{verbatim}
 \end{routine}

%%%%% routine dswunph.f %%%%%
\begin{routine}{dswunph.f}
\begin{verbatim}
      subroutine dswunph(unit,unphys)
c_______________________________________________________________________
c  write reasons for unphys<>0 to specified unit.
c  input:
c    unit - logical unit to write on (integer)
c  author: paolo gondolo (gondolo@lpthe.jussieu.fr) 1994
c=======================================================================
\end{verbatim}
 \end{routine}

%%%%% routine dswvertx.f %%%%%
\begin{routine}{dswvertx.f}
\begin{verbatim}
************************************************************************
      subroutine dswvertx(unit)
c_______________________________________________________________________
c  write out a table of the vertices constants.
c  input:
c    unit - logical unit to write on (integer)
c  common:
c    'dssusy.h' - file with susy common blocks
c  author: paolo gondolo (gondolo@lpthe.jussieu.fr) 1994
c=======================================================================
\end{verbatim}
 \end{routine}

%%%%% routine g4p.f %%%%%
\begin{routine}{g4p.f}
\begin{verbatim}
      function g4p(kp1,kp2,kp3,kp4)
c_______________________________________________________________________
c  function returning the 4-particle vertex
c  author: paolo gondolo (pxg26@po.cwru.edu) 2001
c=======================================================================
\end{verbatim}
 \end{routine}

\newpage
\chapter[suspect: mSUGRA interface (suspect) to DarkSUSY]{\codeb{src/suspect}:\\ mSUGRA interface (suspect) to DarkSUSY}
\label{ch:src-suspect}

\section{Routine headers -- fortran files}

%%%%% routine dssuspecterr.f %%%%%
\begin{routine}{dssuspecterr.f}
\begin{verbatim}
      subroutine dssuspecterr(unit,unphys,errmess)
c_______________________________________________________________________
c  write reasons for unphys<>0 in suspect output to specified unit.
c  input:
c    unit - logical unit to write on (integer)
c  author: paolo gondolo (gondolo@lpthe.jussieu.fr) 1994
c  adapted from suspect by: piero ullio, ullio@sissa.it, september 2001
c=======================================================================
\end{verbatim}
 \end{routine}

%%%%% routine dssuspectsugra.f %%%%%
\begin{routine}{dssuspectsugra.f}
\begin{verbatim}
      subroutine dssuspectsugra(unphys,errmess)
c_______________________________________________________________________
c  interface subroutine between Suspect & Darksusy in mSUGRA case
c  this routine should be called instead of dssusy
c  
c  call for each given set of mSUGRA variables: 
c     m0var,mhfvar,a0var,tgbetavar,sgnmuvar
c  given in the sugrainput common block; additional standard inputs for 
c  Suspect are set in this routine as specified in the body below. 
c
c  if error checks in Suspect are all ok, the routine sets up: 
c  1) SM inputs in Darksusy according to SM standard inputs in Suspect
c  2) Darksusy SUSY variables and soft terms 
c  3) full mass spectrum and mixing matrices according to Suspect output
c  4) global constants needed to run Darksusy (relic density + detection 
c     rates + constraint checks), set with dssuconst
c  5) interaction vertices, set with dsvertx
c  6) particle widths, set with dshwidths + plus a few set explicitly in
c     this routine
c  7) prepares the rate routines for a new model, calling dsprep
c
c  in output if unphys=0, everything is ok
c  if unphys < 0, one or more errmess(10) are <0, check error message
c  with dssuspecterr.f
c
c  expanded from a version by Emmanuel NEZRI, nezri@in2p3.fr,
c  February 2001  
c
c  author: Piero Ullio, ullio@sissa.it, September 2001
c=======================================================================
\end{verbatim}
 \end{routine}

%%%%% routine suspect2.f %%%%%
\begin{routine}{suspect2.f}
\begin{verbatim}
       subroutine SUSPECT2(iknowl,input,ichoice,errmess)
c  VERSION 2.002  
c: last changes : September 10 2001
c+++++++++++++++++++++++++++++++++++++++++++++++++++++++++++
c  J.-L. Kneur, A. Djouadi, G. Moultaka
c see home page: 
c http://www.lpm.univ-montp2.fr:7082/~kneur/
c for manual, updated info and maintenance
c++++++++++++++++++++++++++++++++++++++++++++++++++++++++++++
c
c||||||||||||||||||||||||||||||||||||||||||||||||||||||||||||
c  Calculates the MSSM mass spectrum (charginos, neutralinos, 
c  squarks, sleptons and  Higgs bosons (h,H,A,H+));   
c  including RG evolution of parameters, with different
c  options on models, approximations used, etc (see below).
c   (with some routines taken from 
c   already existing codes, in particular Higgs masses from 
c   M. Carena, C. Quiros, C. Wagner available routine;
c  some Higgs-related routines also common with "HDECAY" code.)
c|||||||||||||||||||||||||||||||||||||||||||||||||||||||||||||||
c EXAMPLE OF CALL: SEE THE ACCOMPANYING FILE suspect2_call.f
c where new (2) version options and detailed calling examples are
c given  
c------------------------------------------------------
c * Notations, definitions, analytic expressions based (mostly) on
c    a mixture of: ** Castano,Ramond,Piard Phys. Rev. D49(1994)4882. 
c                  ** Barger,Berger,Ohmann Phys.Rev. D49(1994)4908.
c EXCEPT for some SIGN CONVENTIONS changed (see conventions below!)
c|||||||||||||||||||||||||||||||||||||||||||||||||||||||||||||
c------DEFINITION OF PARAMETERS AND CONVENTIONS USED ----    |
c|||||||||||||||||||||||||||||||||||||||||||||||||||||||||||||
c  INPUT parameters: there are 3 CLASS of "input" parameters 
c  1) *** OPTION FLAG PARAMETERS (driven from input file) ***
c  =ICHOICE(1--10) (see suspect2_call.f for more explanations)
c
c 2)``Standard Model" parameters (not to be changed normally)
C ALFINV:   1/ALPHA(MZ): QED Coupling (at MZ scale, MSbar scheme)
c  ( reference latest value is ALFINV = 127.938 )
C SW2:       sin^2(theta)_W(MZ) in the MSbar scheme
c  (reference value: SW2= .23117 for MTOP =175 GeV) 
C ALPHAS:  VALUE FOR ALPHA_S(M_Z) (at the MZ scale)
c  (reference value: ALPHAS= .119 )
C MT:   TOP POLE MASS (reference value is MT= 174.3 GeV)
c MB:   BOttom pole mass (ref. value 4.62 GeV)
c MC:   Charm pole mass  (ref. value 1.4 GeV)
c
c  3) MSSM models physical parameters:
c  m0, m1/2, A0, tan(beta), sign(MU) in minimal SUGRA;
c  or arbitrary soft-breaking terms in non-universal models
c (see input file suspect2.in for more details and examples)
c|||||||||||||||||||||||||||||||||||||||||||||||||||||||||
c  **** IMPORTANT: MU SIGN (AND OTHER) CONVENTIONS ***** :
c|||||||||||||||||||||||||||||||||||||||||||||||||||||||||
c  1) WE DEFINE THE SUPERPOTENTIAL with the sign of MU conventions
c  as: W = MU (H_u . H_d) +.. =(def)= MU *eps(i,j) H^i_u H^j_d +..;
c   eps(1,2) = 1 = -eps(2,1)
c  where H_u, H_d are chiral (Higgs, Higgsinos)_u,d SUPERfields; 
c  2) WE DEFINE  the susy-breaking MU-term in the scalar potential
c  with the convention: V = B* MU eps(i,j) h^i_u * h^j_d (h_u,d are now
c  ordinary scalar fields, h_u=(h^1_u, h^2_u)=(h^{+}_u, h^0_u), 
c  h_d=(h^1_d, h^2_d)=(h^0_d, h^{-}_d)
c 
c   MU signs in relevant terms then follow as:
c  
c  -Sfermions: mixing terms =  m_LR = A_i - MU *(TBETA or 1/TBETA)
c  where A_i = A_b, A_tau or A_top
c  - Chargino:  +MU in mass matrix mixing terms
c  - Neutralinos:  -MU in mass matrix mixing terms;
c  - ALSO, Higgs potential minimization condition 
c  (radiative SU(2)xU(1) breaking) takes then the following form:
c  MZ^2/2 =(m1^2 - m2^2 * tbeta^2)/(tbeta^2 -1)
c   B* MU = (m1^2+m2^2)/2 * sin( 2 beta)
c  (where m1^2 =(m_phi_d)^2 +MU^2 +one-loop corrections
c         m2^2 =(m_phi_u)^2 +MU^2 +one-loop corrections )

c   OTHER RELEVANT CONVENTIONS :
c  3) sign of M1,M2,M3:  -M_i * Gbar G in Lagrangien (i.e.
c   ``normal" fermion mass signs), where G,Gbar are gaugino fields. 
c  4)  TGBETA = vu/vd (Q=MZ); (INPUT);
c  5)  v = sqrt(vu^2+vd^2) = 1/sqrt(2*sqrt(2)*GF) =174.** GeV
c    i.e. MW^2 = g2^2 * v^2/2 and there are NO 1/sqrt(2) 
c    factor in phi_u,d: <phi_u> = vu ; <phi_d> = vd 
c 
c -------------------------------------------------------
c  Main RG relevant variables: 
c  y(n) = vector containing all (RG evolving) parameters, 
c  at various possible scales depending on evolution stages.
c  n = number of RG-evolved parameters (may be different from
c  the initial  free parameters)
c  Those RG-evolving, scale-dependent parameters are:
c  y(1) = g1^2   U(1) gauge coupling
c  y(2) = g2^2   SU(2)_L gauge coupling
c  y(3) = g3^2 = 4*pi*alphas  SU(3) gauge coupling
c
c  y(4) = Y_tau  tau Yukawa coupling 
c  y(5) = Y_b    bottom  Yukawa coupling
c  y(6) = Y_top  top Yukawa coupling
c
c  y(7) = Ln(vu)  Logarithm of vu 
c  y(8) = Ln(vd)  Logarithm of vd 
c
c  y(9) = A_tau 
c  y(10)= A_b
c  y(11)=A_top
c
c  y(12) = (m_phi_u)^2  scalar phi_u "mass" term (in potential)
c  y(13) = (m_phi_d)^2  scalar phi_d "mass" term (in potential)
c  
c  y(14) = MTAUR^2 right-handed Stau Lagrangian mass^2 term
c  y(15) = MSL^2 left-handed Stau lagrangian mass^2 term
c  y(16) = MBR^2 right-handed Sbottom lagrangian mass^2 term
c  y(17) = MTR^2 right-handed Stop Lagrangian mass^2 term 
c  y(18) = MSQ^2 left-handed Stop Lagrangian mass^2 term
c
c  y(19) = B  The (dimensionful) B parameter in scalar mixing 
c
c  y(20) = Ln(M1) Log of Bino mass term
c  y(21) = Ln(M2) Log of Wino mass term
c  y(22) = Ln(M3) Log of gluino mass term
c
c  y(23) = Ln(MU) Log of the MU parameter, as defined above.
c
c  y(24) = MER^2 right-handed Selectron(Smuon) Lagrangian mass^2
c  y(25) = MEL^2 left-handed Selctron(Smuon) Lagrangian mass^2
c  y(26) = MDR^2 right-handed Sdown(Sstrange) Lagrangian mass^2
c  y(27) = MUR^2 right-handed Sup(Scharm) Lagrangian mass^2
c  y(28) = MUQ^2 left-handed Sup(Scharm) Lagrangian mass^2
c
c******************************************************
c  PROGRAM  COMMAND LINES START HERE 
c******************************************************
\end{verbatim}
 \end{routine}

\newpage
\chapter[xcern: CERN routines needed by DarkSUSY]{\codeb{src/xcern}:\\ CERN routines needed by DarkSUSY}
\label{ch:src-xcern}

\section{Routine headers -- fortran files}

%%%%% routine besj064.f %%%%%
\begin{routine}{besj064.f}
\begin{verbatim}
*
* $Id: besj064.F,v 1.1.1.1 1996/04/01 15:01:59 mclareni Exp $
*
* $Log: besj064.F,v $
* Revision 1.1.1.1  1996/04/01 15:01:59  mclareni
* Mathlib gen
*
*
      FUNCTION DBESJ0(X)
\end{verbatim}
 \end{routine}

%%%%% routine bsir364.f %%%%%
\begin{routine}{bsir364.f}
\begin{verbatim}
*
* $Id: bsir364.F,v 1.1.1.1 1996/04/01 15:02:07 mclareni Exp $
*
* $Log: bsir364.F,v $
* Revision 1.1.1.1  1996/04/01 15:02:07  mclareni
* Mathlib gen
*
*
      FUNCTION DBSIR3(X,NU)
\end{verbatim}
 \end{routine}

%%%%% routine dbzejy.f %%%%%
\begin{routine}{dbzejy.f}
\begin{verbatim}
      SUBROUTINE DBZEJY(A,N,MODE,REL,X)

C     Computes the first n positive (in the case Jo'(x) the first n
C     non-negative) zeros of the Bessel functions
C               Ja(x), Ya(x), Ja'(x), Ya'(x),
C     where a >= 0 and ' = d/dx.
C
C     Based on Algol procedures published in
C
C     N.M. TEMME, An algorithm with Algol 60 program for the compu-
C     tation of the zeros of ordinary Bessel functions and those of
C     their derivatives, J. Comput. Phys. 32 (1979) 270-279, and
C
C     N.M. TEMME, On the numerical evaluation of the ordinary Bessel
C     function of the second kind, J. Comput. Phys. 21 (1976) 343-350.

\end{verbatim}
 \end{routine}

%%%%% routine ddilog.f %%%%%
\begin{routine}{ddilog.f}
\begin{verbatim}
CDECK  ID>, DDILOG.
      DOUBLE PRECISION FUNCTION DDILOG(X)
C
C          FROM CERN PROGRAM LIBRARY
C
\end{verbatim}
 \end{routine}

%%%%% routine dgadap.f %%%%%
\begin{routine}{dgadap.f}
\begin{verbatim}

c#######################################################################
c
c   one- and two-dimensional adaptive gaussian integration routines.
c
c **********************************************************************

      subroutine dgadap(a0,b0,f,eps0,sum)
c
c   purpose           - integrate a function f(x)
c   method            - adaptive gaussian
c   usage             - call gadap(a0,b0,f,eps,sum)
c   parameters  a0    - lower limit (input,real)
c               b0    - upper limit (input,real)
c               f     - function f(x) to be integrated. must be
c                       supplied by the user. (input,real function)
c               eps0  - desired relative accuracy. if sum is small eps
c                       will be absolute accuracy instead. (input,real)
c               sum   - calculated value for the integral (output,real)
c   precision         - single (see below)
c   req'd prog's      - f
c   author            - t. johansson, lund univ. computer center, 1973
c   reference(s)      - the australian computer journal,3 p.126 aug. -71
c
c  made real*8 by j. edsjo 97-01-17
\end{verbatim}
 \end{routine}

%%%%% routine drkstp.f %%%%%
\begin{routine}{drkstp.f}
\begin{verbatim}
      SUBROUTINE DRKSTP(N,H,X,Y,SUB,W)
No header found.
\end{verbatim}
 \end{routine}

%%%%% routine eisrs1.f %%%%%
\begin{routine}{eisrs1.f}
\begin{verbatim}
CDECK  ID>, EISRS1. 
      SUBROUTINE EISRS1(NM,N,AR,WR,ZR,IERR,WORK)
C     ALL EIGENVALUES AND CORRESPONDING EIGENVECTORS OF A REAL
C     SYMMETRIC MATRIX
C     FROM CERN PROGRAM LIBRARY
C
\end{verbatim}
 \end{routine}

%%%%% routine gpindp.f %%%%%
\begin{routine}{gpindp.f}
\begin{verbatim}
      real*8 function gpindp(a,b,epsin,epsout,func,iop)
c
c     parameters
c
c     a       = lower boundary
c     b       = upper boundary
c     epsin   = accuracy required for the approxination
c     epsout  = improved error estimate for the approximation
c     func    = function routine for the function func(x).to be de-
c               clared external in the calling routine
c     iop     = option parameter , iop=1 , modified romberg algorithm,
c                                          ordinary case
c                                  iop=2 , modified romberg algorithm,
c                                          cosine transformed case
c                                  iop=3 , modified clenshaw-curtis al
c                                          gorithm
c
c     parameters in common block / gpint /
c
c     tend    = upper bound for value of integral
c     umid    = lower bound for value of integralc
c     n       = the number of integrand values used in the calculation
c     line    = line no in romberg table (related to n through
c               n-1=2**(line-1) , applicable only for iop=1 or 2)
c     iout    = element no in line (applicable only for iop=1 or 2)
c     jop     = option parameter , jop=0 , no printing of intermediate
c                                          calculations
c                                  jop=1 , print intermediate calcula-
c                                          tions
c     kop     = option parameter , kop=0 , no time estimate
c                                  kop=1 , estimate time
c     t       = time used for calculation in msec.
c
c     integration parameters
c
c     nupper  = 9 , corresponds to 1024 sub-intervals for the unfolded
c               integral.the max.no of function evaluations thus beeing
c               1025.the highest end-point approximation is thus using
c               1024 intervals while the highest mid-point approxima-
c               tion is using 512 intervals.
c
c     input/output parameters
c
\end{verbatim}
 \end{routine}

%%%%% routine mtlprt.f %%%%%
\begin{routine}{mtlprt.f}
\begin{verbatim}
*
* Dummy routine to easily integrate bsir364.f with DarkSUSY
* Author: Joakim Edsjo, edsjo@physto.se
* Date: September 13, 2000
*
      SUBROUTINE MTLPRT(NAME,ERC,TEXT)
\end{verbatim}
 \end{routine}

%%%%% routine tql2.f %%%%%
\begin{routine}{tql2.f}
\begin{verbatim}
CDECK  ID>, TQL2.   
      SUBROUTINE TQL2(NM,N,D,E,Z,IERR)
C     FROM CERN PROGRAM LIBRARY
\end{verbatim}
 \end{routine}

%%%%% routine tred2.f %%%%%
\begin{routine}{tred2.f}
\begin{verbatim}
CDECK  ID>, TRED2.  
      SUBROUTINE TRED2(NM,N,A,D,E,Z)
C     FROM CERN PROGRAM LIBRARY
\end{verbatim}
 \end{routine}

\newpage
\chapter[xcmlib: CMLIB routines needed by DarkSUSY]{\codeb{src/xcmlib}:\\ CMLIB routines needed by DarkSUSY}
\label{ch:src-xcmlib}

\section{Routine headers -- fortran files}

%%%%% routine d1mach.f %%%%%
\begin{routine}{d1mach.f}
\begin{verbatim}

      real*8 function d1mach(i)
\end{verbatim}
 \end{routine}

%%%%% routine dqagse.f %%%%%
\begin{routine}{dqagse.f}
\begin{verbatim}
* ======================================================================
* nist guide to available math software.
* fullsource for module dqagse from package cmlib.
* retrieved from camsun on wed oct  8 08:26:30 1997.
* ======================================================================
      subroutine dqagse(f,a,b,epsabs,epsrel,limit,result,abserr,neval,
     1   ier,alist,blist,rlist,elist,iord,last)
c***begin prologue  dqagse
c***date written   800101   (yymmdd)
c***revision date  830518   (yymmdd)
c***category no.  h2a1a1
c***keywords  (end point) singularities,automatic integrator,
c             extrapolation,general-purpose,globally adaptive
c***author  piessens, robert, applied math. and progr. div. -
c             k. u. leuven
c           de doncker, elise, applied math. and progr. div. -
c             k. u. leuven
c***purpose  the routine calculates an approximation result to a given
c            definite integral i = integral of f over (a,b),
c            hopefully satisfying following claim for accuracy
c            abs(i-result).le.max(epsabs,epsrel*abs(i)).
c***description
c
c        computation of a definite integral
c        standard fortran subroutine
c        real*8 version
c
c        parameters
c         on entry
c            f      - real*8
c                     function subprogram defining the integrand
c                     function f(x). the actual name for f needs to be
c                     declared e x t e r n a l in the driver program.
c
c            a      - real*8
c                     lower limit of integration
c
c            b      - real*8
c                     upper limit of integration
c
c            epsabs - real*8
c                     absolute accuracy requested
c            epsrel - real*8
c                     relative accuracy requested
c                     if  epsabs.le.0
c                     and epsrel.lt.max(50*rel.mach.acc.,0.5d-28),
c                     the routine will end with ier = 6.
c
c            limit  - integer
c                     gives an upperbound on the number of subintervals
c                     in the partition of (a,b)
c
c         on return
c            result - real*8
c                     approximation to the integral
c
c            abserr - real*8
c                     estimate of the modulus of the absolute error,
c                     which should equal or exceed abs(i-result)
c
c            neval  - integer
c                     number of integrand evaluations
c
c            ier    - integer
c                     ier = 0 normal and reliable termination of the
c                             routine. it is assumed that the requested
c                             accuracy has been achieved.
c                     ier.gt.0 abnormal termination of the routine
c                             the estimates for integral and error are
c                             less reliable. it is assumed that the
c                             requested accuracy has not been achieved.
c            error messages
c                         = 1 maximum number of subdivisions allowed
c                             has been achieved. one can allow more sub-
c                             divisions by increasing the value of limit
c                             (and taking the according dimension
c                             adjustments into account). however, if
c                             this yields no improvement it is advised
c                             to analyze the integrand in order to
c                             determine the integration difficulties. if
c                             the position of a local difficulty can be
c                             determined (e.g. singularity,
c                             discontinuity within the interval) one
c                             will probably gain from splitting up the
c                             interval at this point and calling the
c                             integrator on the subranges. if possible,
c                             an appropriate special-purpose integrator
c                             should be used, which is designed for
c                             handling the type of difficulty involved.
c                         = 2 the occurrence of roundoff error is detec-
c                             ted, which prevents the requested
c                             tolerance from being achieved.
c                             the error may be under-estimated.
c                         = 3 extremely bad integrand behaviour
c                             occurs at some points of the integration
c                             interval.
c                         = 4 the algorithm does not converge.
c                             roundoff error is detected in the
c                             extrapolation table.
c                             it is presumed that the requested
c                             tolerance cannot be achieved, and that the
c                             returned result is the best which can be
c                             obtained.
c                         = 5 the integral is probably divergent, or
c                             slowly convergent. it must be noted that
c                             divergence can occur with any other value
c                             of ier.
c                         = 6 the input is invalid, because
c                             epsabs.le.0 and
c                             epsrel.lt.max(50*rel.mach.acc.,0.5d-28).
c                             result, abserr, neval, last, rlist(1),
c                             iord(1) and elist(1) are set to zero.
c                             alist(1) and blist(1) are set to a and b
c                             respectively.
c
c            alist  - real*8
c                     vector of dimension at least limit, the first
c                      last  elements of which are the left end points
c                     of the subintervals in the partition of the
c                     given integration range (a,b)
c
c            blist  - real*8
c                     vector of dimension at least limit, the first
c                      last  elements of which are the right end points
c                     of the subintervals in the partition of the given
c                     integration range (a,b)
c
c            rlist  - real*8
c                     vector of dimension at least limit, the first
c                      last  elements of which are the integral
c                     approximations on the subintervals
c
c            elist  - real*8
c                     vector of dimension at least limit, the first
c                      last  elements of which are the moduli of the
c                     absolute error estimates on the subintervals
c
c            iord   - integer
c                     vector of dimension at least limit, the first k
c                     elements of which are pointers to the
c                     error estimates over the subintervals,
c                     such that elist(iord(1)), ..., elist(iord(k))
c                     form a decreasing sequence, with k = last
c                     if last.le.(limit/2+2), and k = limit+1-last
c                     otherwise
c
c            last   - integer
c                     number of subintervals actually produced in the
c                     subdivision process
c***references  (none)
c***routines called  d1mach,dqelg,dqk21,dqpsrt
c***end prologue  dqagse
c
\end{verbatim}
 \end{routine}

%%%%% routine dqagseb.f %%%%%
\begin{routine}{dqagseb.f}
\begin{verbatim}
* ======================================================================
* nist guide to available math software.
* fullsource for module dqagse from package cmlib.
* retrieved from camsun on wed oct  8 08:26:30 1997.
* ======================================================================
      subroutine dqagseb(f,a,b,epsabs,epsrel,limit,result,abserr,neval,
     1   ier,alist,blist,rlist,elist,iord,last)
c***begin prologue  dqagse
c***date written   800101   (yymmdd)
c***revision date  830518   (yymmdd)
c***category no.  h2a1a1
c***keywords  (end point) singularities,automatic integrator,
c             extrapolation,general-purpose,globally adaptive
c***author  piessens, robert, applied math. and progr. div. -
c             k. u. leuven
c           de doncker, elise, applied math. and progr. div. -
c             k. u. leuven
c***purpose  the routine calculates an approximation result to a given
c            definite integral i = integral of f over (a,b),
c            hopefully satisfying following claim for accuracy
c            abs(i-result).le.max(epsabs,epsrel*abs(i)).
c***description
c
c        computation of a definite integral
c        standard fortran subroutine
c        real*8 version
c
c        parameters
c         on entry
c            f      - real*8
c                     function subprogram defining the integrand
c                     function f(x). the actual name for f needs to be
c                     declared e x t e r n a l in the driver program.
c
c            a      - real*8
c                     lower limit of integration
c
c            b      - real*8
c                     upper limit of integration
c
c            epsabs - real*8
c                     absolute accuracy requested
c            epsrel - real*8
c                     relative accuracy requested
c                     if  epsabs.le.0
c                     and epsrel.lt.max(50*rel.mach.acc.,0.5d-28),
c                     the routine will end with ier = 6.
c
c            limit  - integer
c                     gives an upperbound on the number of subintervals
c                     in the partition of (a,b)
c
c         on return
c            result - real*8
c                     approximation to the integral
c
c            abserr - real*8
c                     estimate of the modulus of the absolute error,
c                     which should equal or exceed abs(i-result)
c
c            neval  - integer
c                     number of integrand evaluations
c
c            ier    - integer
c                     ier = 0 normal and reliable termination of the
c                             routine. it is assumed that the requested
c                             accuracy has been achieved.
c                     ier.gt.0 abnormal termination of the routine
c                             the estimates for integral and error are
c                             less reliable. it is assumed that the
c                             requested accuracy has not been achieved.
c            error messages
c                         = 1 maximum number of subdivisions allowed
c                             has been achieved. one can allow more sub-
c                             divisions by increasing the value of limit
c                             (and taking the according dimension
c                             adjustments into account). however, if
c                             this yields no improvement it is advised
c                             to analyze the integrand in order to
c                             determine the integration difficulties. if
c                             the position of a local difficulty can be
c                             determined (e.g. singularity,
c                             discontinuity within the interval) one
c                             will probably gain from splitting up the
c                             interval at this point and calling the
c                             integrator on the subranges. if possible,
c                             an appropriate special-purpose integrator
c                             should be used, which is designed for
c                             handling the type of difficulty involved.
c                         = 2 the occurrence of roundoff error is detec-
c                             ted, which prevents the requested
c                             tolerance from being achieved.
c                             the error may be under-estimated.
c                         = 3 extremely bad integrand behaviour
c                             occurs at some points of the integration
c                             interval.
c                         = 4 the algorithm does not converge.
c                             roundoff error is detected in the
c                             extrapolation table.
c                             it is presumed that the requested
c                             tolerance cannot be achieved, and that the
c                             returned result is the best which can be
c                             obtained.
c                         = 5 the integral is probably divergent, or
c                             slowly convergent. it must be noted that
c                             divergence can occur with any other value
c                             of ier.
c                         = 6 the input is invalid, because
c                             epsabs.le.0 and
c                             epsrel.lt.max(50*rel.mach.acc.,0.5d-28).
c                             result, abserr, neval, last, rlist(1),
c                             iord(1) and elist(1) are set to zero.
c                             alist(1) and blist(1) are set to a and b
c                             respectively.
c
c            alist  - real*8
c                     vector of dimension at least limit, the first
c                      last  elements of which are the left end points
c                     of the subintervals in the partition of the
c                     given integration range (a,b)
c
c            blist  - real*8
c                     vector of dimension at least limit, the first
c                      last  elements of which are the right end points
c                     of the subintervals in the partition of the given
c                     integration range (a,b)
c
c            rlist  - real*8
c                     vector of dimension at least limit, the first
c                      last  elements of which are the integral
c                     approximations on the subintervals
c
c            elist  - real*8
c                     vector of dimension at least limit, the first
c                      last  elements of which are the moduli of the
c                     absolute error estimates on the subintervals
c
c            iord   - integer
c                     vector of dimension at least limit, the first k
c                     elements of which are pointers to the
c                     error estimates over the subintervals,
c                     such that elist(iord(1)), ..., elist(iord(k))
c                     form a decreasing sequence, with k = last
c                     if last.le.(limit/2+2), and k = limit+1-last
c                     otherwise
c
c            last   - integer
c                     number of subintervals actually produced in the
c                     subdivision process
c***references  (none)
c***routines called  d1mach,dqelg,dqk21b,dqpsrt
c***end prologue  dqagse
c
\end{verbatim}
 \end{routine}

%%%%% routine dqelg.f %%%%%
\begin{routine}{dqelg.f}
\begin{verbatim}
      subroutine dqelg(n,epstab,result,abserr,res3la,nres)
c***begin prologue  dqelg
c***refer to  dqagie,dqagoe,dqagpe,dqagse
c***routines called  d1mach
c***revision date  830518   (yymmdd)
c***keywords  convergence acceleration,epsilon algorithm,extrapolation
c***author  piessens, robert, applied math. and progr. div. -
c             k. u. leuven
c           de doncker, elise, applied math. and progr. div. -
c             k. u. leuven
c***purpose  the routine determines the limit of a given sequence of
c            approximations, by means of the epsilon algorithm of
c            p.wynn. an estimate of the absolute error is also given.
c            the condensed epsilon table is computed. only those
c            elements needed for the computation of the next diagonal
c            are preserved.
c***description
c
c           epsilon algorithm
c           standard fortran subroutine
c           real*8 version
c
c           parameters
c              n      - integer
c                       epstab(n) contains the new element in the
c                       first column of the epsilon table.
c
c              epstab - real*8
c                       vector of dimension 52 containing the elements
c                       of the two lower diagonals of the triangular
c                       epsilon table. the elements are numbered
c                       starting at the right-hand corner of the
c                       triangle.
c
c              result - real*8
c                       resulting approximation to the integral
c
c              abserr - real*8
c                       estimate of the absolute error computed from
c                       result and the 3 previous results
c
c              res3la - real*8
c                       vector of dimension 3 containing the last 3
c                       results
c
c              nres   - integer
c                       number of calls to the routine
c                       (should be zero at first call)
c***end prologue  dqelg
c
\end{verbatim}
 \end{routine}

%%%%% routine dqk21.f %%%%%
\begin{routine}{dqk21.f}
\begin{verbatim}
      subroutine dqk21(f,a,b,result,abserr,resabs,resasc)
c***begin prologue  dqk21
c***date written   800101   (yymmdd)
c***revision date  830518   (yymmdd)
c***category no.  h2a1a2
c***keywords  21-point gauss-kronrod rules
c***author  piessens, robert, applied math. and progr. div. -
c             k. u. leuven
c           de doncker, elise, applied math. and progr. div. -
c             k. u. leuven
c***purpose  to compute i = integral of f over (a,b), with error
c                           estimate
c                       j = integral of abs(f) over (a,b)
c***description
c
c           integration rules
c           standard fortran subroutine
c           real*8 version
c
c           parameters
c            on entry
c              f      - real*8
c                       function subprogram defining the integrand
c                       function f(x). the actual name for f needs to be
c                       declared e x t e r n a l in the driver program.
c
c              a      - real*8
c                       lower limit of integration
c
c              b      - real*8
c                       upper limit of integration
c
c            on return
c              result - real*8
c                       approximation to the integral i
c                       result is computed by applying the 21-point
c                       kronrod rule (resk) obtained by optimal addition
c                       of abscissae to the 10-point gauss rule (resg).
c
c              abserr - real*8
c                       estimate of the modulus of the absolute error,
c                       which should not exceed abs(i-result)
c
c              resabs - real*8
c                       approximation to the integral j
c
c              resasc - real*8
c                       approximation to the integral of abs(f-i/(b-a))
c                       over (a,b)
c***references  (none)
c***routines called  d1mach
c***end prologue  dqk21
c
\end{verbatim}
 \end{routine}

%%%%% routine dqk21b.f %%%%%
\begin{routine}{dqk21b.f}
\begin{verbatim}
      subroutine dqk21b(f,a,b,result,abserr,resabs,resasc)
c***begin prologue  dqk21b
c***date written   800101   (yymmdd)
c***revision date  830518   (yymmdd)
c***category no.  h2a1a2
c***keywords  21-point gauss-kronrod rules
c***author  piessens, robert, applied math. and progr. div. -
c             k. u. leuven
c           de doncker, elise, applied math. and progr. div. -
c             k. u. leuven
c***purpose  to compute i = integral of f over (a,b), with error
c                           estimate
c                       j = integral of abs(f) over (a,b)
c***description
c
c           integration rules
c           standard fortran subroutine
c           real*8 version
c
c           parameters
c            on entry
c              f      - real*8
c                       function subprogram defining the integrand
c                       function f(x). the actual name for f needs to be
c                       declared e x t e r n a l in the driver program.
c
c              a      - real*8
c                       lower limit of integration
c
c              b      - real*8
c                       upper limit of integration
c
c            on return
c              result - real*8
c                       approximation to the integral i
c                       result is computed by applying the 21-point
c                       kronrod rule (resk) obtained by optimal addition
c                       of abscissae to the 10-point gauss rule (resg).
c
c              abserr - real*8
c                       estimate of the modulus of the absolute error,
c                       which should not exceed abs(i-result)
c
c              resabs - real*8
c                       approximation to the integral j
c
c              resasc - real*8
c                       approximation to the integral of abs(f-i/(b-a))
c                       over (a,b)
c***references  (none)
c***routines called  d1mach
c***end prologue  dqk21b
c
\end{verbatim}
 \end{routine}

%%%%% routine dqpsrt.f %%%%%
\begin{routine}{dqpsrt.f}
\begin{verbatim}
      subroutine dqpsrt(limit,last,maxerr,ermax,elist,iord,nrmax)
c***begin prologue  dqpsrt
c***refer to  dqage,dqagie,dqagpe,dqawse
c***routines called  (none)
c***revision date  810101   (yymmdd)
c***keywords  sequential sorting
c***author  piessens, robert, applied math. and progr. div. -
c             k. u. leuven
c           de doncker, elise, applied math. and progr. div. -
c             k. u. leuven
c***purpose  this routine maintains the descending ordering in the
c            list of the local error estimated resulting from the
c            interval subdivision process. at each call two error
c            estimates are inserted using the sequential search
c            method, top-down for the largest error estimate and
c            bottom-up for the smallest error estimate.
c***description
c
c           ordering routine
c           standard fortran subroutine
c           real*8 version
c
c           parameters (meaning at output)
c              limit  - integer
c                       maximum number of error estimates the list
c                       can contain
c
c              last   - integer
c                       number of error estimates currently in the list
c
c              maxerr - integer
c                       maxerr points to the nrmax-th largest error
c                       estimate currently in the list
c
c              ermax  - real*8
c                       nrmax-th largest error estimate
c                       ermax = elist(maxerr)
c
c              elist  - real*8
c                       vector of dimension last containing
c                       the error estimates
c
c              iord   - integer
c                       vector of dimension last, the first k elements
c                       of which contain pointers to the error
c                       estimates, such that
c                       elist(iord(1)),...,  elist(iord(k))
c                       form a decreasing sequence, with
c                       k = last if last.le.(limit/2+2), and
c                       k = limit+1-last otherwise
c
c              nrmax  - integer
c                       maxerr = iord(nrmax)
c***end prologue  dqpsrt
c
\end{verbatim}
 \end{routine}

\newpage
\chapter[xfeynhiggs: FeynHiggs interface to DarkSUSY]{\codeb{src/xfeynhiggs}:\\ FeynHiggs interface to DarkSUSY}
\label{ch:src-xfeynhiggs}

\section{Routine headers -- fortran files}

%%%%% routine dsfeynhiggs.f %%%%%
\begin{routine}{dsfeynhiggs.f}
\begin{verbatim}
       subroutine dsfeynhiggs(hwar,HM,mh,hc,halpha,drho,
     &  ATop,ABot,inmt,inmb,my,M2,mqtl,mqtr,mqbl,mqbr,
     &  mgluino,mh3,tanb,inmsbarselec)

c -----------------------------------------------------------------       
c Implementation of FeynHiggs in DarkSUSY by S. Heinemeyer, 06/13/02
c
c All names of subroutines and functions that clashed with the
c full FeynHiggs names have _fh appended to them. In most cases,
c they are probably the same routines though, but to be on the
c safe side, the names were changed.
c Output: mh is lighter scalar Higgs mass, HM is heavier Higgs mass 
c -----------------------------------------------------------------

c --------------------------------------------------------------
c
c     FeynHiggs
c     =========
c      
c       Calculation of the masses of the neutral CP-even
c       Higgs bosons in the MSSM
c       
c       Authors: Sven Heinemeyer (one-, two-loop part, new renormalization)
c                Andreas Dabelstein (one-loop part)
c                Markus Frank (new renormalization)
c       
c       Based on hep-ph/9803277, hep-ph/9807423, hep-ph/9812472,
c                hep-ph/9903404, hep-ph/9910283
c       by S. Heinemeyer, W. Hollik, G. Weiglein
c       and on hep-ph/0001002
c       by M. Carena, H. Haber, S. Heinemeyer, W. Hollik,
c          C. Wagner and G. Weiglein
c       new non-log O(alpha_t^2) corrections taken from hep-ph/0112177
c       by A. Brignole, G. Degrassi, P. Slavich and F. Zwirner
c
c       new renormalization implemented based on hep-ph/0202166
c       by M. Frank, S. Heinemeyer, W. Hollik and G. Weiglein
c      
c       In case of problems or questions,
c       contact Sven Heinemeyer
c       email: Sven.Heinemeyer@physik.uni-muenchen.de
c       
c       FeynHiggs homepage:
c       http://www.feynhiggs.de
c
c --------------------------------------------------------------
c --------------------------------------------------------------
c
c Warnings implemented by .......
c
c                   Bit  Value
c  Bits of hwar:    0 -  1:   Potential numerical problems at 1-loop
c                   1 -  2:   Potential numerical problems at 2-loop
c                   2 -  4:   Error with not used H2 mass expression 1-loop
c                   3 -  8:   Error with not used H2 mass expression 2-loop
c                   4 -  16:  Error with not used H2 mass expression 2-loop
c                   5 -  32:  1-loop Higgs sector not OK
c                   6 -  64:  2-loop Higgs sector not OK
c                   7 -  128: Stop or sbottom masses not OK
c---------------------------------------------------------------



\end{verbatim}
 \end{routine}

%%%%% routine dsfeynhiggsdummy.f %%%%%
\begin{routine}{dsfeynhiggsdummy.f}
\begin{verbatim}
       subroutine dsfeynhiggs(hwar,HM,mh,hc,halpha,drho,
     &  ATop,ABot,inmt,inmb,my,M2,mqtl,mqtr,mqbl,mqbr,
     &  mgluino,mh3,tanb,inmsbarselec)

c -----------------------------------------------------------------       
c Dummy routine for those folks who do not have FeynHiggs
c -----------------------------------------------------------------

\end{verbatim}
 \end{routine}

%%%%% routine FeynHiggsSub_ds.f %%%%%
\begin{routine}{FeynHiggsSub\_ds.f}
\begin{verbatim}

      subroutine feynhiggssub(mh1,mh2,mh12,mh22)

\end{verbatim}
 \end{routine}

%%%%% routine Hhmasssr2_ds.f %%%%%
\begin{routine}{Hhmasssr2\_ds.f}
\begin{verbatim}
c     %%%%%%%%%%%%%%%%%%%%%% geaendert! %%%%%%%%%%%%%%%%%%%%%%%%%
      DOUBLE PRECISION FUNCTION DELTA (EPSILON,MUEE,MASS)
C
\end{verbatim}
 \end{routine}

\newpage
\chapter[xhdecay: HDecay interface to DarkSUSY]{\codeb{src/xhdecay}:\\ HDecay interface to DarkSUSY}
\label{ch:src-xhdecay}

\section{Routine headers -- fortran files}

%%%%% routine dshdecay.f %%%%%
\begin{routine}{dshdecay.f}
\begin{verbatim}
c----------------------------------------------------------------------
c   Interface between DarkSUSY and HDECAY.
c   HDECAY is called and the results transferred to the DarkSUSY
c   common blocks. 
c   Author: Joakim Edsjo, edsjo@physto.se
c   Date: September 12, 2002
c----------------------------------------------------------------------

      subroutine dshdecay
\end{verbatim}
 \end{routine}

%%%%% routine hdecay.f %%%%%
\begin{routine}{hdecay.f}
\begin{verbatim}
C Modified by Joakim Edsjo 2002-09-12 to interface it with DarkSUSY.
C See README.TXT for more details

C         Last modification on July 11th 2001 by M.S.
C ==================================================================
C ================= PROGRAM HDECAY: COMMENTS =======================
C ==================================================================
C
C                       ***************
C                       * VERSION 2.0 *
C                       ***************
C
C
C  This program calculates the total decay widths and the branching 
C  ratios of the C Standard Model Higgs boson (HSM) as well as those 
C  of the neutral (HL= the light CP-even, HH= the heavy CP-even, HA= 
C  the pseudoscalar) and the charged (HC) Higgs bosons of the Minimal
C  Supersymmetric extension of the Standard Model (MSSM). It includes:
C
C - All the decay channels which are kinematically allowed and which
C   have branching ratios larger than 10**(-4). 
C
C - All QCD corrections to the fermionic and gluonic decay modes.
C   Most of these corrections are mapped into running masses in a
C   consistent way with some freedom for including high order terms. 
C
C - Below--threshold three--body decays with off--shell top quarks
C   or ONE off-shell gauge boson, as well as some decays with one
C   off-shell Higgs boson in the MSSM. 
C
C - Double off-shell decays: HSM,HL,HH --> W*W*,Z*Z* -->4 fermions,
C   which could be important for Higgs masses close to MW or MZ.
C
C - In the MSSM, the radiative corrections with full squark mixing and 
C   uses the RG improved values of Higgs masses and couplings with the 
C   main NLO corrections implemented (based on M.Carena, M. Quiros and
C   C.E.M. Wagner, Nucl. Phys. B461 (1996) 407, hep-ph/9508343). 
C
C - In the MSSM, all the decays into CHARGINOS, NEUTRALINOS, SLEPTONS 
C   and SQUARKS (with mixing in the stop and sbottom sectors). 
C
C - Chargino, slepton and squark loops in the 2 photon decays and squark
C   loops in the gluonic decays (including QCD corrections). 
C
C  ===================================================================
C  This program has been written by A.Djouadi, J.Kalinowski and M.Spira.
C  For details on how to use the program see: Comp. Phys. Commun. 108
C  (1998) 56, hep-ph/9704448. For any question, comment, suggestion or
C  complaint, please contact us at:
C          djouadi@lpm.univ-montp2.fr
C          kalino@fuw.edu.pl
C          Michael.Spira@cern.ch


C ================ IT USES AS INPUT PARAMETERS:
C
C   IHIGGS: =0: CALCULATE BRANCHING RATIOS OF SM HIGGS BOSON
C           =1: CALCULATE BRANCHING RATIOS OF MSSM h BOSON
C           =2: CALCULATE BRANCHING RATIOS OF MSSM H BOSON
C           =3: CALCULATE BRANCHING RATIOS OF MSSM A BOSON
C           =4: CALCULATE BRANCHING RATIOS OF MSSM H+ BOSON
C           =5: CALCULATE BRANCHING RATIOS OF ALL MSSM HIGGS BOSONS
C
C TGBET:    TAN(BETA) FOR MSSM
C MABEG:    START VALUE OF M_A FOR MSSM AND M_H FOR SM
C MAEND:    END VALUE OF M_A FOR MSSM AND M_H FOR SM
C NMA:      NUMBER OF ITERATIONS FOR M_A
C ALS(MZ):  VALUE FOR ALPHA_S(M_Z)
C MSBAR(1): MSBAR MASS OF STRANGE QUARK AT SCALE Q=1 GEV
C MC:       CHARM POLE MASS
C MB:       BOTTOM POLE MASS
C MT:       TOP POLE MASS
C MTAU:     TAU MASS
C MMUON:    MUON MASS
C ALPH:     INVERSE QED COUPLING
C GF:       FERMI CONSTANT
C GAMW:     W WIDTH
C GAMZ:     Z WIDTH
C MZ:       Z MASS
C MW:       W MASS
C VUS:      CKM PARAMETER V_US
C VCB:      CKM PARAMETER V_CB
C VUB/VCB:  RATIO V_UB/V_CB
C 1ST AND 2ND GENERATION:
C MSL1:      SUSY BREAKING MASS PARAMETERS OF LEFT HANDED SLEPTONS 
C MER1:      SUSY BREAKING MASS PARAMETERS OF RIGHT HANDED SLEPTONS 
C MQL1:      SUSY BREAKING MASS PARAMETERS OF LEFT HANDED SUPS
C MUR1:      SUSY BREAKING MASS PARAMETERS OF RIGHT HANDED SUPS
C MDR1:      SUSY BREAKING MASS PARAMETERS OF RIGHT HANDED SDOWNS 
C 3RD GENERATION:
C MSL:      SUSY BREAKING MASS PARAMETERS OF LEFT HANDED STAUS 
C MER:      SUSY BREAKING MASS PARAMETERS OF RIGHT HANDED STAUS 
C MSQ:      SUSY BREAKING MASS PARAMETERS OF LEFT HANDED STOPS
C MUR:      SUSY BREAKING MASS PARAMETERS OF RIGHT HANDED STOPS
C MDR:      SUSY BREAKING MASS PARAMETERS OF RIGHT HANDED SBOTTOMS 
C AL:       STAU TRILINEAR SOFT BREAKING TERMS 
C AU:       STOP TRILINEAR SOFT BREAKING TERMS.
C AD:       SBOTTOM TRILINEAR SOFT BREAKING TERMS.
C MU:       SUSY HIGGS MASS PARAMETER
C M2:       gaugino MASS PARAMETER. 
C
C NNLO (M): =0: USE O(ALPHA_S) FORMULA FOR POLE MASS --> MSBAR MASS
C           =1: USE O(ALPHA_S**2) FORMULA FOR POLE MASS --> MSBAR MASS
C
C ON-SHELL: =0: INCLUDE OFF_SHELL DECAYS H,A --> T*T*, A --> Z*H,
C               H --> W*H+,Z*A, H+ --> W*A, W*H, T*B
C           =1: EXCLUDE THE OFF-SHELL DECAYS ABOVE
C
C ON-SH-WZ: =0: INCLUDE DOUBLE OFF-SHELL PAIR DECAYS PHI --> W*W*,Z*Z*
C           =1: INCLUDE ONLY SINGLE OFF-SHELL DECAYS PHI --> W*W,Z*Z
C
C IPOLE:    =0 COMPUTES RUNNING HIGGS MASSES (FASTER) 
C           =1 COMPUTES POLE HIGGS MASSES 
C
C OFF-SUSY: =0: INCLUDE DECAYS (AND LOOPS) INTO SUPERSYMMETRIC PARTICLES
C           =1: EXCLUDE DECAYS (AND LOOPS) INTO SUPERSYMMETRIC PARTICLES
C
C INIDEC:   =0: PRINT OUT SUMS OF CHARGINO/NEUTRALINO/hdsfermion DECAYS
C           =1: PRINT OUT INDIVIDUAL CHARGINO/NEUTRALINO/hdsfermion DECAYS
C
C NF-GG:    NUMBER OF LIGHT FLAVORS INCLUDED IN THE GLUONIC DECAYS 
C            PHI --> GG* --> GQQ (3,4 OR 5)
C           

C =====================================================================
C =========== BEGINNING OF THE SUBROUTINE FOR THE DECAYS ==============
C !!!!!!!!!!!!!! Any change below this line is at your own risk!!!!!!!!
C =====================================================================

      SUBROUTINE HDEC(TGBET)
\end{verbatim}
 \end{routine}
%%%%%%%%%%%%%%%%%%%%%%%%%%%%%%%%%%%%%%%%%%%%%%%%%%%%%%%%%%%%
%%% Here comes src/docs/E01-Acknowledgements.tex %%%
%%%%%%%%%%%%%%%%%%%%%%%%%%%%%%%%%%%%%%%%%%%%%%%%%%%%%%%%%%%%
%%%%%%%%%%%%%%%%%%%%%%%%%%%%%%%%%%%%%%%%%%%%%%%%%%%%%%%%%%%%%%%%%%%%%

\chapter*{Acknowledgements}
\addcontentsline{toc}{chapter}{Acknowledgements}

P. Gondolo created \ds\ in 1994, took care of its organization,
arranged it for release, and prepared the documentation. He contributed
\cite{bergstrom96} the routines on the supersymmetric spectrum and mixing, the
original calculation of the neutralino relic density without coannihilations,
the direct detection rates and the accelerator bounds. P. Gondolo and J.
Edsj\"o \cite{edsjo97} included coannihilations in the relic density routines.
J. Edsj\"o contributed the package for the neutrino--induced muons from the Sun
and the Earth \cite{edsjo95+}, and organized the routines for annihilations in
the galactic halo, incorporating the code for the gamma-ray continuum by
himself, for the antiprotons \cite{bergstrom99} and the gamma-ray lines
\cite{bergstrom97+} by P. Ullio, and for the positrons by E.  Baltz
\cite{baltz99}. Finally, \ds\ includes adapted versions of (1)
routines by Carena, Quir\'os and Wagner on the Higgs boson masses, (2) routines
from CMLIB (URL: http://www.netlib.org), specifically \ft{dqagse} and its
dependencies, (3) routines from CERNLIB, specifically \ft{gpindp} by X and
\ft{gadap} by T. Johansson.

%%%%%%%%%%%%%%%%%%%%%%%%%%%%%%%%%%%%%%%%%%%%%%%%%%%%%%%%%%%%
%%% Here comes src/docs/E02-References.tex %%%
%%%%%%%%%%%%%%%%%%%%%%%%%%%%%%%%%%%%%%%%%%%%%%%%%%%%%%%%%%%%
%%%%%%%%%%%%%%%%%%%%%%%%%%%%%%%%%%%%%%%%%%%%%%%%%%%%%%%%%%%%%%%%%%%%
%%%%%                       REFERENCES                         %%%%%
%%%%%%%%%%%%%%%%%%%%%%%%%%%%%%%%%%%%%%%%%%%%%%%%%%%%%%%%%%%%%%%%%%%%

\addcontentsline{toc}{chapter}{Bibliography}

\begin{thebibliography}{99}
  
\bibitem{dspaper}
  P.~Gondolo, J.~Edsj{\"o}, L.~Bergstr{\"o}m, P.~Ullio and E.A.~Baltz,
  astro-ph/???????. 

\bibitem{dsoriginal}
  The original papers for the different processes in \ds\ are\\
  \begin{description}
    \item[General MSSM, direct detction] L.~Bergstr{\"o}m and
      P.~Gondolo, ???.
    \item[Relic density] P.~Gondolo and G.~Gelmini, Nucl.\ Phys. ???;
    J.~Edsj{\"o} and P.~Gondolo, Phys.\ Rev.\ {\bfseries D??} (1997) ???.
    \item[Neutrino telescopes]
       L.~Bergstr{\"o}m, J. Edsj{\"o} and P. Gondolo, Phys.\ Rev.\
      {\bfseries D??} (????) ???.
    \item[Positrons]
      E.A.~Baltz and J.~Edsj{\"o}, Phys.\ Rev.\ {\bfseries D??} (????) ???.
    \item[Antiprotons]
      L.~Bergstr{\"o}m, J.~Edsj{\"o} and P.~Ullio, ApJ ??? (????) ???.
    \item[Gamma lines]
      L.~Bergstr{\"o}m and P.~Ullio, ???, x2.
    \item[Continuous gammas]
      L.~Bergstr{\"o}m, J.~Edsj{\"o} and P.~Ullio, ???.
  \end{description}

\bibitem{jkg} G.~Jungman, M.~Kamionkowski and K.~Griest, Phys.\ Rep.\
  {\bf 267} (1996) 195.

\bibitem{mssm} H.E. Haber and G.L. Kane, Phys.\ Rep.\ 117 (1985) 75;
  J.F. Gunion and H.E. Haber, Nucl.\ Phys.\ B272 (1986) 1 [Erratum: ibid.\ B402
  91993) 567]; H.E. Haber and D. Wyler, Nucl.\ Phys.\ B323 (1989) 267.

\bibitem{HaberKane} H.E. Haber and G.L. Kane, Phys.\ Rep.\ 117 (1985) 75.

\bibitem{GuHa86}
  J.F.~Gunion and H.E. Haber, Nucl.\ Phys.\ B272 (1986) 1 [Erratum: ibid.\ B402
  91993) 567].

\bibitem{archange}
  Derived from the rules in Fig.~83 in Ref.~\cite{HaberKane} or
  directly from the Lagrangian.

\bibitem{Mandl}
  F.~Mandl and G.~Shaw, {\em Quantum Field Theory}, John Wiley \&
  Sons, 1984.

\bibitem{dimopoulos95} S. Dimopoulos and D. Sutter, Nucl.\ Phys.\ B465
  (1995) 23.
  
\bibitem{ellis91} J. Ellis, G. Ridolfi and F. Zwirner, Phys.\ Lett.\ B257
  (1991) 83; ibid.\ B262 (1991) 477.
  
\bibitem{brignole91} A. Brignole, J. Ellis, G. Ridolfi and F. Zwirner, Phys.\ 
  Lett.\ B271 (1991) 123.
  
\bibitem{drees92} M. Drees, M. Nojiri, Phys.\ Rev.\ D45 (1992) 2482.

\bibitem{carena95} M. Carena, Espinosa, M. Quir\'os, and C. Wagner, Phys.\
  Lett.\ B355 (1995) 209.
  
\bibitem{carena96} M. Carena, M. Quir\'os, and C. Wagner, Nucl.\ Phys.\ B461
  (1996) 407.

\bibitem{drees97} M. Drees, M. Nojiri, Yamada, (1997) astro-ph/970129

\bibitem{bertolini91} S. Bertolini, F. Borzumati, A. Masiero and G. Ridolfi,
Nucl. Phys. B353 (1991) 591.

\bibitem{abbiendi99} Abbiendi et al. (OPAL Collab.), Europ.\ Phys.\ J.\ C7
  (1999) 407.

\bibitem{gao99} Gao and Gay (ALEPH Collab.), in ``High Energy Physics 99,''
  Tampere, Finland, July 1999.
  
\bibitem{carr98} J. Carr et al. (ALEPH Collab.), talk to LEPC, 31 March 1998
  (URL: http://alephwww.cern.ch/ALPUB/seminar/carrlepc98/index.html).
  
\bibitem{acciarri96} Acciarri et al. (L3 Collab.), Phys.\ Lett.\ B377 (1996)
  289.
  
\bibitem{decamp92} Decamp et al. (ALEPH Collab.), Phys.\ Rep.\ 216 (1992) 253.
  
\bibitem{hidaka91} Hidaka, Phys.\ Rev.\ D44 (1991) 927.
  
\bibitem{acciarri95} Acciarri et al. (L3 Collab.), Phys.\ Lett. B350 (1995)
  109.
  
\bibitem{buskulic96} Buskulic et al. (ALEPH Collab.), Zeitschrift f\"ur Physik
  C72 (1996) 549.
  
\bibitem{acciarri98} Acciarri et al. (L3 Collab.), Europ.\ Phys.\ J.\ C4 (1998)
  207.
  
\bibitem{abbiendi99b} Abbiendi et al. (OPAL Collab.), Europ.\ Phys.\ J.\ C8
  (1999) 255.
  
\bibitem{abachi95} Abachi et al. (D0 Collab.), Phys.\ Rev.\ Lett.\ 75 (1995)
  618.

\bibitem{abe97} Abe et al. (CDF Collab.), Phys.\ Rev.\ D56 (1997) R1357.
  
\bibitem{abe92} Abe et al. (CDF Collab.), Phys.\ Rev.\ Lett.\ 69 (1992) 3439.
  
\bibitem{abe96} Abe et al. (CDF Collab.), Phys.\ Rev.\ Lett.\ 76 (1996) 2006.
  
\bibitem{barate98} Barate et al. (ALEPH Collab.), Phys.\ Lett.\ B433 (1998)
  176.
  
\bibitem{pdg99} C. Caso et al. (Particle Data Group), Europ.\ Phys.\ J.\ C3
  (1998) 1, and 1999 partial update for edition 2000 (URL: http://pdg.lbl.gov)

\bibitem{haber97} H.E. Haber, in {\it Perspectives on Higgs Physics II}, ed.\
  G. Kane (World Scientific, Singapore, 1997). 
  
\bibitem{bergstrom96} L. Bersgstr\"om and P. Gondolo, Astropart.\ Phys.\ 5
  (1996) 263.

\bibitem{edsjo97} J. Edsj\"o and P. Gondolo, Phys.\ Rev.\ D56 (1997) 1879.
  
\bibitem{edsjo95+} J. Edsj\"o, Diploma Thesis, Uppsala University preprint
  TSL/ISV-93-0091 (1993); J. Edsj\"o, Nucl.\ Phys.\ Proc.\ Suppl.\ 43 (1995)
  265; J. Edsj\"o and P. Gondolo, Phys.\ Lett.\ B357 (1995) 595; L.
  Bergstr\"om, J. Edsj\"o, and P. Gondolo, Phys.\ Rev.\ D55 (1997) 1765;
  ibid.\ D58 (1998) 103519.

\bibitem{bergstrom99} L. Bergstr\"om, J. Edsj\"o, and P. Ullio,
  astro-ph/9902012. 

\bibitem{bergstrom97+} L. Bergstr\"om and P. Ullio, Nucl.\ Phys.\ B504 (1997)
  27; Phys.\ Rev.\ D57 (1998) 1962.
  
\bibitem{baltz99} E. Baltz and J. Edsj\"o, Phys.\ Rev.\ D59 (1999) 023511.

\bibitem{triangle}
N.~Bahcall, J.P.~Ostriker, S.~Perlmutter and P.J.~Steinhardt,
Science {\bf 284}, 1481 (1999).

\bibitem{lbreview} L. Bergstr\"om, Rep. Prog. Phys. {\bf 63} (2000) 793.

\bibitem{goldberg} H.~Goldberg, Phys.\ Rev.\ Lett.\ {\bf 50} (1983)
1419.

\bibitem{krauss}L.M.~Krauss, Nucl.\ Phys.\ {\bf B227} (1983) 556.

\bibitem{ellis} J.~Ellis et al., Nucl.\ Phys.\ {\bf B238}
   (1984) 453.

\bibitem{raffelt} G. Raffelt, Nucl. Phys. Proc. Suppl. {\bf 77} (1999)
456.

\bibitem{bg} L.~Bergstr{\"o}m and P.~Gondolo, Astrop.\ Phys. {\bf 5}
(1996) 263.

\bibitem{haberkane}
H.E. Haber and G.L. Kane, Phys. Rep. {\bf 117} (1985) 75;
J.F. Gunion and H.E. Haber, Nucl. Phys. {\bf B272} (1986) 1 [Erratum-ibid.
{\bf B402} (1993) 567].

\bibitem{dimo} S. Dimopoulos and D. Sutter, Nucl. Phys.
{\bf B465} (1995 23.

\bibitem{feynhiggs1} S. Heinemeyer, W. Hollik and G. Weiglein,
     Comp. Phys. Comm. {\bf 124} (2000) 76; hep-ph/0002213.

\bibitem{feynhiggs2} S. Heinemeyer, W. Hollik and G. Weiglein, Phys.
Rev.  {\bf D58} (1998) 091701; Eur.  Phys.  J. {\bf C9} (1999) 343;
Phys.  Lett.  {\bf B455} (1999) 179.

\bibitem{zwirner} J. Ellis, G. Ridolfi and F. Zwirner, Phys.\ Lett.\ {\bf B257}
   (1991) 83; ibid.\ {\bf B262} (1991) 477;
  A. Brignole, J. Ellis, G. Ridolfi and F. Zwirner, Phys.\
   Lett.\ {\bf B271} (1991) 123.

\bibitem{NeuLoop1}
M.~Drees, M.M.~Nojiri, D.P.~Roy and Y.~Yamada, Phys. Rev. {\bf D56} (1997)
276 [hep-ph/9701219].

\bibitem{NeuLoop2}
D.~Pierce and A.~Papadopoulos, Phys.\ Rev.\ {\bf D50} (1994) 565,
Nucl.\ Phys.\ {\bf B430} (1994) 278; A.B.~Lahanas, K.~Tamvakis and
N.D.~Tracas, Phys.\ Lett.\ {\bf B324} (1994) 387.

\bibitem{B-functions}
M.~Drees, K.~Hagiwara and A.~Yamada, Phys.\ Rev.\ {\bf D45} (1992)
1725.

\bibitem{PDG} Particle Data Group,
D.E. Groom et al, The European Physical Journal {\bf C15} (2000) 1.

\bibitem{bertolini}
S. Bertolini, F. Borzumati, A. Masiero and G. Ridolfi,
Nucl. Phys. B353 (1991) 591.

\bibitem{higgsbound} M. Kado (Aleph Collaboration), Talk given
at the XXXVth Recontre de Moriond, CERN/ALEPH PUB-2000-6.

\bibitem{CMBdetexp}
P. de Bernardis et al. Nature {\bf 404} (2000) 995;
S. Hanany et al., astro-ph/0005123.
\bibitem{cmbdettheory}
\rfprep\nnn  Lange A E et al.;2000;astro-ph/0005004
\rfprep\nn Tegmark M\dualand\nn Zaldarriaga M;2000;astro-ph/0004393
\rfprep\nn  Balbi A et al.;2000;astro-ph/0005124
\rfprep\nn Hu W, \nn Fukugita M,
\nn Zaldarriaga M\multiand\nn Tegmark M;2000;astro-ph/0006436
\rfprep\nn Jaffe A et al.;2000;astro-ph/0007333
\rfprepend\nn Kinney W, \nn Melchiorri A\multiand\nn Riotto 
A;2000;astro-ph/0007375
\bibitem{sn1a} S. Perlmutter et al., Astrophys. J. {\bf 517} (1999) 565;
P.M. Garnavich et al., Astrophys. J.  {\bf 509} (1998) 74.
\bibitem{griestseckel} K.~Griest and D.~Seckel, Phys.\ Rev.\ {\bf 
D43} (1991) 3191.


\bibitem{relcalc}
   K.~Griest, Phys.\ Rev.\ {\bf D38} (1988) 2357 [erratum ibid {\bf
   D39} (1989) 3802]; J.~Scherrer and M.S.~Turner, Phys.\ Rev.\ {\bf
   D33} (1986) 1585 [erratum ibid {\bf D34} (1986) 3263]; M.~Srednicki,
   R.~Watkins and K.A.~Olive, Nucl.\ Phys.\ {\bf B310} (1988) 693;
   K.~Griest, M.~Kamionkowski and M.S.~Turner, Phys.\ Rev.\ {\bf D41}
   (1990) 3565; G.B.~Gelmini, P.~Gondolo, and E.~Roulet, Nucl.\ Phys.\
   {\bf B351} (1991) 623; A.~Bottino et al., Astropart.\ Phys.\ {\bf 1}
   (1992) 61, ibid.\ {\bf 2} (1994) 67; R.~Arnowitt and P.~Nath, Phys.\
   Lett.\ {\bf B299} (1993) 58, {\bf B307} (1993) 403(E), Phys.\ Rev.\
   Lett.\ {\bf 70} (1993) 3696; H.~Baer and M.~Brhlik, Phys.\ Rev.\
   {\bf D53} (1996) 597.

\bibitem{McDonald}
J.~McDonald, K.A.~Olive and M.~Srednicki, Phys.\ Lett.\ {\bf B283}
(1992) 80.

\bibitem{MizutaYamaguchi}
S.~Mizuta and M.~Yamaguchi, Phys.\ Lett.\ {\bf B298} (1993) 120.

\bibitem{SWO}
M.~Srednicki, R.~Watkins and K.A.~Olive, Nucl.\ Phys.\ {\bf B310} (1988)
693.

\bibitem{DreesNojiri}
M.~Drees and M.~Nojiri, Phys.\ Rev.\ {\bf D47} (1993) 376.

\bibitem{KolbTurner}
E.W.~Kolb and M.S.~Turner, \emph{The Early Universe},
Addison-Wesley (1990).

\bibitem{GondoloGelmini}
P. Gondolo and G. Gelmini, Nucl. Phys. B360 (1991) 145.

\bibitem{coann} J.~Edsj{\"o} and P.~Gondolo,
\prd{56}{1997}{1879} [\hepph{9704361}].

\bibitem{ellisfalk}
J.~Ellis, T.~Falk and K.~A.~Olive,
Phys.\ Lett.\  {\bf B444} (1998) 367;
J.~Ellis, T.~Falk, K.~A.~Olive and M.~Srednicki,
Astropart.\ Phys.\  {\bf 13} (2000) 181.

\bibitem{reduce}
{\sc Reduce} 3.5. A.C.~Hearn, RAND, 1993.

\bibitem{BEU} L.~Bergstr\"om, J.~Edsj\"o and P.~Ullio,
astro-ph/9804050, Phys.Rev. D58 (1998) 083507.

\bibitem{lpj}  L.~Bergstr{\"om}, P.~Ullio and J.~Buckley,
Astrop.\ Phys.\ in press, astro-ph/9712318.

\bibitem{beg} L.~Bergstr{\"o}m, J.~Edsj{\"o} and P.~Gondolo,
\prd{55}{1997}{1765}.

\bibitem{eg} J.~Edsj{\"o} and P.~Gondolo, Phys.\ Lett.\ {\bf B357}
   (1995) 595.

\bibitem{joakimthesis}
J.~Edsj{\"o}, PhD Thesis, hep-ph/9704384.

\bibitem{dkpop}
L.~Bergstrom, T.~Damour, J.~Edsjo, L.~M.~Krauss and P.~Ullio,
%``Implications of a new solar system population of neutralinos on 
indirect detection rates,''
JHEP {\bf 9908}, 010 (1999)
[hep-ph/9905446].
%%CITATION = HEP-PH 9905446;%%

\bibitem{goodmanwitten} M.W. Goodman and E. Witten, Phys. Rev. {\bf D31}
(1985) 3059.


\bibitem{bottino} A. Bottino et al., Phys. Lett. B402 (1997) 113.

\bibitem{Gould87} A.~Gould, \apj{321}{1987}{571}.

\bibitem{Gould91}
A.~Gould, \apj{368}{1991}{610}

\bibitem{Gould92}
A.~Gould, \apj{388}{1992}{338}.

\bibitem{EllisFlores} J. Ellis and R. Flores, Nucl. Phys. B307 (1988)
883; Phys. Lett B263 (1991) 259.

\bibitem{engel} J. Engel, Phys. Lett. {\bf B264} (1991) 114.

\bibitem{efflagrange}
K. Griest, Phys. Rev. D28 (1988) 2357; R. Barbieri, M. Frigeni and
G.F. Giudice, Nucl. Phys. B313 (1989) 725; G. Gelmini, P. Gondolo and
E. Roulet, Nucl. Phys. B351 (1991) 623; M. Kamionkowski, Phys. Rev.
D44 (1991) 3021; A. Bottino et al., Astropart. Phys. 2 (1994) 77.

\bibitem{Gasser} J. Gasser, H. Leutwyler and M.E. Sainio,
Phys. Lett. B253 (1991) 252.

\bibitem{SMC} D. Adams et al,     Phys. Lett. B357 (1995) 248.

\bibitem{jaffe} R.L. Jaffe and A. Manohar, Nucl. Phys.
{\bf 337} (1990) 509.

\bibitem{EngelVogel} J. Engel and P. Vogel, Phys. Rev. D40 (1989) 3132;
J. Engel, S. Pittel and P. Vogel, Int. J. Mod. Phys. E1 (1992) 1.

\bibitem{pythia}
T.~Sj\"{o}strand, \cpc{82}{1994}{74};
T.~Sj\"{o}strand, {\em PYTHIA 5.7 and JETSET 7.4. Physics and Manual},
CERN-TH.7112/93, \hepph{9508391} (revised version).


\bibitem{RS}
S.~Ritz and D.~Seckel, Nucl.\ Phys.\ {\bf B304} (1988) 877.

\bibitem{Edpre}

J.~Edsj\"{o}, Diploma Thesis, Uppsala University preprint
TSL/ISV-93-0091 (ISSN 0284-2769), can be downloaded from
\texttt{http://www.physto.se/\~{ }edsjo/articles/index.html}.

J.~Edsj\"{o}, in {\em Trends in Astroparticle Physics}, Stockholm,
Sweden, 1994, eds.\ L.~Bergstr\"om, P.~Carlson, P.O.~Hulth and
H.~Snellman, Nucl.\ Phys.\ (Proc.\ Suppl.) {\bf B43} (1995) 265.


\bibitem{Angdist}
J.~Edsj{\"o} and P.~Gondolo, Phys.~Lett.~{\bf B357} (1995) 595.

\bibitem{neutrinos}
L.~Krauss, \emph{Cold dark matter candidates and the solar neutrino
problem}, Harvard preprint HUTP-85/A008a (1985);\\
W.H.~Press and D.N~Spergel, \apj{296}{1985}{679};\\
J.~Silk, K.~Olive and M.~Srednicki, \prl{55}{1985}{257};\\
L.~Krauss, M.~Srednicki and F.~Wilczek, \prd{33}{1986}{2079};\\
T.~Gaisser, G.~Steigman and S.~Tilav, \prd{34}{1986}{2206};\\
K.~Griest and S.~Seckel, \npb{283}{1987}{681},
erratum \ibid{296}{1988}{1034};\\
L.M.~Krauss, K.~Freese, D.N.~Spergel and W.H.~Press, \apj{299}{1985}{1001};\\
J.~Hagelin, K.~Ng and K.~Olive, \plb{180}{1987}{375};\\
K.~Freese, \plb{167}{1986}{295};\\
M.~Kamionkowski, \prd{44}{1991}{3021};\\
F.~Halzen, T.~Stelzer and M. Kamionkowski, \prd{45}{1992}{4439};\\
A.~Bottino, V.~de Alfaro, N.~Fornengo, G.~Mignola and
M.~Pignone, \plb{265}{1991}{57};
\noindent A.~Bottino, N.~Fornengo, G.~Mignola, L.~Moscoso,
\app{3}{1995}{65} [\hepph{9408391}];\\
R.~Gandhi, J.L.~Lopez, D.V.~Nanopoulos, K.~Yuan and A. Zichichi,
\prd{49}{1994}{3691} [\astroph{9309048}];\\
L.~Bergstr{\"o}m, J.~Edsj{\"o} and P.~Gondolo,
\prd{55}{1997}{1765} [\hepph{9607237}];
L.~Bergstrom, J.~Edsjo and P.~Gondolo,
%``Indirect detection of dark matter in km-size neutrino telescopes,''
Phys.\ Rev.\  {\bf D58}, 103519 (1998)
[hep-ph/9806293].
%%CITATION = HEP-PH 9806293;%%

\bibitem{halzen} F. Halzen, Comments Nucl. Part. Phys. {\bf 22} (1997) 155.

\bibitem{neuprod}

G.F.~Giudice and E.~Roulet, Nucl.\ Phys.\ {\bf B316} (1989) 429.

F.~Halzen, T.~Stelzer and M.~Kamionkowski, Phys.\ Rev.\ {\bf D45} (1992) 4439.

M.~Drees, G.~Jungman, M.~Kamionkowski and M.M.~Nojiri, Phys.\ Rev.\ {\bf D49}
(1994) 636.

R.~Gandhi, J.L.~Lopez, D.V.~Nanopoulos, K.~Yuan and A.~Zichichi,
Phys.\ Rev.\ {\bf D49} (1994) 3691.

A.~Bottino, N.~Fornengo, G.~Mignola and L.~Moscoso, Astropart.\ Phys.\
{\bf 3} (1995) 65.

G.~Jungman and M.~Kamionkowski, Phys.\ Rev.\ {\bf D51} (1995) 328.

V.~Berezinsky, A.~Bottino, J.~Ellis, N.~Fornengo, G.~Mignola and
S.~Scopel, hep-ph/9603342.

\bibitem{begnu2}L. Bergstr\"om, J. Edsj\"o and P. Gondolo, Phys. Rev.
{\bf D58}

\bibitem{kamsad} M. Kamionkowski, G. Jungman, K. Griest and
B. Sadoulet, Phys. Rev. Lett. {\bf 74} (1995) 5174.

\bibitem{EG} J. Edsj\"o and P. Gondolo, Phys. Lett. {\bf B357} (1995) 595.

\bibitem{BEK} L. Bergstr\"om, J. Edsj\"o and M. Kamionkowski, Astropart.
Phys. {\bf 7} (1997) 147.

\bibitem{dk1}
T.~Damour and L.M.~Krauss,  \prl{81}{1998}{5726} [\astroph{9806165}].

\bibitem{dk2}
T.~Damour and L.M.~Krauss,  \prd{59}{1999}{063509} [\astroph{9807099}].

\bibitem{otherpop}
G.~Steigman, C.L.~Sarazin, H.~Quintana and J.~Faulkner, \apj{83}{1978}{1050};\\
K.~Griest, \prd{37}{1988}{2703};\\
A.~Gould, J.A.~Frieman and K.~Freese, \prd{39}{1989}{1029};\\
J.I.~Collar, \prd{59}{1999}{063514} [\astroph{9808058}].

\bibitem{Gould91}
A.~Gould, \apj{368}{1991}{610}.

\bibitem{gouldnew}
A.~Gould and S.~M.~Khairul Alam,
%``Can heavy WIMPs be captured by the earth?,''
astro-ph/9911288.
%%CITATION = ASTRO-PH 9911288;%%

\bibitem{jnb}
J.N.~Bahcall and M.H. Pinsonneault,
\rmp{64}{1992}{885}.


\bibitem{EncBrit}
\emph{The Earth: its properties, composition, and structure.}
Britannica CD, Version 99 \copyright 1994--1999.
Encyclop{\ae}dia Britannica, Inc.

\bibitem{pbar} L. Bergstr\"om, J. Edsj\"o and P. Ullio, Astrophys. J. {\bf 526}
(1999) 215.

\bibitem{gaisserpbar}J.W. Bieber et al., Phys. Rev. Lett. {\bf 83} (1999) 674.

\bibitem{ub} Ullio, P.\ \& Bergstr\"om, L. 1998,
Phys.\ Rev., {D57},  1962.

\bibitem{bua}Drees, M., Jungman, G., Kamionkowski, M.
\& Nojiri, M.M. 1994, Phys. Rev., D49, 636.

\bibitem{lp} L.~Bergstr{\"o}m and P.~Ullio, Nucl.\ Phys.\ {\bf B504}
(1997) 27; see also Z. Bern, P. Gondolo and M. Perelstein,
Phys. Lett. {\bf B411} (1997) 86.

\bibitem{navarro} J.F.~Navarro, C.S.~Frenk and S.D.M.~White,
Ap.\ J.\ {\bf 462} (1996) 563.

\bibitem{clumpy} Bergstr{\"o}m, L., Edsj{\"o}, J., Gondolo, P.\ \&
Ullio, P. 1999, Phys.\ Rev., {D59}, 043506.

\bibitem{pieroclumpy} P. Ullio, astro-ph/9904086.

\bibitem{Berezinskii} Berezinskii, V.S., Bulanov, S.,
Dogiel, V., Ginzburg, V.\ \& Ptuskin, V. 1990, {\em Astrophysics
of cosmic rays}, North-Holland, Amsterdam.

\bibitem{Gaisserbook} Gaisser, T.K. 1990, {\em Cosmic rays and
particle physics}, Cambridge University Press, Cambridge.

\bibitem{Chardonnet}
Chardonnet, P., Mignola, G., Salati, P.\ \&
Taillet, R. 1996, Phys.\ Lett., {B384}, 161.

\bibitem{bottinolast} Bottino, A., Donato, F.,
Fornengo, N. \& Salati, P. 1998, Phys.\ Rev.\ {D58} 123503.

\bibitem{fisk}
Fisk, L.A. 1971, J. Geophys. Res., {76}, {221}.

\bibitem{baltz}  E.A. Baltz and J. Edsj{\"o},
Phys.\ Rev. {\bf D59} (1999) 023511.

\bibitem{bub} Bergstr\"om, L.,
Ullio, P.\ \& Buckley, J.H.\ 1998, Astrop.\ Phys.\ {9}, 137.

\bibitem{wlg}
W.R.~Webber, M.A.~Lee and M.~Gupta, \apj{390}{1992}{96}.
\bibitem{kamturner}
M.~Kamionkowski and M.~S.~Turner, \prd{43}{1991}{1774}.

\bibitem{energy-loss}
M.S.~Longair, {\em High Energy Astrophysics}, (Cambridge University Press, New
York, 1994), Vol.\ 2, Chap.\ 19.
\bibitem{MoskStrong98}
I.V.~Moskalenko and A.W.~Strong, \apj{493}{1998}{694}.

\bibitem{binney} W. Dehnen and J. Binney, astro-ph/9612059 (1997).
\bibitem{carlberg} R. Carlberg, Astrophys. J. {\bf 433} (1994) 468.

\bibitem{kravtsov} A.V.~Kravtsov et al., Ap.\ J.\ in press, astro-ph/9708176.

\bibitem{moore} B. Moore et al., astro-ph/9709051, Astrophys. J. Lett.,
submitted.
\bibitem{bere} V.S.~Berezinsky, A.V.~Gurevich and K.P.~Zybin,
Phys. Lett. {\bf B294} (1992) 221.
\bibitem{flores} R.A. Flores and J.R. Primack, Astrophys. J. {\bf 427} (1994)
L1.
\bibitem{bs} A. Burkert and J. Silk, astro-ph/9707343 (1997).
\bibitem{makino} T. Fukushige and J. Makino, Astrophys. J. {\bf 487} (1997) L9.
\bibitem{evans}N.W. Evans and J.L. Collett, astro-ph/9702085.
\bibitem{self} %\cite{Spergel:2000mh}
D.~N.~Spergel and P.~J.~Steinhardt,
%``Observational evidence for self-interacting cold dark matter,''
Phys.\ Rev.\ Lett.\  {\bf 84} (2000) 3760.
%%CITATION = ASTRO-PH 9909386;%%

\bibitem{annihil} %\cite{Kaplinghat:2000vt}
M.~Kaplinghat, L.~Knox and M.~S.~Turner,
%``Annihilating the cold dark matter cusp crisis,''
astro-ph/0005210.
%%CITATION = ASTRO-PH 0005210;%%

\bibitem{gs}
P.~Gondolo and J.~Silk,
%``Dark matter annihilation at the galactic center,''
Phys.\ Rev.\ Lett.\  {\bf 83} (1999) 1719
[astro-ph/9906391].
%%CITATION = ASTRO-PH 9906391;%%
                                %
\bibitem{spike}
P.~Gondolo,
%``Either neutralino dark matter or cuspy dark halos,''
hep-ph/0002226.
%%CITATION = HEP-PH 0002226;%%

\bibitem{pierothesis} P. Ullio, PhD thesis, Physics Department,
Stockholm University, 1999.

\bibitem{kochanek} C.S. Kochanek, Astrophys. J. {\bf 457} (1996) 228.
\bibitem{lin} D.N.C. Lin, B.F.~Jones and A.R.~Klemola, Astrophys. J.
{\bf 439} (1995) 652.
\bibitem{kerr} F.J. Kerr and D. Lynden-Bell, MNRAS {\bf 221} (1986) 1023.
\bibitem{reid} M.J. Reid, ARA\&A {\bf 31} (1993) 345.
\bibitem{oll} R.P. Olling and M.R. Merrifield, aspro-ph/9711157, to 
appear in proceedings of the Workshop on Galactic Halos, Santa Cruz, 
August 1997 (ASP conference Series).
\bibitem{kui} K. Kuijken and G. Gilmore Astrophys. J. {\bf 367} (1991) L9.
\bibitem{gould} A. Gould, MNRAS {\bf 244} (1990) 25.


\bibitem{oldcontga}
J. Silk and M.~Srednicki, Phys. Rev. Lett {\bf 53} (1984) 624;\\
J.~Silk and H.~Bloemen, Astrophys. J. {\bf 313} (1987) L47;\\
S.~Rudaz and F.W.~Stecker, Astrophys. J. {\bf 325} (1988) 16;\\
F.W.~Stecker and A.~Tylka, Astrophys. J. {\bf 343} (1989) 169;\\
H.-U. Bengtsson, P. Salati and J. Silk, Nucl Phys.
{\bf B346} (1990) 129;\\
E.~Diehl, G.L.~Kane, C.~Kolda and J.D.~Wells, Phys. Rev. {\bf D52} (1994) 
4223;\\
P.~Chardonnet, P.~Salati, J.~Silk, I.~Grenier, and G.~Smoot,
Astrophys. J. {\bf 454} (1995) 774.

\bibitem{charm}
M.~Srednicki, S.~Theisen and J.~Silk, Phys.\ Rev.\ Lett.\ {\bf 56}, 
263 (1986); Erratum-ibid. {\bf 56}, 1883 (1986);\\
S.~Rudaz, Phys.\ Rev.\ Lett.\ {\bf 56}, 2128 (1986).

\bibitem{oldlines}
L. Bergstr\"om and H. Snellman, Phys. Rev. {\bf D37} (1988) 3737;\\ 
S. Rudaz, Phys. Rev. {\bf D39} (1989) 3549; \\
G.F. Giudice and K. Griest, Phys. Rev. {\bf D40} (1989) 2549;\\ 
A. Bouquet, P. Salati and J. Silk, Phys. Rev. {\bf D40} (1989) 3168;\\ 
V. Berezinsky, A. Bottino and V. de Alfaro, Phys. Lett. {\bf B274} (1992) 
122;\\
M. Urban et al., Phys. Lett. {\bf B293} (1992) 149;\\
L. Bergstr\"om and J. Kaplan, Astropart. Phys. {\bf 2} (1994) 261.

% F. Stecker and A. Tylka, Astrophys. J. {\bf 336} (1989) L51; this is pbar

\bibitem{jkline}
G. Jungman and M. Kamionkowski, Phys. Rev. {\bf D51} (1995) 3121.

\bibitem{fujikawa}
K.~Fujikawa, Phys.\ Rev. {\bf D7} (1973) 393.

\bibitem{galga}
M.S.~Turner, Phys.\ Rev. {\bf D34} (1986) 1921;\\
J.R.~Ipser and P.~Sikivie, Phys.\ Rev. {\bf D35} (1987) 3695;\\
K.~Freese and J.~Silk, Phys.\ Rev. {\bf D40} (1989) 3828;\\
V. Berezinsky, A. Bottino and G.~Mignola, Phys. Lett. {\bf B325} (1994) 136.

\bibitem{clumpyga}
G.~Lake, Nature {\bf 346} (1990) 39;\\
J.~Silk and A.~Stebbins, Astrophys. J. {\bf 411} (1993) 439;\\
C.~Calcaneo-Roldan and B.~Moore, Phys.\ Rev. {\bf D62} (2000) 123005.

\bibitem{gahalo}
L. Bergstr\"om, J. Edsj\"o and P.~Ullio, Phys.\ Rev.\ D {\bf 58}, 
(1998) 083507.

\bibitem{clumpybeg}
L. Bergstr\"om, J. Edsj\"o and C. Gunnarsson, Phys.\ Rev.\ D {\bf 63}, 
083515 (2001).

\bibitem{extergal}
E.~A.~Baltz, C.~Briot, P.~Salati, R.~Taillet and J.~Silk,
Phys.\ Rev.\ D {\bf 61}, 023514 (2000).

\bibitem{extragal}
D.B. Cline and Y.-T. Gao,  Astronomy and Astrophys. {\bf 231} (1990) L23;\\
Y.-T. Gao, F.W. Stecker and D.B. Cline, Astronomy and Astrophys. {\bf 249},
1 (1991).

\bibitem{extragalbeu}
L. Bergstr\"om, J. Edsj\"o and P.~Ullio, 
Phys.\ Rev.\ Lett. (2001) in press.


\bibitem{GleesonAxford} Gleeson, L.J.\ \&
   Axford, W.I. 1967, ApJ, {149}, L115.

\bibitem{kamion91}M. Kamionkowski, \prd{44}{1991}{3021}.

\bibitem{thermal-freeze-out}
X.~Chen, M.~Kamionkowski and X.~Zhang, Phys.\ Rev.\ {\bfseries D64}
(2001) 021302;
S.~Hofmann, D.J.~Schwarz and H.~Stocker, Phys.\ Rev.\ {\bfseries D64}
(2001) 083507.

\bibitem{feynhiggs}
Feynhiggs.

\bibitem{feynhiggsfast}
FeynHiggsFast.

\bibitem{higgsqcd}
higgsqcd.

\bibitem{chardonnay}
chardonnay.

\bibitem{bottinopbar}
bottinopbar.

\bibitem{linejk}
linejk.

\bibitem{Gould321}
Gould321.

\bibitem{paoloprivate}
P.~Gondolo, private communication.

\bibitem{GriestSeckel}
K.~Griest and D.~Seckel, Phys.\ Rev.\ {\bf D43} (1991) 3191.

\bibitem{earthcomp}
W.F.~Mcdonough, Treatise on Geochemistry, Vol 2, Elsevier, 2003. (The values for the Earth composition are very close to those in The Encyclopedia of
Geochemistry, Eds. Marshall and Fairbridge, Klower Acadmic Publ,
1998.)

\bibitem{form2f}
Perl script \codeb{form2f} to convert from \code{Form} output
to \code{Fortran} output, written by J.~Edsj\"o.

\bibitem{ggulio}
L.~Bergstr\"om and P.~Ullio, Nucl.\ Phys.\ {\bfseries B504} (1997) 27.

\bibitem{zgullio}
P.~Ullio and L.~Bergstr\"om, Phys.\ Rev.\ {\bfseries D57} (1998) 1962.

\bibitem{bub}
L.~Bergstr\"om, P.~Ullio and J.~Buckley, Astrop.\ Phys.\ {\bfseries 9}
(1998) 137.

\bibitem{sfcoann}
J.~Edsj\"o, ...
{\bfseries SFCOANN}.

\bibitem{bsgsm} P.~Gambino and M.~Misiak, Nucl.~Phys. {\bf B611} (2001) 338.
%\emph{Quark mass effects in anti-B --> X(s) gamma}
%Updated experimental values in the hep-ph version, hep-ph/0104034.

\bibitem{bsgmagic} A.~J.~Buras, A.~Czarnecki, M.~Misiak and J.~Urban,  Nucl.~Phys.  {\bf B631} (2002) 219.
%\emph{Completing the NLO QCD calculation of anti-B --> X(s) gamma}

\bibitem{bsgh2} M.~Ciuchini, G.~Degrassi, P.~Gambino and G.~F.~Giudice, Nucl.~Phys. {\bf B527}  (1998) 21.
%\emph{Next-to-leading QCD corrections to B --> X(s) gamma: Standard Model and two Higgs %doublet model} 

\bibitem{bsgtan} G.~Degrassi, P.~Gambino and G.~F.~Giudice, JHEP {\bf 0012} (2000) 009.
%\emph{B --> X(s) gamma in supersymmetry: large contributions beyond the leading order}

\bibitem{bsgsusy} M.~Ciuchini, G.~Degrassi, P.~Gambino and G.~F.~Giudice, Nucl.~Phys. {\bf B534}  (1998) 3.
%\emph{Next-to-leading QCD corrections to B --> X(s) gamma in supersymmetry}

\bibitem{bsgcompare} K. Okumura and L. Roszkowski, hep-ph/0212007, Proceedings SUSY02.
%\emph{ B --> X(s) gamma in minimal supersymmetric standard model with general flavor mixing}

\bibitem{pdg02} 
  K.~Hagiwara et al., Phys.~Rev. {\bf D66} (2002) 010001.

\bibitem{dbar} 
 F. Donato, N. Fornengo and P. Salati, Phys. Rev. {\bf D62} (2000)
 043003.

\bibitem{earth-diff}
J.~Lundberg and J.~Edsj\"o, Phys.\ Rev.\ {\bfseries D69}�(2004)
123505. [astro-ph/0401113]

\bibitem{gould-diff}
A.~Gould, 
\emph{Gravitational diffusion of solar system WIMPs},
Astrophys.\ J.\ {\bfseries 368} (1991) 610.

\bibitem{farinella}
P.~Farinella, C.~Froeschl\'e,
  C.~Froeschl\'e, R.~Gonczi, G.~Hahn, A.~Morbidelli and G.B.~Velsecchi,
  \emph{Asteroids falling into the Sun},
  Nature {\bfseries 371} (1994) 314).

\bibitem{gould-conserv}
A.~Gould and S.M.K~Alam,
\emph{Can heavy WIMPs be captured by the Earth?},
Astrophys.\ J.\ {\bfseries 549} (2001) 72.

\bibitem{bp2000}
{\bfseries BP2000 solar model.}


\end{thebibliography}

%%%%%%%%%%%%%%%%%%%%%%%%%%%%%%%%%%%%%%%%%%%%%%%%%%%%%%%%%%%%%%%%%%%%


\end{document}
